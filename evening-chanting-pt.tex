\documentclass[
  babelLanguage=portuguese,
  final,
  %showtrims,
  %showwirebinding,
]{chantingbook}

\usepackage{local}
\definecolor{textbody}{gray}{0}

\title{Livro de Cânticos}
\subtitle{}

\begin{document}

\frontmatter

\thispagestyle{empty}

{\centering
\mbox{}
\vfill

\parttitlefont\color{chaptertitle}
%\addfontfeature{LetterSpace=2.0}

Livro de Cânticos

\vspace*{4\baselineskip}

\vfill

\mbox{}
}

\mainmatter

\usePsMarksPartOnly

\artopttrue

\chapter{Dedicação de ofertas}

[Yo so] bha꜕gavā a꜕rahaṃ sammāsambuddho\\
Svākkhā꜓to yena bha꜕gava꜓tā dhammo\\
Supaṭi꜕panno yassa bha꜕gava꜕to sāvaka꜕saṅgho\\
Tam-ma꜓yaṃ bha꜕gavantaṃ sa꜕dhammaṃ sa꜕saṅghaṃ\\
Imehi꜓ sakkārehi꜕ yathārahaṃ āropi꜕tehi a꜕bhi꜓pūja꜕yāma\\
Sādhu꜓ no bhante bha꜕gavā su꜕cira-parinibbu꜕topi\\
Pacchi꜓mā-ja꜕na꜓tānu꜓kampa꜕-mānasā\\
Ime sakkāre dugga꜕ta꜕-paṇṇākāra꜓-bhūte pa꜕ṭiggaṇhātu\\
Amhā꜓kaṃ dīgha꜕rattaṃ hi꜕tāya su꜕khāya\\
Arahaṃ sammāsambuddho bha꜕gavā\\
Buddhaṃ bha꜕gavantaṃ a꜕bhi꜓vādemi

[Svākkhā꜓to] bha꜕gava꜓tā dhammo\\
Dhammaṃ namassāmi

[Supaṭi꜕panno] bha꜕gava꜕to sāvaka꜕saṅgho\\
Sa꜓ṅghaṃ na꜕māmi

\chapter{Homenagem preliminar}

\begin{leader}
  [Ha꜓nda mayaṃ buddhassa꜕ bhagavato pubbabhāga-namakā꜕raṃ karomase]
\end{leader}

Namo tassa bha꜕gava꜕to araha꜕to sa꜓mmāsa꜓mbuddha꜕ssa

\instr{Três vezes}

\artoptfalse

\chapter[Metta Sutta]{Metta Sutta}

\begin{leader}
  [Ha꜓nda mayaṃ metta-sutta-gāthā꜓yo b꜕haṇāmase]
\end{leader}

[Karaṇīyam-attha-kusalena]\\
Yan-taṃ santaṃ padaṃ abhisamecca\\
Sakko ujū ca suhujū ca\\
Suvaco c'assa mudu anatimānī

Santussako ca subharo ca\\
Appakicco ca sallahuka-vutti\\
Sant'indriyo ca nipako ca\\
Appagabbho kulesu ananugiddho

Na ca khuddaṃ samācare kiñci\\
Yena viññū pare upavadeyyuṃ\\
Sukhino vā khemino hontu\\
Sabbe sattā bhavantu sukhit'attā

Ye keci pāṇa-bhūt'atthi\\
Tasā vā thāvarā vā anavasesā\\
Dīghā vā ye mahantā vā\\
Majjhimā rassakā aṇuka-thūlā

Diṭṭhā vā ye ca adiṭṭhā\\
Ye ca dūre vasanti avidūre\\
Bhūtā vā sambhavesī vā\\
Sabbe sattā bhavantu sukhit'attā

\enlargethispage{\baselineskip}

Na paro paraṃ nikubbetha\\
Nātimaññetha katthaci naṃ kiñci\\
Byārosanā paṭighasaññā\\
Nāññam-aññassa dukkham-iccheyya

\clearpage

Mātā yathā niyaṃ puttaṃ\\
Āyusā eka-puttam-anurakkhe\\
Evam'pi sabba-bhūtesu\\
Mānasam-bhāvaye aparimāṇaṃ

Mettañca sabba-lokasmiṃ\\
Mānasam-bhāvaye aparimāṇaṃ\\
Uddhaṃ adho ca tiriyañca\\
Asambādhaṃ averaṃ asapattaṃ

Tiṭṭhañ-caraṃ nisinno vā\\
Sayāno vā yāvat'assa vigata-middho\\
Etaṃ satiṃ adhiṭṭheyya\\
Brahmam-etaṃ vihāraṃ idham-āhu

Diṭṭhiñca anupagamma\\
Sīlavā dassanena sampanno\\
Kāmesu vineyya gedhaṃ\\
Na hi jātu gabbha-seyyaṃ punaretī'ti

\chapter[Metta Sutta]{Metta Sutta}

\begin{leader}
  [Cantemos agora as palavras do Buddha\\ sobre o Amor e a Compaixão]
\end{leader}

Eis o que se deve fazer\\
Para cultivar a bondade\\
E seguir a via da paz:\\
Ser capaz e ser honesto,\\
Franco e gentil no falar.\\
Humilde e não arrogante,\\
Contente, facilmente satisfeito,\\
Aliviado de deveres e frugal no seu caminho.

Pacífico e sereno, sábio e inteligente,\\
Sem orgulho, sem exigência por natureza.\\
Que ele nada faça\\
Que os sábios possam vir a reprovar.\\
Desejando: Na alegria e na segurança,\\
Que todos os seres sejam felizes.\\
Quaisquer que sejam os seres vivos,\\
Fracos, fortes, sem excepção\\
Dos maiores aos mais pequenos,\\
Visíveis ou invisíveis,\\
Estejam perto ou estejam longe,\\
Nascidos ou por nascer ---\\
Que todos os seres sejam felizes!

Que ninguém engane ninguém,\\
Ou despreze alguém em que estado fôr.\\
Que ninguém por raiva ou má-fé,\\
Deseje mal a alguém.\\
Assim como uma Mãe protege o filho,\\
Com sua vida, seu único filho,\\
Assim de coração infinito,\\
Se deveria estimar todo o ser vivo;\\
Irradiando ternura por todo o mundo:\\
Acima ao mais alto céu,\\
E abaixo às profundezas;\\
Irradiante e sem limites,\\
Livre de ódio e má-fé.\\
Seja parado ou a andar,\\
Sentado ou deitado,\\
Livre de torpor,\\
Esta é uma lembrança a manter.

Diz-se esta ser a sublime permanência.\\
O puro de coração, com clareza de visão,\\
Ao não insistir em ideias fixas,\\
Liberto dos desejos dos sentidos,\\
Não voltará a nascer neste mundo.

\chapter{Reflexões sobre a Partilha de Bençãos}

\begin{leader}
  [Ha꜓nda mayaṃ uddissanādhiṭṭhāna-gāthā꜓yo b꜕haṇāmase]
\end{leader}

[Iminā puñña꜕kammena] u꜕pajjhāyā gu꜕ṇutta꜕rā\\
Ācariyūpa꜕kārā ca꜕ mātāpitā ca꜕ ñāta꜕kā\\
Suriyo candimā rājā gu꜕ṇavantā na꜕rāpi꜕ ca꜕\\
Brahma-mārā ca꜕ indā ca꜕ loka꜕pālā ca꜕ deva꜕tā\\
Yamo mittā ma꜕nussā ca majjhattā veri꜕kāpi꜕ ca꜕\\
Sa꜕bbe sattā sukhī hontu puññāni pa꜕ka꜕tāni꜕ me\\
Sukhañca tividhaṃ dentu꜕ khippaṃ pāpetha꜕ voma꜕taṃ\\
Iminā puññakammena iminā uddi꜕ssena꜕ ca꜕\\
Khipp'āhaṃ su꜕la꜕bhe ceva taṇhūpādāna꜕-cheda꜕naṃ\\
Ye santāne hīnā dhammā yāva꜕ nibbāna꜕to ma꜕maṃ\\
Nassantu sabba꜕dā yeva yattha꜕ jāto bha꜕ve bha꜕ve\\
Ujucittaṃ sa꜕ti꜕paññā sallekho vi꜕ri꜕yamhinā\\
Mārā labhantu nokāsaṃ kātuñca vi꜕ri꜕yes꜕u me\\
Buddhādhipa꜕va꜕ro nātho dhammo nātho va꜕rutta꜕mo\\
Nātho pacceka꜕buddho ca꜕ saṅgho nāthotta꜕ro ma꜕maṃ\\
Tesottamānubhāvena mārokāsaṃ la꜕bhantu꜕ mā.

\chapter{Reflexões sobre a Partilha de Bençãos}

\enlargethispage{\baselineskip}

\begin{leader}
  [Cantemos agora as Reflexões sobre a Partilha de Bençãos]
\end{leader}

Através do bem que resulta da minha prática,\\
Que os meus mestres e guias espirituais de grande virtude,\\
A minha mãe, o meu pai e os meus familiares,\\
O Sol e a Lua, e todos os líderes virtuosos do mundo,\\
Que os Deuses mais elevados e as forças do mal,\\
Seres celestiais, espíritos guardiões da Terra e o Senhor da Morte,\\
Aqueles que são amigáveis, indiferentes ou hostis,\\
Que todos os seres recebam as bênçãos da minha vida.\\
Que brevemente cheguem à Tripla Bênção, e superem a morte.

Através do bem que resulta da minha prática,\\
E através desta partilha,\\
Que todos os desejos e apegos rapidamente cessem\\
Assim como os estados prejudiciais da mente.

Até realizar o Nibbana,\\
Em qualquer tipo de nascimento, que eu tenha uma mente justa,\\
Com consciência e sabedoria, austeridade e vigor.\\
Que as forças ilusórias não controlem,\\
nem enfraqueçam a minha decisão.

O Buddha é o meu excelente refúgio,\\
Insuperável é a proteção do Dhamma,\\
O Buddha solitário é o meu Nobre exemplo,\\
O Sangha é o meu maior suporte.

Que através desta supremacia\\
Desapareçam a escuridão e a ilusão.

\chapter[Cinco Temas]{Cinco temas para frequentemente relembrar}

\begin{leader}
  [Ha꜓nda mayaṃ abhiṇha-paccave꜕kkhaṇa-pāṭhaṃ bhaṇāmase]
\end{leader}

\sidepar{Homens}%
[Jarā-dhammomhi꜕] jaraṃ a꜕na꜕tīto

\sidepar{Mulheres}%
[Jarā-dhammāmhi꜕] jaraṃ a꜕na꜕tītā

\begin{english}
  Envelhecer faz parte da minha natureza,\\
  não estou para além do envelhecimento
\end{english}

\sidepar{h.}%
Byādhi꜓-dhammomhi꜕ byādhiṃ a꜕na꜕tīto

\sidepar{m.}%
Byādhi꜓-dhammāmhi꜕ byādhiṃ a꜕na꜕tītā

\begin{english}
  Adoecer faz parte da minha natureza,\\
  não estou para além da doença
\end{english}

\sidepar{h.}%
Ma꜕raṇa-dhammomhi꜕ ma꜕raṇaṃ a꜕na꜕tīto

\sidepar{m.}%
Ma꜕raṇa-dhammāmhi꜕ ma꜕raṇaṃ a꜕na꜕tītā

\begin{english}
  Morrer faz parte da minha natureza,\\
  não estou para além da morte
\end{english}

Sa꜕bbehi me pi꜕yehi ma꜕nāpehi꜕ nānābhāvo vi꜕nābhāvo

\begin{english}
  Tudo aquilo que tenho, que me é querido e que amo,\\
  é impermanente e separar-se-á de mim
\end{english}

\sidepar{h.}%
Kammassa꜕komhi kamma꜓dāyādo kamma꜕yoni kamma꜕bandhu kammapa꜕ṭisa꜓ra꜕ṇo\\
Yaṃ kammaṃ ka꜕rissāmi, kalyāṇaṃ vā pāpa꜕kaṃ vā, tassa꜕ dāyādo bha꜕vissāmi

\clearpage

\sidepar{m.}%
Kammassa꜕kāmhi kamma꜓dāyādā kamma꜕yoni kamma꜕bandhu kammapa꜕ṭisa꜓ra꜕ṇā\\
Yaṃ kammaṃ ka꜕rissāmi, kalyāṇaṃ vā pāpa꜕kaṃ vā, tassa꜕ dāyādā bha꜕vissāmi

\begin{english}
  Sou o dono do meu kamma, herdeiro do meu kamma, nascido do\\
  meu kamma, relacionado com o meu kamma, vivo segundo o meu kamma.\\
  Todo o kamma que criar, para o bem e para o mal, dele serei o herdeiro.
\end{english}

Evaṃ amhehi꜕ a꜕bhiṇhaṃ pacca꜕vekkhi꜓tabbaṃ

\begin{english}
  \prul{Assim} devemos frequentemente relembrar.
\end{english}

\artopttrue

\chapter{Homenagem de encerramento}

\delegateSetUseNext

[Arahaṃ] sammāsambuddho bha꜕gavā\\
Buddhaṃ bha꜕gavantaṃ a꜕bhi꜓vādemi

[Svākkhā꜓to] bha꜕gava꜓tā dhammo\\
Dhammaṃ namassāmi

[Supaṭi꜕panno] bha꜕gava꜕to sāvaka꜕saṅgho\\
Sa꜓ṅghaṃ na꜕māmi

\part{Formal Requests}

\artopttrue

\clearpage
\chapter[Three Refuges \& the Five Precepts]{Requesting the Three Refuges\newline \& the Five Precepts}

\begin{instruction}
  After bowing three times, with hands joined in añjali,\\
  recite the appropriate request.
\end{instruction}

\section{For a group from a monk}

\begin{twochants}
Mayaṃ bhante tisaraṇena sa꜕ha & pañca sī꜓lāni yā꜕cāma\\
Dutiyampi mayaṃ bhante tisaraṇena sa꜕ha & pañca sī꜓lāni yā꜕cāma\\
Tatiyampi mayaṃ bhante tisaraṇena sa꜕ha & pañca sī꜓lāni yā꜕cāma\\
\end{twochants}

\section{For oneself from a monk}

\begin{twochants}
Ahaṃ bhante tisaraṇena sa꜕ha & pañca sī꜓lāni yā꜕cāmi\\
Dutiyampi ahaṃ bhante tisaraṇena sa꜕ha & pañca sī꜓lāni yā꜕cāmi\\
Tatiyampi ahaṃ bhante tisaraṇena sa꜕ha & pañca sī꜓lāni yā꜕cāmi
\end{twochants}

\section{Translation}

\begin{english}
  We/I, Venerable Sir, request the Three Refuges and the Five Precepts.

  For the second time,\\
  we/I, Venerable Sir, request the Three Refuges and the Five Precepts.

  For the third time,\\
  we/I, Venerable Sir, request the Three Refuges and the Five Precepts.
\end{english}

\clearpage
\chapter{Taking the Three Refuges}%{{{1

\begin{instruction}
  Repeat, after the leader has chanted the first three lines
\end{instruction}

Namo tassa bhagavato arahato sammāsambuddhassa\\
Namo tassa bhagavato arahato sammāsambuddhassa\\
Namo tassa bhagavato arahato sammāsambuddhassa

\begin{english}
  Homage to the Blessed, Noble, and Perfectly Enlightened One.\\
  Homage to the Blessed, Noble, and Perfectly Enlightened One.\\
  Homage to the Blessed, Noble, and Perfectly Enlightened One.
\end{english}

Buddhaṃ saraṇaṃ gacchāmi\\
Dhammaṃ saraṇaṃ gacchāmi\\
Saṅghaṃ saraṇaṃ gacchāmi

\begin{english}
  To the Buddha I go for refuge.\\
  To the Dhamma I go for refuge.\\
  To the Saṅgha I go for refuge.
\end{english}

Dutiyampi buddhaṃ saraṇaṃ gacchāmi\\
Dutiyampi dhammaṃ saraṇaṃ gacchāmi\\
Dutiyampi saṅghaṃ saraṇaṃ gacchāmi

\begin{english}
  For the second time, to the Buddha I go for refuge.\\
  For the second time, to the Dhamma I go for refuge.\\
  For the second time, to the Saṅgha I go for refuge.
\end{english}

Tatiyampi buddhaṃ saraṇaṃ gacchāmi\\
Tatiyampi dhammaṃ saraṇaṃ gacchāmi\\
Tatiyampi saṅghaṃ saraṇaṃ gacchāmi

\clearpage

\begin{english}
  For the third time, to the Buddha I go for refuge.\\
  For the third time, to the Dhamma I go for refuge.\\
  For the third time, to the Saṅgha I go for refuge.
\end{english}

\begin{instruction}
  Leader:
\end{instruction}

[Tisaraṇa-gamanaṃ niṭṭhitaṃ]

\begin{english}
  This completes the going to the Three Refuges.
\end{english}

\begin{instruction}
  Response:
\end{instruction}

Āma bhante

\begin{english}
  Yes, Venerable Sir/Sister/Friend.
\end{english}

\chapter{The Five Precepts}%{{{1

\begin{instruction}
  Repeat each precept after the leader
\end{instruction}

\begin{precept}
  \setcounter{enumi}{0}
  \item Pāṇātipātā vera꜓maṇī sikkhā꜓padaṃ sa꜓mādi꜕yāmi
\end{precept}

\begin{english}
  I undertake the precept to refrain from taking the life of any living~creature.
\end{english}

\begin{precept}
  \setcounter{enumi}{1}
  \item Adinnādānā vera꜓maṇī sikkhā꜓padaṃ sa꜓mādi꜕yāmi
\end{precept}

\begin{english}
  I undertake the precept to refrain from taking that which is not given.
\end{english}

\begin{precept}
  \setcounter{enumi}{2}
  \item Kāmesu micchā꜓cārā vera꜓maṇī sikkhā꜓padaṃ sa꜓mādi꜕yāmi
\end{precept}

\begin{english}
  I undertake the precept to refrain from sexual misconduct.
\end{english}

\begin{precept}
  \setcounter{enumi}{3}
  \item Musā꜓vādā vera꜓maṇī sikkhā꜓padaṃ sa꜓mādi꜕yāmi
\end{precept}

\enlargethispage{\baselineskip}

\begin{english}
  I undertake the precept to refrain from lying.
\end{english}

\clearpage

\begin{precept}
  \setcounter{enumi}{4}
  \item Surāmeraya-majja-pamādaṭṭhā꜓nā vera꜓maṇī sikkhā꜓padaṃ sa꜓mādi꜕yāmi
\end{precept}

\begin{english}
  I undertake the precept to refrain from consuming intoxicating drink and drugs which lead to carelessness.
\end{english}

\begin{instruction}
  Leader:
\end{instruction}

[Imāni pañca sikkhā꜓padāni\\
Sī꜓lena suga꜕tiṃ yanti\\
Sī꜓lena bhoga꜕sa꜓mpadā\\
Sī꜓lena nibbu꜕tiṃ yanti\\
Tasmā꜓ sī꜓laṃ viso꜓dhaye]

\begin{english}
  These are the Five Precepts;\\
  virtue is the source of happiness,\\
  virtue is the source of true wealth,\\
  virtue is the source of peacefulness ---\\
  Therefore let virtue be purified.
\end{english}

\begin{instruction}
  Response:
\end{instruction}

Sādhu, sādhu, sādhu

\begin{instruction}
  Bow three times
\end{instruction}


\end{document}
