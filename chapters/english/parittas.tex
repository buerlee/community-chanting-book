% vim: foldmethod=marker foldlevel=0 foldtext=FoldText()

\chapter{Invitation to the Devas}% {{{1

\firstline{Pharitvāna mettaṃ samettā bhadantā}
\firstline{Samantā cakka-vāḷesu}

\begin{paritta}
\sidepar{A.}%
Pharitvāna mettaṃ samettā bhadantā\\
Avikkhitta-cittā parittaṃ bhaṇantu

\sidepar{B.}%
Samantā cakka-vāḷesu\\
Atr'āgacchantu devatā

Sagge kāme ca rūpe\\
Giri-sikhara-taṭe c'antalikkhe vimāne\\
Dīpe raṭṭhe ca gāme\\
Taru-vana-gahane geha-vatthumhi khette\\
Bhummā c'āyantu devā\\
Jala-thala-visame yakkha-gandhabba-nāgā\\
Tiṭṭhantā santike yaṃ\\
Muni-vara-vacanaṃ sādhavo me suṇantu

Dhammassavana-kālo ayam-bhadantā

\instr{Three times, or}

Buddha-dassana-kālo ayam-bhadantā\\
Dhammassavana-kālo ayam-bhadantā\\
Saṅgha-payirūpāsana-kālo ayam-bhadantā
\end{paritta}

\chapter{Pubba-bhāga-nama-kāra-pāṭho}% {{{1

\firstline{Namo tassa Bhagavato}

\begin{paritta}
Namo tassa bhagavato arahato sammā-sambuddhassa\\
Namo tassa bhagavato arahato sammā-sambuddhassa\\
Namo tassa bhagavato arahato sammā-sambuddhassa
\end{paritta}

\clearpage

\chapter{Saraṇa-gamana-pāṭho}% {{{1

\firstline{Buddhaṃ saraṇaṃ gacchāmi}

\begin{paritta}
Buddhaṃ saraṇaṃ gacchāmi\\
Dhammaṃ saraṇaṃ gacchāmi\\
Saṅghaṃ saraṇaṃ gacchāmi

Dutiyam pi buddhaṃ saraṇaṃ gacchāmi\\
Dutiyam pi dhammaṃ saraṇaṃ gacchāmi\\
Dutiyam pi saṅghaṃ saraṇaṃ gacchāmi

Tatiyam pi buddhaṃ saraṇaṃ gacchāmi\\
Tatiyam pi dhammaṃ saraṇaṃ gacchāmi\\
Tatiyam pi saṅghaṃ saraṇaṃ gacchāmi
\end{paritta}

\chapter{Nama-kāra-siddhi-gāthā}% {{{1

\firstline{Yo cakkhumā moha-malāpakaṭṭho}

\begin{paritta}
Yo cakkhumā moha-malāpakaṭṭho\\
Sāmaṃ va buddho sugato vimutto\\
Mārassa pāsā vinimocayanto\\
Pāpesi khemaṃ janataṃ vineyyaṃ\\
Buddhaṃ varan-taṃ sirasā namāmi\\
Lokassa nāthañ-ca vināyakañ-ca\\
Tan-tejasā te jaya-siddhi hotu\\
Sabb'antarāyā ca vināsamentu

Dhammo dhajo yo viya tassa satthu\\
Dassesi lokassa visuddhi-maggaṃ\\
Niyyāniko dhamma-dharassa dhārī\\
Sāt'āvaho santi-karo suciṇṇo\\
Dhammaṃ varan-taṃ sirasā namāmi\\
Mohappadālaṃ upasanta-dāhaṃ\\
Tan-tejasā te jaya-siddhi hotu\\
Sabb'antarāyā ca vināsamentu

Saddhamma-senā sugatānugo yo\\
Lokassa pāpūpakilesa-jetā\\
Santo sayaṃ santi-niyojako ca\\
Svākkhāta-dhammaṃ viditaṃ karoti\\
Saṅghaṃ varan-taṃ sirasā namāmi\\
Buddhānubuddhaṃ sama-sīla-diṭṭhiṃ\\
Tan-tejasā te jaya-siddhi hotu\\
Sabb'antarāyā ca vināsamentu
\end{paritta}

\chapter{Sambuddhe}% {{{1

\firstline{Sambuddhe aṭṭhavīsañca}

\begin{twochants}
Sambuddhe aṭṭhavīsañca & dvādasañca sahassake\\
Pañca-sata-sahassāni & namāmi sirasā ahaṃ\\
Tesaṃ dhammañca saṅghañca & ādarena namāmihaṃ\\
Namakārānubhāvena & hantvā sabbe upaddave\\
Anekā antarāyāpi & vinassantu asesato\\
Sambuddhe pañca-paññāsañca & catuvīsati sahassake\\
Dasa-sata-sahassāni & namāmi sirasā ahaṃ\\
Tesaṃ dhammañca saṅghañca & ādarena namāmihaṃ\\
Namakārānubhāvena & hantvā sabbe upaddave\\
Anekā antarāyāpi & vinassantu asesato\\
Sambuddhe navuttarasate & aṭṭhacattāḷīsa sahassake\\
\end{twochants}

\clearpage

\begin{twochants}
Vīsati-sata-sahassāni & namāmi sirasā ahaṃ\\
Tesaṃ dhammañca saṅghañca & ādarena namāmihaṃ\\
Namakārānubhāvena & hantvā sabbe upaddave\\
Anekā antarāyāpi & vinassantu asesato\\
\end{twochants}

\chapter{Namo-kāra-aṭṭhaka}% {{{1

\firstline{Namo arahato sammā}

\begin{twochants}
Namo arahato sammā & sambuddhassa mahesino\\
Namo uttama-dhammassa & svākkhātass'eva ten'idha\\
Namo mahā-saṅghassāpi & visuddha-sīla-diṭṭhino\\
Namo omāty-āraddhassa & ratanattayassa sādhukaṃ\\
Namo omakātītassa & tassa vatthuttayassa-pi\\
Namo-kārappabhāvena & vigacchantu upaddavā\\
Namo-kārānubhāvena & suvatthi hotu sabbadā\\
Namo-kārassa tejena & vidhimhi homi tejavā\\
\end{twochants}

\chapter{Maṅgala-sutta}% {{{1

\firstline{Asevanā ca bālānaṃ}
\enlargethispage{\baselineskip}

\begin{paritta}
Asevanā ca bālānaṃ\\
Paṇḍitānañ-ca sevanā\\
Pūjā ca pūjanīyānaṃ\\
Etam maṅgalam-uttamaṃ

Paṭirūpa-desa-vāso ca\\
Pubbe ca kata-puññatā\\
Atta-sammā-paṇidhi ca\\
Etam maṅgalam-uttamaṃ

\clearpage

Bāhu-saccañ-ca sippañ-ca,\\
Vinayo ca susikkhito\\
Subhāsitā ca yā vācā\\
Etam maṅgalam-uttamaṃ

Mātā-pitu-upaṭṭhānaṃ\\
Putta-dārassa saṅgaho\\
Anākulā ca kammantā\\
Etam maṅgalam-uttamaṃ

Dānañ-ca dhamma-cariyā ca\\
Ñātakānañ-ca saṅgaho\\
Anavajjāni kammāni\\
Etam maṅgalam-uttamaṃ

Āratī viratī pāpā\\
Majja-pānā ca saññamo\\
Appamādo ca dhammesu\\
Etam maṅgalam-uttamaṃ

Gāravo ca nivāto ca\\
Santuṭṭhī ca kataññutā\\
Kālena dhammassavanaṃ\\
Etam maṅgalam-uttamaṃ

Khantī ca sovacassatā\\
Samaṇānañ-ca dassanaṃ\\
Kālena dhamma-sākacchā\\
Etam maṅgalam-uttamaṃ

Tapo ca brahma-cariyañ-ca\\
Ariya-saccāna-dassanaṃ\\
Nibbāna-sacchikiriyā ca\\
Etam maṅgalam-uttamaṃ

Phuṭṭhassa loka-dhammehi\\
Cittaṃ yassa na kampati\\
Asokaṃ virajaṃ khemaṃ\\
Etam maṅgalam-uttamaṃ

Etādisāni katvāna\\
Sabbattham-aparājitā\\
Sabbattha sotthiṃ gacchanti\\
Tan-tesaṃ maṅgalam-uttaman-ti

\suttaref{(Sn.vv.258-269; Khp.V)}
\end{paritta}

\chapter{Ratana-sutta}% {{{1

\firstline{Yaṅkiñci vittaṃ idha vā huraṃ vā}

\begin{paritta}

Yaṅkiñci vittaṃ idha vā huraṃ vā\\
Saggesu vā yaṃ ratanaṃ paṇītaṃ\\
Na no samaṃ atthi tathāgatena\\
Idam-pi buddhe ratanaṃ paṇītaṃ\\
Etena saccena suvatthi hotu

Khayaṃ virāgaṃ amataṃ paṇītaṃ\\
Yad-ajjhagā sakya-munī samāhito\\
Na tena dhammena sam'atthi kiñci\\
Idam-pi dhamme ratanaṃ paṇītaṃ\\
Etena saccena suvatthi hotu

\clearpage

Yam buddha-seṭṭho parivaṇṇayī suciṃ\\
Samādhim-ānantarikaññam-āhu\\
Samādhinā tena samo na vijjati\\
Idam-pi dhamme ratanaṃ paṇītaṃ\\
Etena saccena suvatthi hotu

Ye puggalā aṭṭha sataṃ pasaṭṭhā\\
Cattāri etāni yugāni honti\\
Te dakkhiṇeyyā sugatassa sāvakā\\
Etesu dinnāni mahapphalāni\\
Idam-pi saṅghe ratanaṃ paṇītaṃ\\
Etena saccena suvatthi hotu

Ye suppayuttā manasā daḷhena\\
Nikkāmino gotama-sāsanamhi\\
Te patti-pattā amataṃ vigayha\\
Laddhā mudhā nibbutiṃ bhuñjamānā\\
Idam-pi saṅghe ratanaṃ paṇītaṃ\\
Etena saccena suvatthi hotu

Khīṇaṃ purāṇaṃ navaṃ n'atthi sambhavaṃ\\
Viratta-citt'āyatike bhavasmiṃ\\
Te khīṇa-bījā aviruḷhi-chandā\\
Nibbanti dhīrā yathā'yam padīpo\\
Idam-pi saṅghe ratanaṃ paṇītaṃ\\
Etena saccena suvatthi hotu

\suttaref{(Sn.vv.224-241; Khp.VI)}
\end{paritta}

\clearpage

\chapter{Karaṇīya-metta-sutta}% {{{1

\firstline{Karaṇīya m-attha-kusalena}

\begin{paritta}
Karaṇīya m-attha-kusalena\\
Yan-taṃ santaṃ padaṃ abhisamecca\\
Sakko ujū ca suhujū ca\\
Suvaco c'assa mudu anatimānī

Santussako ca subharo ca\\
Appakicco ca sallahuka-vutti\\
Sant'indriyo ca nipako ca\\
Appagabbho kulesu ananugiddho

Na ca khuddaṃ samācare kiñci\\
Yena viññū pare upavadeyyuṃ\\
Sukhino vā khemino hontu\\
Sabbe sattā bhavantu sukhit'attā

Ye keci pāṇa-bhūt'atthi\\
Tasā vā thāvarā vā anavasesā\\
Dīghā vā ye mahantā vā\\
Majjhimā rassakā aṇuka-thūlā

Diṭṭhā vā ye ca adiṭṭhā\\
Ye ca dūre vasanti avidūre\\
Bhūtā vā sambhavesī vā\\
Sabbe sattā bhavantu sukhit'attā

Na paro paraṃ nikubbetha\\
Nātimaññetha katthaci naṃ kiñci\\
Byārosanā paṭīgha-saññā\\
Nāññam-aññassa dukkham-iccheyya

Mātā yathā niyaṃ puttaṃ\\
Āyusā eka-puttam-anurakkhe\\
Evam pi sabba-bhūtesu\\
Mānasam-bhāvaye aparimāṇaṃ

Mettañ-ca sabba-lokasmiṃ\\
Mānasam-bhāvaye aparimāṇaṃ\\
Uddhaṃ adho ca tiriyañ-ca\\
Asambādhaṃ averaṃ asapattaṃ

Tiṭṭhañ-caraṃ nisinno vā\\
Sayāno vā yāvat'assa vigata-middho\\
Etaṃ satiṃ adhiṭṭheyya\\
Brahmam-etaṃ vihāraṃ idham-āhu

Diṭṭhiñ-ca anupagamma\\
Sīlavā dassanena sampanno\\
Kāmesu vineyya gedhaṃ\\
Na hi jātu gabbha-seyyaṃ punar-etī-ti

\suttaref{(Sn.vv.143-152; Khp.IX)}
\end{paritta}

\chapter{Khandha-parittaṃ}% {{{1

\firstline{Virūpakkhehi me mettaṃ}

\enlargethispage{\baselineskip}

% TODO: Virūp_a_kkhehi correct?

\begin{twochants}
Virūpakkhehi me mettaṃ & mettaṃ erāpathehi me\\
Chabyā-puttehi me mettaṃ & mettaṃ kaṇhā-gotamakehi ca\\
Apādakehi me mettaṃ & mettaṃ di-pādakehi me\\
Catuppadehi me mettaṃ & mettaṃ bahuppadehi me\\
Mā maṃ apādako hiṃsi & mā maṃ hiṃsi di-pādako\\
Mā maṃ catuppado hiṃsi & mā maṃ hiṃsi bahuppado\\
\end{twochants}

\clearpage

\begin{twochants}
Sabbe sattā sabbe pāṇā & sabbe bhūtā ca kevalā\\
Sabbe bhadrāni passantu & mā kiñci pāpam-āgamā\\
Appamāṇo buddho & appamāṇo dhammo\\
Appamāṇo saṅgho & pamāṇavantāni siriṃsapāni\\
Ahi-vicchikā sata-padī & uṇṇā-nābhī sarabhū mūsikā\\
Katā me rakkhā katā me parittā & paṭikkamantu bhūtāni\\
So'haṃ namo bhagavato & namo sattannaṃ\\
Sammā-sambuddhānaṃ & \\
\end{twochants}

\suttaref{(A.II.72-73; Vin.II.110; J.144)}

\chapter{Mora-parittaṃ}% {{{1

\firstline{Udet'ayañ-cakkhumā eka-rājā}

\begin{paritta}
Udet'ayañ-cakkhumā eka-rājā\\
Harissa-vaṇṇo paṭhavippabhāso\\
Taṃ taṃ namassāmi\\
Harissa-vaṇṇaṃ paṭhavippabhāsaṃ\\
Tay'ajja guttā viharemu divasaṃ\\
Ye brāhmaṇā vedagu sabba-dhamme\\
Te me namo\\
Te ca maṃ pālayantu\\
Nam'atthu buddhānaṃ\\
Nam'atthu bodhiyā\\
Namo vimuttānaṃ\\
Namo vimuttiyā\\
Imaṃ so parittaṃ katvā\\
Moro carati esanā'ti\\
Apet'ayañ-cakkhumā eka-rājā\\
Harissa-vaṇṇo paṭhavippabhāso\\
Taṃ taṃ namassāmi\\
Harissa-vaṇṇaṃ paṭhavippabhāsaṃ\\
Tay'ajja guttā viharemu rattiṃ\\
Ye brāhmaṇā vedagu sabba-dhamme\\
Te me namo\\
Te ca maṃ pālayantu\\
Nam'atthu buddhānaṃ\\
Nam'atthu bodhiyā\\
Namo vimuttānaṃ\\
Namo vimuttiyā\\
Imaṃ so parittaṃ katvā\\
Moro vāsam-akappayī'ti

\suttaref{(J.159)}
\end{paritta}

\chapter{Vaṭṭaka-parittaṃ}% {{{1

\firstline{Atthi loke sīla-guṇo}

\begin{paritta}
Atthi loke sīla-guṇo\\
Saccaṃ soceyy'anuddayā\\
Tena saccena kāhāmi\\
Sacca-kiriyam-anuttaraṃ\\
Āvajjitvā dhamma-balaṃ\\
Saritvā pubbake jine\\
Sacca-balam-avassāya\\
Sacca-kiriyam-akās'ahaṃ\\
Santi pakkhā apattanā\\
Santi pādā avañcanā\\
Mātā pitā ca nikkhantā\\
Jāta-veda paṭikkama\\
Saha sacce kate mayhaṃ\\
Mahā-pajjalito sikhī\\
Vajjesi soḷasa karīsāni\\
Udakaṃ patvā yathā sikhī\\
Saccena me samo n'atthi\\
Esā me sacca-pāramī-ti

\suttaref{(Cariyapiṭaka vv.319-322)}
\end{paritta}

\chapter{Buddha-dhamma-saṅgha-guṇā}% {{{1

\firstline{Iti pi so bhagavā}

\begin{paritta}
Iti pi so bhagavā\\
Arahaṃ sammā-sambuddho\\
Vijjā-caraṇa-sampanno\\
Sugato loka-vidū\\
Anuttaro purisa-damma-sārathi\\
Satthā devamanussānaṃ\\
Buddho bhagavā-ti

Svākkhāto bhagavatā dhammo\\
Sandiṭṭhiko akāliko ehi-passiko\\
Opanayiko paccattaṃ veditabbo viññūhī-ti

Supaṭipanno bhagavato sāvaka-saṅgho\\
Uju-paṭipanno bhagavato sāvaka-saṅgho\\
Ñāya-paṭipanno bhagavato sāvaka-saṅgho\\
Sāmīci-paṭipanno bhagavato sāvaka-saṅgho\\
Yad-idaṃ cattāri purisa-yugāni\\
Aṭṭha purisa-puggalā\\
Esa bhagavato sāvaka-saṅgho\\
Āhuneyyo pāhuneyyo dakkhiṇeyyo añjali-karaṇīyo\\
Anuttaraṃ puññakkhettaṃ lokassā-ti
\end{paritta}

\chapter{Āṭānāṭiya-parittaṃ}% {{{1

\firstline{Vipassissa nam'atthu}

\begin{twochants}
Vipassissa nam'atthu & cakkhumantassa sirīmato\\
Sikhissa pi nam'atthu & sabba-bhūtānukampino\\
Vessabhussa nam'atthu & nhātakassa tapassino\\
Nam'atthu kakusandhassa & māra-senappamaddino\\
Konāgamanassa nam'atthu & brāhmaṇassa vusīmato\\
Kassapassa nam'atthu & vippamuttassa sabbadhi\\
Aṅgīrasassa nam'atthu & sakya-puttassa sirīmato\\
Yo imaṃ dhammam-adesesi & sabba-dukkhāpanūdanaṃ\\
Ye cāpi nibbutā loke & yathā-bhūtaṃ vipassisuṃ\\
Te janā apisuṇā & mahantā vīta-sāradā\\
Hitaṃ deva-manussānaṃ & yaṃ namassanti gotamaṃ\\
Vijjā-caraṇa-sampannaṃ & mahantaṃ vīta-sāradaṃ\\
Vijjā-caraṇa-sampannaṃ & buddhaṃ vandāma gotaman-ti\\
\end{twochants}

\clearpage

\chapter{N'atthi me saraṇaṃ aññaṃ}% {{{1

\firstline{N'atthi me saraṇaṃ aññaṃ}

\begin{paritta}
N'atthi me saraṇaṃ aññaṃ\\
Buddho me saraṇaṃ varaṃ\\
Etena sacca-vajjena\\
Hotu te jaya-maṅgalaṃ\\
N'atthi me saraṇaṃ aññaṃ\\
Dhammo me saraṇaṃ varaṃ\\
Etena sacca-vajjena\\
Hotu te jaya-maṅgalaṃ\\
N'atthi me saraṇaṃ aññaṃ\\
Saṅgho me saraṇaṃ varaṃ\\
Etena sacca-vajjena\\
Hotu te jaya-maṅgalaṃ
\end{paritta}

\chapter{Yaṅkiñci ratanaṃ loke}% {{{1

\firstline{Yaṅkiñci ratanaṃ loke}

\begin{twochants}
Yaṅkiñci ratanaṃ loke & vijjati vividhaṃ puthu\\
Ratanaṃ buddha-samaṃ n'atthi & tasmā sotthī bhavantu te/me\\
Yaṅkiñci ratanaṃ loke & vijjati vividhaṃ puthu\\
Ratanaṃ dhamma-samaṃ n'atthi & tasmā sotthī bhavantu te/me\\
Yaṅkiñci ratanaṃ loke & vijjati vividhaṃ puthu\\
Ratanaṃ saṅgha-samaṃ n'atthi & tasmā sotthī bhavantu te/me\\
\end{twochants}

\clearpage

\chapter{Sakkatvā}% {{{1

\firstline{Sakkatvā buddha-ratanaṃ}

\begin{paritta}
Sakkatvā buddha-ratanaṃ\\
Osathaṃ uttamaṃ varaṃ\\
Hitaṃ deva-manussānaṃ\\
Buddha-tejena sotthinā\\
Nassant'upaddavā sabbe\\
Dukkhā vūpasamentu te/me

Sakkatvā dhamma-ratanaṃ\\
Osathaṃ uttamaṃ varaṃ\\
Pariḷāhūpasamanaṃ\\
Dhamma-tejena sotthinā\\
Nassant'upaddavā sabbe\\
Bhayā vūpasamentu te/me

Sakkatvā saṅgha-ratanaṃ\\
Osathaṃ uttamaṃ varaṃ\\
Āhuneyyaṃ pāhuneyyaṃ\\
Saṅgha-tejena sotthinā\\
Nassant'upaddavā sabbe\\
Rogā vūpasamentu te/me
\end{paritta}

\chapter{Aṅguli-māla-parittaṃ}% {{{1

\firstline{Yato'haṃ bhagini ariyāya jātiyā jāto}

\begin{paritta}
Yato'haṃ bhagini ariyāya jātiyā jāto\\
Nābhijānāmi sañcicca pāṇaṃ jīvitā voropetā\\
Tena saccena sotthi te hotu sotthi gabbhassa

\suttaref{(M.II.103)}
\end{paritta}

\clearpage

\chapter{Bojjh'aṅga-parittaṃ}% {{{1

\firstline{Bojjh'aṅgo sati-saṅkhāto}

\begin{twochants}
Bojjh'aṅgo sati-saṅkhāto & dhammānaṃ vicayo tathā\\
Viriyam-pīti-passaddhi & bojjh'aṅgā ca tathā'pare\\
Samādh'upekkha-bojjh'aṅgā & satt'ete sabba-dassinā\\
Muninā sammad-akkhātā & bhāvitā bahulī-katā\\
Saṃvattanti abhiññāya & nibbānāya ca bodhiyā\\
Etena sacca-vajjena & sotthi te hotu sabbadā\\
Ekasmiṃ samaye nātho & moggallānañ-ca kassapaṃ\\
Gilāne dukkhite disvā & bojjh'aṅge satta desayi\\
Te ca taṃ abhinanditvā & rogā mucciṃsu taṅ-khaṇe\\
Etena sacca-vajjena & sotthi te hotu sabbadā\\
Ekadā dhamma-rājā pi & gelaññenābhipīḷito\\
Cundattherena tañ-ñeva & bhaṇāpetvāna sādaraṃ\\
Sammoditvā ca ābādhā & tamhā vuṭṭhāsi ṭhānaso\\
Etena sacca-vajjena & sotthi te hotu sabbadā\\
Pahīnā te ca ābādhā & tiṇṇannam-pi mahesinaṃ\\
Magg'āhata-kilesā va & pattānuppatti-dhammataṃ\\
Etena sacca-vajjena & sotthi te hotu sabbadā\\
\end{twochants}

\suttaref{(cf. S.V.80f)}

\clearpage

\chapter{Abhaya-parittaṃ}% {{{1

\firstline{Yan-dunnimittaṃ avamaṅgalañ-ca}

\begin{paritta}
Yan-dunnimittaṃ avamaṅgalañ-ca\\
Yo cāmanāpo sakuṇassa saddo\\
Pāpaggaho dussupinaṃ akantaṃ\\
Buddhānubhāvena vināsamentu

Yan-dunnimittaṃ avamaṅgalañ-ca\\
Yo cāmanāpo sakuṇassa saddo\\
Pāpaggaho dussupinaṃ akantaṃ\\
Dhammānubhāvena vināsamentu

Yan-dunnimittaṃ avamaṅgalañ-ca\\
Yo cāmanāpo sakuṇassa saddo\\
Pāpaggaho dussupinaṃ akantaṃ\\
Saṅghānubhāvena vināsamentu
\end{paritta}

\chapter{Devatā-uyyojana-gāthā}% {{{1

\firstline{Dukkhappattā ca niddukkhā}

\begin{twochants}
Dukkhappattā ca niddukkhā & bhayappattā ca nibbhayā\\
Sokappattā ca nissokā & hontu sabbe pi pāṇino\\
Ettāvatā ca amhehi & sambhataṃ puñña-sampadaṃ\\
Sabbe devānumodantu & sabba-sampatti-siddhiyā\\
Dānaṃ dadantu saddhāya & sīlaṃ rakkhantu sabbadā\\
Bhāvanābhiratā hontu & gacchantu devatā-gatā\\\relax
[Sabbe buddhā] balappattā & paccekānañ-ca yaṃ balaṃ\\
Arahantānañ-ca tejena & rakkhaṃ bandhāmi sabbaso\\
\end{twochants}

\enlargethispage{\baselineskip}

\suttaref{(MJG)}

\clearpage

\chapter{Jaya-maṅgala-aṭṭha-gāthā}% {{{1

\firstline{Bāhuṃ sahassam-abhinimmita sāvudhan-taṃ}

\begin{paritta}
Bāhuṃ sahassam-abhinimmita sāvudhan-taṃ\\
Grīmekhalaṃ udita-ghora-sasena-māraṃ\\
Dān'ādi-dhamma-vidhinā jitavā mun'indo\\
Tan-tejasā bhavatu te jaya-maṅgalāni

Mārātirekam-abhiyujjhita-sabba-rattiṃ\\
Ghoram-pan'āḷavakam-akkhama-thaddha-yakkhaṃ\\
Khantī-sudanta-vidhinā jitavā mun'indo\\
Tan-tejasā bhavatu te jaya-maṅgalāni

Nāḷāgiriṃ gaja-varaṃ atimatta-bhūtaṃ\\
Dāv'aggi-cakkam-asanīva sudāruṇan-taṃ\\
Mett'ambu-seka-vidhinā jitavā mun'indo\\
Tan-tejasā bhavatu te jaya-maṅgalāni

Ukkhitta-khaggam-atihattha-sudāruṇan-taṃ\\
Dhāvan-ti-yojana-path'aṅguli- mālavantaṃ\\
Iddhī'bhisaṅkhata-mano jitavā mun'indo\\
Tan-tejasā bhavatu te jaya-maṅgalāni

Katvāna kaṭṭham-udaraṃ iva gabbhinīyā\\
Ciñcāya duṭṭha-vacanaṃ jana-kāya majjhe\\
Santena soma-vidhinā jitavā mun'indo\\
Tan-tejasā bhavatu te jaya-maṅgalāni

Saccaṃ vihāya-mati-saccaka-vāda-ketuṃ\\
Vādābhiropita-manaṃ ati-andha-bhūtaṃ\\
Paññā-padīpa-jalito jitavā mun'indo\\
Tan-tejasā bhavatu te jaya-maṅgalāni

Nandopananda-bhujagaṃ vibudhaṃ mah'iddhiṃ\\
Puttena thera-bhujagena damāpayanto\\
Iddhūpadesa-vidhinā jitavā mun'indo\\
Tan-tejasā bhavatu te jaya-maṅgalāni

Duggāha-diṭṭhi-bhujagena sudaṭṭha-hatthaṃ\\
Brahmaṃ visuddhi-jutim-iddhi-bakābhidhānaṃ\\
Ñāṇāgadena vidhinā jitavā mun'indo\\
Tan-tejasā bhavatu te jaya-maṅgalāni

Etā pi buddha-jaya-maṅgala-aṭṭha-gāthā\\
Yo vācano dina-dine saratem-atandī\\
Hitvān'aneka-vividhāni c'upaddavāni\\
Mokkhaṃ sukhaṃ adhigameyya naro sapañño
\end{paritta}

\chapter{Jaya-parittaṃ}% {{{1

\firstline{Mahā-kāruṇiko nātho}

\begin{twochants}
Mahā-kāruṇiko nātho & hitāya sabba-pāṇinaṃ\\
Pūretvā pāramī sabbā & patto sambodhim-uttamaṃ\\
Etena sacca-vajjena & hotu te jaya-maṅgalaṃ\\
Jayanto bodhiyā mūle & sakyānaṃ nandi-vaḍḍhano\\
Evaṃ tvaṃ vijayo hohi & jayassu jaya-maṅgale\\
Aparājita-pallaṅke & sīse paṭhavi-pokkhare\\
Abhiseke sabba-buddhānaṃ & aggappatto pamodati\\
Sunakkhattaṃ sumaṅgalaṃ & supabhātaṃ suhuṭṭhitaṃ\\
\end{twochants}

\clearpage

\begin{twochants}
Sukhaṇo sumuhutto ca & suyiṭṭhaṃ brahma-cārisu\\
Padakkhiṇaṃ kāya-kammaṃ & vācā-kammaṃ padakkhiṇaṃ\\
Padakkhiṇaṃ mano-kammaṃ & paṇidhi te padakkhiṇā\\
Padakkhiṇāni katvāna & labhant'atthe padakkhiṇe\\
\end{twochants}

\suttaref{(MJG; A.I.294)}

\chapter{Bhavatu sabba-maṅgalaṃ}% {{{1

\firstline{Bhavatu sabba-maṅgalaṃ}

\begin{paritta}
Bhavatu sabba-maṅgalaṃ\\
Rakkhantu sabba-devatā\\
Sabba-buddhānubhāvena\\
Sadā sotthī bhavantu te/me

Bhavatu sabba-maṅgalaṃ\\
Rakkhantu sabba-devatā\\
Sabba-dhammānunbhāvena\\
Sadā sotthī bhavantu te/me

Bhavatu sabba-maṅgalaṃ\\
Rakkhantu sabba-devatā\\
Sabba-saṅghānubhāvena\\
Sadā sotthī bhavantu te/me
\end{paritta}

\suttaref{(MJG)}

\cleartoverso

\chapter{The Twenty-Eight Buddhas' Protection}% {{{1

\vspace*{5pt}

{\setlength{\parskip}{0pt}%
\soloinstr{Solo introduction}

\begin{soloonechants}
We will now recite the discourse given by the Great Hero\\
(the Buddha), as a protection for virtue-loving human beings.\\
Against harm from all evil-doing, malevolent nonhumans who are\\
displeased with the Buddha's Teachings.\\
\end{soloonechants}%
}

\vspace*{-2pt}

\setEnglishTextSize{12}{21}{\parskip}
\englishText

\begin{onechants}
[Homage to all Buddhas,] the mighty who have arisen:\\
Taṇhaṅkara, the great hero, Medhaṅkara, the renowned,\\
Saraṇaṅkara, who guarded the world, Dīpaṅkara, the light-bearer,\\
Koṇḍañña, liberator of people, Maṅgala, great leader of people,\\
Sumana, kindly and wise, Revata, increaser of joy,\\
Sobhita, perfected in virtues, Anomadassī, greatest of beings,\\
Paduma, illuminer of the world, Nārada, true charioteer,\\
Padumuttara, most excellent of beings, Sumedha, the unequalled one,\\
Sujāta, summit of the world,  Piyadassī, great leader of men,\\
Atthadassī, the compassionate, Dhammadassī, destroyer of darkness,\\
Siddhattha, unequalled in the world,  and Tissa, speaker of Truth,\\
Phussa, bestower of blessings, Vipassī, the incomparable,\\
Sikhī, the bliss-bestowing teacher, Vessabhū, giver of happiness,\\
Kakusandha, the caravan leader, Koṇāgamana, abandoner of ills,\\
Kassapa, perfect in glory, Gotama, chief of the Sakyans.\\
\end{onechants}

\clearpage

\chapter{Āṭānāṭiya Paritta (long version)}% {{{1

\firstline{Namo me sabbabuddhānaṃ}

\paliText

\begin{leader}
\soloinstr{Solo introduction}

\begin{solotwochants}
Appasannehi nāthassa & sāsane sādhusammate\\
Amanussehi caṇḍehi & sadā kibbisakāribhi\\
Parisānañca-tassannam & ahiṃsāya ca guttiyā\\
Yandesesi mahāvīro & parittantam bhaṇāma se.\\
\end{solotwochants}
\end{leader}

\begin{twochants}
[Namo me sabbabuddhānaṃ] & uppannānaṃ mahesinaṃ\\
Taṇhaṅkaro mahāvīro & medhaṅkaro mahāyaso\\
Saraṇaṅkaro lokahito & dīpaṅkaro jutindharo\\
Koṇḍañño janapāmokkho & maṅgalo purisāsabho\\
Sumano sumano dhīro & revato rativaḍḍhano\\
Sobhito guṇasampanno & anomadassī januttamo\\
Padumo lokapajjoto & nārado varasārathī\\
Padumuttaro sattasāro & sumedho appaṭipuggalo\\
Sujāto sabbalokaggo & piyadassī narāsabho\\
Atthadassī kāruṇiko & dhammadassī tamonudo\\
Siddhattho asamo loke & tisso ca vadataṃ varo\\
Phusso ca varado buddho & vipassī ca anūpamo\\
Sikhī sabbahito satthā & vessabhū sukhadāyako\\
Kakusandho satthavāho & koṇāgamano raṇañjaho\\
Kassapo sirisampanno & gotamo sakyapuṅgavo\\
\end{twochants}

% TODO: Punctuation of the Pali is missing from here onwards. Either
% add, or remove from the above as well.

\clearpage

\englishText

\begin{onechants}
These and all self-enlightened Buddhas are also peerless ones,\\
All the Buddhas together, all of mighty power,\\
All endowed with the Ten Powers, attained to highest knowledge,\\
All of these are accorded the supreme place of leadership.\\
They roar the lion's roar with confidence among their followers,\\
They observe with the divine eye, unhindered, all the world.\\
The leaders endowed with the eighteen kinds of Buddha-Dhamma,\\
The thirty-two major and eighty minor marks of a great being,\\
Shining with fathom-wide haloes, all these elephant-like sages,\\
All these omniscient Buddhas, conquerors free of corruption,\\
Of mighty brilliance, mighty power, of mighty wisdom, mighty strength,\\
Of mighty compassion and wisdom, bearing bliss to all,\\
Islands, guardians and supports, shelters and caves for all beings,\\
Resorts, kinsmen and comforters, benevolent givers of refuge,\\
These are all the final resting place for the world with its deities.\\
With my head at their feet I salute these greatest of humans.\\
With both speech and thought I venerate those Tathāgatas,\\
Whether lying down, seated or standing, or walking anywhere.\\
May they ever guard your happiness, the Buddhas, bringers of peace,\\
And may you, guarded by them, at peace, freed from all fear,\\
Released from all illness, safe from all torments,\\
Having transcended hatred, may you gain cessation.\\
\end{onechants}

\clearpage

\paliText

\begin{twochants}
Ete caññe ca sambuddhā & anekasatakoṭayo\\
Sabbe buddhā asamasamā & sabbe buddhā mahiddhikā\\
Sabbe dasabalūpetā & vesārajjehupāgatā\\
Sabbe te paṭijānanti & āsabhaṇṭhānamuttamaṃ\\
Sīhanādaṃ nadantete & parisāsu visāradā\\
Brahmacakkaṃ pavattenti & loke appaṭivattiyaṃ\\
Upetā buddhadhammehi & aṭṭhārasahi nāyakā\\
Dvattiṃsa-lakkhaṇūpetā & sītyānubyañjanādharā\\
Byāmappabhāya suppabhā & sabbe te muṇikuñjarā\\
Buddhā sabbaññuno ete & sabbe khīṇāsavā jinā\\
Mahappabhā mahātejā & mahāpaññā mahabbalā\\
Mahākāruṇikā dhīrā & sabbesānaṃ sukhāvahā\\
Dīpā nāthā patiṭṭhā & ca tāṇā leṇā ca pāṇinaṃ\\
Gatī bandhū mahassāsā & saraṇā ca hitesino\\
Sadevakassa lokassa & sabbe ete parāyanā\\
Tesāhaṃ sirasā pāde & vandāmi purisuttame\\
Vacasā manasā ceva & vandāmete tathāgate\\
Sayane āsane ṭhāne & gamane cāpi sabbadā\\
Sadā sukhena rakkhantu & buddhā santikarā tuvaṃ\\
Tehi tvaṃ rakkhito santo & mutto sabbabhayena ca\\
Sabba-rogavinimutto & sabba-santāpavajjito\\
Sabba-veramatikkanto & nibbuto ca tuvaṃ bhava\\
\end{twochants}

\clearpage

\englishText

\begin{onechants}
By the power of their truth, their virtue and love,\\
May they protect and guard you in health and happiness.\\
In the Eastern quarter are beings of great power,\\
May they protect and guard you in health and happiness.\\
In the Southern quarter are deities of great power,\\
May they protect and guard you in health and happiness.\\
In the Western quarter are dragons of great power,\\
May they protect and guard you in health and happiness.\\
In the Northern quarter are spirits of great power,\\
May they protect and guard you in health and happiness.\\
In the East is Dhataraṭṭha, in the South is Viruḷhaka,\\
In the West is Virūpakkha, Kuvera rules the North.\\
These Four Mighty Kings, far-famed guardians of the world,\\
May they all be your protectors in health and happiness.\\
Sky-dwelling and earth-dwelling gods and dragons of great power,\\
May they all be your protectors in health and happiness.\\
For me there is no other refuge, the Buddha is my excellent refuge:\\
By this declaration of truth may the blessings of victory be yours.\\
For me there is no other refuge, the Dhamma is my excellent refuge:\\
By this declaration of truth may the blessings of victory be yours.\\
For me there is no other refuge, the Sangha is my excellent refuge:\\
By this declaration of truth may the blessings of victory be yours.\\
\end{onechants}

\clearpage

\paliText

\begin{twochants}
Tesaṃ saccena sīlena & khantimettābalena ca\\
Tepi tumhe%
% FIXME: footnote
\footnote{If chanting for oneself, change \textit{tumhe} to \textit{amhe} here and in the lines below.}
anurakkhantu & ārogyena sukhena ca\\
Puratthimasmiṃ disābhāge & santi bhūtā mahiddhikā\\
Tepi tumhe anurakkhantu & ārogyena sukhena ca\\
Dakkhiṇasmiṃ disābhāge & santi devā mahiddhikā\\
Tepi tumhe anurakkhantu & ārogyena sukhena ca\\
Pacchimasmiṃ disābhāge & santi nāgā mahiddhikā\\
Tepi tumhe anurakkhantu & ārogyena sukhena ca\\
Uttarasmiṃ disābhāge & santi yakkhā mahiddhikā\\
Tepi tumhe anurakkhantu & ārogyena sukhena ca\\
Purimadisaṃ dhataraṭṭho & dakkhiṇena viruḷhako\\
Pacchimena virūpakkho & kuvero uttaraṃ disaṃ\\
Cattāro te mahārājā & lokapālā yasassino\\
Tepi tumhe anurakkhantu & ārogyena sukhena ca\\
Ākāsaṭṭhā ca bhummaṭṭhā & devā nāgā mahiddhikā\\
Tepi tumhe anurakkhantu & ārogyena sukhena ca\\
Natthi me saraṇaṃ aññaṃ & buddho me saraṇaṃ varaṃ\\
Etena saccavajjena & hotu te%
% FIXME: footnote
\footnote{If chanting for oneself, change \textit{te} to \textit{me} here and in the lines below.}
jayamaṅgalaṃ\\
Natthi me saraṇaṃ aññaṃ & dhammo me saraṇaṃ varaṃ\\
Etena saccavajjena & hotu te jayamaṅgalaṃ\\
Natthi me saraṇaṃ aññaṃ & saṅgho me saraṇaṃ varaṃ\\
Etena saccavajjena & hotu te jayamaṅgalaṃ\\
\end{twochants}

\clearpage

\englishText

\begin{onechants}
Whatever jewel may be found in the world, however splendid,\\
There is no jewel equal to the Buddha, therefore may you be blessed.\\
Whatever jewel may be found in the world, however splendid,\\
There is no jewel equal to the Dhamma, therefore may you be blessed.\\
Whatever jewel may be found in the world, however splendid,\\
There is no jewel equal to the Sangha, therefore may you be blessed.\\
If you venerate the Buddha jewel, the supreme, excellent protection,\\
Which benefits gods and humans, then in safety, by the Buddha's power,\\
All dangers will be prevented, your sorrows will pass away.\\
If you venerate the Dhamma jewel, the supreme, excellent protection,\\
Which calms all fevered states, then in safety, by the Dhamma's power,\\
All dangers will be prevented, your fears will pass away.\\
If you venerate the Sangha jewel, the supreme, excellent protection,\\
Worthy of gifts and hospitality, then in safety, by the Sangha's power,\\
All dangers will be prevented, your sicknesses will pass away.\\
May all calamities be avoided, may all illness pass away,\\
May no dangers threaten you, may you be happy and long-lived,\\
Greeted kindly and welcome everywhere.\\
May four things accrue to you: long life, beauty, bliss, and strength.\\
\end{onechants}

\clearpage

\paliText

\begin{twochants}
Yaṅkiñci ratanaṃ loke & vijjati vividhaṃ puthu\\
Ratanaṃ buddhasamaṃ & natthi tasmā sotthī bhavantu te\\
Yaṅkiñci ratanaṃ loke & vijjati vividhaṃ puthu\\
Ratanaṃ dhammasamaṃ & natthi tasmā sotthī bhavantu te\\
Yaṅkiñci ratanaṃ loke & vijjati vividhaṃ puthu\\
Ratanaṃ saṅghasamaṃ & natthi tasmā sotthī bhavantu te\\
Sakkatvā buddharatanaṃ & osathaṃ uttamaṃ varaṃ\\
Hitaṃ devamanussānaṃ & buddhatejena sotthinā\\
Nassantupaddavā sabbe & dukkhā vūpasamentu te\\
Sakkatvā dhammaratanaṃ & osathaṃ uttamaṃ varaṃ\\
Pariḷāhūpasamanaṃ & dhammatejena sotthinā\\
Nassantupaddavā sabbe & bhayā vūpasamentu te\\
Sakkatvā saṅgharatanaṃ & osathaṃ uttamaṃ varaṃ\\
Āhuneyyaṃ pāhuneyyaṃ & saṅghatejena sotthinā\\
Nassantupaddavā sabbe & rogā vūpasamentu te\\
Sabbītiyo vivajjantu & sabbarogo vinassatu\\
Mā te bhavatvantarāyo & sukhī dīghāyuko bhava\\
Abhivādanasīlissa & niccaṃ vuḍḍhāpacāyino\\
Cattāro dhammā vaḍḍhanti & āyu vaṇṇo sukhaṃ balaṃ\\
\end{twochants}

\setEnglishTextSize{12}{18}{9pt}
\resumeNormalText

% End of parittas.tex
