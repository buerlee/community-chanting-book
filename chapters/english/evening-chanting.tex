% vim: foldmethod=marker foldlevel=0 foldtext=FoldText()

\chapter{Dedication of Offerings}%{{{1

[Yo so] bha꜕gavā a꜕rahaṃ sammāsambuddho\\
Svākkhā꜓to yena bha꜕gava꜓tā dhammo\\
Supaṭi꜕panno yassa bha꜕gava꜕to sāvaka꜕saṅgho\\
Tam-ma꜓yaṃ bha꜕gavantaṃ sa꜕dhammaṃ sa꜕saṅghaṃ\\
Imehi꜓ sakkārehi꜕ yathārahaṃ āropi꜕tehi a꜕bhi꜓pūja꜕yāma\\
Sādhu꜓ no bhante bha꜕gavā su꜕cira-parinibbu꜕topi\\
Pacchi꜓mā-ja꜕na꜓tānu꜓kampa꜕-mānasā\\
Ime sakkāre dugga꜕ta꜕-paṇṇākāra꜓-bhūte pa꜕ṭiggaṇhātu\\
Amhā꜓kaṃ dīgha꜕rattaṃ hi꜕tāya su꜕khāya\\
Arahaṃ sammāsambuddho bha꜕gavā\\
Buddhaṃ bha꜕gavantaṃ a꜕bhi꜓vādemi \instr{Bow}

[Svākkhā꜓to] bha꜕gava꜓tā dhammo\\
Dhammaṃ namassāmi \instr{Bow}

[Supaṭi꜕panno] bha꜕gava꜕to sāvaka꜕saṅgho\\
Sa꜓ṅghaṃ na꜕māmi \instr{Bow}

%}}}1
\clearpage

\chapter{Dedication of Offerings}%{{{1

[To the Ble꜕ssed One,] the꜕ Lord, who fu꜓lly a꜕ttained\\
\vin perfect enli꜕ghtenment,\\
To the꜕ Teaching, which he expo꜕unde꜕d so well,\\
And to the꜕ Blessed One's disci꜓ples who have pra꜕ctised well,\\
To these --- the꜕ Buddha, the꜕ Dhamma, a꜕nd the Sa꜓ngha ---\\
We render wi꜕th offerings our ri꜓ghtful ho꜖mage.\\
It is we꜓ll for us that the Ble꜕ssed One, having attained li꜕bera꜓tion,\\
Still had co꜕mpassion for later ge꜓nera꜖tions.\\
May the꜕se simple o꜓fferings be acce꜕pted\\
For o꜕ur long-lasting benefit and fo꜕r the ha꜓ppiness it giv꜖es us.\\
The꜕ Lord, the꜕ Perfectly Enli꜓ghtened and Ble꜕ssed One ---\\
I꜕ render homage to꜕ the Bu꜓ddha, the Ble꜕ssed One. \instr{Bow}

[The꜕ Teaching,] so co꜕mpletely expla꜓ined by him ---\\
I bo꜖w to꜕ the꜕ Dha꜕mma. \instr{Bow}

[The꜕ Blessed One's disci꜓ples,] who have pra꜕ctised well ---\\
I bo꜖w to꜕ the꜕ Sa꜕ngha. \instr{Bow}

%}}}1
\clearpage

\chapter{Preliminary Homage}%{{{1

\begin{leader}
  [Ha꜓nda mayaṃ buddhassa꜕ bhagavato pubbabhāga-namakā꜕raṃ karomase]
\end{leader}

Namo tassa bha꜕gava꜕to araha꜕to sa꜓mmāsa꜓mbuddha꜕ssa

\instr{Three times}

\chapter{Recollection of the Buddha}%{{{1

\begin{leader}
  [Ha꜓nda mayaṃ buddhānu꜕ssa꜕ti꜕nayaṃ karomase]
\end{leader}

Taṃ kho꜓ pana bha꜕gavantaṃ evaṃ kaly꜓āṇo kitti꜕saddo abbhugga꜕to\\
Itipi so bha꜕gavā a꜕rahaṃ sammāsambuddho\\
Vijjāca꜕raṇa꜓-sampanno su꜕ga꜕to loka꜕vi꜓dū\\
Anu꜓tta꜕ro purisa꜕damma-sārathi\\
\vin satthā deva-ma꜕nussānaṃ buddho bha꜕gavā'ti

%}}}1
\clearpage

\chapter{Preliminary Homage}%{{{1

\begin{leader}
  [No꜓w let us pay preliminary homage to the Bu꜕ddha.]
\end{leader}

Ho꜓ma꜓ge to the꜕ Ble꜕ssed, No꜓ble, a꜕nd Pe꜕rfectly Enli꜓ghtened One.

\instr{Three times}

\chapter{Recollection of the Buddha}%{{{1

\begin{leader}
  [No꜓w let us chant the recollection of the Bu꜕ddha.]
\end{leader}

A꜕ good word of the꜕ Blessed One's re꜕puta꜓tion has spread as fo꜕llows:\\
He, the꜕ Ble꜕ssed One, is indeed the Pu꜓re One,\\
\vin the꜕ Perfectly Enli꜓ghtened One;\\
He is i꜕mpeccable i꜕n conduct and u꜕ndersta꜓nding,\\
\vin the A꜕cco꜓mplished One, the꜕ Knower o꜓f the꜕ Worlds;\\
He trains perfectly tho꜕se who wi꜓sh to꜓ be꜕ trained;\\
\vin he is Teacher of go꜓ds and hu꜕mans; he is Awake and Ho꜕ly.

%}}}1
\clearpage

\chapter{Supreme Praise of the Buddha}%{{{1

\begin{leader}
  [Ha꜓nda mayaṃ buddhābhi꜕gī꜕tiṃ karomase]
\end{leader}

Buddh'vāra꜕ha꜓nta-varatādi꜕gu꜓ṇābhi꜕yutto\\
Suddhābhi꜕ñāṇa-ka꜕ru꜓ṇāhi sa꜓māga꜕tatto\\
Bodhesi꜕ yo su꜕ja꜕na꜓taṃ ka꜕ma꜓laṃ va꜕ sūro\\
Vandāmaha꜓ṃ ta꜕mara꜕ṇaṃ si꜕rasā꜓ ji꜕nendaṃ\\
Buddho yo sabba꜕pāṇīnaṃ sa꜕raṇaṃ khema꜕mutta꜕maṃ\\
Pa꜕ṭhamānussa꜕tiṭṭhānaṃ vandāmi꜕ taṃ si꜓rena꜕haṃ\\
Buddhassā꜓ha꜕smi dāso/dāsī va buddho me sā꜕miki꜓ssaro\\
Buddho dukkhassa꜕ ghātā ca꜕ vidhātā ca꜕ hi꜓tassa꜕ me\\
Buddhassāha꜓ṃ niyyādemi sa꜕rīrañjīvi꜕tañci꜕daṃ\\
% NOTE: mid-line Pali caps for alternative forms
Vandantoha꜓ṃ/Vandantīha꜓ṃ ca꜕rissāmi buddhasseva꜕ su꜓bodhi꜕taṃ\\
Natthi me sa꜕ra꜓ṇaṃ aññaṃ buddho me sa꜕ra꜓ṇaṃ va꜕raṃ\\
Etena sacca꜕vajjena vaḍḍheyyaṃ sa꜕tthu-sā꜓sane\\
Buddhaṃ me vanda꜕mānena/vanda꜕mānāya\\
\vin yaṃ puññaṃ pa꜕su꜓taṃ i꜕dha\\
Sa꜕bbepi anta꜕rāyā me māhe꜓su꜓ṃ ta꜕ssa꜓ teja꜕sā

\begin{instruction}
  Bowing
\end{instruction}

Kāyena vācāya va ceta꜕sā꜓ vā\\
Bu꜓ddhe ku꜕kammaṃ pa꜕kataṃ ma꜕yā yaṃ\\
Bu꜓ddho pa꜕ṭiggaṇhā꜕tu acca꜕yantaṃ\\
Kālantare sa꜓ṃvarituṃ va꜕ bu꜓ddhe

%}}}1
\clearpage

\chapter{Supreme Praise of the Buddha}%{{{1

\begin{leader}
  [No꜓w let us chant the supreme praise of the Bud꜕dha.]
\end{leader}

The꜕ Buddha, the꜕ truly wo꜓rthy one,\\
\vin e꜕ndowed with such e꜓xcellent qua꜕lities,\\
Whose being is composed of pu꜕rity,\\
\vin transcendental wi꜓sdom, and compa꜕ssion,\\
Who has e꜕nlightened the wise like the꜕ sun awa꜓kening the lo꜕tus ---\\
I bow m꜕y head to tha꜕t peaceful chi꜓ef of co꜕nquerors.\\
The꜕ Buddha, who is the꜕ safe, se꜕cure re꜓fuge of al꜖l beings ---\\
As the꜕ first object of re꜕colle꜓ction, I꜕ venerate him with bo꜓wed head.\\
I am indeed the Buddha's se꜕rvant, the꜕ Buddha is my Lo꜓rd a꜕nd Guide.\\
The꜕ Buddha is sorrow's destro꜕yer, who bestows ble꜓ssi꜓ngs o꜕n me.\\
To the꜕ Buddha I de꜓dicate this bo꜕d꜕y and life,\\
And in de꜕votion I wi꜕ll walk the Buddha's Pa꜓th of A꜕wa꜕kening.\\
For me there is no other re꜕fuge, the꜕ Buddha is my e꜓xcelle꜕nt re꜕fuge.\\
By the꜕ utterance of thi꜕s Truth, may I grow in the Ma꜓ste꜕r's Way.\\
By my de꜕votion to the Bu꜕ddha, and the꜕ blessing of this pra꜓ctice ---\\
By i꜕ts power, may a꜕ll obstacles be o꜓ve꜕rcome.

\begin{instruction}
  Bowing
\end{instruction}

By body, speech, o꜕r mind,\\
For whatever wro꜕ng action I have co꜕mmitted towards the Bu꜕ddha,\\
May my a꜕cknowledgement of fault be acce꜓pted,\\
That i꜕n future there may be re꜕straint regarding the Bu꜕ddha.

%}}}1
\clearpage

\chapter{Recollection of the Dhamma}%{{{1

\begin{leader}
  [Ha꜓nda mayaṃ dhammānu꜕ssa꜕ti꜕nayaṃ karomase]
\end{leader}

Svākkhā꜓to bha꜕gava꜓tā dhammo\\
Sa꜓ndiṭṭhi꜕ko a꜕kāli꜕ko ehi꜕passi꜕ko\\
Opanayi꜕ko pa꜕cca꜕ttaṃ vedi꜓ta꜕bbo viññūhī'ti

% TODO: 'ti with apostrophe here. This is clearer to read. But either
% this way consistently or nowhere.

\chapter{Supreme Praise of the Dhamma}%{{{1

\begin{leader}
  [Ha꜓nda mayaṃ dhammābhi꜕gī꜕tiṃ karomase]
\end{leader}

Svākkhā꜓ta꜕tādi꜕guṇa-yoga꜕-va꜓sena꜕ seyyo\\
Yo magga꜕pāka-pa꜕riyatti꜕-vi꜓mokkha꜕-bhedo\\
Dhammo ku꜕loka-pa꜕ta꜓nā ta꜕da꜓-dhāri꜕-dhārī\\
Vandāmaha꜓ṃ ta꜕maha꜕raṃ va꜕radha꜓mma꜕metaṃ\\
Dhammo yo sabba꜕pāṇīnaṃ sa꜕raṇaṃ khema꜕mutta꜕maṃ\\
Du꜕tiyānussa꜕tiṭṭhānaṃ vandāmi꜕ taṃ si꜓rena꜕haṃ\\
Dhammassā꜓ha꜕smi dāso/dāsī va dhammo me sā꜕miki꜓ssaro\\
Dhammo dukkhassa꜕ ghātā ca꜕ vidhātā ca꜕ hi꜓tassa꜕ me\\
Dhammassāha꜓ṃ niyyādemi sa꜕rīrañjīvi꜕tañci꜕daṃ\\
Vandantoha꜓ṃ/Vandantīha꜓ṃ ca꜕rissāmi dhammasseva꜕ su꜓dhamma꜕taṃ\\
Natthi me sa꜕ra꜓ṇaṃ aññaṃ dhammo me sa꜕ra꜓ṇaṃ va꜕raṃ\\
Etena sacca꜕vajjena vaḍḍheyyaṃ sa꜕tthu-sā꜓sane\\
Dhammaṃ me vanda꜕mānena/vanda꜕mānāya\\
\vin yaṃ puññaṃ pa꜕su꜓taṃ i꜕dha\\
Sa꜕bbepi anta꜕rāyā me māhe꜓su꜓ṃ ta꜕ssa꜓ teja꜕sā

%}}}1
\clearpage

\chapter{Recollection of the Dhamma}%{{{1

\begin{leader}
  [No꜓w let us chant the recollection of the Dha꜕mma.]
\end{leader}

The꜕ Dhamma is we꜕ll expo꜓unded by the Ble꜕ssed One,\\
Apparent here a꜕nd now, timeless, e꜕ncouraging inve꜕stiga꜓tion,\\
Leading i꜕nwards, to be e꜕xperienced indi꜕vidually by꜓ the꜕ wise.

\chapter{Supreme Praise of the Dhamma}%{{{1

\begin{leader}
  [No꜓w let us chant the supreme praise of the Dha꜕mma.]
\end{leader}

It i꜕s excellent be꜕cause it is `we꜕ll expo꜓unded,'\\
And it can be di꜕vided into꜕ Path and Fruit, Learning and Li꜕bera꜓tion.\\
The꜕ Dhamma holds those who u꜕phold it from fa꜕lli꜕ng i꜕nto꜕ delu꜓sion.\\
I revere the꜕ excellent Te꜓aching, that which removes da꜕rkness ---\\
The꜕ Dhamma, which is the su꜕preme, se꜕cure re꜓fuge of al꜖l beings ---\\
As the꜕ Second Object of Re꜕colle꜓ction, I꜕ venerate it with bo꜓wed head.\\
I am indeed the Dhamma's se꜕rvant, the꜕ Dhamma is my Lo꜓rd a꜕nd Guide.\\
The꜕ Dhamma is sorrow's destro꜕yer, and it bestows ble꜓ssi꜓ngs o꜕n me.\\
To the꜕ Dhamma I de꜓dicate this bo꜕d꜕y and life,\\
And in de꜕votion I wi꜕ll walk this excellent wa꜓y o꜕f Truth.\\
For me there is no other re꜕fuge, the꜕ Dhamma is my e꜓xcelle꜕nt re꜕fuge.\\
By the꜕ utterance of thi꜕s Truth, may I grow in the Ma꜓ste꜕r's Way.\\
By my de꜕votion to the Dha꜕mma, and the꜕ blessing of this pra꜓ctice ---\\
By i꜕ts power, may a꜕ll obstacles be o꜓ve꜕rcome.

%}}}1
\clearpage

\begin{instruction}%{{{1
  Bowing
\end{instruction}

Kāyena vācāya va ceta꜕sā꜓ vā\\
Dha꜓mme ku꜕kammaṃ pa꜕kataṃ ma꜕yā yaṃ\\
Dha꜓mmo pa꜕ṭiggaṇhā꜕tu acca꜕yantaṃ\\
Kālantare sa꜓ṃvarituṃ va꜕ dha꜓mme

\chapter{Recollection of the Sangha}%{{{1

\begin{leader}
  [Ha꜓nda mayaṃ saṅghānu꜕ssa꜕ti꜕nayaṃ karomase]
\end{leader}

Supaṭi꜕panno bha꜕gava꜕to sāvaka꜕saṅgho\\
Ujupaṭi꜕panno bha꜕gava꜕to sāvaka꜕saṅgho\\
Ñāyapaṭi꜕panno bha꜕gava꜕to sāvaka꜕saṅgho\\
Sā꜓mīci꜕pa꜕ṭi꜕panno bha꜕gava꜕to sāvaka꜕saṅgho\\
Yadidaṃ cattāri purisa꜕yugāni aṭṭha꜓ purisa꜕pugga꜕lā\\
Esa bha꜕gava꜕to sāvaka꜕saṅgho\\
Āhu꜕ṇeyyo pāhu꜕ṇeyyo dakkhi꜕ṇeyyo añja꜕li-ka꜕ra꜓ṇīyo\\
Anu꜓tta꜕raṃ puññakkhe꜕ttaṃ lokassā'ti

\chapter{Supreme Praise of the Sangha}%{{{1

\begin{leader}
  [Ha꜓nda mayaṃ saṅghābhi꜕gī꜕tiṃ karomase]
\end{leader}

Sa꜕ddhammajo supaṭipatti꜕-gu꜓ṇādi꜕yutto\\
Yoṭṭhabbi꜕dho ari꜓yapugga꜕la꜓-saṅgha꜕seṭṭho\\
Sī꜓lādi꜕dhamma-pa꜕varāsa꜕ya꜓-kāya꜕-citto\\
Vandāmaha꜓ṃ ta꜕m-ari꜕yāna꜕-gaṇa꜓ṃ su꜕suddhaṃ\\
Sa꜓ṅgho yo sabba꜕pāṇīnaṃ sa꜕raṇaṃ khema꜕mutta꜕maṃ\\
Ta꜕tiyānussa꜕tiṭṭhānaṃ vandāmi꜕ taṃ si꜓rena꜕haṃ

%}}}1
\clearpage

\begin{instruction}%{{{1
  Bowing
\end{instruction}

By body, speech, o꜕r mind,\\
For whatever wro꜕ng action I have co꜕mmitted towards the Dha꜕mma,\\
May my a꜕cknowledgement of fault be acce꜓pted,\\
That i꜕n future there may be re꜕straint regarding the Dha꜕mma.

\chapter{Recollection of the Sangha}%{{{1

\begin{leader}
  [No꜓w let us chant the recollection of the Sa꜕ngha.]
\end{leader}

They are the꜕ Blessed One's disci꜓ples, who have pra꜕ctised well,\\
Who have practised dire꜕ctly,\\
Who have practised insi꜓ghtfully,\\
Those who pra꜓ctise with inte꜕grity ---\\
That is the fo꜕ur pairs, the꜕ eight kinds of no꜓ble꜕ beings ---\\
These are the꜕ Blessed One's disci꜓ples.\\
Such ones a꜕re worthy of gifts, worthy of ho꜕spita꜓lity,\\
\vin worthy of o꜕fferings, worthy o꜓f re꜕spect;\\
They give o꜕ccasion for i꜕nco꜕mparable go꜓odness to ari꜕se i꜕n the world.

%}}}1
\chapter{Supreme Praise of the Sangha}%{{{1

\begin{leader}
  [No꜓w let us chant the supreme praise of the Sa꜕ngha.]
\end{leader}

Born o꜕f the Dha꜓mma, tha꜕t Sangha which has pra꜓cti꜕sed well,\\
The field of the꜕ Sangha formed o꜕f eight kinds of no꜓ble꜕ beings,\\
Guided in body a꜕nd mind b꜕y excellent mora꜓lity and vi꜕rtue.\\
I revere that a꜕ssembly of no꜓ble beings pe꜕rfected in pu꜕rity.\\
The꜕ Sangha, which is the su꜕preme, se꜕cure re꜓fuge of al꜖l beings ---\\
As the꜕ Third Object of Re꜕colle꜓ction, I꜕ venerate it with bo꜓wed head.

%}}}1
\enlargethispage{\baselineskip}
\clearpage

Saṅghassā꜓ha꜕smi dāso/dāsī va saṅgho me sā꜕miki꜓ssaro\\%{{{
Sa꜓ṅgho dukkhassa꜕ ghātā ca꜕ vi꜓dhātā ca꜕ hi꜓tassa꜕ me\\
Saṅghassāha꜓ṃ niyyādemi sa꜕rīrañjīvi꜕tañci꜕daṃ\\
Vandantoha꜓ṃ/Vandantīha꜓ṃ ca꜕rissāmi saṅghassopa꜕ṭi꜓panna꜕taṃ\\
Natthi me sa꜕ra꜓ṇaṃ aññaṃ saṅgho me sa꜕ra꜓ṇaṃ va꜕raṃ\\
Etena sacca꜕vajjena vaḍḍheyyaṃ sa꜕tthu-sā꜓sane\\
Sa꜓ṅghaṃ me vanda꜕mānena/vanda꜕mānāya\\
\vin yaṃ puññaṃ pa꜕su꜓taṃ i꜕dha\\
Sa꜕bbepi anta꜕rāyā me māhe꜓su꜓ṃ ta꜕ssa꜓ teja꜕sā

\begin{instruction}
  Bowing
\end{instruction}

Kāyena vācāya va ceta꜕sā꜓ vā\\
Sa꜓ṅghe ku꜕kammaṃ pa꜕kataṃ ma꜕yā yaṃ\\
Sa꜓ṅgho pa꜕ṭiggaṇhā꜕tu acca꜕yantaṃ\\
Kālantare sa꜓ṃvarituṃ va꜕ sa꜓ṅghe

\vfill

\begin{instruction}
  At this time meditation is practised in silence, sometimes followed by a Dhamma talk, and ending with the following:
\end{instruction}

\chapter{Closing Homage}%{{{1

[Arahaṃ] sammāsambuddho bha꜕gavā\\
Buddhaṃ bha꜕gavantaṃ a꜕bhi꜓vādemi \instr{Bow}

[Svākkhā꜓to] bha꜕gava꜓tā dhammo\\
Dhammaṃ namassāmi \instr{Bow}

[Supaṭi꜕panno] bha꜕gava꜕to sāvaka꜕saṅgho\\
Sa꜓ṅghaṃ na꜕māmi \instr{Bow}

%}}}1
\clearpage

I am indeed the Sangha's se꜕rvant, the꜕ Sangha is my Lo꜓rd a꜕nd Guide.\\%{{{
The꜕ Sangha is sorrow's destro꜕yer and it bestows ble꜓ssi꜓ngs o꜕n me.\\
To the꜕ Sangha I de꜓dicate this bo꜕dy꜕ and life,\\
And in de꜕votion I wi꜕ll walk the well-practised wa꜓y of the꜕ Sa꜕ngha.\\
For me there is no other re꜕fuge, the꜕ Sangha is my e꜓xcelle꜕nt re꜕fuge.\\
By the꜕ utterance of thi꜕s truth, may I grow in the Ma꜓ste꜕r's Way.\\
By my de꜕votion to the Sa꜕ngha, and the꜕ blessing o꜕f this pra꜓ctice ---\\
By i꜕ts power, may a꜕ll obstacles be o꜓ve꜕rcome.

\begin{instruction}
  Bowing
\end{instruction}

By body, speech, o꜕r mind,\\
For whatever wro꜕ng action I have co꜕mmitted towards the Sa꜕ngha,\\
May my a꜕cknowledgement of fault be acce꜓pted,\\
That i꜕n future there may be re꜕straint regarding the Sa꜕ngha.

\vfill

\begin{instruction}
  At this time meditation is practised in silence, sometimes followed by a Dhamma talk, and ending with the following:
\end{instruction}

\chapter{Closing Homage}%{{{1

[The꜕ Lord,] the꜕ Perfectly Enli꜓ghtened and Ble꜕ssed One ---\\
I꜕ render homage to꜕ the Bu꜓ddha, the Ble꜕ssed One. \instr{Bow}

[The꜕ Teaching,] so co꜕mpletely expla꜓ined by him ---\\
I bo꜖w to꜕ the꜕ Dha꜕mma. \instr{Bow}

[The꜕ Blessed One's disci꜓ples,] who have pra꜕ctised well ---\\
I bo꜖w to꜕ the꜕ Sa꜕ngha. \instr{Bow}

%}}}1

% End of evening-chanting.tex
