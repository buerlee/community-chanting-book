% vim: foldmethod=marker foldlevel=0 foldtext=FoldText()

\chPali
\chapter[Sharing and Aspiration]{Verses of Sharing and Aspiration}% {{{1

\begin{leader}
  [Ha꜓nda mayaṃ uddissanādhiṭṭhāna-gāthā꜓yo b꜕haṇāmase]
\end{leader}

[Iminā puñña꜕kammena] u꜕pajjhāyā gu꜕ṇutta꜕rā\\
Ācariyūpa꜕kārā ca꜕ mātāpitā ca꜕ ñāta꜕kā\\
Suriyo candimā rājā gu꜕ṇavantā na꜕rāpi꜕ ca꜕\\
Brahma-mārā ca꜕ indā ca꜕ loka꜕pālā ca꜕ deva꜕tā\\
Yamo mittā ma꜕nussā ca majjhattā veri꜕kāpi꜕ ca꜕\\
Sa꜕bbe sattā sukhī hontu puññāni pa꜕ka꜕tāni꜕ me\\
Sukhañca tividhaṃ dentu꜕ khippaṃ pāpetha꜕ voma꜕taṃ\\
Iminā puññakammena iminā uddi꜕ssena꜕ ca꜕\\
Khippāhaṃ su꜕la꜕bhe ceva taṇhūpādāna꜕-cheda꜕naṃ\\
Ye santāne hīnā dhammā yāva꜕ nibbāna꜕to ma꜕maṃ\\
Nassantu sabba꜕dā yeva yattha꜕ jāto bha꜕ve bha꜕ve\\
Ujucittaṃ sa꜕ti꜕paññā sallekho vi꜕ri꜕yamhinā\\
Mārā labhantu nokāsaṃ kātuñca vi꜕ri꜕yes꜕u me\\
Buddhādhipa꜕va꜕ro nātho dhammo nātho va꜕rutta꜕mo\\
Nātho pacceka꜕buddho ca꜕ saṅgho nāthotta꜕ro ma꜕maṃ\\
Tesottamānubhāvena mārokāsaṃ la꜕bhantu꜕ mā.

\chEnglish
\chapter[Sharing and Aspiration]{Verses of Sharing and Aspiration}% {{{1

\begin{leader}
[No꜓w let us chant the verses of sharing and aspira꜕tion.]
\end{leader}

Through the꜕ goodness that ari꜓ses from my pra꜕ctice,\\
May m꜕y spiritual te꜓achers and \prul{guides} of great vi꜕rtue,\\
M꜕y mother, my fa꜓ther, and my re꜕latives,\\
The Sun a꜕nd the꜓ Moon, and a꜕ll \prul{virtuous} le꜓aders of the꜕ world,\\
May the꜕ highest gods and evil fo꜕rces,\\
Cele꜕stia꜓l beings, \prul{guardian} spi꜓rits of the꜕ Earth, and the Lo꜕rd o꜓f Death,\\
May \prul{those} who꜕ are frie꜓ndly, indifferent, or ho꜕stile,\\
May a꜕ll beings receive the ble꜓ssi꜓ngs of m꜕y life,\\
May the꜕y soon attain the thre꜓efo꜕ld bliss and realise the De꜕athless.\\
Through the꜕ goodness that ari꜓ses from my pra꜕ctice,\\
And through this act of sha꜕ring,\\
May all de꜕sires and atta꜓chments quickl꜖y cease\\
And a꜕ll harmful sta꜕tes o꜓f mind.\\
Until I realize Nibbā꜕na,\\
In every kind o꜕f birth, may I ha꜕ve an u꜓pri꜕ght mind,\\
Wi꜕th mindfulness and wi꜕sdom, auste꜓rity and vi꜕gour.\\
May the꜕ forces of delu꜓sion not ta꜕ke hold nor weaken m꜖y re꜓solve.\\
The꜕ Buddha is my e꜓xcellent re꜕fuge,\\
Unsu꜕rpassed is the prote꜓ction of the Dhamma꜕,\\
The꜕ Solitary Bu꜓ddha is my no꜕ble guide,\\
The꜕ Sangha is my supre꜓me su꜕pport.\\
Through the꜕ supreme po꜓we꜓r of a꜕ll these,\\
Ma꜕y darkness and delu꜓sion be di꜕spelled.

\chapter[Sharing of Merit]{Verses on the Sharing of Merit}% {{{1

\begin{leader}
  [Ha꜓nda mayaṃ sa꜕bba-patti-dāna-gāthā꜓yo bha꜕ṇāmase]
\end{leader}

Puññass'i꜕dāni꜓ ka꜕ta꜕ssa yān'aññāni꜓ ka꜕tāni꜓ me\\
Tesa꜓ñ-ca꜕ bhāgi꜕no ho꜓ntu꜕ sa꜕ttānantā꜕ppa꜕māṇa꜓kā

\begin{english}
  May whate꜕ve꜕r li꜕vi꜕ng be꜕ings,\\
  Without measure wi꜓tho꜕ut end\\
  Parta꜕ke o꜕f al꜕l th꜕e me꜕rit\\
  From the good deeds I꜓ ha꜕ve done:
\end{english}

Ye pi꜕yā gu꜓ṇavantā ca꜕ mayhaṃ mātā-pi꜕tā-da꜓yo\\
Diṭṭhā me cāpy꜕-adiṭṭhā vā aññe majjh꜓att꜕a-veri꜓no

\begin{english}
  Those loved an꜓d full of go꜕odness,\\
  My mother and my fa꜓th꜕er dear,\\
  Beings seen by me and tho꜕se u꜕nseen,\\
  Those neutral an꜓d a꜕verse,
\end{english}

Sa꜕ttā tiṭṭha꜓nti꜕ lokasmiṃ te-bhummā ca꜕tu꜕-yoni꜓kā\\
Pañc'eka꜕-ca꜕tu꜕-vokārā sa꜓ṃsa꜕rantā bh꜕avābha꜕ve

\begin{english}
  Beings esta꜕bli꜕shed i꜕n th꜕e world\\
  From the three planes and four gro꜓unds o꜕f birth,\\
  With five a꜕ggr꜕ega꜕tes o꜕r on꜕e o꜕r fo꜕ur,\\
  Wand'ring on from re꜓alm to꜕ realm,
\end{english}

Ñātaṃ ye pa꜕tti꜓-dānam-me a꜕nu꜓modantu꜕ te sa꜕yaṃ\\
Ye c'imaṃ nappa꜕jānanti devā tesa꜓ṃ ni꜕veda꜕yuṃ

\begin{english}
  Those who kno꜕w m꜕y ac꜕t of꜕ de꜕di꜕ca꜕tion,\\
  May they all rejo꜓ice in꜕ it\\
  And as for those yet u꜕na꜕ware,\\
  May the devas le꜓t th꜕em know.
\end{english}

Ma꜓yā dinnāna-puññānaṃ a꜕nu꜓moda꜕na-he꜓tu꜕nā\\
Sa꜕bbe sa꜕ttā sa꜕dā ho꜓ntu꜕ a꜕verā su꜕kh꜕a-jīvi꜓no\\
Kh꜓ema꜓ppa꜕dañ-ca꜕ pa꜕ppontu꜕ tesā꜓sā꜓ si꜕jjha꜕taṃ su꜕bhā

\begin{english}
  By rejo꜕ici꜕ng in꜕ my꜕ sha꜕ring\\
  May all beings liv꜓e a꜕t ease,\\
  In fre꜕edo꜕m fro꜕m ho꜕st꜕il꜕ity,\\
  May their good wishes b꜓e fu꜕lfilled\\
  And may they a꜕ll re꜕ach sa꜕fety.
\end{english}

\chPali
\chapter[Loving-Kindness]{The Buddha's Words on Loving-Kindness}% {{{1

% TODO: Handa mayam
\begin{leader}
  [No꜓w let us chant the Buddha's words on loving-ki꜕ndness.]
\end{leader}

[Karaṇīyam-attha-kusalena,]\\
Yan-taṃ santaṃ padaṃ abhisamecca;\\
Sakko ujū ca suhujū ca,\\
Suvaco c'assa mudu anatimānī,

Santussako ca subharo ca,\\
Appakicco ca sallahuka-vutti;\\
Sant'indriyo ca nipako ca,\\
Appagabbho kulesu ananugiddho.

Na ca khuddaṃ samācare kiñci,\\
Yena viññū pare upavadeyyuṃ;\\
Sukhino vā khemino hontu,\\
Sabbe sattā bhavantu sukhit'attā:

Ye keci pāṇa-bhūt'atthi,\\
Tasā vā thāvarā vā anavasesā;\\
Dīghā vā ye mahantā vā,\\
Majjhimā rassakā aṇuka-thūlā.

Diṭṭhā vā ye ca adiṭṭhā,\\
Ye ca dūre vasanti avidūre;\\
Bhūtā vā sambhavesī vā,\\
Sabbe sattā bhavantu sukhit'attā.

\chEnglish
\chapter[Loving-Kindness]{The Buddha's Words on Loving-Kindness}% {{{1

\begin{leader}
  [No꜓w let us chant the Buddha's words on loving-ki꜕ndness.]
\end{leader}

[This is what should be꜕ done]\\
By one who꜕ is ski꜓lled in go꜕odness\\
And who knows the pa꜕th of peace:\\
Let them be꜕ able and u꜓pright,\\
Stra꜕ightforward and gentle꜓ i꜕n speech,

Humble and not conce꜕ited,\\
Co꜕ntented and e꜓asily sa꜕tisfied,\\
Unburdened with du꜕ties and frugal i꜓n the꜕ir ways.\\
Peaceful and calm, a꜕nd wise and ski꜓lful,\\
No꜕t proud and dema꜓nding in na꜕ture.

Let them not do꜕ the sli꜓ghte꜕st thing\\
That the꜕ wise would late꜓r re꜕prove,\\
Wishing: In gladness a꜕nd in sa꜓fety,\\
May a꜕ll beings be꜓ a꜕t ease.

Whatever livi꜕ng beings there ma꜕y be,\\
Whether the꜕y are we꜓ak o꜕r strong, omi꜕tting none,\\
The great or the mi꜕ghty, medium, sho꜓rt, o꜕r small,

The seen and the u꜕nseen,\\
Those living near and fa꜓r a꜕way,\\
Those born and to꜕ be꜓ born,\\
May a꜕ll beings be꜓ a꜕t ease.

\clearpage

Na paro paraṃ nikubbetha,\\% {{{1
Nātimaññetha katthaci naṃ kiñci;\\
Byārosanā paṭighasaññā,\\
Nāññam-aññassa dukkham-iccheyya.

Mātā yathā niyaṃ puttaṃ,\\
Āyusā eka-puttam-anurakkhe;\\
Evam pi sabba-bhūtesu,\\
Mānasam-bhāvaye aparimāṇaṃ.

Mettañ-ca sabba-lokasmiṃ,\\
Mānasam-bhāvaye aparimāṇaṃ;\\
Uddhaṃ adho ca tiriyañ-ca,\\
Asambādhaṃ averaṃ asapattaṃ.

Tiṭṭhañ-caraṃ nisinno vā,\\
Sayāno vā yāvat'assa vigata-middho;\\
Etaṃ satiṃ adhiṭṭheyya,\\
Brahmam-etaṃ vihāraṃ idham-āhu.

Diṭṭhiñ-ca anupagamma,\\
Sīlavā dassanena sampanno;\\
Kāmesu vineyya gedhaṃ,\\
Na hi jātu gabbha-seyyaṃ punar-etī-ti.

\clearpage

Let none dece꜖ive a꜕no꜕ther\\% {{{1
Or de꜕spise any꜕ being in a꜓ny꜕ state.\\
Let none through anger or i꜕ll-will\\
Wish ha꜓rm upon ano꜕ther.

Even as a꜕ mother protects with he꜕r life\\
Her child, her o꜕nly꜓ child,\\
So with a bo꜓undless heart\\
Should o꜕ne cherish all li꜓vi꜕ng beings,\\
Radiating ki꜓ndness over the꜕ enti꜓re꜕ world:

Spreading upwards to the ski꜓es\\
And do꜕wnwa꜕rds to꜕ the꜓ depths,\\
Outwards and unbo꜕unded,\\
Fre꜕ed from ha꜓tre꜓d and i꜕ll-will.

Whether standing or wa꜕lking, seated, \\
Or ly꜓i꜕ng down --- free from dro꜕wsiness ---\\
One should su꜕stain this re꜕colle꜓ction.\\
This is said to꜕ be the꜕ subli꜓me abi꜕ding.

By not holding to fi꜕xed views,\\
The꜕ pure-he꜓arte꜕d one, having clarity of vi꜕sion,\\
Being freed fro꜕m all se꜓nse-desires,\\
Is not bo꜓rn a꜓gain into꜕ this world.

\chPali
\chapter[Universal Well-Being]{Reflection on Universal Well-Being}% {{{1

\begin{leader}
[Ha꜓nda mayam mettāpharaṇaṃ ka꜕romase]
\end{leader}

[Aha꜓ṃ sukhito ho꜓mi,]\\
niddukkho ho꜓mi,\\
a꜕vero ho꜓mi,\\
a꜕byāpajjho ho꜓mi,\\
a꜕nīgho ho꜓mi,\\
sukhī꜓ attānaṃ pa꜕riha꜓rāmi.

Sa꜕bbe sa꜕ttā sukhitā ho꜓ntu,\\
sa꜕bbe sa꜕ttā averā ho꜓ntu,\\
sa꜕bbe sa꜕ttā abyāpajjhā ho꜓ntu,\\
sa꜕bbe sa꜕ttā anīghā ho꜓ntu,\\
sa꜕bbe sa꜕ttā sukhī꜓\\
a꜕ttānaṃ pa꜕riha꜓rantu.

Sa꜕bbe sa꜕ttā sabbadukkhā pamucca꜓ntu.\\
Sa꜕bbe sa꜕ttā laddha-sa꜓mpa꜕tti꜓to mā vigaccha꜓ntu.

Sa꜕bbe sa꜕ttā kammassa꜕kā kamma꜓dāyādā kamma꜓yonī\\
\vin kamma꜓bandhū kammapa꜕ṭisa꜓ra꜕ṇā,

yaṃ kammaṃ ka꜕rissa꜓nti,\\
kalyāṇaṃ vā pāpa꜕kaṃ vā,\\
tassa꜕ dāyādā bha꜕vissa꜓nti.

\chEnglish
\chapter[Universal Well-Being]{Reflection on Universal Well-Being}% {{{1

\begin{leader}
[Now let us chant the reflections on universal we꜕ll-being.]
\end{leader}

[May I abi꜕de in we꜓ll-being,]\\
In fre꜕edom fro꜓m a꜕ffliction,\\
In fre꜕edom fro꜓m ho꜕sti꜓lity,\\
In fre꜕edom fro꜓m i꜕ll-will,\\
In fre꜕edom fro꜓m a꜕nxi꜓ety,\\
And may I \prul{mainta꜖in} we꜖ll-be꜓ing in m꜕yself.

May everyone abi꜕de in we꜓ll-being,\\
In fre꜕edom fro꜓m ho꜕sti꜓lity,\\
In fre꜕edom fro꜓m i꜕ll-will,\\
In fre꜕edom fro꜓m a꜕nxi꜓ety, and may they\\
\prul{Mainta꜖in} we꜖ll-be꜓ing in the꜕mselves.

May \prul{all} beings be rele꜖ased fro꜕m a꜕ll su꜓ffering.\\
And may they not be pa꜓rted from the꜕\\
\vin \prul{good fo꜓rtune} they have a꜕ttained.

When they act upon inte꜕ntion,\\
\prul{All} beings a꜕re the o꜓wners of their a꜕ction and inhe꜓rit its re꜕sults.\\
Their futu꜓re is born from such a꜕ction, compa꜓nion to such a꜕ction,\\
And its \prul{results} will be꜖ the꜓ir home.

\prul{All} a꜓ctions with inte꜕ntion,\\
Be the꜓y \prul{skilful} or ha꜓rmful ---\\
Of su꜕ch \prul{acts} they will be꜕ the꜓ heirs.

\chPali
\chapter[Divine Abidings]{Suffusion With the Divine Abidings}% {{{1

\begin{leader}
  [Ha꜓nda mayaṃ caturappamaññā obhāsanaṃ karomase]
\end{leader}

[Mettā-sa꜕ha꜕ga꜕tena] cetasā ekaṃ disaṃ pha꜕ri꜕tv꜕ā viha꜕ra꜕ti\\
Ta꜕thā dutiyaṃ ta꜕thā tatiyaṃ ta꜕thā ca꜕tutthaṃ\\
Iti uddhamadho tiriyaṃ sabba꜕dhi꜕ sabbatta꜕tāya\\
Sabbāvantaṃ lokaṃ mettā-sa꜕ha꜕ga꜕tena cetasā\\
Vipulena mahagga꜕tena appa꜕māṇena a꜕verena a꜕byāpajjhena\\
\vin pha꜕ri꜕tv꜕ā viha꜕ra꜕ti

Karuṇā-sa꜕ha꜕ga꜕tena cetasā ekaṃ disaṃ pha꜕ri꜕tv꜕ā viha꜕ra꜕ti\\
Ta꜕thā dutiyaṃ ta꜕thā tatiyaṃ ta꜕thā ca꜕tutthaṃ\\
Iti uddhamadho tiriyaṃ sabba꜕dhi꜕ sabbatta꜕tāya\\
Sabbāvantaṃ lokaṃ ka꜕ru꜕ṇā-sa꜕ha꜕ga꜕tena cetasā\\
Vipulena mahagga꜕tena appa꜕māṇena a꜕verena a꜕byāpajjhena\\
\vin pha꜕ri꜕tv꜕ā viha꜕ra꜕ti

Muditā-sa꜕ha꜕ga꜕tena cetasā ekaṃ disaṃ pha꜕ri꜕tv꜕ā viha꜕ra꜕ti\\
Ta꜕thā dutiyaṃ ta꜕thā tatiyaṃ ta꜕thā ca꜕tutthaṃ\\
Iti uddhamadho tiriyaṃ sabba꜕dhi꜕ sabbatta꜕tāya\\
Sabbāvantaṃ lokaṃ mu꜕di꜕tā-sa꜕ha꜕ga꜕tena cetasā\\
Vipulena mahagga꜕tena appa꜕māṇena a꜕verena a꜕byāpajjhena\\
\vin pha꜕ri꜕tv꜕ā viha꜕ra꜕ti

\instr{Continue}

\chEnglish
\chapter[Divine Abidings]{Suffusion With the Divine Abidings}% {{{1

\begin{leader}
  [No꜖w let us make the Four Boundless Qualities shi꜖ne forth.]
\end{leader}

[I wi꜓ll abide] pervading on꜕e quarter with a heart imbued\\
\vin with loving-ki꜓ndness;\\
Likewi꜓se the second, likewi꜓se the third, likewise th꜕e fourth;\\
So above and be꜕low, around and ev꜓ery꜕where; and to a꜓ll a꜓s to m꜕yself.\\
I wi꜓ll abide pervading the all-encompa꜓ssing world with a heart \\
\vin imbued with loving-ki꜕ndness; abu꜓ndant, exa꜓lted,\\
\vin imme꜕asurable, without ho꜕stility, and without i꜕ll-will.

I wi꜓ll abide pervading on꜕e quarter with a heart imbued\\
\vin with compa꜓ssion;\\
Likewi꜓se the second, likewi꜓se the third, likewise th꜕e fourth;\\
So above and be꜕low, around and ev꜓ery꜕where; and to a꜓ll a꜓s to m꜕yself.\\
I wi꜓ll abide pervading the all-encompa꜓ssing world with a heart \\
\vin imbued with compa꜕ssion; abu꜓ndant, exa꜓lted,\\
\vin imme꜕asurable, without ho꜕stility, and without i꜕ll-will.

I wi꜓ll abide pervading on꜕e quarter with a heart imbued\\
\vin with gla꜓dness;\\
Likewi꜓se the second, likewi꜓se the third, likewise th꜕e fourth;\\
So above and be꜕low, around and ev꜓ery꜕where; and to a꜓ll a꜓s to m꜕yself.\\
I wi꜓ll abide pervading the all-encompa꜓ssing world with a heart \\
\vin imbued with gla꜕dness; abu꜓ndant, exa꜓lted,\\
\vin imme꜕asurable, without ho꜕stility, and without i꜕ll-will.

\instr{Continue}

\clearpage

Upekkhā-saha꜕ga꜕te꜕na cetasā ekaṃ disaṃ pha꜕ri꜕tv꜕ā viha꜕ra꜕ti\\% {{{1
Ta꜕thā dutiyaṃ ta꜕thā tatiyaṃ ta꜕thā ca꜕tutthaṃ\\
Iti uddhamadho tiriyaṃ sabba꜕dhi꜕ sabbatta꜕tāya\\
Sabbāvantaṃ lokaṃ u꜕pe꜕kkhā-sa꜕ha꜕ga꜕tena cetasā\\
Vipulena mahagga꜕tena appa꜕māṇena a꜕verena a꜕byāpajjhena\\
\vin pha꜕ri꜕tv꜕ā viha꜕ra꜕tī'ti

\clearpage

I wi꜓ll abide pervading on꜕e quarter with a heart imbued\\% {{{1
\vin with equanimi꜓ty;\\
Likewi꜓se the second, likewi꜓se the third, likewise th꜕e fourth;\\
So above and be꜕low, around and ev꜓ery꜕where; and to a꜓ll a꜓s to m꜕yself.\\
I wi꜓ll abide pervading the all-encompa꜓ssing world with a heart \\
\vin imbued with equani꜕mity; abu꜓ndant, exa꜓lted,\\
\vin imme꜕asurable, without ho꜕stility, and withou꜕t ill-will.

\chPali
\chapter{The Highest Blessings}% {{{1

% TODO: Handa mayam
\begin{leader}
  [Now let us chant the verses on the Highest Bl꜕essings]
\end{leader}

[Evam-me sutaṃ: Ekaṃ samayaṃ bhagavā,]\\
Sāvatthiyaṃ viharati, jeta-vane anāthapiṇḍikassa ārāme.

Atha kho aññatarā devatā abhikkantāya rattiyā abhikkanta-vaṇṇā\\
kevala-kappaṃ jetavanaṃ obhāsetvā, yena bhagavā ten’upasaṅkami.\\
Upasaṅkamitvā bhagavantaṃ abhivādetvā ekam-antaṃ aṭṭhāsi.\\
Ekam-antaṃ ṭhitā kho sā devatā bhagavantaṃ gāthāya ajjhabhāsi:

Bahū devā manussā ca,\\
Maṅgalāni acintayuṃ;\\
Ākaṅkhamānā sotthānaṃ,\\
Brūhi maṅgalam-uttamaṃ.

[Asevanā ca bālānaṃ,]\\
Paṇḍitānañ-ca sevanā;\\
Pūjā ca pūjanīyānaṃ,\\
Etam maṅgalam-uttamaṃ.

Paṭirūpa-desa-vāso ca,\\
Pubbe ca kata-puññatā;\\
Atta-sammā-paṇidhi ca,\\
Etam maṅgalam-uttamaṃ.

\chEnglish
\chapter{The Highest Blessings}% {{{1

\begin{leader}
  [Now let us chant the verses on the Highest Bl꜕essings]
\end{leader}

[Thus have I heard that the Ble꜕ssed One]\\
Was staying at Sā꜓va꜕tthī,\\
Residing at the Jeta's Grove\\
In Anāthapi꜓ṇḍika꜕'s Park.

Then in the dark of the night, a ra꜓dia꜕nt de꜕va\\
Illuminated all Je꜓ta꜕'s Grove.\\
She bowed down low before the Ble꜕ssed One\\
Then standing to one si꜓de she꜕ said:

`Devas are concerned for ha꜓ppiness\\
And ever lo꜓ng fo꜕r peace.\\
The same is true for hu꜓mankind.\\
What then are the hi꜓ghest ble꜕ssings?'

`A꜕vo꜕iding those of foo꜕lish ways,\\
A꜕sso꜕ciating wi꜓th the꜕ wise,\\
And ho꜕nouring those wo꜓rthy of ho꜕nour.\\
These are the hi꜓ghest ble꜕ssings.

`Li꜕ving in places of suitable kinds,\\
With the frui꜕ts of past goo꜕d deeds\\
A꜕nd gui꜕ded by the ri꜓ghtfu꜕l way.\\
These are the hi꜓ghest ble꜕ssings.

\clearpage

Bāhu-saccañ-ca sippañ-ca,\\% {{{1
Vinayo ca susikkhito;\\
Subhāsitā ca yā vācā,\\
Etam maṅgalam-uttamaṃ.

Mātā-pitu-upaṭṭhānaṃ,\\
Putta-dārassa saṅgaho;\\
Anākulā ca kammantā,\\
Etam maṅgalam-uttamaṃ.

Dānañ-ca dhamma-cariyā ca,\\
Ñātakānañ-ca saṅgaho;\\
Anavajjāni kammāni,\\
Etam maṅgalam-uttamaṃ.

Āratī viratī pāpā,\\
Majja-pānā ca saññamo;\\
Appamādo ca dhammesu,\\
Etam maṅgalam-uttamaṃ.

Gāravo ca nivāto ca,\\
Santuṭṭhī ca katañ-ñutā;\\
Kālena dhammassavanaṃ,\\
Etam maṅgalam-uttamaṃ.

Khantī ca sovacassatā,\\
Samaṇānañ-ca dassanaṃ;\\
Kālena dhamma-sākacchā,\\
Etam maṅgalam-uttamaṃ.

\clearpage

`Acco꜕mplished in le꜕arni꜕ng a꜕nd cra꜕ftsman's skills,\\% {{{1
With di꜕scipline, highl꜕y trained,\\
And speech that is true and ple꜓asant to꜕ hear.\\
These are the hi꜓ghest ble꜕ssings.

`Provi꜕ding for mother and father's support\\
And che꜓rishing family,\\
And ways of work that ha꜓rm no꜕ being,\\
These are the hi꜓ghest ble꜕ssings.

`Generosity and a r꜕ighteous life,\\
Offering help to re꜓lati꜕ves and kin,\\
And acting in ways that le꜓ave no꜕ blame.\\
These are the hi꜓ghest ble꜕ssings.

`Steadfast in re꜕straint, and shunning e꜓vil ways,\\
Avo꜕iding into꜓xicants that du꜕ll the mind,\\
And heedfu꜕lness in all things tha꜓t arise.\\
These are the hi꜓ghest ble꜕ssings.

`Respe꜕ctfulness and being of humble ways,\\
Contentment and gra꜓titude,\\
And hearing the Dha꜕mma fre꜓quentl꜕y taught.\\
These are the hi꜓ghest ble꜕ssings.

`Patience and will꜕ingness to accept one's faults,\\
Seeing venerated se꜓ekers of the꜕ truth,\\
And sharing o꜕ften the wo꜓rds of Dha꜕mma.\\
These are the hi꜓ghest ble꜕ssings.

\clearpage

Tapo ca brahma-cariyañ-ca,\\% {{{1
Ariya-saccāna-dassanaṃ;\\
Nibbāna-sacchikiriyā ca,\\
Etam maṅgalam-uttamaṃ.

Phuṭṭhassa loka-dhammehi,\\
Cittaṃ yassa na kampati;\\
Asokaṃ virajaṃ khemaṃ,\\
Etam maṅgalam-uttamaṃ.

Etādisāni katvāna,\\
Sabbattham-aparājitā;\\
Sabbattha sotthiṃ gacchanti,\\
Tan-tesaṃ maṅgalam-uttaman-ti.

% TODO: sutta ref, (Maṅgala Sutta)
% [Sn. vv. 258---269; Khp.V]

\clearpage

`Ardent, co꜕mmitted to the Ho꜕ly Life,\\% {{{1
Seeing for onese꜓lf the Noble꜕ Truths\\
And the realization of Nibbā꜕na.\\
These are the hi꜓ghest ble꜕ssings.

`Although in co꜕nta꜕ct wi꜕th th꜕e world,\\
Unsha꜕ken the mi꜓nd r꜕emains\\
Beyond all so꜕rrow, spo꜓tless, se꜕cure.\\
These are the hi꜓ghest ble꜕ssings.

`The꜕y who live by fo꜕llowing this path\\
Know vi꜕ctory whe꜓rever the꜕y go,\\
And every place for them i꜕s safe.\\
These are the hi꜓ghest ble꜕ssings.'

% TODO: sutta ref, (Maṅgala Sutta)
% [Sn. vv. 258---269; Khp.V]

\chapter[The Unconditioned]{Reflection on the Unconditioned}% {{{1

\begin{leader}
  [Ha꜓nda mayaṃ nibbāna-sutta-pāṭhaṃ bha꜕ṇāmase]
\end{leader}

Atthi bhi꜓kkha꜕ve a꜕jātaṃ a꜓bhūtaṃ a꜕kataṃ a꜕sa꜓ṅkh꜕ataṃ

\begin{english}
  There is an U꜕nborn, Un꜕or꜓iginated, Un꜕create꜓d an꜕d Unformed.
\end{english}

N꜕o cetaṃ bhi꜓kkha꜕ve a꜕bhavissa\\
A꜕jātaṃ a꜓bhūtaṃ a꜕kataṃ a꜕sa꜓nkh꜕ataṃ

\begin{english}
  If there was not this U꜕nborn, this Un꜕or꜓iginated, this Un꜕creat꜓ed, thi꜕s~Unformed,
\end{english}

Na꜕ yidaṃ jātassa꜕ bhūtassa ka꜕tassa sa꜓ṅkh꜕atassa nissaraṇaṃ paññāye꜓tha

\begin{english}
  Fr꜓eedom from the world of th꜕e born, th꜕e ori꜓ginated, th꜕e create꜓d, th꜕e formed would no꜕t be po꜓ssible.
\end{english}

Ya꜕smā ca kho bhi꜓kkh꜕ave atthi a꜕jātaṃ a꜓bhūtaṃ a꜕kataṃ a꜕sa꜓ṅkha꜕taṃ

\begin{english}
  But since there is an U꜕nborn, U꜕nori꜓ginated, U꜕ncreate꜓d an꜕d Unformed,
\end{english}

Ta꜕smā jātass꜕a bhūtassa ka꜕tassa sa꜓ṅkha꜕tassa nissaraṇaṃ paññāyati

\begin{english}
  Therefore is fre꜓edom po꜕ssible from the world of th꜕e born, th꜕e ori꜓ginated, th꜕e create꜓d an꜕d the formed.
\end{english}

% TODO: sutta ref, (The Nibbāna Sutta)

\chapter{Just as Rivers}% {{{1

% TODO: leader?

Yathā vāri-vahā꜓ pūrā pa꜕ripūrenti sāgaraṃ

\begin{english}
  Just as ri꜕ve꜕rs fu꜕ll o꜕f wa꜕ter Entirely fill u꜓p th꜕e sea
\end{english}

Evam-eva i꜓to dinnaṃ pe꜕tānaṃ u꜕pakappa꜕ti

\begin{english}
  So will what's he꜕re be꜕en gi꜕ven Bring blessi꜓ngs to dep꜕art꜕ed sp꜕irits.
\end{english}

Icchitaṃ pa꜕tthitaṃ tu꜓mhaṃ

\begin{english}
  May all your ho꜕pes a꜕nd a꜕ll yo꜕ur lo꜕ngings
\end{english}

Khippam-eva sami꜓jjhatu

\begin{english}
  Come true in n꜓o lo꜕ng time.
\end{english}

Sa꜕bbe pūrentu sa꜓ṅkappā

\begin{english}
  May all your w꜕ish꜕es b꜕e fu꜕lfilled
\end{english}

Cando paṇṇa-raso꜓ yathā

\begin{english}
  Like on the fi꜕fte꜕enth da꜕y th꜕e moon
\end{english}

Maṇi jot꜕i-raso꜓ yathā

\begin{english}
  or like a bright and sh꜓ini꜕ng gem.
\end{english}

Sabb'ītiyo vivajja꜓ntu

\begin{english}
  May all misfo꜕rtu꜕nes b꜕e a꜕vo꜕ided,
\end{english}

S꜕abba-rogo vinassa꜕tu

\begin{english}
  May all il꜕lne꜕ss b꜕e di꜕spelled,
\end{english}

Mā te bha꜕vatv-antarā꜓yo

\begin{english}
  May you n꜓ever me꜕et wi꜕th da꜕ngers,
\end{english}

Sukhī꜓ dīgh'āyu꜕ko bha꜕va

\begin{english}
  May you be ha꜕pp꜕y an꜕d li꜕ve long.
\end{english}

A꜕bhivādana-sī꜓lissa꜕ niccaṃ vu꜕ḍḍhāpa꜕cāyi꜕no\\

\begin{english}
  For those who ar꜕e r꜕esp꜕ectful,\\
  Who always ho꜓nour the e꜕lders,
\end{english}

C꜕attāro dhammā vaḍḍha꜓nti\\
Āyu꜓ vaṇṇo su꜕khaṃ\\
Balaṃ

\begin{english}
  Four are th꜓e qu꜕al꜕iti꜕es wh꜕ich wi꜕ll in꜕crease:\\
  Life, be꜓auty, ha꜕ppi꜕ne꜕ss\\
  An꜕d strength.
\end{english}

Bhavatu sa꜕bba꜕-maṅg꜓alaṃ

\begin{english}
  May every ble꜕ssi꜕ng co꜕me t꜕o be
\end{english}

Rakkha꜓ntu sa꜕bba꜕-deva꜓tā

\begin{english}
  And all good spirits gu꜓ard yo꜕u well.
\end{english}

\enlargethispage{2\baselineskip}

Sa꜕bba-bu꜓ddhānu꜓bhāvena

\begin{english}
  Through the po꜕we꜕r o꜕f a꜕ll Bu꜕ddhas
\end{english}

Sa꜕dā so꜕tthi꜓ bhavantu꜕ te

\begin{english}
  May you a꜕lwa꜕ys b꜕e a꜕t ease.
\end{english}

Bhavatu sa꜕bba꜕-maṅg꜓alaṃ

\begin{english}
  May every ble꜕ssi꜕ng co꜕me t꜕o be
\end{english}

Rakkha꜓ntu sa꜕bba꜕-deva꜓tā

\begin{english}
  And all good spirits gu꜓ard yo꜕u well.
\end{english}

Sa꜕bba-dha꜓mmānu꜓bhāvena

\begin{english}
  Through the po꜕we꜕r o꜕f a꜕ll Dha꜕mmas
\end{english}

Sa꜕dā so꜕tthi꜓ bhavantu꜕ te

\begin{english}
  May you a꜕lwa꜕ys b꜕e a꜕t ease.
\end{english}

Bhavatu sa꜕bba꜕-maṅg꜓alaṃ

\begin{english}
  May every ble꜕ssi꜕ng co꜕me t꜕o be
\end{english}

Rakkha꜓ntu sa꜕bba꜕-deva꜓tā

\begin{english}
  And all good spirits gu꜓ard yo꜕u well.
\end{english}

Sa꜕bba-sa꜓ṅghānu꜓bhāvena

\begin{english}
  Through the po꜕we꜕r o꜕f a꜕ll Sa꜕nghas
\end{english}

Sa꜕dā so꜕tthi꜓ bhavantu꜕ te

\begin{english}
  May you a꜕lwa꜕ys b꜕e a꜕t ease.
\end{english}

\clearpage

\chapter[Four Requisites]{Reflection on the Four Requisites}% {{{1

\begin{leader}
  [Ha꜓nda mayaṃ taṅkhaṇika-paccave꜕kkhaṇa-pāṭhaṃ bhaṇāmase]
\end{leader}

[Paṭisaṅkhā] yoniso cīva꜕raṃ pa꜕ṭise꜓vāmi, yāvadeva sī꜓tassa꜕\\
pa꜕ṭighātāya, uṇhassa pa꜕ṭighātāya, ḍaṃsa-maka꜕sa꜕-vātāta꜕pa꜕-siriṃsapa-\\
-samphassānaṃ pa꜕ṭighātāya, yāvadeva hiri꜓kopina-pa꜕ṭicchāda꜕natthaṃ.

\begin{english}
  Wisely reflecting, I use the꜕ robe: only to ward o꜕ff cold, to ward o꜕ff heat, to ward off the touch o꜕f flies, mo꜕squitoes, wind, bu꜕rni꜕ng and cre꜓eping things, only for the sa꜓ke of mo꜕desty.
\end{english}

[Paṭisaṅkhā] yoniso piṇḍa꜕pātaṃ pa꜕ṭise꜓vāmi, neva da꜕vāya, na ma꜕dāya, na maṇḍa꜕nāya, na꜕ vi꜓bhūsa꜕nāya, yāvadeva i꜓massa꜕ kāyassa꜕ ṭhi꜕tiyā, yāpa꜕nāya, vihiṃsū꜕para꜓ti꜕yā, brahmaca꜕ri꜓yānugga꜕hāya, iti purāṇañca꜕ veda꜓naṃ pa꜕ṭiha꜓ṅkhāmi, navañca꜕ veda꜓naṃ na uppādessāmi, yātrā ca꜕ me bhavissati a꜕navajjatā ca꜕ phāsuvihāro cā'ti.

\begin{english}
  Wisely reflecting, I use a꜕lmsfood: not fo꜕r fun, not for ple꜕asure, not for fa꜕ttening, not for beautifica꜓tion, only for the꜕ maintenance and no꜓urishment of this bo꜕dy, for keeping it he꜕althy, for helping with the Ho꜓ly Life; thinki꜕ng thus, `I will allay hu꜓nger without overe꜕ating, so that I may co꜕ntinue to live bla꜓melessly and a꜕t ease.'
\end{english}

[Paṭisaṅkhā] yoniso senāsa꜕naṃ pa꜕ṭise꜓vāmi, yāvadeva sī꜓tassa꜕\\
pa꜕ṭighātāya, uṇhassa pa꜕ṭighātāya, ḍaṃsa-maka꜕sa꜕-vātāta꜕pa꜕-siriṃsapa-\\
-samphassānaṃ pa꜕ṭighātāya, yāvadeva utupa꜕rissaya vi꜕nodanaṃ pa꜕ṭisa꜓llānārāmatthaṃ.

\begin{english}
  Wisely reflecting, I use the lo꜕dging: only to ward o꜕ff cold, to ward o꜕ff heat, to ward off the touch o꜕f flies, mo꜕squitoes, wind, bu꜕rni꜕ng and cre꜓eping things, only to remove the꜕ danger from we꜕ather, and fo꜕r living in seclu꜓sion.
\end{english}

[Paṭisaṅkhā] yoniso gi꜕lāna-pacca꜕ya꜕-bhesajja-pa꜕rikkhāraṃ pa꜕ṭise꜓vāmi, yāvadeva uppa꜓nnānaṃ veyyābādhi꜕kānaṃ veda꜕nānaṃ pa꜕ṭighātāya, a꜕byāpajjha-pa꜕ramatāyā ti.

\begin{english}
  Wisely reflecting, I use su꜕pports for the sick and me꜕dicinal re꜓quisites: only to ward off pa꜕inful fe꜓elings that have ari꜕sen, for the꜕ maximum freedom from di꜕sease.
\end{english}

\chapter[Five Subjects]{Five Subjects for Frequent Recollection}% {{{1

% TODO: the male-female variations are confusing the reading

% NOTE: mid-line Pali caps for alternative forms

\begin{leader}
  [Ha꜓nda mayaṃ abhiṇha-paccave꜕kkhaṇa-pāṭhaṃ bhaṇāmase]
\end{leader}

[Jarā-dhammomhi꜕/Jarā-dhammāmhi꜕] jaraṃ a꜕na꜕tīto/a꜕na꜕tītā

\begin{english}
  I am of the nature to꜕ age, I have not go꜓ne beyond a꜕geing.
\end{english}

Byādhi꜓-dhammomhi꜕/Byādhi꜓-dhammāmhi꜕ byādhiṃ a꜕na꜕tīto/a꜕na꜕tītā

\begin{english}
  I am of the nature to si꜕cken, I have not go꜓ne beyond si꜕ckness.
\end{english}

Ma꜕raṇa-dhammomhi꜕/Ma꜕raṇa-dhammāmhi꜕ ma꜕raṇaṃ a꜕na꜕tīto/a꜕na꜕tītā

\begin{english}
  I am of the nature to꜕ die, I have not go꜓ne beyond d꜕ying.
\end{english}

Sa꜕bbehi me pi꜕yehi ma꜕nāpehi꜕ nānābhāvo vi꜕nābhāvo

\begin{english}
  All that i꜕s mine, be꜕loved and ple꜓asing,\\
  will become o꜕therwise, will become se꜓parated fro꜕m me.
\end{english}

Kammassa꜕komhi/Kammassa꜕kāmhi kamma꜓dāyādo/kamma꜓dāyādā kamma꜕yoni kamma꜕bandhu kammapa꜕ṭisa꜓ra꜕ṇo/kammapa꜕ṭisa꜓ra꜕ṇā. Yaṃ kammaṃ ka꜕rissāmi, kalyāṇaṃ vā pāpa꜕kaṃ vā, tassa꜕ dāyādo/dāyādā bha꜕vissāmi

\begin{english}
  I am the꜕ owner of my ka꜕mma, heir to my ka꜕mma, born of my ka꜕mma,\\
  related to my ka꜕mma, abide suppo꜓rted by my ka꜕mma.\\
  Whatever kamma I sha꜕ll do, for good or fo꜕r ill, of that I will be꜕ the꜓ heir.
\end{english}

Evaṃ amhehi꜕ a꜕bhiṇhaṃ pacca꜕vekkhi꜓tabbaṃ

\begin{english}
  Thus we sho꜕uld frequently re꜓co꜕llect.
\end{english}

\chapter[Ten Subjects]{Ten Subjects for Frequent Recollection by One Who Has Gone Forth}% {{{1

\enlargethispage{\baselineskip}

\begin{leader}
  [Ha꜓nda mayaṃ pabbajita-abhiṇha-\\
  -paccave꜕kkhaṇa-pāṭhaṃ bhaṇāmase]
\end{leader}

[Dasa i꜕me bhikkhave] dhammā pabba꜕jitena a꜕bhiṇhaṃ pacca꜕vekkhi꜓tabbā. ka꜕ta꜕me dasa?

\begin{english}
  Bhikkhus, there are te꜕n dhammas which should be re꜕flected upon again and a꜕gain by one who ha꜕s go꜓ne forth. What a꜕re these ten?
\end{english}

`Vevaṇṇi꜕yamhi ajjhūpa꜕ga꜕to' ti pabba꜕jitena a꜕bhiṇhaṃ pacca꜕vekkhi꜓tabbaṃ.

\begin{english}
  `I am no꜕ longer li꜓ving according to꜕ worldly aims and va꜕lues.'\\
  This should be re꜕flected upon again and a꜕gain\\
  by one who ha꜕s go꜓ne forth.
\end{english}

`Parapaṭi꜕baddhā me jīvi꜓kā' ti pabba꜕jitena a꜕bhiṇhaṃ pacca꜕vekkhi꜓tabbaṃ.

\begin{english}
  `My very꜕ life is susta꜓ined through the gifts of o꜕thers.'\\
  This should be re꜕flected upon again and a꜕gain\\
  by one who ha꜕s go꜓ne forth.
\end{english}

`Añño me ākappo ka꜕ra꜕ṇīyo' ti pabba꜕jitena a꜕bhiṇhaṃ pacca꜕vekkhi꜓tabbaṃ.

\begin{english}
  `I shou꜕ld strive to aba꜓ndon my former ha꜕bits.'\\
  This should be re꜕flected upon again and a꜕gain\\
  by one who ha꜕s go꜓ne forth.
\end{english}

\clearpage

`Kacci nu꜕ kho me attā sīla꜕to na u꜕pavadatī' ti pabba꜕jitena a꜕bhiṇhaṃ pacca꜕vekkhi꜓tabbaṃ.

\begin{english}
  `Does re꜕gret over my co꜓nduct arise in m꜕y mind?'\\
  This should be re꜕flected upon again and a꜕gain\\
  by one who ha꜕s go꜓ne forth.
\end{english}

`Kacci nu꜕ kho maṃ a꜕nuvicca viññū sabrahma꜓cārī sīla꜕to na u꜕pavadantī' ti pabba꜕jitena a꜕bhiṇhaṃ pacca꜕vekkhi꜓tabbaṃ.

\begin{english}
  `Could m꜕y spiritual compa꜓nions find fault with my co꜕nduct?'\\
  This should be re꜕flected upon again and a꜕gain\\
  by one who ha꜕s go꜓ne forth.
\end{english}

`Sa꜕bbehi me pi꜕yehi ma꜕nāpehi꜕ nānābhāvo vi꜕nābhāvo' ti pabba꜕jitena abhiṇhaṃ pacca꜕vekkhi꜓tabbaṃ.

\begin{english}
  `All that i꜕s mine, be꜕loved and ple꜓asing,\\
  will become o꜕therwise, will become se꜓parated from me.'\\
  This should be re꜕flected upon again and a꜕gain\\
  by one who ha꜕s go꜓ne forth.
\end{english}

`Kammassa꜕komhi kamma꜓dāyādo kamma꜕yoni kamma꜕bandhu kammapa꜕ṭisa꜓raṇo, yaṃ kammaṃ ka꜕rissāmi, kalyāṇaṃ vā pāpa꜕kaṃ vā, tassa꜕ dāyādo bha꜕vissāmī' ti pabba꜕jitena a꜕bhiṇhaṃ pacca꜕vekkhi꜓tabbaṃ.

\begin{english}
  `I am the꜕ owner of my ka꜕mma, heir to my ka꜕mma, born of my ka꜕mma,\\
  re꜕lated to my ka꜕mma, abide suppo꜓rted by my ka꜕mma;\\
  whatever kamma I sha꜕ll do, for good or fo꜕r ill, of that I will be꜕ the꜓ heir.'\\
  This should be re꜕flected upon again and a꜕gain\\
  by one who ha꜕s go꜓ne forth.
\end{english}

\clearpage

`Kathambhūtassa꜕ me rattindi꜕vā vīti꜕pa꜓tantī' ti pabba꜕jitena a꜕bhiṇhaṃ pacca꜕vekkhi꜓tabbaṃ.

\begin{english}
  `The꜕ days and nights are re꜕lentlessly pa꜓ssing;\\
  ho꜕w well am I spe꜓ndi꜓ng m꜕y time?'\\
  This should be re꜕flected upon again and a꜕gain\\
  by one who ha꜕s go꜓ne forth.
\end{english}

`Kacci nu꜕ kho'haṃ suññā꜓gāre abhira꜕māmī' ti pabba꜕jitena a꜕bhiṇhaṃ pacca꜕vekkhi꜓tabbaṃ.

\begin{english}
  `Do I delight in so꜓litude or not?'\\
  This should be re꜕flected upon again and a꜕gain\\
  by one who ha꜕s go꜓ne forth.
\end{english}

`Atthi nu꜕ kho me uttari-ma꜕nussa-dhammā alamariya꜕-ñāṇa-dassana-viseso adhiga꜕to, so'haṃ pacchi꜓me kāle sa꜕brahmacārīhi꜕ puṭṭho na maṅku bha꜕vissāmī' ti pabba꜕jitena a꜕bhiṇhaṃ pacca꜕vekkhi꜓tabbaṃ.

\begin{english}
  `Has m꜕y practice borne fruit with freedom or i꜓nsight\\
  so that at the e꜕nd of my life I need not feel a꜕shamed\\
  when questioned b꜕y my spi꜓ri꜓tual compa꜕nions?'\\
  This should be re꜕flected upon again and a꜕gain\\
  by one who ha꜕s go꜓ne forth.
\end{english}

Ime kho bhikkha꜓ve da꜕sa꜕ dhammā pabba꜕jitena a꜕bhiṇhaṃ pacca꜕vekkhitabbā' ti.

\begin{english}
  Bhikkhus, these are the te꜕n dhammas to be re꜕flected upon again and a꜕gain by one who ha꜕s go꜓ne forth.
\end{english}

\chapter[Thirty-Two Parts]{Reflection on the Thirty-Two Parts}% {{{1

\begin{leader}
  [Ha꜓nda mayaṃ dvattiṃsākāra-pāṭhaṃ bhaṇāmase]
\end{leader}

[Ayaṃ kho] me kāyo uddhaṃ pāda꜕ta꜕lā adho kesamatthakā\\
ta꜕ca꜕pa꜕ri꜕yanto pūro nānappa꜕kārassa꜕ a꜕su꜕ci꜕no

\begin{english}
  This, which is my body, from the soles of the feet up, and down from the crown of the head, is a sealed bag of skin filled with unattractive things.
\end{english}

Atthi imasmiṃ kāye

\begin{english}
  In this body there are:
\end{english}

{\centering
\setArrayStrech{1}

\begin{tabular}{ r l }
kesā            & \tr{hair of the head} \\
lomā            & \tr{hair of the body} \\
nakhā           & \tr{nails} \\
dantā           & \tr{teeth} \\
taco            & \tr{skin} \\
maṃsaṃ          & \tr{flesh} \\
nahārū          & \tr{sinews} \\
aṭṭhī           & \tr{bones} \\
aṭṭhimiñjaṃ     & \tr{bone marrow} \\
vakkaṃ          & \tr{kidneys} \\
hadayaṃ         & \tr{heart} \\
yakanaṃ         & \tr{liver} \\
kilomakaṃ       & \tr{membranes} \\
pihakaṃ         & \tr{spleen} \\
papphāsaṃ       & \tr{lungs} \\
\end{tabular}

\clearpage

\begin{tabular}{ r l }
antaṃ           & \tr{bowels} \\
antaguṇaṃ       & \tr{entrails} \\
udariyaṃ        & \tr{undigested food} \\
karīsaṃ         & \tr{excrement} \\
pittaṃ          & \tr{bile} \\
semhaṃ          & \tr{phlegm} \\
pubbo           & \tr{pus} \\
lohitaṃ         & \tr{blood} \\
sedo            & \tr{sweat} \\
medo            & \tr{fat} \\
assu            & \tr{tears} \\
vasā            & \tr{grease} \\
kheḷo           & \tr{spittle} \\
siṅghāṇikā      & \tr{mucus} \\
lasikā          & \tr{oil of the joints} \\
muttaṃ          & \tr{urine} \\
matthaluṅgan'ti & \tr{brain} \\
\end{tabular}

\restoreArrayStretch
}

Evam-ayaṃ me kāyo uddhaṃ pāda꜕ta꜕lā adho kesamatthakā\\
ta꜕ca꜕pa꜕ri꜕yanto pūro nānappa꜕kārassa꜕ a꜕su꜕ci꜕no

\begin{english}
  This, then, which is my body, from the soles of the feet up, and down from the crown of the head, is a sealed bag of skin filled with unattractive things.
\end{english}

% }}}1

% End of reflections-and-recollections-p1.tex
