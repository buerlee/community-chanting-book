% vim: foldmethod=marker foldlevel=0 foldtext=FoldText()

\chapter[Off-putting qualities of requisites]{Reflection on the off-putting qualities of requisites}% {{{1

\begin{leader}
  [Ha꜓nda mayaṃ dhātu-paṭikūla-paccavekkhaṇa-pāṭhaṃ bhaṇāmase]
\end{leader}

[Yathā꜓ pa꜕ccayaṃ] pava꜓tt꜕amānaṃ dhātu꜕-ma꜓tta꜕m-ev'etaṃ

\begin{english}
  Composed of only e꜓lements acco꜕rdin꜕g to꜕ ca꜕use꜕s an꜕d co꜕ndi꜕tions
\end{english}

Yad i꜓daṃ cī꜓varaṃ ta꜕d upa꜕bhuñja꜓ko c꜕a pu꜕gga꜕lo

\begin{english}
  Are these ro꜕bes an꜕d so꜕ is꜕ th꜕e pe꜕rso꜕n we꜕aring them;
\end{english}

\begin{twochants}
  Dhātu-ma꜓tta꜕ko & \tr{Merely e꜕lements,} \\
  Ni꜓ssa꜕tto & \tr{Not a be꜕ing,} \\
  Ni꜓jjīvo & \tr{Without a꜕ soul} \\
  Su꜓ñño & \tr{And e꜕mpty꜕ o꜕f self.} \\
\end{twochants}

S꜕abbāni pa꜕na imāni cī꜓varāni a꜕jigu꜓ccha꜕nīyāni

\begin{english}
  None of th꜓ese robes ar꜕e inna꜕tel꜕y re꜕pu꜕lsive
\end{english}

Imaṃ pūti꜓-kāyaṃ pa꜕tvā

\begin{english}
  But touching this u꜓nclean bo꜕dy
\end{english}

A꜕tiviya jigu꜓ccha꜕nīyāni jāyanti

\begin{english}
  They beco꜕me di꜕sgu꜕sti꜕ng in꜕deed.
\end{english}

\clearpage

Yathā꜓ pa꜕ccayaṃ pava꜓tt꜕amānaṃ dhātu꜕-ma꜓tta꜕m-ev'etaṃ

\begin{english}
  Composed of only e꜓lements acco꜕rdin꜕g to꜕ ca꜕use꜕s an꜕d co꜕ndi꜕tions
\end{english}

Yad i꜓daṃ piṇḍa꜓pāto ta꜕d upa꜕bhuñja꜓ko c꜕a pu꜕gga꜕lo

\begin{english}
  Is this a꜕lmsfo꜕od an꜕d s꜕o i꜕s th꜕e pe꜕rso꜕n ea꜕ting it;
\end{english}

\begin{twochants}
  Dhātu-ma꜓tta꜕ko & \tr{Merely e꜕lements,} \\
  Ni꜓ssa꜕tto & \tr{Not a be꜕ing,} \\
  Ni꜓jjīvo & \tr{Without a꜕ soul} \\
  Su꜓ñño & \tr{And e꜕mpty꜕ o꜕f self.} \\
\end{twochants}

S꜕abbo pa꜕nāyaṃ piṇḍa꜓-pāto a꜕jigu꜓ccha꜕nīyo

\begin{english}
  None of th꜓is almsfood is inna꜕tel꜕y re꜕pu꜕lsive
\end{english}

Imaṃ pūti꜓-kāyaṃ pa꜕tvā

\begin{english}
  But touching this u꜓nclean bo꜕dy
\end{english}

A꜕tiviya jigu꜓ccha꜕nīyo jāyati

\begin{english}
  It beco꜕mes di꜕sgu꜕sti꜕ng in꜕deed.
\end{english}

Yathā꜓ pa꜕ccayaṃ pava꜓tt꜕amānaṃ dhātu꜕-ma꜓tta꜕m-ev'etaṃ

\begin{english}
  Composed of only e꜓lements acco꜕rdin꜕g to꜕ ca꜕use꜕s an꜕d co꜕ndi꜕tions
\end{english}

Yad i꜓daṃ senā꜓sanaṃ ta꜕d upa꜕bhuñja꜓ko c꜕a pu꜕gga꜕lo

\begin{english}
  Is this dwe꜕lli꜕ng an꜕d s꜕o i꜕s th꜕e pe꜕rso꜕n u꜕sing it;
\end{english}

\begin{twochants}
  Dhātu-ma꜓tta꜕ko & \tr{Merely e꜕lements,} \\
  Ni꜓ssa꜕tto & \tr{Not a be꜕ing,} \\
  Ni꜓jjīvo & \tr{Without a꜕ soul} \\
  Su꜓ñño & \tr{And e꜕mpty꜕ o꜕f self.} \\
\end{twochants}

S꜕abbāni pa꜕na imāni senā꜓sanāni a꜕jigu꜓ccha꜕nīyāni

\begin{english}
  None of the꜓se dwellings are inna꜕tel꜕y re꜕pu꜕lsive
\end{english}

Imaṃ pūti꜓-kāyaṃ pa꜕tvā

\begin{english}
  But touching this u꜓nclean bo꜕dy
\end{english}

A꜕tiviya jigu꜓ccha꜕nīyāni jāyanti

\begin{english}
  They beco꜕me di꜕sgu꜕sti꜕ng in꜕deed.
\end{english}

Yathā꜓ pa꜕ccayaṃ pava꜓tt꜕amānaṃ dhātu꜕-ma꜓tta꜕m-ev'etaṃ

\begin{english}
  Composed of only e꜓lements acco꜕rdin꜕g to꜕ ca꜕use꜕s an꜕d co꜕ndi꜕tions
\end{english}

Yad i꜓daṃ gi꜕lāna-pacca꜕ya꜕-bhesajja-pa꜕rikkhāro ta꜕d upa꜕bhuñja꜓ko c꜕a pu꜕gga꜕lo

\begin{english}
  Is this m꜕e꜕di꜕ci꜕na꜕l requ꜕is꜕ite an꜕d s꜕o i꜕s th꜕e pe꜕rso꜕n tha꜕t ta꜕kes it;
\end{english}

\begin{twochants}
  Dhātu-ma꜓tta꜕ko & \tr{Merely e꜕lements,} \\
  Ni꜓ssa꜕tto & \tr{Not a be꜕ing,} \\
  Ni꜓jjīvo & \tr{Without a꜕ soul} \\
  Su꜓ñño & \tr{And e꜕mpty꜕ o꜕f self.} \\
\end{twochants}

S꜕abbo pa꜕nāyaṃ gi꜕lāna-pacca꜕ya꜕-bhesajja-pa꜕rikkhāro a꜕jigu꜓ccha꜕nīyo

\begin{english}
  None of th꜓is medicinal re꜕qui꜕si꜕te is inna꜕tel꜕y re꜕pu꜕lsive
\end{english}

Imaṃ pūti꜓-kāyaṃ pa꜕tvā

\begin{english}
  But touching this u꜓nclean bo꜕dy
\end{english}

A꜕tiviya jigu꜓ccha꜕nīyo jāyati

\begin{english}
  It beco꜕mes di꜕sgu꜕sti꜕ng in꜕deed.
\end{english}

\chapter{Reflection on impermanence}% {{{1

\begin{leader}
  [Handa mayaṃ aniccānussati-pāṭhaṃ bhaṇāmase]
\end{leader}

\begin{twochants}

[Sa꜕bbe sa꜓ṅkhā꜓rā a꜕ni꜓ccā] &
\tr{All conditioned things are impe꜕rmanent;} \\

Sa꜕bbe sa꜓ṅkhā꜓rā du꜕kkhā &
\tr{All conditioned things are du꜕kkha;} \\

Sa꜕bbe dhammā a꜕na꜓ttā &
\tr{Everything is vo꜕id o꜕f self.} \\

A꜕ddhuvaṃ jīvi꜓taṃ &
\tr{Life is no꜕t fo꜕r sure;} \\

Dhuvaṃ ma꜓ra꜕ṇaṃ &
\tr{Dea꜕th i꜕s fo꜕r sure;} \\

A꜕vassaṃ mayā mari꜓ta꜕bbaṃ &
\tr{It is i꜓nevitable tha꜕t I꜕'ll die;} \\

Ma꜕raṇa-pa꜕riyosā꜓naṃ me jīvi꜓taṃ &
\tr{Death is th꜓e culmina꜕ti꜕on o꜕f my꜕ life;} \\

Jīvitaṃ me ani꜓ya꜕taṃ &
\tr{My life is unce꜕rtain;} \\

Maraṇaṃ me ni꜓ya꜕taṃ &
\tr{My dea꜕th is꜕ ce꜕rtain.} \\

Vata &
\tr{I꜕ndeed,} \\

A꜕yaṃ kāyo &
\tr{This bo꜕dy} \\

A꜕ciraṃ &
\tr{Wi꜕ll soon} \\

A꜕peta꜕-viññāṇo &
\tr{Be void of co꜓nsci꜕ousness} \\

Chu꜕ḍḍho &
\tr{And ca꜕st a꜕way.} \\

A꜕dhise꜓ssa꜕ti &
\tr{I꜕t wi꜕ll lie} \\

Pa꜕ṭha꜕viṃ &
\tr{On꜕ th꜕e ground} \\

Ka꜕liṅga꜓raṃ i꜕va &
\tr{Just like a ro꜓tten꜕ log,} \\

Ni꜕ratthaṃ &
\tr{Comple꜕tel꜕y vo꜕id o꜕f use.} \\

\end{twochants}

\clearpage

\begin{twochants}

Aniccā vata sa꜓ṅkhā꜓rā &
\tr{Truly co꜓nditioned thin꜕gs ca꜕nno꜕t last,} \\

U꜕ppāda-vaya-dha꜓mmi꜕no &
\tr{Their nature is to ri꜕se an꜕d fall,} \\

U꜕ppajjitvā nirujjh꜓anti &
\tr{Having a꜓risen thi꜕ngs mu꜕st cease,} \\

Tesa꜓ṃ vūpa꜕sa꜕mo sukho &
\tr{Their st꜕illin꜕g i꜕s tr꜕ue ha꜕ppiness.} \\

\end{twochants}

\artopttrue
\chapter{Verses on the Burden}% {{{1

\begin{leader}
  [Ha꜓nda mayaṃ bhāra-su꜕tta-gāthā꜓yo bha꜕ṇāmase]
\end{leader}

\begin{twochants}
Bhārā ha꜕ve pañcakkha꜓ndhā & bhāra-hāro ca pu꜓gga꜕lo \\
Bhā꜕r'ādānaṃ du꜕kkhaṃ loke꜓ & bhāra-nikkhe꜓pa꜕naṃ su꜕khaṃ \\
\end{twochants}

\begin{english}
  The five aggregates inde꜕ed ar꜕e bu꜕rdens,\\
  The beast of burden tho꜓ugh i꜕s man.\\
  In this world to ta꜕ke u꜕p bu꜕rde꜕ns i꜕s du꜕kkha.\\
  Putting the꜓m down brings ha꜓ppi꜕ness.
\end{english}

\begin{twochants}
Nikkhipi꜕tvā ga꜕ruṃ bhā꜓raṃ & aññaṃ bhāraṃ anā꜓di꜕ya \\
Sa꜕mūlaṃ taṇhaṃ a꜕bbuyha & nicchāto pa꜕ri꜕nibbu꜕to \\
\end{twochants}

\begin{english}
  A heavy burden ca꜕st a꜕way,\\
  Not taking on ano꜓th꜕er load,\\
  With cravi꜓ng pulled out fro꜕m th꜕e root,\\
  Desire꜓s stilled on꜕e i꜕s re꜕leased.
\end{english}

\artoptfalse
\chapter{True and False Refuges}% {{{1

\begin{leader}
  [Ha꜓nda mayaṃ khemākhema-sa꜕raṇa-gamana-\\
  -pa꜕ridīpikā-gāthā꜓yo bha꜕ṇāmase]
\end{leader}

\begin{twochants}
  Bahuṃ ve sa꜕ra꜓ṇaṃ yanti꜕ & pa꜕bba꜕tāni va꜕nāni꜓ ca \\
  Ārāma-rukkha꜕-cetyāni & manussā꜓ bha꜕ya꜕-tajji꜕tā \\
\end{twochants}

\begin{english}
  To many re꜕fu꜕ge꜕s th꜕ey go --\\
  To mountain slopes and fo꜓re꜕st glades,\\
  To pa꜕rkla꜕nd shri꜕nes an꜕d sa꜕cre꜕d sites --\\
  People overco꜓me by꜕ fear.
\end{english}

\begin{twochants}
  N'etaṃ kho sa꜕ra꜓ṇaṃ khemaṃ & n'etaṃ sa꜕raṇam-u꜓tt꜕amaṃ \\
  N'etaṃ sa꜕raṇam-āgamma & sa꜕bba-dukkhā꜓ pa꜕mucca꜕ti \\
\end{twochants}

\begin{english}
  Such a refuge is no꜕t se꜕cure,\\
  Such a refuge is no꜓t su꜕preme,\\
  Such a꜓ refuge do꜕es no꜕t bring\\
  Complete release from suf꜕fe꜕ring.
\end{english}

\begin{twochants}
  Yo ca꜕ Buddhañ-ca꜕ Dhammañ-ca꜕ & sa꜓ṅghañ-ca꜕ sa꜓ra꜕ṇaṃ ga꜕to \\
  Ca꜕ttāri a꜕riya-saccāni & sa꜕mmappaññāya꜓ pa꜕ss꜕ati \\
\end{twochants}

\begin{english}
  Whoe꜕ve꜕r go꜕es t꜕o r꜕efuge\\
  In the Tr꜓iple꜕ Gem\\
  Sees with ri꜓ght disce꜕rnment\\
  The Fo꜕ur No꜕bl꜕e Truths:
\end{english}

\begin{twochants}
  Dukkhaṃ dukkha-sa꜕muppādaṃ & dukkhassa ca꜕ a꜕ti꜕kka꜕maṃ \\
  A꜕riyañ-c'a꜕ṭṭh'a꜓ṅgi꜕kaṃ maggaṃ & dukkhūpasa꜕ma꜕-gāmi꜓naṃ \\
\end{twochants}

\begin{english}
  Suffering an꜕d it꜕'s o꜕rigin\\
  And that which li꜓es be꜕yond --\\
  The Nob꜕le E꜕ightfo꜕ld Path\\
  That leads th꜓e way to su꜕ff'r꜕ing's end.
\end{english}

\begin{twochants}
  Etaṃ kho sa꜕ra꜓ṇaṃ khemaṃ & etaṃ sa꜕raṇam-u꜓tta꜕maṃ \\
  Etaṃ sa꜕raṇam-āgamma & sa꜕bba-dukkhā꜓ pa꜕mucca꜕ti \\
\end{twochants}

\begin{english}
  Such a꜓ refuge i꜕s se꜕cure,\\
  Such a refuge i꜓s su꜕preme,\\
  Such a refuge tr꜕uly꜕ brings\\
  Complete r꜓elease from all su꜕ffe꜕ring.
\end{english}

\chapter{Verses on the riches of a Noble One}% {{{1

\begin{leader}
  [Ha꜓nda mayaṃ a꜕riya-dhana-gāthā꜓yo bha꜕ṇāmase]
\end{leader}

\begin{twochants}
  Yassa꜕ sa꜕ddhā Tathā꜓ga꜕te & a꜕ca꜕lā su꜕pa꜕tiṭṭhi꜓tā \\
  Sī꜓lañ-ca꜕ yassa꜕ kalyāṇaṃ & a꜕riya-kantaṃ pasa꜓ṃsi꜕taṃ \\
\end{twochants}

\begin{english}
  One whose faith in the Tathā꜕gata\\
  Is unshaken and esta꜓bli꜕shed well,\\
  Whose virtue is be꜕autiful,\\
  The Noble Ones enjo꜓y an꜕d praise;
\end{english}

\begin{twochants}
  Sa꜓ṅghe pa꜕sā꜕do yass'atthi & uju-bhūtañ-ca da꜓ss꜕anaṃ \\
  A꜕daliddo-t꜕i taṃ āhu꜕ & a꜕moghaṃ ta꜕ssa꜕ jīvi꜓taṃ \\
\end{twochants}

\begin{english}
  Whose trust is in꜕ th꜕e Sa꜕ngha,\\
  Who sees things rightly a꜓s th꜕ey are,\\
  It is sa꜕id th꜕at no꜕t i꜕n vain\\
  And undeluded i꜓s th꜕eir life.
\end{english}

\begin{twochants}
  Tasmā sa꜕ddhañ-ca꜕ sī꜓lañ-ca꜕ & pasādaṃ dhamma-da꜓ssa꜕naṃ \\
  A꜕nuyuñjetha medhāvī & sa꜕raṃ buddhāna sā꜓sa꜕naṃ \\
\end{twochants}

\begin{english}
  To virtu꜓e and to꜕ faith,\\
  To trust to se꜓ein꜕g truth,\\
  To these the wise devo꜕te th꜕emselves,\\
  The Buddha꜓'s teaching in꜕ th꜕eir mind.
\end{english}

\chapter{Verses on the Three Characteristics}% {{{1

\begin{leader}
  [Ha꜓nda mayaṃ ti-lakkhaṇ'ādi-gāthā꜓yo bha꜕ṇāmase]
\end{leader}

\begin{twochants}
  Sa꜕bbe sa꜓ṅkhā꜓rā a꜕ni꜓ccā-t꜕i & yadā paññāya꜓ pa꜕ssa꜕ti \\
  Atha nibbinda꜕ti dukkhe & esa꜕ maggo vi꜓su꜕ddh꜓iyā \\
\end{twochants}

\begin{english}
  `Impermanent are all condi꜕tio꜕ned things' --\\
  When with wisdom th꜓is i꜕s seen\\
  One feels we꜕ary꜕ o꜕f a꜕ll du꜕kkha;\\
  This is the path to pu꜓r꜕ity.
\end{english}

\begin{twochants}
  Sa꜕bbe sa꜓ṅkhā꜓rā du꜕kkhā-t꜕i & yadā paññāya꜓ pa꜕ssa꜕ti \\
  Atha nibbinda꜕ti dukkhe & esa꜕ maggo vi꜓su꜕ddh꜓iyā \\
\end{twochants}

\begin{english}
  `Dukkha are all condi꜕tio꜕ned things' --\\
  When with wisdom th꜓is i꜕s seen\\
  One feels we꜕ary꜕ o꜕f a꜕ll du꜕kkha;\\
  This is the path to pu꜓r꜕ity.
\end{english}

\begin{twochants}
  Sa꜕bbe dhammā ana꜓ttā-ti꜕ & yadā paññāya꜓ pa꜕ssa꜕ti \\
  Atha nibbinda꜕ti dukkhe & esa꜕ maggo vi꜓su꜕ddh꜓iyā \\
\end{twochants}

\begin{english}
  `There is no self in a꜕nything' --\\
  When with wisdom th꜓is i꜕s seen\\
  One feels we꜕ary꜕ o꜕f a꜕ll du꜕kkha;\\
  This is the path to pu꜓ri꜕ty.
\end{english}

\begin{twochants}
  A꜕ppa꜕kā te manusse꜓su꜕ & ye janā pāra-gāmi꜓no \\
  A꜕thāyaṃ i꜕ta꜕rā pajā & tīram-evānudhā꜓va꜕ti \\
\end{twochants}

\begin{english}
  Few amongst huma꜕nkind\\
  Are those who g꜓o b꜕eyond\\
  Yet there are the ma꜕ny folks\\
  Ever wand'ring o꜕n th꜕is shore.
\end{english}

\begin{twochants}
  Ye ca꜕ kho sammad-akkhāte & dhamme dhammānuva꜓tt꜕ino \\
  Te ja꜕nā pā꜕ram-essanti & ma꜕ccu-dheyyaṃ sud'u꜓tta꜕raṃ \\
\end{twochants}

\begin{english}
  Wherever Dha꜕mm꜕a i꜕s we꜕ll-taught,\\
  Those who train in li꜓ne wi꜕th it\\
  Are the ones who wi꜕ll cr꜕oss o꜕ver\\
  The realm o꜓f death so ha꜕rd t꜕o flee.
\end{english}

\begin{twochants}
  Kaṇhaṃ dhammaṃ vi꜕ppahā꜓ya & su꜕kkaṃ bhāvetha꜕ paṇḍi꜓to \\
  Okā a꜕noka꜕m-āgamma & viveke ya꜕tth꜕a dūramaṃ \\
  Ta꜕trābh꜕irat꜕im-iccheyya & hi꜕tvā kāme a꜕kiñc꜓ano \\
\end{twochants}

\begin{english}
  Abandoning the da꜕rke꜕r states,\\
  The wise pursu꜓e th꜕e bright;\\
  From the floo꜕ds dr꜕y la꜕nd th꜕ey reach\\
  Living wi꜓thdrawn so ha꜕rd t꜕o do.\\
  Such rare de꜓light on꜕e sho꜕uld de꜕si꜕re,\\
  Sense pleasu꜓res cast awa꜕y,\\
  No꜕t ha꜕vin꜕g a꜕nything.
\end{english}

\chapter{Verses on a Shining Night of Prosperity}% {{{1

\begin{leader}
  [Ha꜓nda mayaṃ bhadd'eka-ratta꜕-gāthā꜓yo bha꜕ṇāmase]
\end{leader}

\begin{twochants}
  A꜕tītaṃ nānvāga꜕meyya & nappa꜕ṭikaṅkhe꜓ a꜕nāga꜓taṃ \\
  Ya꜕d a꜕tītam-pa꜕hīnan-taṃ & a꜕ppattañ-c꜕a a꜕nāga꜕taṃ \\
\end{twochants}

\begin{english}
  One should not revi꜕ve th꜕e past\\
  Nor speculate on wha꜓t's t꜕o come;\\
  The past is l꜕eft be꜕hind,\\
  The futu꜓re is un-r꜓ea꜕lised.
\end{english}

\begin{twochants}
  Paccu꜕ppannañ-ca꜕ yo dhammaṃ & tattha tattha vi꜓pa꜕ss꜕ati \\
  Asa꜓ṃhi꜕raṃ asa꜓ṅku꜕ppaṃ & taṃ viddhām-a꜕nu꜕brūhaye \\
\end{twochants}

\begin{english}
  In every presently ari꜕se꜕n state\\
  There just there one cle꜓arly꜕ sees;\\
  Unmoved una꜕gi꜕ta꜕ted,\\
  Such insight i꜓s on꜕e's strength.
\end{english}

\begin{twochants}
  A꜕jj'eva ki꜕cca꜕m-ātappaṃ & ko jaññā ma꜓ra꜕ṇaṃ su꜕ve \\
  Na hi no sa꜓ṅga꜕ran-tena & mahā-senena ma꜓cc꜕unā \\
\end{twochants}

\begin{english}
  Ardently doing one's ta꜕sk t꜕oday,\\
  Tomorrow who knows de꜓ath ma꜕y come;\\
  Facing the mighty ho꜕rdes o꜕f death,\\
  Indeed o꜓ne cannot str꜕ike a꜕ deal.
\end{english}

\begin{twochants}
  Evaṃ vihārim-ātāpiṃ & a꜕ho-rattam-a꜕tandi꜓taṃ \\
  Taṃ ve bha꜕dd'eka꜕-ratto-ti & santo ā꜕ci꜕kkha꜕te muni \\
\end{twochants}

\begin{english}
  To dwell with e꜕ne꜕rgy꜕ a꜕roused\\
  Thus for a night of no꜓n-de꜕cline,\\
  That is a `night of shi꜕ni꜕ng pr꜕osperity'\\
  So it was taught by the Peacefu꜕l Sage.
\end{english}

\chapter{Verses on the Buddha's First Exclamation}% {{{1

\begin{leader}
  [Ha꜓nda mayaṃ paṭhama-bu꜕ddha-bhāsi꜕ta-gāthāyo bh꜕aṇāmase]
\end{leader}

\begin{twochants}
  A꜕neka꜕-jāti꜕-sa꜓ṃsā꜓raṃ & sa꜕ndhāviss꜓aṃ a꜕nibbi꜕saṃ \\
  Ga꜕ha-kā꜕raṃ ga꜕vesa꜓nto & dukkhā jāt꜕i pu꜕nappu꜕naṃ \\
\end{twochants}

\begin{english}
  For many lifetimes in the ro꜕und o꜕f birth,\\
  Wandering on e꜓ndle꜕ssly,\\
  For the bu꜕ilde꜕r o꜕f th꜕is ho꜕use I꜕ searched --\\
  How painful is repe꜓ated꜕ birth
\end{english}

\begin{twochants}
  Ga꜕ha-kā꜕raka꜕ diṭṭho꜓'si & pu꜕na gehaṃ na kā꜓hasi \\
  Sa꜕bbā te phāsu꜕kā bhaggā & gaha-kūṭa꜓ṃ vi꜕saṅkh꜕ataṃ \\
  Visa꜓ṅkhā꜕ra-ga꜕taṃ ci꜕ttaṃ & taṇhānaṃ kh꜕aya꜕m-ajjh꜕agā \\
\end{twochants}

\begin{english}
  House-builder yo꜕u've be꜕en seen,\\
  Another home you w꜓ill no꜕t build,\\
  All your ra꜕fte꜕rs ha꜕ve be꜕en snapped,\\
  Dismantle꜓d is your ri꜕dge-pole;\\
  The non-co꜓nstructing mi꜕nd\\
  Ha꜕s co꜕me to꜕ cra꜕vi꜕ng's end.
\end{english}

\chapter{Verses on Respect for the Dhamma}% {{{1

\begin{leader}
  [Ha꜓nda mayaṃ dhamma-gā꜕rav'ādi꜕-gāthā꜓yo bha꜕ṇāmase]
\end{leader}

\begin{twochants}
  Ye ca꜕ atītā sa꜓mbuddhā & ye ca꜕ buddhā a꜕nāga꜓tā \\
  Yo c'eta꜕rahi sambuddho & ba꜕hunnaṃ so꜕ka꜕-nāsa꜕no \\
\end{twochants}

\begin{english}
  All the Buddhas o꜕f th꜕e past,\\
  All the Buddhas ye꜓t to꜕ come,\\
  The Buddha of this cu꜕rre꜕nt age --\\
  Dispe꜓llers of mu꜕ch so꜕rrow.
\end{english}

\begin{twochants}
  Sa꜕bbe sa꜕ddhamma-gar꜓uno & vi꜕ha꜕riṃsu vi꜕ha꜕ranti ca \\
  A꜕tho pi viha꜕riss꜓anti & esā buddhāna꜓ dha꜕mma꜕tā \\
\end{twochants}

\begin{english}
  Those having lived or li꜕vi꜕ng now,\\
  Those livi꜓ng in the fu꜕ture,\\
  All do reve꜕re th꜕e Tru꜕e Dha꜕mma --\\
  That is th꜓e nature o꜕f a꜕ll Bu꜕ddhas.
\end{english}

\begin{twochants}
  Tasmā h꜕i atta-kāmena & mahattam-abhika꜓ṅkh꜕atā \\
  Sa꜕ddhammo ga꜕ru꜓-kāta꜕bbo & s꜕araṃ buddhāna sā꜓sa꜕naṃ \\
\end{twochants}

\begin{english}
  Therefore de꜓siring on꜕e's ow꜕n welfare,\\
  Pursui꜓ng greatest a꜕spi꜕ra꜕tions,\\
  One should reve꜕re th꜕e Tr꜕ue Dha꜕mma,\\
  Reco꜓llecting th꜕e Bu꜕ddha꜕'s te꜕aching.
\end{english}

\begin{twochants}
  Na h꜕i dhammo a꜕dhammo ca & ubho s꜕ama-vipāki꜓no \\
  A꜕dhammo nirayaṃ neti & dh꜕ammo pāpeti꜕ su꜕gga꜕tiṃ \\
\end{twochants}

\begin{english}
  What is true Dhamma an꜕d wh꜕at not\\
  Will never have the sa꜓me re꜕sults,\\
  While lack of Dha꜕mma꜕ le꜕ads t꜕o he꜕ll-realms\\
  True Dhamma꜓ takes one o꜕n a꜕ go꜕od course.
\end{english}

Dhammo ha꜕ve rakkha꜕ti꜕ dhamma꜓-cāriṃ\\
Dhammo su꜕ciṇṇo su꜕kham-āvahāti\\
Esā꜓'ni꜕saṃso dhamme su꜕ciṇṇe

\begin{english}
  The Dhamma guards who li꜕ves i꜕n li꜓ne wi꜕th its\\
  And leads to ha꜕ppi꜕ne꜕ss whe꜕n pra꜕cti꜕sed well --\\
  This is th꜓e blessing of we꜕ll-pr꜕acti꜕sed Dha꜕mma.
\end{english}

\chapter{Verses on the training code}% {{{1

\begin{leader}
  [Ha꜓nda mayaṃ ovāda-pā꜕ṭi꜕mokkha-gāthā꜓yo bha꜕ṇāmase]
\end{leader}

\begin{instruction}
  Version One
\end{instruction}

\begin{tabular}{@{}p{0.5\linewidth} p{0.5\linewidth}@{}}

Sa꜕bb꜕a-pāpa꜕ss꜕a a꜕ka꜕ra꜓ṇaṃ &
\tr{Avoidance of all e꜕vil ways;} \\

Ku꜕salassūpasa꜓mpa꜕dā &
\tr{Commitment to what's wh꜓olly good;} \\

Sa꜕ci꜕tta-pa꜕ri꜓yoda꜓pa꜕naṃ &
\tr{Purifica꜕tion of one's mind:} \\

Etaṃ buddhāna sā꜓sa꜕naṃ &
\tr{Just this is what the Bu꜓ddhas teach.} \\

Kha꜓ntī pa꜕ramaṃ ta꜕po tīti꜕kkhā &
\tr{Pa꜕tience is the cl꜕eansing flame;} \\

Nibbānaṃ pa꜕ramaṃ\newline va꜕dant꜕i buddhā &
\tr{Nibbāna's supre꜓me,\newline the Bu꜓ddhas say.} \\

Na h꜕i pa꜕bbaji꜕to pa꜕rūpaghātī &
\tr{Ha꜕rming others, you're n꜓o recluse;} \\

Sa꜕maṇo ho꜓ti pa꜕raṃ vihe꜓ṭha꜕yanto &
\tr{A trouble-maker's no꜕ samana.} \\

A꜕nūpa꜕vādo a꜕nūpa꜕ghāto &
\tr{To neither insult nor cau꜕se wounds;} \\

Pā꜕ṭimokkhe꜓ ca꜕ sa꜓ṃva꜕ro &
\tr{To live restrai꜓ned by training rules;} \\

Mattaññu꜕tā ca꜕\newline bhatta꜕smiṃ &
\tr{To know what's eno꜓ugh\newline when taking food;} \\

Pa꜕ntañ-ca꜕ saya꜓n'āsa꜕naṃ &
\tr{To dw꜕ell alone in a qu꜓iet place;} \\

A꜕dhici꜕tte ca꜕ āyogo &
\tr{And devo꜕tion to the hi꜓gher mind:} \\

Etaṃ buddhāna sā꜓sa꜕naṃ &
\tr{Every Buddha te꜓aches this.} \\

\end{tabular}

\clearpage

\begin{instruction}
  Version Two
\end{instruction}

\begin{tabular}{@{}p{0.5\linewidth} p{0.5\linewidth}@{}}

Sabba-pāpa꜕ss꜕a a꜕ka꜕ra꜓ṇaṃ &
\tr{Not do꜕in꜕g a꜕ny꜕ e꜕vil;} \\

Kusalassūpasa꜓mpa꜕dā &
\tr{To be committed to꜕ th꜕e good;} \\

Sa꜕citta-pa꜕ri꜓yoda꜓pa꜕naṃ &
\tr{To pu꜕ri꜕fy꜕ on꜕e's mind:} \\

Etaṃ buddhāna sā꜓sa꜕naṃ &
\tr{These are th꜓e teachings o꜕f al꜕l Bu꜕ddhas.} \\

Kha꜓ntī\newline pa꜕ramaṃ\newline ta꜕po tīti꜕kkhā &
\tr{Patient e꜓ndurance\newline is the highest pra꜕cti꜕ce,\newline bu꜕rni꜕ng ou꜕t de꜕fi꜕lements;} \\

Nibbānaṃ pa꜕ramaṃ\newline vadant꜕i buddhā &
\tr{The Buddha꜓s say\newline Nibbā꜕na꜕ i꜕s su꜕preme.} \\

Na h꜕i pa꜕bbaji꜕to\newline pa꜕rūpaghātī &
\tr{Not a renu꜕nci꜕ant\newline is꜕ on꜕e wh꜕o in꜕ju꜕res o꜕thers;} \\

Sa꜕maṇo ho꜓ti pa꜕raṃ\newline vihe꜓ṭha꜕yanto &
\tr{Whoever troubl꜓es others\newline ca꜕n't b꜕e ca꜓lled a꜕ monk.} \\

A꜕nūpa꜕vādo a꜕nūpa꜕ghāto &
\tr{Not to insu꜕lt an꜕d no꜕t t꜕o i꜕njure;} \\

Pāṭimokkhe꜓ ca꜕ sa꜓ṃva꜕ro &
\tr{To live restrained by tra꜕ini꜕ng rules;} \\

Mattaññu꜕tā ca꜕ bhatta꜕smiṃ &
\tr{Knowing one's me꜕asure a꜕t t꜕he meal;} \\

Pantañ-ca꜕ saya꜓n'āsa꜕naṃ &
\tr{Retreating to a lo꜓ne꜕ly place;} \\

A꜕dhici꜕tte ca꜕ āyogo &
\tr{Devoti꜓on to the hi꜕ghe꜕r mind:} \\

Etaṃ buddhāna sā꜓sa꜕naṃ &
\tr{These are the tea꜕chi꜕ngs o꜕f al꜕l Bu꜕ddhas.} \\

\end{tabular}

\chapter{Verses on the Last Instructions}% {{{1

\begin{leader}
  [Ha꜓nda mayaṃ pacchima-ovāda-gāthā꜓yo bha꜕ṇāmase]
\end{leader}

Handa dāni bhi꜓kkha꜕ve āmant꜕ayāmi꜓ vo

\begin{english}
  Now bhikkhus I decl꜕are t꜕o you,
\end{english}

Vaya-dhammā sa꜓ṅkhā꜓rā

\begin{english}
  Change is th꜓e nature o꜕f co꜕ndi꜕tio꜕ned things;
\end{english}

A꜕ppamādena sa꜓mpā꜕detha

\begin{english}
  Perfect yo꜓urselves no꜕t be꜕i꜕ng ne꜕gligent:
\end{english}

Ayaṃ tathā꜓ga꜕tassa pa꜕cchi꜓mā vācā

\begin{english}
  These are the Tathā꜓ga꜕ta's fi꜕na꜕l words.
\end{english}

\chapter{The Teaching on Mindfulness of Breathing}% {{{1

\begin{leader}
  [Ha꜓nda mayam ānāpānass꜕ati-sutta-pāṭhaṃ bha꜕ṇāmase]
\end{leader}

Ānāpāna꜓ssa꜕ti bhi꜓kkha꜕ve bhāvi꜓tā bahu꜕lī-ka꜕tā

\begin{english}
  Bhikkhus wh꜕en mindfulness of bre꜓athing is de꜕veloped and cu꜕ltiva꜓ted
\end{english}

Mahappha꜕lā ho꜓ti mahā꜓-nisa꜓ṃsā

\begin{english}
  It is of gre꜕at fruit and great be꜕nefit;
\end{english}

Ānāpāna꜓ssa꜕ti bhi꜓kkha꜕ve bhāvi꜓tā bahu꜕lī-ka꜕tā

\begin{english}
  Wh꜕en mindfulness of bre꜓athing is de꜕veloped and cu꜕ltiva꜓ted
\end{english}

Ca꜕ttāro sati꜓pa꜕ṭṭhāne pa꜕ri꜓pū꜕reti

\begin{english}
  It fu꜕lfills the Four Foundations of Mi꜕ndfu꜕lness;
\end{english}

Ca꜕ttāro sa꜕tipa꜕ṭṭhānā bhāvi꜓tā bahu꜕lī-ka꜕tā

\begin{english}
  When th꜕e Four Foundations of Mi꜓ndfulness are de꜕veloped and cu꜕ltiva꜓ted
\end{english}

Sa꜕tta-bojjhaṅge pa꜕ri꜓pū꜕renti

\begin{english}
  They fu꜕lfill the Seven Factors of Awa꜕kening;
\end{english}

Sa꜕tta-bojjhaṅgā bhāvi꜓tā bahu꜕lī-ka꜕tā

\begin{english}
  When th꜕e Seven Factors of Awa꜓kening are de꜕veloped and cu꜕ltiva꜓ted
\end{english}

Vijjā-vimuttiṃ pa꜕ri꜓pū꜕renti

\begin{english}
  They fu꜕lfill true knowledge and deli꜕verance.
\end{english}

Kathaṃ bhāvi꜓tā ca bhi꜓kkha꜕ve ānāpāna꜓ss꜕ati ka꜕thaṃ bahu꜕lī-ka꜕tā

\begin{english}
  An꜕d how bhikkhus is mindfulness of bre꜓athing de꜕veloped and cu꜕ltiva꜓ted
\end{english}

Mahappha꜕lā ho꜓ti mahā꜓-nisa꜓ṃsā

\begin{english}
  So that it is of gre꜕at fruit and great be꜕nefit?
\end{english}

Idha bhi꜓kkha꜕ve bhikkhu

\begin{english}
  Here bhikkhus a bhi꜕kkhu,
\end{english}

Arañña꜓-ga꜕to vā

\begin{english}
  Gone to꜕ the fo꜓rest,
\end{english}

Rukkha-mūla꜓-ga꜕to vā

\begin{english}
  To the fo꜕ot o꜕f a꜕ tree
\end{english}

Suññāgāra꜓-ga꜕to vā

\begin{english}
  Or to an em꜓pty꜕ hut.
\end{english}

N꜕isīdati pallaṅkaṃ ābhuji꜓tv꜕ā

\begin{english}
  Si꜕ts down having cro꜕ssed hi꜕s legs,
\end{english}

Ujuṃ kāyaṃ pa꜕ṇidhāya pa꜕rimukhaṃ sa꜕tiṃ u꜕paṭṭha꜕petvā

\begin{english}
  Sets his bo꜕dy꜕ e꜕rect having established mi꜓ndfulness in fro꜕nt o꜕f him.
\end{english}

So sa꜕to'va a꜕ssasa꜕ti sa꜕to'va pa꜕ssa꜕sa꜕ti

\begin{english}
  Ever mi꜓ndful he bre꜕athes in; mindful h꜕e bre꜕athes out.
\end{english}

Dīghaṃ vā assa꜕sa꜓nto dīghaṃ a꜕ssasā꜓mī-ti pa꜕jānāti

\begin{english}
  Breathing i꜓n long, he꜕ knows `I bre꜕athe i꜕n long';
\end{english}

Dīghaṃ vā pa꜕ssa꜕santo dīghaṃ pa꜕ssasā꜓mī-ti pa꜕jānāti

\begin{english}
  Breathing ou꜕t long, he꜕ knows `I bre꜕athe ou꜕t long';
\end{english}

Rassaṃ vā a꜕ssa꜕santo rassaṃ a꜕ssasā꜓mī-ti pa꜕jānāti

\begin{english}
  Breathing i꜓n short, h꜕e knows `I bre꜕athe i꜕n short';
\end{english}

Rassaṃ vā pa꜕ssa꜕santo rassaṃ pa꜕ssasā꜓mī-ti pa꜕jānāti

\begin{english}
  Breathing ou꜕t short, h꜕e knows `I bre꜕athe ou꜕t short'.
\end{english}

Sabba꜕-kāya-pat꜕isa꜓ṃvedī a꜕ssasi꜕ssāmī-ti si꜕kkh꜕ati

\begin{english}
  He tra꜕ins thus: `I shall breathe i꜓n experiencing the whole bo꜕dy'.
\end{english}

Sabba꜕-kāya-pat꜕isa꜓ṃvedī pa꜕ssasi꜕ssāmī-ti si꜕kkh꜕ati

\begin{english}
  He tra꜕ins thus: `I shall breathe ou꜕t e꜕xpe꜕ri꜕enci꜕ng th꜕e who꜕le bo꜕dy'.
\end{english}

Passa꜕mbhayaṃ kāya꜕-sa꜓ṅkhāraṃ a꜕ssasi꜕ssāmī-ti si꜕kkh꜕ati

\begin{english}
  He tra꜕ins thus: `I shall breathe i꜓n tranquillising the bodily forma꜕tions'.
\end{english}

Passa꜕mbhayaṃ kāya꜕-sa꜓ṅkhāraṃ pa꜕ssasi꜕ssāmī-ti si꜕kkh꜕ati

\begin{english}
  He tra꜕ins thus: `I shall breathe ou꜕t tra꜕nqui꜕ll꜕isi꜕ng th꜕e bo꜕dily fo꜕rmations'.
\end{english}

Pīti꜕-pati꜕sa꜓ṃvedī a꜕ssasi꜕ssāmī-ti si꜕kkh꜕ati

\begin{english}
  He tra꜕ins thus: `I shall breathe i꜓n experiencing ra꜕pture'.
\end{english}

Pīti꜕-pati꜕sa꜓ṃvedī pa꜕ssasi꜕ssāmī-ti si꜕kkh꜕ati

\begin{english}
  He tra꜕ins thus: `I shall breathe ou꜕t e꜕xpe꜕ri꜕enci꜕ng ra꜕pture'.
\end{english}

Sukh꜕a-pati꜕sa꜓ṃvedī a꜕ssasi꜕ssāmī-ti si꜕kkh꜕ati

\begin{english}
  He tra꜕ins thus: `I shall breathe i꜓n experiencing ple꜕asure'
\end{english}

Sukh꜕a-pati꜕sa꜓ṃvedī pa꜕ssasi꜕ssāmī-ti si꜕kkh꜕ati

\begin{english}
  He tra꜕ins thus: `I shall breathe ou꜕t e꜕xpe꜕ri꜕enci꜕ng ple꜕asure'.
\end{english}

Citta꜕-sa꜓ṅkhāra-pati꜕sa꜓ṃvedī a꜕ssasi꜕ssāmī-ti si꜕kkh꜕ati

\begin{english}
  He tra꜕ins thus: `I shall breathe i꜓n experiencing the mental forma꜕tions'.
\end{english}

Citta꜕-sa꜓ṅkhāra-pati꜕sa꜓ṃvedī pa꜕ssasi꜕ssāmī-ti si꜕kkh꜕ati

\begin{english}
  He tra꜕ins thus: `I shall breathe ou꜕t e꜕xpe꜕ri꜕enci꜕ng th꜕e me꜕nta꜕l fo꜕rma꜕tions'.
\end{english}

Passa꜕mbhayaṃ citta꜕-sa꜓ṅkhāraṃ a꜕ssasi꜕ssāmī-ti si꜕kkh꜕ati

\begin{english}
  He tra꜕ins thus: `I shall breathe i꜓n tranquillising the mental forma꜕tions'.
\end{english}

Passa꜕mbhayaṃ citt꜕a-sa꜓ṅkhāraṃ pa꜕ssasi꜕ssāmī-ti si꜕kkh꜕ati

\begin{english}
  He tra꜕ins thus: `I shall breathe ou꜕t tra꜕nqu꜕ill꜕isi꜕ng th꜕e me꜕nta꜕l fo꜕rma꜕tions'.
\end{english}

Citta꜕-pati꜕sa꜓ṃvedī a꜕ssasi꜕ssāmī-ti si꜕kkh꜕ati

\begin{english}
  He tra꜕ins thus: `I shall breathe i꜓n experiencing th꜕e mind'.
\end{english}

Citta꜕-pati꜕sa꜓ṃvedī pa꜕ssasi꜕ssāmī-ti si꜕kkh꜕ati

\begin{english}
  He tra꜕ins thus: `I shall breathe ou꜕t e꜕xpe꜕ri꜕enci꜕ng th꜕e mind'.
\end{english}

A꜕bhippa꜕moda꜓yaṃ cittaṃ a꜕ssasi꜕ssāmī-ti si꜕kkh꜕ati

\begin{english}
  He tra꜕ins thus: `I shall breathe i꜓n gladdening th꜕e mind'.
\end{english}

A꜕bhippa꜕moda꜓yaṃ cittaṃ pa꜕ssasi꜕ssāmī-ti si꜕kkh꜕ati

\begin{english}
  He tra꜕ins thus: `I shall breathe ou꜕t gl꜕adde꜕ni꜕ng th꜕e mind'.
\end{english}

Sa꜕māda꜓haṃ cittaṃ a꜕ssasi꜕ssāmī-ti si꜕kkh꜕ati

\begin{english}
  He tra꜕ins thus: `I shall breathe i꜓n concentrating th꜕e mind'
\end{english}

Sa꜕māda꜓haṃ cittaṃ pa꜕ssasi꜕ssāmī-ti si꜕kkh꜕ati

\begin{english}
  He tra꜕ins thus: `I shall breathe ou꜕t co꜕nce꜕ntr꜕ati꜕ng th꜕e mind'.
\end{english}

Vimoca꜓yaṃ cittaṃ a꜕ssasi꜕ssāmī-ti si꜕kkh꜕ati

\begin{english}
  He tra꜕ins thus: `I shall breathe i꜓n liberating th꜕e mind'.
\end{english}

Vimoca꜓yaṃ cittaṃ pa꜕ssasi꜕ssāmī-ti si꜕kkh꜕ati

\begin{english}
  He tra꜕ins thus: `I shall breathe ou꜕t li꜕be꜕ra꜕ti꜕ng th꜕e mind'.
\end{english}

Aniccānupa꜕ssī a꜕ssasi꜕ssāmī-ti si꜕kkh꜕ati

\begin{english}
  He tra꜕ins thus: `I shall breathe i꜓n contemplating impe꜕rmanence'.
\end{english}

Aniccānupa꜕ssī pa꜕ssasi꜕ssāmī-ti si꜕kkh꜕ati

\begin{english}
  He tra꜕ins thus: `I shall breathe ou꜕t co꜕nte꜕mpla꜕ti꜕ng i꜕mpe꜕rmanence'.
\end{english}

Virāgānupa꜕ssī a꜕ssasi꜕ssāmī-ti si꜕kkh꜕ati

\begin{english}
  He tra꜕ins thus: `I shall breathe i꜓n contemplating the fading away of pa꜕ssions'.
\end{english}

Virāgānupa꜕ssī pa꜕ssasi꜕ssāmī-ti si꜕kkh꜕ati

\begin{english}
  He tra꜕ins thus: `I shall breathe ou꜕t co꜕nte꜕mpl꜕ati꜕ng th꜕e fa꜕di꜕ng aw꜕ay o꜕f pa꜕ssions'.
\end{english}

Nirodhānupa꜕ssī a꜕ssasi꜕ssāmī-ti si꜕kkh꜕ati

\begin{english}
  He tra꜕ins thus: `I shall breathe i꜓n contemplating cessa꜕tion'.
\end{english}

Nirodhānupa꜕ssī pa꜕ssasi꜕ssāmī-ti si꜕kkh꜕ati

\begin{english}
  He tra꜕ins thus: `I shall breathe ou꜕t co꜕nte꜕mpl꜕ati꜕ng ce꜕ss꜕ation'.
\end{english}

Pa꜕ṭiniss꜕aggānupa꜕ssī a꜕ssasi꜕ssāmī-ti si꜕kkh꜕ati

\begin{english}
  He tra꜕ins thus: `I shall breathe i꜓n contemplating reli꜕nquishment'.
\end{english}

Pa꜕ṭinissa꜕ggānupa꜕ssī pa꜕ssasi꜕ssāmī-ti si꜕kkh꜕ati

\begin{english}
  He tra꜕ins thus: `I shall breathe ou꜕t co꜕nte꜕mpl꜕ati꜕ng re꜕li꜕nquishment'.
\end{english}

Evaṃ bhāvi꜓tā kho bhi꜓kkha꜕ve ānāpāna꜓ss꜕ati evaṃ bahu꜕lī-ka꜕tā

% TODO: check \cD : Bhikkhus that is ho꜕w mindfulness of bre꜓athing is d\꜕cD eveloped and cu꜕ltiva꜓ted

\begin{english}
  Bhikkhus that is ho꜕w mindfulness of bre꜓athing is developed and cu꜕ltiva꜓ted
\end{english}

Mahappha꜕lā ho꜓ti mahā꜓-nisa꜓ṃsā-ti

\begin{english}
  So that it is of gr꜕eat fruit and great be꜕nefit.
\end{english}

% }}}1

% End of reflections-and-recollections-p2.tex
