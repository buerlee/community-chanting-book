% vim: foldmethod=marker foldlevel=0 foldtext=FoldText()

\chapter{Dedication of Offerings}                   % {{{1

%\suttaref{Trad.}%
[Yo so] bha꜕gavā a꜕rahaṃ sammāsambuddho

\begin{english}
To the Ble꜕ssed One, the꜕ Lord, who fu꜓lly a꜕ttained perfect enli꜕ghtenment,
\end{english}

Svākkhā꜓to yena bha꜕gava꜓tā dhammo

\begin{english}
To the꜕ Teaching which he expo꜕unde꜕d so well,
\end{english}

Supaṭi꜕panno yassa bha꜕gava꜕to sāvaka꜕saṅgho

\begin{english}
And to the꜕ Blessed One's disci꜓ples who have pra꜕ctised well,
\end{english}

Tam-ma꜓yaṃ bha꜕gavantaṃ sa꜕dhammaṃ sa꜕saṅghaṃ

\begin{english}
To these --- the꜕ Buddha, the꜕ Dhamma, a꜕nd the Sa꜓ngha ---
\end{english}

Imehi꜓ sakkārehi꜕ yathārahaṃ āropi꜕tehi a꜕bhi꜓pūja꜕yāma

\begin{english}
We render wi꜕th offerings our ri꜓ghtful ho꜖mage.
\end{english}

Sādhu꜓ no bhante bha꜕gavā su꜕cira-parinibbu꜕topi

\begin{english}
It is we꜓ll for us that the Ble꜕ssed One, having attained li꜕bera꜓tion,
\end{english}

Pacchi꜓mā-ja꜕na꜓tānu꜓kampa꜕-mānasā

\begin{english}
Still had co꜕mpassion for later ge꜓nera꜖tions.
\end{english}

Ime sakkāre dugga꜕ta꜕-paṇṇākāra꜓-bhūte pa꜕ṭiggaṇhātu

\begin{english}
May the꜕se simple off꜓erings be acce꜕pted
\end{english}

Amhā꜓kaṃ dīgha꜕rattaṃ hi꜕tāya su꜕khāya

\begin{english}
For o꜕ur long-lasting benefit and fo꜕r the ha꜓ppiness it giv꜖es us.
\end{english}

%}}}1
\clearpage

Arahaṃ sammāsambuddho bha꜕gavā              % {{{1

\begin{english}
The꜕ Lord, the꜕ Perfectly Enli꜓ghtened and Ble꜕ssed One ---
\end{english}

Buddhaṃ bha꜕gavantaṃ a꜕bhi꜓vādemi

\begin{english}
  I꜕ render homage to꜕ the Bu꜓ddha, the Ble꜕ssed One.
  \instr{Bow}
\end{english}

[Svākkhā꜓to] bha꜕gava꜓tā dhammo

\begin{english}
  The꜕ Teaching so co꜕mpletely expla꜓ined by him ---
\end{english}

Dhammaṃ namassāmi

\begin{english}
  I bo꜖w to꜕ the꜕ Dha꜕mma.
  \instr{Bow}
\end{english}

[Supaṭi꜕panno] bha꜕gava꜕to sāvaka꜕saṅgho

\begin{english}
The꜕ Blessed One's disci꜓ples who have pra꜕ctised well ---
\end{english}

Sa꜓ṅghaṃ na꜕māmi

\begin{english}
  I bo꜖w to꜕ the꜕ Sa꜕ngha.
  \instr{Bow}
\end{english}

\chapter{Preliminary Homage}            % {{{1

\begin{leader}
  [Ha꜓nda mayaṃ buddhassa꜕ bha꜕gavato pubbabhāga-namakā꜕raṃ karomase]
\end{leader}

\begin{english}
  [No꜓w let us pay preliminary homage to the Bu꜕ddha.]
\end{english}

\vspace{\baselineskip}

%\suttaref{Cf. DN 21}%
Namo tassa bha꜕gava꜕to araha꜕to sa꜓mmāsa꜓mbuddha꜕ssa

\instr{Three times}

\begin{english}
  Ho꜓ma꜓ge to the꜕ Ble꜕ssed, No꜓ble, a꜕nd Pe꜕rfectly Enli꜓ghtened One.

  \instr{Three times}
\end{english}

%}}}1
\clearpage

\chapter{Homage to the Buddha}                % {{{1

\begin{leader}
  [Ha꜓nda mayaṃ buddhābhi꜕tthu꜕tiṃ karomase]
\end{leader}

\begin{english}
  [No꜓w let us chant in praise of the Bu꜕ddha.]
\end{english}

%\suttaref{DN 2}%
Yo so tathā꜓ga꜕to a꜕rahaṃ sammāsambuddho

\begin{english}
  The Tathāga꜕ta is the Pu꜓re One, the꜕ Perfectly Enli꜓ghtened One.
\end{english}

Vijjāca꜕raṇa꜓-sampanno

\begin{english}
  He is i꜕mpeccable i꜕n conduct and u꜕ndersta꜓nding,
\end{english}

Su꜕ga꜕to

\begin{english}
  The A꜕cco꜓mplished One,
\end{english}

Loka꜕vi꜓dū

\begin{english}
  The꜕ Knower o꜓f the꜕ Worlds.
\end{english}

Anu꜓tta꜕ro purisa꜕damma-sārathi

\begin{english}
  He trains perfectly tho꜕se who wi꜓sh to꜓ be꜕ trained.
\end{english}

Satthā deva-ma꜕nussānaṃ

\begin{english}
  He is Teacher of go꜓ds and hu꜕mans.
\end{english}

Buddho bha꜕gavā

\begin{english}
  He is awake and ho꜕ly.
\end{english}

Yo imaṃ lokaṃ sa꜕devakaṃ sa꜕mārakaṃ sa꜕brahma꜕kaṃ

\begin{english}
  In this world with i꜕ts gods, demons, a꜕nd kind spi꜓rits,
\end{english}

Sassa꜓maṇa-brāhmaṇiṃ pa꜕jaṃ sa꜕deva-ma꜕nussa꜓ṃ sa꜕yaṃ a꜕bhiññā sacchika꜕tv꜓ā pa꜕vedesi

\begin{english}
  Its se꜓ekers and sa꜕ges, ce꜕lestial and hu꜕ma꜓n beings, he has by \\de꜕ep insight reve꜓aled the꜕ Truth.
\end{english}

Yo dhammaṃ dese꜓si ā꜕di꜓-kalyāṇaṃ majjhe꜓-ka꜕lyāṇaṃ \\pa꜕riyosāṇa-k꜕alyāṇaṃ

\begin{english}
  He has pointed out the Dha꜕mma: beautiful in the begi꜕nning, \\beautiful in the mi꜓ddle, beautiful i꜕n the꜓ end.
\end{english}

Sāttha꜓ṃ sa꜕byañjanaṃ kevala-pa꜕ripuṇṇaṃ pa꜕risuddhaṃ \\brahma-ca꜕ri꜓yaṃ pa꜕kāsesi

\begin{english}
  He has explained the Spi꜓ri꜓tu꜕al Life of co꜕mplete pu꜓rity in its \\e꜓ssence and conve꜕ntions.
\end{english}

Tam-aha꜓ṃ bha꜕gavantaṃ a꜕bhi꜓pūja꜕yāmi tam-aha꜓ṃ bha꜕gavantaṃ \\si꜕rasā꜓ na꜕māmi

\begin{english}
  I chant m꜕y praise to꜕ the Ble꜓ssed One, I bow m꜕y head to꜕ \\the꜓ Ble꜓ssed One.
  \instr{Bow}
\end{english}

%}}}1
\clearpage

\chapter{Homage to the Dhamma}                      % {{{1

\begin{leader}
  [Ha꜓nda mayaṃ dhammābhi꜕tthu꜕tiṃ karomase]
\end{leader}

\begin{english}
  [No꜓w let us chant in praise of the Dha꜕mma.]
\end{english}

%\suttaref{DN 16}%
Yo so svākkhā꜓to bha꜕gava꜓tā dhammo

\begin{english}
  The꜕ Dhamma is we꜕ll expo꜓unded by the Ble꜕ssed One,
\end{english}

Sa꜓ndiṭṭhi꜕ko

\begin{english}
  Apparent here a꜕nd now,
\end{english}

A꜕kāli꜕ko

\begin{english}
  Timeless,
\end{english}

Ehi꜕passi꜕ko

\begin{english}
  E꜕ncouraging inve꜕stiga꜓tion,
\end{english}

Opanayi꜕ko

\begin{english}
  Leading i꜕nwards,
\end{english}

Pa꜕cca꜕ttaṃ vedi꜓ta꜕bbo viññūhi% {{{1

\begin{english}
  To be e꜕xperienced indi꜕vidually by꜓ the꜕ wise.
\end{english}

Tam-aha꜓ṃ dhammaṃ a꜕bhi꜓pūja꜕yāmi tam-aha꜓ṃ dhammaṃ \\si꜕rasā꜓ na꜕māmi

\begin{english}
  I chant m꜕y praise to꜕ this Te꜓aching, I bow m꜕y head\\ to꜕ thi꜓s Truth.
  \instr{Bow}
\end{english}

\clearpage

\chapter{Homage to the Sangha}                      % {{{1

\begin{leader}
  [Ha꜓nda mayaṃ saṅghābhi꜕tthu꜕tiṃ karomase]
\end{leader}

\begin{english}
  [No꜓w let us chant in praise of the Sa꜕ngha.]
\end{english}

%\suttaref{DN 16}%
Yo so supaṭi꜕panno bha꜕gava꜕to sāvaka꜕saṅgho

\begin{english}
  They are the꜕ Blessed One's disci꜓ples, who have pra꜕ctised well,
\end{english}

Ujupaṭi꜕panno bha꜕gava꜕to sāvaka꜕saṅgho

\begin{english}
  Who have practised dire꜕ctly,
\end{english}

Ñāyapaṭi꜕panno bha꜕gava꜕to sāvaka꜕saṅgho

\begin{english}
  Who have practised insi꜓ghtfully,
\end{english}

Sā꜓mīci꜕pa꜕ṭi꜕panno bha꜕gava꜕to sāvaka꜕saṅgho

\begin{english}
  Those who pra꜓ctise with inte꜕grity ---
\end{english}

Yadidaṃ cattāri purisa꜕yugāni aṭṭha꜓ purisa꜕pugga꜕lā

\begin{english}
  That is the fo꜕ur pairs, the꜕ eight kinds of no꜓ble꜕ beings ---
\end{english}

Esa bha꜕gava꜕to sāvaka꜕saṅgho

\begin{english}
  These are the꜕ Blessed One's disci꜓ples.
\end{english}

Āhu꜕ṇeyyo

\begin{english}
  Such ones a꜕re worthy of gifts,
\end{english}

Pāhu꜕ṇeyyo

\begin{english}
  Worthy of ho꜕spita꜓lity,
\end{english}

\clearpage

Dakkhi꜕ṇeyyo

\begin{english}
  Worthy of o꜕fferings,
\end{english}

Añja꜕li-ka꜕ra꜓ṇīyo

\begin{english}
  Worthy o꜓f re꜕spect;
\end{english}

Anu꜓tta꜕raṃ puññakkhe꜕ttaṃ lokassa

\begin{english}
  They give o꜕ccasion for i꜕ncomparable go꜓odness to ari꜕se \\i꜕n the world.
\end{english}

Tam-aha꜓ṃ saṅghaṃ a꜕bhi꜓pūja꜕yāmi tam-aha꜓ṃ saṅghaṃ \\si꜕rasā꜓ na꜕māmi

\begin{english}
  I chant m꜕y praise to꜕ this Sa꜓ngha, I bow m꜕y head to꜕\\ thi꜓s Sa꜓ngha.
  \instr{Bow}
\end{english}

%}}}1
\clearpage

\chapter{Salutation to the Triple Gem}              % {{{1

\begin{leader}
  [Ha꜓nda mayaṃ ratanattaya-paṇāma-gāthā꜓yo ceva\\
  sa꜓ṃvega-parikittana-pāṭhañca꜕ bhaṇāmase]
\end{leader}

\begin{english}
  [No꜓w let us chant our salutation to the Tri꜕ple Gem and a passage \\to arouse u꜓rgency]
\end{english}

%\suttaref{Trad.}%
Buddho su꜕suddho ka꜕ruṇāmaha꜓ṇṇavo

\begin{english}
  The Bu꜕ddha, absolutel꜕y pure, with o꜓cean-like compa꜕ssion,
\end{english}

Yoccanta꜕-suddhabba꜕ra-ñāṇa꜕-loca꜕no

\begin{english}
  Possessing the꜕ clear sight of wi꜕sdom,
\end{english}

Lokassa꜕ pāpūpa꜕ki꜓lesa꜕-ghāta꜕ko

\begin{english}
  Destroyer o꜕f worldly self-corru꜓ption ---
\end{english}

Vandāmi꜓ buddhaṃ a꜕ha꜓m-āda꜕rena꜕ taṃ

\begin{english}
  Devote꜓dly i꜕ndeed, that Buddha I꜓ re꜕vere.
\end{english}

Dhammo pa꜕dīpo vi꜕ya tassa꜕ satthu꜕no

\begin{english}
  The Teaching of the꜕ Lord, like a꜕ lamp,
\end{english}

Yo magga꜓pākāma꜕ta꜕-bheda꜕-bhinna꜕ko

\begin{english}
  Illuminating the꜕ Path and its Fruit: the De꜕athless,
\end{english}

Lokuttaro yo ca꜕ ta꜕dattha꜕-dīpa꜕no

\begin{english}
  That which is beyo꜓nd the condi꜕tioned world ---
\end{english}

Vandāmi꜓ dhammaṃ a꜕ha꜓m-āda꜕rena꜕ taṃ

\begin{english}
  Devote꜓dly i꜕ndeed, that Dhamma I꜓ re꜕vere.
\end{english}

Sa꜓ṅgho su꜕khettābhyati-khe꜕tta-sa꜓ññito

\begin{english}
  The꜕ Sangha, the mo꜕st fertile gro꜓und for cultiva꜕tion,
\end{english}

Yo diṭṭha꜓santo su꜕ga꜕tānu꜕bodha꜕ko

\begin{english}
  Those who have reali꜕zed peace, awakened after the꜕ \\Acco꜓mplished One,
\end{english}

Lolappa꜕hīno a꜕ri꜓yo su꜕medha꜕so

\begin{english}
  No꜓ble a꜕nd wise, all longing aba꜕ndoned ---
\end{english}

Vandāmi꜓ saṅghaṃ a꜕ha꜓m-āda꜕rena꜕ taṃ

\begin{english}
  Devote꜓dly i꜕ndeed, that Sangha I꜓ re꜕vere.
\end{english}

Iccevam-ekanta꜕bhi꜓pūja꜕-neyya꜕kaṃ vatthuttayaṃ \\vanda꜕ya꜕tābhi꜕saṅkha꜕taṃ

\begin{english}
  This salutation should be꜕ made to tha꜓t which is wo꜕rthy.
\end{english}

Puññaṃ ma꜕yā yaṃ ma꜕ma꜕ sabbu꜕padda꜕vā mā ho꜓ntu꜕ ve tassa꜕ pa꜕bhāva꜕siddhi꜕yā

\begin{english}
  Through the꜕ power of su꜕ch good ac꜓tion, may a꜕ll obstacles di꜓sa꜕ppear.
\end{english}

Idha tathā꜓ga꜕to loke u꜕ppanno a꜕rahaṃ sammāsambuddho

\begin{english}
  One who knows things as the꜕y are has come into꜓ thi꜕s world; \\and he is an A꜕rahant, a꜕ perfectly Awa꜓kened being,
\end{english}

Dhammo ca꜕ desi꜕to niyyāni꜕ko u꜕pa꜕sa꜕miko pa꜕rinibbāni꜕ko sa꜓mbodha꜕gāmī su꜕ga꜕tappa꜕vedi꜕to

\begin{english}
  Purifying the꜕ way leading o꜕ut o꜕f de꜕lu꜕sion, calming and di꜕recting to pe꜓rfect peace, and leading to enli꜕ghtenment --- this Way he ha꜓s ma꜕de known.
\end{english}

Ma꜓yantaṃ dhammaṃ su꜕tvā evaṃ jānāma

\begin{english}
  Having heard the꜕ Teaching, we kno꜕w this:
\end{english}

%\suttaref{DN 22}%
Jātipi꜕ dukkhā

\begin{english}
  Birth is du꜕kkha,
\end{english}

Jarāpi꜕ dukkhā

\begin{english}
  Ageing is du꜕kkha,
\end{english}

Ma꜕raṇampi꜕ dukkhaṃ

\begin{english}
  And death is du꜕kkha;
\end{english}

So꜓ka-pa꜕rideva-dukkha꜕-domanassu꜕pāyāsā꜓pi꜕ dukkhā

\begin{english}
  So꜓rrow, lamenta꜕tion, pain, grief, and de꜕spair are du꜕kkha;
\end{english}

Appiyehi꜕ sa꜓mpa꜕yogo dukkho

\begin{english}
  Associ꜕ation with the꜕ di꜕sliked is du꜕kkha;
\end{english}

Piyehi꜕ vi꜓ppa꜕yogo dukkho

\begin{english}
  Sepa꜓ration from the꜕ liked is du꜕kkha;
\end{english}

Yampiccha꜓ṃ na꜕ labhati tampi꜕ dukkhaṃ

\begin{english}
  Not attaining one's wi꜓shes is du꜕kkha.
\end{english}

Sa꜓ṅkhittena pañcu꜕pādānakkha꜓ndhā dukkhā

\begin{english}
  In brief, the꜕ five focuses of iden꜓tity are du꜕kkha.
\end{english}

Seyya꜕thīdaṃ

\begin{english}
  These are as fo꜕llows:
\end{english}

Rūpūpādāna꜕kkha꜓ndho

\begin{english}
  Attachment t꜕o form,
\end{english}

Vedanūpādāna꜕kkha꜓ndho

\begin{english}
  Attachment to fe꜓eling,
\end{english}

Sa꜓ññūpādāna꜕kkha꜓ndho

\begin{english}
  Attachment to perce꜕ption,
\end{english}

Sa꜓ṅkhā꜓rūpādāna꜕kkha꜓ndho

\begin{english}
  Attachment to mental formati꜓ons,
\end{english}

Viññāṇūpādāna꜕kkha꜓ndho

\begin{english}
  Attachment to꜕ sense-co꜓nsciousness.
\end{english}

%\suttaref{Trad.}%
Yesaṃ pa꜕riññāya

\begin{english}
  For the꜕ co꜕mplete understa꜓nding of this,
\end{english}

Dha꜕ramāno so꜓ bha꜕gavā evaṃ ba꜕hulaṃ sā꜓va꜕ke vi꜕neti

\begin{english}
  The꜕ Blessed One in꜕ his li꜓fetime frequently i꜕nstructed his disci꜓ples \\in just thi꜕s way.
\end{english}

Evaṃ bhāgā ca꜕ panassa bha꜕gava꜕to sā꜓va꜕kesu a꜕nusā꜓sa꜕nī ba꜕hulā pa꜕vatta꜕ti

\begin{english}
  In addition, he fu꜕rthe꜕r i꜕nstru꜕cted:
\end{english}

%\suttaref{SN 22.90}%

Rūpaṃ a꜕niccaṃ

\begin{english}
  Form is impe꜕rmanent,
\end{english}

Vedanā a꜕niccā

\begin{english}
  Feeling is impe꜕rmanent,
\end{english}

Sa꜓ññā a꜕niccā

\begin{english}
  Perce꜓ption is impe꜕rmanent,
\end{english}

Sa꜓ṅkhā꜓rā a꜕niccā

\begin{english}
  Mental formations are impe꜕rmanent,
\end{english}

Viññāṇaṃ a꜕niccaṃ

\begin{english}
  Sense-co꜓nsciousness is impe꜕rmanent;
\end{english}

Rūpaṃ a꜕nattā

\begin{english}
  Form is no꜕t-self,
\end{english}

Vedanā a꜕nattā

\begin{english}
  Feeling is no꜕t-self,
\end{english}

Sa꜓ññā a꜕nattā

\begin{english}
  Perce꜓ption is no꜕t-self,
\end{english}

Sa꜓ṅkhā꜓rā a꜕nattā

\begin{english}
  Mental formations are no꜕t-self,
\end{english}

Viññāṇaṃ a꜕nattā

\begin{english}
  Sense-co꜓nsciousness is no꜕t-self;
\end{english}

Sa꜕bbe sa꜓ṅkhā꜓rā a꜕niccā

\begin{english}
  All conditions are t꜕ransient,
\end{english}

Sa꜕bbe dhammā a꜕nattā'ti

\begin{english}
  There is no꜕ self in the cre꜕ated or the u꜕ncre꜓ated.
\end{english}

%\suttaref{MN 29}%
Te ma꜓yaṃ otiṇṇāmha-jāti꜕yā ja꜕rāmaraṇena

\begin{english}
  All of us are bound b꜕y birth, ageing, a꜕nd death,
\end{english}

So꜓kehi꜕ pa꜕ridevehi꜕ dukkhe꜓hi꜕ domanassehi꜕ u꜕pāyāsehi

\begin{english}
  By so꜓rrow, lamenta꜕tion, pain, grief, and de꜕spair,
\end{english}

Dukkho꜓tiṇṇā dukkha꜕pa꜕retā

\begin{english}
  Bo꜓und by dukkha and obstru꜕cte꜕d b꜕y du꜕kkha.
\end{english}

Appevanāmi꜓massa꜕ kevalassa꜕ dukkhakkha꜓ndhassa꜕ anta꜕kiri꜓yā \\paññāyethā'ti

\begin{english}
  Let us all a꜕spire to co꜕mplete fre꜓edom from su꜕ffering.
\end{english}

\begin{instruction}
  The following is chanted only by the monks and nuns.
\end{instruction}

%\suttaref{Trad.}%
Ci꜓ra꜓pari꜕nibbutampi꜓ taṃ bha꜕gava꜓ntaṃ uddissa a꜕raha꜓ntaṃ sammāsambuddhaṃ

\begin{english}
  Remembering the Ble꜕ssed One, the꜕ Noble Lord, a꜕nd \\Perfectly Enli꜓ghtened One, who long ago attained Pa꜕ri꜕ni꜕bbā꜕na,
\end{english}

Saddhā a꜕gārasmā anagāri꜓yaṃ pabba꜕ji꜕tā

\begin{english}
  We have gone forth wi꜕th faith from home to ho꜓melessness,
\end{english}

Tasmi꜓ṃ bha꜕gavati brahma-ca꜕ri꜓yaṃ ca꜕rāma

\begin{english}
  And like the Ble꜕ssed One, we practise the Ho꜓ly꜕ Life,
\end{english}

% NOTE: mid-line Pali caps for alternative forms
Bhikkhū꜓naṃ/Sīladharī꜓naṃ si꜓kkhāsā꜕jīva꜕-samāpannā

\begin{english}
  Being fully e꜕quipped with the꜕ bhikkhus'/nuns' sy꜓stem of tra꜕ining.
\end{english}

Taṃ no brahma-ca꜕ri꜓yaṃ imassa꜕ kevalassa꜕ dukkhakkha꜓ndhassa꜕ anta꜕kiri꜓yāya sa꜓ṃva꜓tta꜕tu

\begin{english}
  May this Ho꜕ly Life lead us to the꜕ end of this who꜓le mass\\ of su꜕ffering.
\end{english}

\clearpage

\begin{instruction}
  An alternative version of the preceding section, which can be chanted by laypeople as well.
\end{instruction}

Ci꜓ra꜓pari꜕nibbutampi꜓ taṃ bha꜕gava꜓ntaṃ saraṇaṃ ga꜕tā

\begin{english}
  The Ble꜕ssed One, who long ago attained Parinibbā꜓na, is our re꜕fuge.
\end{english}

Dha꜓mmañca sa꜓ṅghañca

\begin{english}
  So too are the Dha꜓mma and the Sa꜕ngha.
\end{english}

Tassa bha꜕gavato sā꜓sanaṃ yathā꜓sati yathā꜓balaṃ manasika꜕roma a꜕nupaṭipa꜓jjāma

\begin{english}
  Attentively we fo꜓llow the pathway of that Ble꜕ssed One, with all of \\our mi꜓ndfulness a꜕nd strength.
\end{english}

Sā꜓ sā꜓ no pa꜕ṭi꜓patti

\begin{english}
  May then the cultiva꜓tion of this pra꜕ctice
\end{english}

Imassa꜕ kevalassa꜕ dukkhakkha꜓ndhassa꜕ anta꜕kiri꜓yāya sa꜓ṃva꜓tta꜕tu

\begin{english}
  Lead us to the꜕ end of eve꜓ry kind of su꜕ffering.
\end{english}

%}}}1
\clearpage

\chapter{Closing Homage}                            % {{{1

%\suttaref{Trad.}%
[Arahaṃ] sammāsambuddho bha꜕gavā

\begin{english}
  The꜕ Lord, the꜕ Perfectly Enli꜓ghtened and Ble꜕ssed One -
\end{english}

Buddhaṃ bha꜕gavantaṃ a꜕bhi꜓vādemi

\begin{english}
  I꜕ render homage to꜕ the Bu꜓ddha, the Ble꜕ssed One.
  \instr{Bow}
\end{english}

[Svākkhā꜓to] bha꜕gava꜓tā dhammo

\begin{english}
  The꜕ Teaching, so co꜕mpletely expla꜓ined by him ---
\end{english}

Dhammaṃ namassāmi

\begin{english}
  I bo꜖w to꜕ the꜕ Dha꜕mma.
  \instr{Bow}
\end{english}


[Supaṭi꜕panno] bha꜕gava꜕to sāvaka꜕saṅgho

\begin{english}
  The꜕ Blessed One's disci꜓ples, who have pra꜕ctised well ---
\end{english}

Sa꜓ṅghaṃ na꜕māmi

\begin{english}
  I bo꜖w to꜕ the꜕ Sa꜕ngha.
  \instr{Bow}
\end{english}

%}}}1

% End of morning-chanting.tex
