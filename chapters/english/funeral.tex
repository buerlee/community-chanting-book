% vim: foldmethod=marker foldlevel=0 foldtext=FoldText()

\chapter{Dhamma-saṅgaṇī-mātikā}% {{{1

\firstline{Kusalā dhammā}

\begin{paritta}
Kusalā dhammā\\
Akusalā dhammā\\
Abyākatā dhammā

Sukhāya vedanāya sampayuttā dhammā\\
Dukkhāya vedanāya sampayuttā dhammā\\
Adukkhamasukhāya vedanāya sampayuttā dhammā

Vipākā dhammā\\
Vipāka-dhamma-dhammā\\
N'eva vipāka na vipāka-dhamma-dhammā

Upādinn'upādāniyā dhammā\\
Anupādinn'upādāniyā dhammā\\
Anupādinnānupādāniyā dhammā

Saṅkiliṭṭha-saṅkilesikā dhammā\\
Asaṅkiliṭṭha-saṅkilesikā dhammā\\
Asaṅkiliṭṭhāsaṅkilesikā dhammā

Savitakka-savicārā dhammā\\
Avitakka-vicāra-mattā dhammā\\
Avitakkāvicārā dhammā

Pīti-saha-gatā dhammā\\
Sukha-saha-gatā dhammā\\
Upekkhā-saha-gatā dhammā

\enlargethispage{\baselineskip}

Dassanena pahātabbā dhammā\\
Bhāvanāya pahātabbā dhammā\\
N'eva dassanena na bhāvanāya pahātabbā dhammā

\clearpage

Dassanena pahātabba-hetukā dhammā\\
Bhāvanāya pahātabba-hetukā dhammā\\
N'eva dassanena na bhāvanāya pahātabba-hetukā dhammā

Ācaya-gāmino dhammā\\
Apacaya-gāmino dhammā\\
N'evācaya-gāmino nāpacaya-gāmino dhammā

Sekkhā dhammā\\
Asekkhā dhammā\\
N'eva sekkhā nāsekkhā dhammā

Parittā dhammā\\
Mahaggatā dhammā\\
Appamāṇā dhammā

Paritt'ārammaṇā dhammā\\
Mahaggat'ārammaṇā dhammā\\
Appamāṇ'ārammaṇā dhammā

Hīnā dhammā\\
Majjhimā dhammā\\
Paṇītā dhammā

Micchatta-niyatā dhammā\\
Sammatta-niyatā dhammā\\
Aniyatā dhammā

Magg'ārammaṇā dhammā\\
Magga-hetukā dhammā\\
Maggādhipatino dhammā

\clearpage

Uppannā dhammā\\
Anuppannā dhammā\\
Uppādino dhammā

Atītā dhammā\\
Anāgatā dhammā\\
Paccuppannā dhammā

Atīt'ārammaṇā dhammā\\
Anāgat'ārammaṇā dhammā\\
Paccuppann'ārammaṇā dhammā

Ajjhattā dhammā\\
Bahiddhā dhammā\\
Ajjhatta-bahiddhā dhammā

Ajjhatt'ārammaṇā dhammā\\
Bahiddh'ārammaṇā dhammā\\
Ajjhatta-bahiddh'ārammaṇā dhammā

Sanidassana-sappaṭighā dhammā\\
Anidassana-sappaṭighā dhammā\\
Anidassanāppaṭighā dhammā
\end{paritta}

\chapter{Paṭṭhāna-mātikā-pāṭho}% {{{1

\firstline{Hetu-paccayo}

\begin{paritta}
Hetu-paccayo, ārammaṇa-paccayo, adhipati-paccayo, anantara-paccayo,
samanantara-paccayo, saha-jāta-paccayo, aññam-añña-paccayo,
nissaya-paccayo, upanissaya-paccayo, pure-jāta-paccayo,
pacchā-jāta-paccayo, āsevana-paccayo, kamma-paccayo, vipāka-paccayo,
āhāra-paccayo, indriya-paccayo, jhāna-paccayo, magga-paccayo,
sampayutta-paccayo, vippayutta-paccayo, atthi-paccayo, n'atthi-paccayo,
vigata-paccayo, avigata-paccayo.
\end{paritta}

\chapter{Vipassanā-bhūmi-pāṭho}% {{{1

\firstline{Pañcakkhandhā rūpakkhandho}

\begin{paritta}

Pañcakkhandhā:

Rūpakkhandho, vedanākkhandho, saññākkhandho, saṅkhārakkhandho,
viññāṇakkhandho

Dvā-das'āyatanāni:

Cakkhv-āyatanaṃ rūp'āyatanaṃ, sot'āyatanaṃ sadd'āyatanaṃ, ghān'āyatanaṃ
gandh'āyatanaṃ, jivh'āyatanaṃ ras'āyatanaṃ, kāy'āyatanaṃ
phoṭṭhabb'āyatanaṃ, man'āyatanaṃ dhamm'āyatanaṃ

Aṭṭhārasa dhātuyo:

Cakkhu-dhātu rūpa-dhātu cakkhu-viññāṇa-dhātu, sota-dhātu sadda-dhātu
sota-viññāṇa-dhātu, ghāna-dhātu gandha-dhātu ghāna-viññāṇa-dhātu,
jivhā-dhātu rasa-dhātu jivhā-viññāṇa-dhātu, kāya-dhātu phoṭṭhabba-dhātu
kāya-viññāṇa-dhātu, mano-dhātu dhamma-dhātu mano-viññāṇa-dhātu

Bā-vīsat'indriyāni:

Cakkhu'ndriyaṃ sot'indriyaṃ ghān'indriyaṃ jivh'indriyaṃ kāy'indriyaṃ
man'indriyaṃ, itth'indriyaṃ puris'indriyaṃ jīvit'indriyaṃ, sukh'indriyaṃ
dukkh'indriyaṃ somanass'indriyaṃ domanass'indriyaṃ upekkh'indriyaṃ,
saddh'indriyaṃ viriy'indriyaṃ sat'indriyaṃ samādh'indriyaṃ
paññ'indriyaṃ, anaññātañ-ñassāmī-t'indriyaṃ aññ'indriyaṃ
aññātāv'indriyaṃ

Cattāri ariya-saccāni:

Dukkhaṃ ariya-saccaṃ, dukkha-samudayo ariya-saccaṃ, dukkha-nirodho
ariya-saccaṃ, dukkha-nirodha-gāminī paṭipadā ariya-saccaṃ

Avijjā-paccayā saṅkhārā, saṅkhāra-paccayā viññāṇaṃ, viññāṇa-paccayā
nāma-rūpaṃ, nāma-rūpa-paccayā saḷ-āyatanaṃ, saḷ-āyatana-paccayā phasso,
phassa-paccayā vedanā, vedanā-paccayā taṇhā, taṇhā-paccayā upādānaṃ,
upādāna-paccayā bhavo, bhava-paccayā jāti, jāti-paccayā jarā-maraṇaṃ
soka-parideva-dukkha-domanass'upāyāsā sambhavanti

Evam-etassa kevalassa dukkhakkhandhassa samudayo hoti

Avijjāya tv-eva asesa-virāga-nirodhā saṅkhāra-nirodho, saṅkhāra-nirodhā
viññāṇa-nirodho, viññāṇa-nirodhā nāma-rūpa-nirodho, nāma-rūpa-nirodhā
saḷ-āyatana-nirodho, saḷ-āyatana-nirodhā phassa-nirodho, phassa-nirodhā
vedanā-nirodho, vedanā-nirodhā taṇhā-nirodho, taṇhā-nirodhā
upādāna-nirodho, upādāna-nirodhā bhava-nirodho, bhava-nirodhā
jāti-nirodho, jāti-nirodhā jarā-maraṇaṃ
soka-parideva-dukkha-domanass'upāyāsā nirujjhanti

Evam-etassa kevalassa dukkhakkhandhassa nirodho hoti

\end{paritta}

\clearpage

\chapter{Paṃsu-kūla for the dead}% {{{1

\firstline{Aniccā vata saṅkhārā}

\begin{paritta}
Aniccā vata saṅkhārā\\
Uppāda-vaya-dhammino\\
Uppajjitvā nirujjhanti\\
Tesaṃ vūpasamo sukho

Sabbe sattā maranti ca\\
Mariṃsu ca marissare\\
Tath'evāhaṃ marissāmi\\
N'atthi me ettha saṃsayo

% NOTE: References are commented out, but keep them in the source for information.
%\suttaref{(D.II.157; S.I.6)}

\end{paritta}

\chapter{Paṃsu-kūla for the living}% {{{1

\firstline{Aciraṃ vat'ayaṃ kāyo}

\begin{paritta}
Aciraṃ vat'ayaṃ kāyo\\
Paṭhaviṃ adhisessati\\
Chuddho apeta-viññāṇo\\
Niratthaṃ va kaliṅgaraṃ

% NOTE: References are commented out, but keep them in the source for information.
%\suttaref{(Dhp.v.41)}

\end{paritta}

