\chapter{Pāli Phonetics and Pronunciation}

Pāli is the original scriptural language of Theravāda Buddhism. It was a spoken
language, closely related to Sanskrit, with no written script of its own. As
written forms have emerged, they have been in the letterings of other languages
(e.g. Devanagari, Sinhalese, Burmese, Khmer, Thai, Roman). The Roman lettering
used here is pronounced as in English, with the following clarifications:

\section{Vowels}

\begin{tabular}{@{} ll @{}}

\begin{tabular}{@{} ll @{}}
  Short & Long\\
  \textbf{a} as in \prul{a}bout &
  \textbf{ā} as in f\prul{a}ther\\
  \textbf{i} as in h\prul{i}t &
  \textbf{ī} as in mach\prul{i}ne\\
  \textbf{u} as in p\prul{u}t &
  \textbf{ū} as in r\prul{u}le\\
  & \textbf{e} as in gr\prul{e}y\\
  & \textbf{o} as in m\prul{o}re\\
\end{tabular} &

\parbox[b][][t]{0.5\linewidth}{%
Exceptions: \textbf{e} and \textbf{o} change to short sounds in syllables
ending in consonants. They are then pronounced as in `g\prul{e}t' and
`\prul{o}x', respectively.} \\

\end{tabular}

\section{Consonants}

\textbf{c} as in an\prul{c}ient (like \prul{ch} but unaspirated)

\textbf{ṃ, ṅ} as \prul{ng} in sa\prul{ng}

\textbf{ñ} as \prul{ny} in ca\prul{ny}on

\textbf{v} rather softer than the English \prul{v}; near \prul{w}

\subsection{Aspirated consonants}

\textbf{bh ch dh ḍh gh jh kh ph th ṭh}

These two-lettered notations with \prul{h} denote an aspirated, airy sound,
distinct from the hard, crisp sound of the single consonant. They should be
considered as one unit.

However, the other combinations with \textbf{h,} i.e., \textbf{lh, mh, ñh,} and
\textbf{vh,} do count as two consonants (for example in the Pāli words
‘ji\textbf{vh}ā’ or ‘mu\textbf{ḷh}o’).

\subsection{Examples}

\textbf{th} as \prul{t} in \prul{t}ongue. (Never pronounced as in `\prul{th}e'.)

\textbf{ph} as \prul{p} in \prul{p}alate. (Never pronounced as in `\prul{ph}oto'.)

These are distinct from the hard, crisp sound of the single consonant, e.g.
\textbf{th} as in `\prul{Th}omas' (not as in `\prul{th}in') or \textbf{ph} as
in `\prul{p}uff' (not as in `\prul{ph}one').

\subsection{Retroflex consonants}

\textbf{ḍ ḍh ḷ ṇ ṭ ṭh}

These retroflex consonants have no English equivalents. They are sounded
by curling the tip of the tongue back against the palate.

\section{Chanting technique}

Once you have grasped the system of Pāli pronunciation and the following
chanting technique, it allows you to chant a text in Pāli from sight
with the correct rhythm.

\textbf{Unstressed syllables} end in a short \textbf{a, i} or
\textbf{u}. All other syllables are stressed. Stressed syllables take
twice the time of unstressed syllables --- rather like two beats in a bar
of music compared to one. This is what gives the chanting its particular
rhythm.

\begin{centering}

{\setlength{\tabcolsep}{1.8pt}%
\begin{tabular}{ccc c ccccc c ccccc c ccccccc}
BUD & · & DHO & \hsp & SU & · & SUD & · & DHO & \hsp & KA & · & RU & · & ṆĀ & \hsp & MA & · & HAṆ & · & ṆA & · & VO\\
 1  &   & 1   &      & ½  &   & 1   &   & 1   &      & ½  &   & ½  &   & 1  &      & ½  &   & 1   &   & ½  &   & 1\\
\end{tabular}%
}

\end{centering}

Two details that are important when separating the syllables:

\textbf{1.} Syllables with double letters get divided in this way:

\begin{centering}

\begin{minipage}{0.8\linewidth}
\begin{multicols}{2}
\setlength{\tabcolsep}{1.8pt}%

\begin{tabular}{rrcccl}
     & A & · & NIC & · & CA   \\
     & ½ &   &  1  &   & ½    \\
(not & A & · & NI  & · & CCA) \\
     & ½ &   & ½   &   & ½    \\
\end{tabular}

\columnbreak

\begin{tabular}{rrcccl}
     & PUG & · & GA  & · & LĀ \\
     &  1  &   &  ½  &   &  1 \\
(not & PU  & · & GGA & · & LĀ)\\
     &  ½  &   &  ½  &   &  1 \\
\end{tabular}

\end{multicols}
\end{minipage}

\end{centering}

They are always enunciated separately, e.g. \textbf{dd} in ‘uddeso’ as
in ‘mad dog’, or \textbf{gg} in ‘maggo’ as in ‘big gun’.

\textbf{2.} \textbf{Aspirated consonants} like \textbf{bh, dh} etc.
count as single consonant and don't get divided (Therefore
\textbf{am·hā·kaṃ}, but \textbf{sa·dham·maṃ}, not \textbf{sad·ham·maṃ}
or, another example: \textbf{Bud·dho} and not \textbf{Bu·ddho}).

Precise pronunciation and correct separation of the syllables is
especially important when someone is interested in learning Pāli and to
understand and memorize the meaning of Suttas and other chants,
otherwise the meaning of it will get distorted.

\textbf{An example to illustrate this:}

The Pāli word ‘\textbf{sukka}’ means ‘bright’; ‘\textbf{sukkha}’ means
‘dry’; ‘\textbf{sukha}’ --- ‘happiness’; ‘\textbf{suka}’ --- ‘parrot’ and
‘\textbf{sūka}’ --- ‘bristles on an ear of barley’.

So if you chant ‘\textbf{sukha}’ with a ‘\textbf{k}’ instead of a
‘\textbf{kh}’, you would chant ‘parrot’ instead of ‘happiness’.

A general rule of thumb for understanding the practice of chanting is to
listen carefully to what the leader and the group are chanting and to
follow, keeping the same pitch, tempo and speed. All voices should blend
together as one.

\section{Punctuation, tonal marks and pauses in this edition}

[Square brackets] indicate parts usually chanted only by the leader, but
chanting customs differ in the various monasteries.

The slash / indicates variations of male of female forms according to
the person chanting them, or singular and plural forms when chanting
alone or in a group.

The cantillation marks indicate changes in pitch, usually a full tone up or down:

\begin{tabular}{llll}
High tone: & no꜓ble & Long low tone: & ho꜖mage\\
Low tone: & ble꜕ssed & Long mid tone: & \prul{guides}\\
\end{tabular}

\section{A note on hyphenation in the text}

As an aid to understanding, some of the longer Pāli words in the text have been
hyphenated into the words from which they are compounded. This does not affect
the pronunciation in any way.

% In order to not suggest unintended pauses in the flow of the chanting, we have
% omitted all punctuation marks (commas, periods, colon and semicolon), although
% for rendering the meaning of the phrases accurately, they would be required.
% The line breaks indicate that a short breathing pause is inserted.

% End of pali-pronunciation.tex
