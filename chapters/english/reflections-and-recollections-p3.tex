% vim: foldmethod=marker foldlevel=0 foldtext=FoldText()

\chapter[The Noble Eightfold Path]{The Teaching on the Noble Eightfold Path}% {{{1

\begin{leader}
  [Handa mayaṃ ariyaṭṭhaṅgika-magga-pāṭham bhaṇāmase]
\end{leader}

Ayam-eva a꜕riyo aṭṭha꜓ṅgi꜕ko maggo

\begin{english}
  This is the No꜕bl꜕e E꜕ightfo꜕ld Path,
\end{english}

\begin{twochants}

Se꜓yyathī꜓daṃ &
\tr{Which is as fo꜕llows:} \\

Sa꜓mmā-diṭṭhi &
\tr{Ri꜕ght View,} \\

Sa꜓mmā-sa꜓ṅka꜕ppo &
\tr{Right Inte꜕ntion} \\

Sa꜓mmā-vācā &
\tr{Ri꜕ght Speech,} \\

Sa꜓mmā-kammanto &
\tr{Right A꜕ction,} \\

Sa꜓mmā-ājīvo &
\tr{Right Li꜓vel꜕ihood,} \\

Sa꜓mmā-vā꜕yāmo &
\tr{Right E꜕ffort,} \\

Sa꜓mmā-sa꜕ti &
\tr{Right Mi꜓ndfu꜕lness,} \\

Sa꜓mmā-sa꜕mādhi &
\tr{Ri꜕ght Co꜕nce꜕ntr꜕ation.} \\

\end{twochants}

Ka꜕tamā ca bhi꜓kkh꜕ave sammā-diṭṭhi

\begin{english}
  And what bhikkhus i꜕s Ri꜕ght View?
\end{english}

\begin{tabular}{@{}p{0.5\linewidth} p{0.5\linewidth}@{}}

Yaṃ kho bhi꜓kkh꜕ave dukkhe ñāṇaṃ &
\tr{Knowledge of su꜕ffering;} \\

Dukkha-sa꜕mu꜕daye ñāṇaṃ &
\tr{Knowledge of the o꜓rigin of su꜕ffering;} \\

Dukkha-ni꜓rodhe ñāṇaṃ &
\tr{Knowledge of the cessa꜕tio꜕n o꜕f su꜕ffe꜕ring;} \\

Dukkha-ni꜓rodha-gāmi꜓ni꜓yā\newline pa꜕ṭipa꜕dāya ñāṇaṃ &
\tr{Knowledge of th꜓e path\newline Leading to the cess꜕ati꜕on o꜕f su꜕ffering:} \\

\end{tabular}

A꜕yaṃ vuccati bhi꜓kkh꜕ave sa꜓mmā-diṭṭhi

\begin{english}
  This bhikkhus is ca꜕lled Ri꜕ght View.
\end{english}

Katamo ca bhi꜓kkh꜕ave sammā-sa꜓ṅka꜕ppo

\begin{english}
  And what bhikkhus is Ri꜕ght I꜕nte꜕ntion?
\end{english}

\begin{twochants}

Nekkhamma-sa꜓ṅka꜕ppo &
\tr{The intention of renu꜕nc꜕ia꜕tion;} \\

A꜕byāpāda-sa꜓ṅka꜕ppo &
\tr{The intention of no꜕n-il꜕l-will;} \\

A꜕vihiṃsā-sa꜓ṅka꜕ppo &
\tr{The intention of non-cru꜓e꜕lty:} \\

\end{twochants}

Ayaṃ vuccati bhi꜓kkh꜕ave sa꜓mmā-sa꜓ṅka꜕ppo

\begin{english}
  This bhikkhus is ca꜕lled Ri꜕ght I꜕nte꜕ntion.
\end{english}

Katamā ca bhi꜓kkh꜕ave sa꜓mmā-vācā

\begin{english}
  And what bhikkhus i꜕s Ri꜕ght Speech?
\end{english}

\begin{twochants}

Musā-vādā vera꜓ma꜕ṇī &
\tr{Abstaining fro꜕m fa꜕lse speech;} \\

Pisuṇāya vācāya vera꜓ma꜕ṇī &
\tr{Abstaini꜓ng from mali꜓cio꜕us speech;} \\

Pharusāya vācāya vera꜓ma꜕ṇī &
\tr{Abstaining fro꜕m ha꜕rsh speech;} \\

Sa꜓mphappa꜕lāpā vera꜓ma꜕ṇī. &
\tr{Abstaining from i꜕dl꜕e cha꜕tter:} \\

\end{twochants}

Ayaṃ vuccati bhi꜓kkh꜕ave sa꜓mmā-vācā

\begin{english}
  This bhikkhus is ca꜕lled Ri꜕ght Speech.
\end{english}

Katamo ca bhi꜓kkh꜕ave sa꜓mmā-kammanto

\begin{english}
  And what bhikkhus i꜕s Ri꜕ght A꜕ction?
\end{english}

\begin{tabular}{@{}p{0.4\linewidth} p{0.6\linewidth}@{}}

Pāṇāti꜕pātā vera꜓ma꜕ṇī &
\tr{Abstaini꜓ng from ki꜕lli꜕ng li꜕vi꜕ng be꜕ings;} \\

A꜕dinnādānā vera꜓ma꜕ṇī &
\tr{Abstaini꜓ng from ta꜕ki꜕ng wh꜕at i꜕s no꜕t gi꜕ven;} \\

Kāmesu꜕-micchā꜓cārā vera꜓ma꜕ṇī &
\tr{Abstaini꜓ng from se꜕xu꜕al mi꜓sco꜕nduct:} \\

\end{tabular}

Ayaṃ vuccati bhi꜓kkh꜕ave sa꜓mmā-kammanto

\begin{english}
  This bhikkhus is ca꜕lled Ri꜕ght Ac꜕tion.
\end{english}

Katamo ca bhi꜓kkha꜕ve sa꜓mmā-ājīvo

\begin{english}
  And what bhikkhus is Right L꜓ivel꜕ihood?
\end{english}

\begin{tabular}{@{}p{0.45\linewidth} p{0.55\linewidth}@{}}

Idha bhi꜓kkh꜕ave a꜕riya-sā꜓va꜕ko\newline
Micchā-ājīvaṃ pa꜕hāya\newline
Sammā-ājī꜓vena jīvi꜓taṃ ka꜕ppeti &

\tr{Here, bhikkhus, a Nobl꜕e Di꜕sc꜕iple,\newline
Having a꜓bandoned wrong li꜓vel꜕ihood,\newline
Earns h꜓is living by ri꜕ght li꜕vel꜕ihood:} \\

\end{tabular}

Ayaṃ vuccati bhi꜓kkh꜕ave sa꜓mmā-ājīvo

\begin{english}
  This bhikkhus is ca꜕lled Ri꜕ght Li꜕vel꜕ihood.
\end{english}

Katamo ca bhi꜓kkh꜕ave sa꜓mmā-vāyāmo

\begin{english}
  And what bhikkhus i꜕s Ri꜕ght E꜕ffort?
\end{english}

Idha bhi꜓kkh꜕ave bhikkhu a꜕nuppannānaṃ pāpa꜕kānaṃ a꜕ku꜕salānaṃ dhammānaṃ anuppādāya

\begin{english}
  Here, bhikkhus, a꜓ bhikkhu awa꜕ke꜕ns zeal for the non-a꜓rising of unari꜕sen, evil unwho꜓leso꜕me states;
\end{english}

Chandaṃ ja꜕neti vāyama꜓ti vī꜓ri꜓yaṃ ārabha꜕ti ci꜕ttaṃ pa꜕ggaṇhā꜓ti pa꜕daha꜕ti

\begin{english}
  He puts forth e꜕ffort, arouses e꜓ne꜕rgy, exerts h꜓is mind an꜕d strives.
\end{english}

U꜕ppannānaṃ pāpa꜕kānaṃ a꜕ku꜕salānaṃ dhammānaṃ pa꜕hānāya

\begin{english}
  He awake꜓ns zeal for the aba꜕ndoning of a꜓risen, evil unwho꜓leso꜕me states;
\end{english}

Chandaṃ ja꜕neti vāyama꜓ti vī꜓ri꜓yaṃ ārabha꜕ti ci꜕ttaṃ pa꜕ggaṇhā꜓ti pa꜕daha꜕ti

\begin{english}
  He puts forth e꜕ffort, arouses e꜓ne꜕rgy, exerts h꜓is mind an꜕d strives.
\end{english}

Anuppannānaṃ ku꜕salānaṃ dhammānaṃ u꜕ppādāya

\begin{english}
  He awake꜓ns zeal for the ari꜕sing of una꜓risen who꜓leso꜕me states;
\end{english}

Chandaṃ ja꜕neti vāyama꜓ti vī꜓ri꜓yaṃ ārabha꜕ti ci꜕ttaṃ pa꜕ggaṇhā꜓ti pa꜕daha꜕ti

\begin{english}
  He puts forth e꜕ffort, arouses e꜓ne꜕rgy, exerts h꜓is mind an꜕d strives.
\end{english}

U꜕ppannānaṃ ku꜕salānaṃ dhammānaṃ ṭh꜓iti꜕yā a꜕sa꜕mmosāya bh꜓iyyobhāvāya vepu꜕llāya bhāva꜓nāya pāri꜓pū꜕riyā

\begin{english}
  He awakens zeal for the conti꜕nuance, non-disa꜓ppearance, stre꜕ngthening, increase and f꜓ulfilment by deve꜓lo꜕pment of ari꜕sen who꜕leso꜕me states;
\end{english}

Chandaṃ ja꜕neti vāyama꜓ti vī꜓ri꜓yaṃ ārabha꜕ti ci꜕ttaṃ pa꜕ggaṇhā꜓ti pa꜕daha꜕ti

\begin{english}
  He puts forth e꜕ffort, arouses e꜓ne꜕rgy, exerts h꜓is mind an꜕d strives:
\end{english}

Ayaṃ vuccati bhi꜓kkh꜕ave sa꜓mmā-vāyāmo

\begin{english}
  This bhikkhus is ca꜕lled Ri꜕ght E꜕ffort.
\end{english}

Katamā ca bhi꜓kkh꜕ave sa꜓mmā-sa꜕ti

\begin{english}
  And what bhikkhus is Right Mi꜓ndfu꜕lness?
\end{english}

Idha bhi꜓kkh꜕ave bhikkhu kāye kāyānupa꜕ssī vi꜓ha꜕rati

\begin{english}
  Here, bhikkhus, a bhi꜕kkhu꜕ a꜕bides conte꜓mplating the bo꜕dy a꜕s a꜕ bo꜕dy,
\end{english}

Ātāpī sa꜓mpa꜕jāno sa꜕timā

\begin{english}
  Ardent, fully꜓ a꜕ware and mi꜕ndful,
\end{english}

Vi꜓neyya loke a꜕bhijjhā-domanassaṃ

\begin{english}
  Having pu꜕t a꜕way co꜕ve꜕to꜕usn꜕ess an꜕d gri꜕ef fo꜕r th꜕e world;
\end{english}

Veda꜕nāsu꜕ veda꜕nānu꜓pa꜕ssī vi꜓ha꜕rati

\begin{english}
  He a꜕bides conte꜓mplating fe꜕eli꜕ngs a꜕s fe꜕elings,
\end{english}

Ātāpī sa꜓mpa꜕jāno sa꜕timā

\begin{english}
  Ardent, fully꜓ a꜕ware and mi꜕ndful,
\end{english}

Vi꜓neyya loke a꜕bhijjhā-domanassaṃ

\begin{english}
  Having pu꜕t a꜕way co꜕ve꜕to꜕usn꜕ess an꜕d gri꜕ef fo꜕r th꜕e world;
\end{english}

Ci꜕tte ci꜕ttānu꜓pa꜕ssī vi꜓ha꜕rati

\begin{english}
  He a꜕bides conte꜓mplating mi꜕nd a꜕s mind,
\end{english}

Ātāpī sa꜓mpa꜕jāno sa꜕timā

\begin{english}
  Ardent, fully꜓ a꜕ware and mi꜕ndful,
\end{english}

Vi꜓neyya loke a꜕bhijjhā-domanassaṃ

\begin{english}
  Having pu꜕t a꜕way co꜕ve꜕to꜕usne꜕ss an꜕d gri꜕ef fo꜕r th꜕e world.
\end{english}

Dhammesu꜕ dhammānu꜓pa꜕ssī vi꜓ha꜕rati

\begin{english}
  He a꜕bides conte꜓mplating mind-o꜕bje꜕cts a꜕s mi꜕nd-o꜕bjects,
\end{english}

Ātāpī sa꜓mpa꜕jāno sa꜕timā

\begin{english}
  Arden,t fully꜓ a꜕ware and mi꜕ndful,
\end{english}

Vi꜓neyya loke a꜕bhijjhā-domanassaṃ

\begin{english}
  Having pu꜕t a꜕way co꜕ve꜕to꜕usn꜕ess an꜕d gri꜕ef fo꜕r th꜕e world:
\end{english}

Ayaṃ vuccati bhi꜓kkh꜕ave sa꜓mmā-sa꜕ti

\begin{english}
  This bhikkhus is ca꜕lled Ri꜕ght Mi꜕ndfu꜕lness.
\end{english}

Katamo ca bhi꜓kkh꜕ave sa꜓mmā-sa꜕mādhi

\begin{english}
  And what bhikkhus is Ri꜕ght Co꜕nce꜕ntr꜕ation?
\end{english}

\begin{twochants}

Idha bhi꜓kkh꜕ave bhikkhu &
\tr{Here, bhikkhus, a bhi꜕kkhu,} \\

Vivicc'eva kāmehi &
\tr{Quite se꜓cluded from se꜕nsu꜕al pl꜕easures,} \\

Vivicca a꜕ku꜕sa꜕lehi dh꜕ammehi &
\tr{Secluded from unwho꜓leso꜕me states,} \\

\end{twochants}

Sa꜕vi꜓ta꜕kkaṃ sa꜕vi꜓cāraṃ viveka꜕-jaṃ pīti꜕-sukhaṃ pa꜕ṭhamaṃ jhānaṃ upasa꜓mpajja vi꜓ha꜕rati

\begin{english}
  Enters u꜓pon and a꜕bides in꜕ th꜕e fi꜕rst Jhā꜕na --\\
  Accompa꜓nied by appli꜕ed an꜕d su꜕stai꜕ned thought,\\
  With raptu꜓re and ple꜕asure bo꜕rn o꜕f se꜕clu꜕sion.
\end{english}

Vi꜓takka-vicārānaṃ vūpa꜕samā

\begin{english}
  With the stilling of appli꜕ed an꜕d su꜕stai꜕ned thought,
\end{english}

Ajjhattaṃ sa꜓mpa꜕sādanaṃ ceta꜕so ekodi꜓bhāvaṃ avi꜓ta꜕kkaṃ avi꜓cāraṃ sa꜕mādhi꜓-jaṃ pīti꜕-sukhaṃ du꜕tiyaṃ jhānaṃ upasa꜓mpa꜕jja vi꜓ha꜕rati

\begin{english}
  He enters u꜓pon and a꜕bides in꜕ th꜕e se꜕co꜕nd Jhā꜕na --\\
  Accompa꜓nied by self-co꜓nf꜕idence and si꜕ngle꜕ne꜕ss o꜕f mind,\\
  Without applie꜕d an꜕d su꜕stai꜕ned thought,\\
  With raptu꜓re and ple꜕asure bo꜕rn o꜕f co꜕nce꜕ntr꜕ation.
\end{english}

\begin{twochants}

Pītiyā ca꜕ vi꜓rāgā &
\tr{With the fadi꜓ng a꜕way as we꜕ll o꜕f ra꜕pture} \\

U꜕pekkhako ca vi꜓ha꜕rati &
\tr{He abides in equani꜓mi꜕ty,} \\

Sa꜕to ca꜕ sa꜓mpa꜕jāno &
\tr{Mindful and fully꜓ a꜕ware,} \\

Su꜕khañ-ca kāyena pa꜕ṭisa꜓ṃvedeti &
\tr{Still fee꜕li꜕ng ple꜕asu꜕re wi꜕th th꜕e bo꜕dy,} \\

\end{twochants}

Yaṃ taṃ a꜕riyā āci꜕kkhanti `u꜕pekkha꜓ko sa꜕timā su꜕kha-vi꜓hā꜕rī'ti tatiyaṃ jhānaṃ u꜕pasa꜓mpa꜕jja vi꜓ha꜕rati

\begin{english}
  He enters u꜓pon and a꜕bides in꜕ th꜕e thi꜕rd Jh꜕āna --\\
  On account o꜓f which the No꜕bl꜕e O꜓nes a꜕nnounce,\\
  `He has a꜓ pleasant abi꜕ding,\\
  With equani꜓mi꜕ty and is mi꜕ndful.'
\end{english}

\begin{twochants}

Sukhassa ca꜕ pahānā &
\tr{With the aba꜓ndoning of ple꜕asure} \\

Dukkhassa ca꜕ pahānā &
\tr{And the aba꜕ndo꜕ni꜕ng o꜕f pain,} \\

\end{twochants}

\clearpage

Pu꜕bb'eva somanassa꜕ domanassā꜓naṃ a꜕tthaṅga꜕mā

\begin{english}
  With the previous disa꜓ppearance of jo꜕y an꜕d grief,
\end{english}

Adukkham-asu꜕khaṃ u꜕pekkhā-sa꜕ti-pā꜕ri꜓su꜕ddhiṃ ca꜕tutthaṃ jhānaṃ u꜕pasa꜓mpa꜕jja vi꜓ha꜕rati

\begin{english}
  He enters u꜓pon and a꜕bides i꜕n th꜕e fou꜕rth Jh꜕āna --\\
  Accompa꜓nied by neither pa꜕in no꜕r-pl꜕easure,\\
  And purity of mi꜓ndfu꜕lness du꜕e to꜕ e꜕qu꜕an꜕imity:
\end{english}

Ayaṃ vuccati bhi꜓kkh꜕ave sa꜓mmā-sa꜕mādhi

\begin{english}
  This bhikkhus is ca꜕lled Ri꜕ght Co꜕nce꜕ntr꜕ation.
\end{english}

Ayam-eva a꜕riyo aṭṭha꜓ṅgi꜕ko maggo

\begin{english}
  This is the No꜕bl꜕e E꜕ightfo꜕ld Path.
\end{english}

\chapter[The Wheel of Dhamma]{Teachings from the Discourse on Setting in Motion the Wheel of Dhamma}% {{{1

\begin{leader}
  [Ha꜓nda mayaṃ dhamma-cakkappavattana su꜕tta-pāṭhaṃ bha꜕ṇāmase]
\end{leader}

Dve me bhi꜓kkha꜕ve antā

\begin{english}
  Bhikkhus, there are these tw꜕o ex꜕tremes
\end{english}

Pabbaji꜓tena na sevi꜓ta꜕bbā

\begin{english}
  That shou꜕ld no꜕t b꜕e pu꜕rsued by one who ha꜓s go꜕ne forth:
\end{english}

Yo cāyaṃ kāmesu꜕ kāma-su꜕kh'alli꜓kānu꜓yogo

\begin{english}
  That is, whatever is tied u꜕p t꜕o se꜕nse pl꜕easures,\\
  Within the re꜕alm o꜕f se꜕nsu꜕a꜕li꜕ty,
\end{english}

\begin{twochants}

Hīno &
\tr{Whi꜕ch i꜕s low,} \\

Gammo &
\tr{Co꜕mmon,} \\

Pothujj꜓ani꜕ko &
\tr{The way of the co꜕mmo꜕n folks,} \\

Anar꜓iyo &
\tr{Not the wa꜕y o꜕f th꜕e No꜓bl꜕e Ones} \\

Anattha-sa꜓ñh꜕ito &
\tr{And po꜕intless;} \\

\end{twochants}

Yo cāyaṃ atta-kilama꜓thānu꜓yogo

\begin{english}
  Then there is whate꜕ve꜕r i꜕s t꜕ied up\\
  With se꜕lf-de꜕pr꜕iva꜕tion,
\end{english}

\begin{twochants}

Dukkho &
\tr{Which is pa꜕inful,} \\

Anar꜓iyo &
\tr{Not the wa꜕y o꜕f th꜕e No꜓bl꜕e Ones} \\

A꜕nattha꜕-sa꜓ñh꜕ito &
\tr{And po꜕intless.} \\

\end{twochants}

Ete te bhi꜓kkha꜕ve ub꜕ho꜕ ante a꜕nupa꜕gamma majjhi꜓mā pa꜕ṭi꜕pa꜕dā tathā꜓ga꜕tena a꜕bhis꜓ambuddhā

\begin{english}
  Bhikkhus, without go꜕ing t꜕o e꜕ith꜕er o꜕f th꜕ese e꜕xtremes,\\
  The Tathā꜓ga꜕ta has u꜕lt꜕ima꜕tel꜕y a꜕wa꜕kened\\
  To a꜓ middle wa꜕y o꜕f pr꜕actice,
\end{english}

\begin{twochants}

Cakkhu-ka꜕ra꜓ṇī &
\tr{Givi꜓ng rise to v꜕ision,} \\

Ñāṇa-ka꜕ra꜓ṇī &
\tr{Ma꜕ki꜕ng fo꜕r i꜕nsight,} \\

U꜕pasa꜕māya &
\tr{Leadi꜓ng t꜕o calm,} \\

A꜕bhiññāya &
\tr{To he꜕ight꜕ened kn꜕owing,} \\

Sa꜓mbodhāya &
\tr{Awa꜕ke꜕ning} \\

Ni꜓bbānāya sa꜓ṃvat꜕tati &
\tr{An꜕d t꜕o Nibbā꜕na.} \\

\end{twochants}

Katamā ca sā bhi꜓kkh꜕ave majjhi꜕mā p꜕aṭ꜕ip꜕adā

\begin{english}
  And what, bhikkhus, i꜕s tha꜕t mi꜕ddl꜕e wa꜕y o꜕f pra꜕ctice?
\end{english}

Ayam-eva a꜕riyo aṭṭha꜓ṅgi꜕ko maggo

\begin{english}
  It is this No꜕bl꜕e Ei꜕ghtfo꜕ld Path,
\end{english}

\begin{twochants}

Se꜓yyathī꜓daṃ &
\tr{Which is as fo꜕llows:} \\

Sa꜓mmā-diṭṭhi &
\tr{Ri꜕ght View,} \\

Sa꜓mmā-sa꜓ṅka꜕ppo &
\tr{Right Inte꜕ntion} \\

Sa꜓mmā-vācā &
\tr{Ri꜕ght Speech,} \\

Sa꜓mmā-kammanto &
\tr{Right A꜕ction,} \\

Sa꜓mmā-ājīvo &
\tr{Right Li꜓vel꜕ihood,} \\

Sa꜓mmā-vā꜕yāmo &
\tr{Right E꜕ffort,} \\

Sa꜓mmā-sa꜕ti &
\tr{Right Mi꜓ndfu꜕lness,} \\

Sa꜓mmā-sa꜕mādhi &
\tr{Ri꜕ght Co꜕nce꜕ntr꜕ation.} \\

\end{twochants}

Ayaṃ kho sā bhi꜓kkh꜕ave majjh꜕imā p꜕aṭ꜕ip꜕adā tathā꜓ga꜕tena abhisa꜓mbuddhā

\begin{english}
  This, bhikkhus, is the mi꜕ddl꜕e wa꜕y o꜕f pr꜕actice\\
  That the Tathā꜓ga꜕ta has u꜕lti꜕ma꜕tel꜕y a꜕wa꜕ke꜕ned to,
\end{english}

\begin{twochants}

Cakkhu-ka꜕ra꜓ṇī &
\tr{Givi꜓ng rise to v꜕ision,} \\

Ñāṇa-ka꜕ra꜓ṇī &
\tr{Ma꜕ki꜕ng fo꜕r i꜕nsight,} \\

U꜕pasa꜕māya &
\tr{Leadi꜓ng t꜕o calm,} \\

A꜕bhiññāya &
\tr{To he꜕ight꜕ened kn꜕owing,} \\

Sa꜓mbodhāya &
\tr{Awa꜕ke꜕ning} \\

Ni꜓bbānāya sa꜓ṃvat꜕tati &
\tr{An꜕d t꜕o Nibbā꜕na.} \\

\end{twochants}

Idaṃ kho pana bhi꜓kkh꜕ave dukkhaṃ a꜕riya꜓-s꜕accaṃ

\begin{english}
  This bhikkhus is the No꜕ble꜕ Tr꜕uth o꜕f du꜕kkha:
\end{english}

\begin{twochants}

Jātipi꜕ dukkhā &
\tr{Birth is du꜕kkha,} \\

Jarāpi꜕ dukkhā &
\tr{Ageing is du꜕kkha} \\

Maraṇampi꜕ dukkhaṃ &
\tr{And death is du꜕kkha;} \\

\end{twochants}

So꜓ka-pa꜕rideva-dukkha꜕-domanassu꜕pāyāsā꜓pi꜕ dukkhā

\begin{english}
  So꜓rrow lamenta꜕tion pain grief and de꜕spair are du꜕kkha,
\end{english}

Appiyehi꜕ sa꜓mpa꜕yogo dukkho

\begin{english}
  Association with the di꜕sliked is du꜕kkha,
\end{english}

Piyehi꜕ vi꜓ppa꜕yogo dukkho

\begin{english}
  Separa꜓tion from th꜕e liked is du꜕kkha,
\end{english}

Yampiccha꜓ṃ na꜕ labhati꜕ tampi꜕ dukkhaṃ

\begin{english}
  Not attaining one's wi꜓shes is du꜕kkha;
\end{english}

Sa꜓ṅkhi꜕ttena pañcu꜕pādānakkha꜓ndhā dukkhā

\begin{english}
  In brief th꜕e five focuses of iden꜓tity are du꜕kkha.
\end{english}

Idaṃ kho pa꜕na bhi꜓kkh꜕ave dukkha-sa꜕mu꜕dayo a꜕riya꜓-sa꜕ccaṃ

\begin{english}
  This bhikkhus is the No꜕bl꜕e Tr꜕uth o꜕f th꜕e cau꜕se o꜕f du꜕kkha:
\end{english}

\begin{twochants}

Yā'yaṃ taṇhā &
\tr{It is this cra꜕ving} \\

Ponobbha꜓vi꜕kā &
\tr{Which lea꜕ds t꜕o re꜕birth,} \\

Nandi꜓-rāga-sa꜕ha꜕ga꜕tā &
\tr{Accompanied by deli꜓ght a꜕nd lust,} \\

Ta꜕tra-ta꜕trābhi꜓nandi꜕nī &
\tr{Delighting now he꜕re, no꜕w there,} \\

Se꜓yyathī꜓daṃ &
\tr{Na꜕mely:} \\

Kāma-taṇhā &
\tr{Craving fo꜕r se꜕nsu꜕a꜕lity,} \\

Bhava-taṇhā &
\tr{Craving t꜓o be꜕come,} \\

Vi꜓bhava-taṇhā &
\tr{Craving no꜕t t꜕o be꜕come.} \\

\end{twochants}

Idaṃ kho pa꜕na bhi꜓kkh꜕ave dukkha-nirodho a꜕riya꜓-sa꜕ccaṃ

\begin{english}
  This bhikkhus is the No꜕bl꜕e Tr꜕uth o꜕f th꜕e ce꜕ssa꜕ti꜕on o꜕f du꜕kkha:
\end{english}

Yo tassāy'eva taṇhāya a꜕sesa-vi꜓rāga-nirodho

\begin{english}
  It is the remainderless fa꜕di꜕ng a꜕wa꜕y an꜕d ce꜕ssa꜕tion of th꜓at very cr꜕aving,
\end{english}

\begin{twochants}

Cāgo &
\tr{Its reli꜓nqu꜕ishment,} \\

Pa꜕ṭini꜓ssa꜕ggo &
\tr{Le꜕tti꜕ng go,} \\

Mutti &
\tr{Re꜕lease,} \\

A꜕nāla꜓yo &
\tr{Without a꜕ny꜕ a꜕tta꜕chment.} \\

\end{twochants}

Idaṃ kho pa꜕na bhi꜓kkh꜕ave dukkha-nirodha꜕-gāmi꜓nī-pa꜕ṭi꜕pa꜕dā a꜕riya꜓-sa꜕ccaṃ

\begin{english}
  This bhikkhus is the No꜕ble꜕ Tru꜕th o꜕f th꜕e wa꜕y o꜕f pra꜕ctice leading to the ce꜓ssation of du꜕kkha:
\end{english}

Ayam-eva a꜕riyo aṭṭh'a꜓ṅgi꜕ko maggo

\begin{english}
  It is just this No꜕ble꜕ E꜕ightfo꜕ld Path,
\end{english}

\begin{twochants}

Se꜓yyathī꜓daṃ &
\tr{Which is as fo꜕llows:} \\

Sa꜓mmā-diṭṭhi &
\tr{Ri꜕ght View,} \\

Sa꜓mmā-sa꜓ṅka꜕ppo &
\tr{Right Inte꜕ntion} \\

Sa꜓mmā-vācā &
\tr{Ri꜕ght Speech,} \\

Sa꜓mmā-kammanto &
\tr{Right A꜕ction,} \\

Sa꜓mmā-ājīvo &
\tr{Right Li꜓vel꜕ihood,} \\

Sa꜓mmā-vā꜕yāmo &
\tr{Right E꜕ffort,} \\

Sa꜓mmā-sa꜕ti &
\tr{Right Mi꜓ndfu꜕lness,} \\

Sa꜓mmā-sa꜕mādhi &
\tr{Ri꜕ght Co꜕nce꜕ntr꜕ation.} \\

\end{twochants}

Idaṃ dukkhaṃ a꜕riya-sa꜕ccan-t꜕i me bhi꜓kkh꜕ave\\
Pubbe a꜕nanussu꜕tesu꜕ dhammesu\\
Cakkhuṃ u꜕da꜓pādi\\
Ñāṇaṃ u꜕da꜓pādi\\
Paññā u꜕da꜓pādi\\
Vijjā u꜕da꜓pādi\\
Āloko u꜕da꜓pādi

\begin{english}
  Bhikkhus, in rega꜕rd t꜕o thi꜕ngs u꜕nhe꜕ard-o꜕f be꜕fore,\\
  Visio꜓n a꜕rose,\\
  I꜕ns꜕ight a꜕rose,\\
  Disce꜕rnm꜕ent a꜕rose,\\
  Knowle꜓dge a꜕rose,\\
  L꜕ight a꜕rose:\\
  This is the No꜕bl꜕e Tru꜕th o꜕f du꜕kkha;
\end{english}

Taṃ kho pa꜕n'idaṃ dukkhaṃ a꜕riya꜓-sa꜕ccaṃ pa꜕riññeyyan-ti

\begin{english}
  Now this No꜕ble꜕ Tr꜕uth o꜕f du꜕kkha\\
  Should be completely u꜓nde꜕rstood;
\end{english}

Taṃ kho pa꜕n'idaṃ dukkhaṃ a꜕riya꜓-sa꜕ccaṃ pa꜕riññātan-ti

\begin{english}
  Now this No꜕ble꜕ Tr꜕uth o꜕f du꜕kkha
  Has be꜕en co꜕mple꜕tel꜕y u꜕nde꜕rstood.
\end{english}

Idaṃ dukkha-sa꜕mu꜕dayo a꜕riya꜓-sa꜕ccan-t꜕i me bhi꜓kkh꜕ave\\
Pubbe a꜕nanussu꜕tesu꜕ dhammesu\\
Cakkhuṃ u꜕da꜓pādi\\
Ñāṇaṃ u꜕da꜓pādi\\
Paññā u꜕da꜓pādi\\
Vijjā u꜕da꜓pādi\\
Āloko u꜕da꜓pādi

\begin{english}
  Bhikkhus, in rega꜕rd t꜕o thi꜕ngs u꜕nhe꜕ard-o꜕f be꜕fore,\\
  Visio꜓n a꜕rose,\\
  I꜕ns꜕ight a꜕rose,\\
  Disce꜕rnm꜕ent a꜕rose,\\
  Knowle꜓dge a꜕rose,\\
  L꜕ight a꜕rose:\\
  This is the No꜕ble꜕ Tru꜕th o꜕f th꜕e ca꜕use o꜕f du꜕kkha.
\end{english}

\clearpage

Taṃ kho pa꜕n'idaṃ dukkha-sa꜕mu꜕dayo a꜕riya꜓-sa꜕ccaṃ pa꜕hāta꜕bban-ti

\begin{english}
  Now this ca꜕use o꜕f du꜕kkha\\
  Sho꜕uld b꜕e a꜕ba꜕ndoned;
\end{english}

Taṃ kho pa꜕n'idaṃ dukkha-sa꜕mu꜕dayo a꜕riya꜓-sa꜕ccaṃ pa꜕hīnan-ti

\begin{english}
  Now this ca꜕use o꜕f du꜕kkha\\
  Ha꜕s be꜕en a꜕ba꜕ndoned.
\end{english}

Idaṃ dukkha꜕-nirodho a꜕riya꜓-sa꜕ccan-t꜕i me bhi꜓kkh꜕ave\\
Pubbe a꜕nanussu꜕tesu꜕ dhammesu\\
Cakkhuṃ u꜕da꜓pādi\\
Ñāṇaṃ u꜕da꜓pādi\\
Paññā u꜕da꜓pādi\\
Vijjā u꜕da꜓pādi\\
Āloko u꜕da꜓pādi

\begin{english}
  Bhikkhus, in rega꜕rd t꜕o thi꜕ngs u꜕nhe꜕ard-o꜕f be꜕fore,\\
  Visio꜓n a꜕rose,\\
  I꜕ns꜕ight a꜕rose,\\
  Disce꜕rnm꜕ent a꜕rose,\\
  Knowle꜓dge a꜕rose,\\
  L꜕ight a꜕rose:\\
  This is the No꜕ble꜕ Tr꜕uth o꜕f th꜕e ce꜕ssa꜕ti꜕on o꜕f du꜕kkha;
\end{english}

Taṃ kho pa꜕n'idaṃ dukkha-nirodho a꜕riya꜓-sa꜕ccaṃ sacch꜕i-kāta꜓bban-ti

\begin{english}
  Now the ce꜓ssation o꜕f du꜕kkha\\
  Should be expe꜕rie꜕nced di꜕re꜕ctly;
\end{english}

Taṃ kho pa꜕n'idaṃ dukkha-nirodho a꜕riya꜓-sa꜕ccaṃ sacch꜕ika꜕tan-ti

\begin{english}
  Now the ce꜓ssation o꜕f du꜕kkha\\
  Ha꜕s be꜕en e꜕xpe꜕rie꜕nced di꜕re꜕ctly.
\end{english}

Idaṃ dukkha꜕-nirodha꜕-gāmi꜓nī-pa꜕ṭi꜕pa꜕dā a꜕riya꜓-sa꜕ccan-t꜕i me bhi꜓kkh꜕ave\\
Pubbe a꜕nanussu꜕tesu꜕ dhammesu\\
Cakkhuṃ u꜕da꜓pādi\\
Ñāṇaṃ u꜕da꜓pādi\\
Paññā u꜕da꜓pādi\\
Vijjā u꜕da꜓pādi\\
Āloko u꜕da꜓pādi

\begin{english}
  Bhikkhus, in rega꜕rd t꜕o thi꜕ngs u꜕nhe꜕ard-o꜕f be꜕fore,\\
  Visio꜓n a꜕rose,\\
  I꜕ns꜕ight a꜕rose,\\
  Disce꜕rnm꜕ent a꜕rose,\\
  Knowle꜓dge a꜕rose,\\
  L꜕ight a꜕rose:\\
  This is the No꜕ble꜕ Tru꜕th o꜕f th꜕e wa꜕y o꜕f pra꜕ctice leading to the ce꜓ssation of du꜕kkha;
\end{english}

Taṃ kho pa꜕n'idaṃ dukkha-nirodha-gāmi꜓nī-pa꜕ṭi꜕pa꜕dā a꜕riya꜓-sa꜕ccaṃ bhāvetabban-ti

\begin{english}
  Now this wa꜕y o꜕f pra꜕ctice leading to the ce꜓ssation of du꜕kkha\\
  Sho꜕uld be꜕ de꜕ve꜕loped;
\end{english}

Taṃ kho pa꜕n'idaṃ dukkha-nirodha-gāmi꜓nī-pa꜕ṭi꜕pa꜕dā a꜕riya꜓-sa꜕ccaṃ bhāvi꜓tan-ti

\begin{english}
  Now this wa꜕y o꜕f pra꜕ctice leading to the ce꜓ssation of du꜕kkha\\
  Ha꜕s be꜕en de꜕ve꜕loped.
\end{english}

Yāva-kī꜕vañ-ca꜕ me bhi꜓kkh꜕ave i꜕mesu꜕ ca꜕tūsu a꜕riya꜓-sa꜕ccesu\\
Evan-t꜕i-pa꜕rivaṭṭaṃ dvādas'ā꜓kā꜓raṃ yathā꜓-bhūtaṃ ñāṇa-dassa꜕naṃ na su꜕vi꜓su꜕ddhaṃ a꜕hosi

\begin{english}
  As long, bhi꜕kkhus, as my knowledge and understa꜕nding,\\
  As it ac꜕tua꜕lly꜕ is,\\
  Of these Four No꜓bl꜕e Truths,\\
  With their three pha꜕se꜕s an꜕d twe꜕lve a꜕spects,\\
  Was no꜕t e꜕nt꜓irel꜕y pure,
\end{english}

N'eva tāvāhaṃ bhi꜓kkh꜕ave sa꜕deva꜓ke loke sa꜕māra꜓ke sa꜕brahma꜓ke\\
Sassamaṇa-brāhmaṇiyā pa꜕jāya sa꜕deva-ma꜕nussā꜓ya\\
Anu꜓tta꜕raṃ sa꜓mmā-sa꜓mbodhiṃ a꜕bhisa꜓mbuddho pa꜕ccaññāsiṃ

\begin{english}
  Did I not cla꜕im, bhi꜕kkhus,\\
  In this world of de꜕vas Mā꜕ra꜕ an꜕d Br꜕ahmā,\\
  Amongst ma꜕nkind with its priests and renu꜓nci꜕ants,\\
  Kings and co꜓mmo꜕ners,\\
  An u꜕lti꜕ma꜕te a꜕wa꜕ke꜕ning\\
  To unsu꜓rpassed pe꜕rfe꜕ct e꜕nli꜓ghte꜕nment.
\end{english}

Ya꜕to ca꜕ kho me bhi꜓kkh꜕ave i꜕mesu꜕ ca꜕tūsu a꜕riya꜓-sa꜕ccesu\\
Evan-t꜕i-pa꜕rivaṭṭaṃ dvādas'ā꜓kā꜓raṃ yathā꜓-bhūtaṃ ñāṇa-dassanaṃ su꜕vi꜓su꜕ddhaṃ ahosi

\begin{english}
  But when, bhi꜕kkhus, my knowledge and understa꜕nding\\
  As it ac꜕tua꜕lly꜕ is,\\
  Of these Four No꜓bl꜕e Truths,\\
  With their three pha꜕se꜕s an꜕d twe꜕lve a꜕spects,\\
  Was inde꜕ed e꜕nt꜓irel꜕y pure,
\end{english}

Athāhaṃ bhi꜓kkh꜕ave sa꜕deva꜓ke loke sa꜕mār꜓ake sa꜕brahma꜓ke\\
Sassamaṇa-brāhmaṇiyā pa꜕jāya sa꜕deva-ma꜕nussā꜓ya\\
Anu꜓tta꜕raṃ sa꜓mmā-sa꜓mbodhiṃ a꜕bhisa꜓mbuddho pa꜕cca꜕ññā꜕siṃ

\begin{english}
  Th꜕en i꜕ndeed did I cla꜕im, bhi꜕kkhus,\\
  In this world of de꜕vas, Mā꜕ra꜕ an꜕d Br꜕ahmā,\\
  Amongst ma꜕nkind with its priests and renu꜓nci꜕ants,\\
  Kings and co꜓mmo꜕ners,\\
  An u꜕lti꜕ma꜕te a꜕wa꜕ke꜕ning\\
  To unsu꜓rpassed, pe꜕rfe꜕ct e꜕nli꜓ghte꜕nment.
\end{english}

Ñāṇañ-ca pana me dass꜕anaṃ u꜕da꜓pādi

\begin{english}
  Now kno꜕wle꜕dge an꜕d u꜕nde꜕rst꜕anding aro꜕se i꜕n me:
\end{english}

A꜕kuppā me vi꜓mutti

\begin{english}
  My release i꜕s unsh꜓akeable,
\end{english}

A꜕yam-ant꜕imā jāti

\begin{english}
  This is my la꜕st birth,
\end{english}

N'atthi꜓dāni pu꜕nabbh꜕avo-ti

\begin{english}
  There won't be a꜕ny꜕ fu꜕rth꜕er be꜕co꜕ming.
\end{english}

\chapter[Striving according to Dhamma]{The Teaching on striving according to Dhamma}% {{{1

\begin{leader}
  [Handa mayaṃ dhamma-pahaṃsāna-pāṭham bhaṇāmase]
\end{leader}

Evaṃ svā꜕kkhāto bhi꜓kkh꜕ave mayā dhammo

\begin{english}
  Bhikkhus, th꜕e Dhamma has thus been we꜕ll expo꜓unded by me,
\end{english}

\begin{twochants}
Uttāno &
\tr{Elu꜕ci꜕da꜕ted,} \\

Vi꜓va꜕ṭo &
\tr{Di꜕sclosed,} \\

Pa꜕kāsi꜓to &
\tr{Re꜕vealed} \\

Chi꜓nna-pi꜕loti꜓ko &
\tr{An꜕d str꜕ipped o꜕f pa꜕tchwork --} \\
\end{twochants}

Alam-eva sa꜕ddhā-pa꜕bbaj꜓itena kula-pu꜕ttena vī꜓riyaṃ ā꜕rabh꜕ituṃ

\begin{english}
  This is enou꜕gh fo꜕r a꜕ cl꜕ansman,\\
  Who has go꜕ne forth out o꜕f faith,\\
  To arou꜕se h꜕is e꜓ne꜓rgy꜕ thus:
\end{english}

Kāmaṃ ta꜕co ca nahā꜓ru c꜕a aṭṭhi c꜕a a꜕vasi꜓ss꜕atu

\begin{english}
  `Willingly let o꜕nly꜕ my꜕ skin, si꜕ne꜕ws a꜕nd bo꜕nes re꜕main,
\end{english}

Sa꜕rīre u꜕pasuss꜓atu maṃsa꜕-lohi꜕taṃ

\begin{english}
  And let th꜓e flesh and blo꜕od i꜕n th꜕is bo꜕dy wi꜕th꜕er a꜕way.
\end{english}

\begin{tabular}{@{}p{0.5\linewidth} p{0.5\linewidth}@{}}

Yaṃ taṃ &
\tr{As long as whate꜕ve꜕r i꜕s t꜕o b꜕e a꜕ttained}\\

Pu꜕risa-thāmena &
\tr{By huma꜓n strength,} \\

Pu꜕risa-vī꜓riyena &
\tr{By human e꜓ne꜕rgy,} \\

\end{tabular}

\begin{tabular}{@{}p{0.5\linewidth} p{0.5\linewidth}@{}}

Pu꜕risa-pa꜕rakk꜕amena &
\tr{B꜕y hu꜕ma꜕n e꜕ffort} \\

Pa꜕tta꜕bbaṃ na taṃ a꜕pāpu꜕ṇitvā &
\tr{Has not be꜓en a꜕ttained,} \\

Vī꜓riyassa sa꜓ṇṭhānaṃ bha꜕vissa꜕tī-ti &
\tr{Let no꜕t m꜕y e꜕ff꜕orts st꜕and still.'} \\

\end{tabular}

Dukkhaṃ bhi꜓kkh꜕ave kusī꜓to vi꜓ha꜕rati

\begin{english}
  Bhikkhus, the laz꜓y person dwe꜕lls i꜕n su꜓ffe꜕ring,
\end{english}

Voki꜕ṇṇo pāpa꜕kehi a꜕ku꜕saleh꜕i dhammehi

\begin{english}
  Soiled by e꜕vi꜕l, u꜕nwho꜕leso꜕me states
\end{english}

Maha꜓ntañ-ca sa꜕da꜕tthaṃ pa꜕ri꜓hāpeti

\begin{english}
  And great is th꜓e personal go꜕od tha꜕t h꜕e ne꜕glects.
\end{english}

Āraddha-vī꜓riyo c꜕a kho bhi꜓kkh꜕ave su꜕khaṃ vi꜓ha꜕rati

\begin{english}
  The ene꜓rgetic pe꜕rs꜕on tho꜕ugh dw꜕ells ha꜕ppi꜕ly,
\end{english}

Pa꜕vivitto pāpa꜕keh꜕i a꜕ku꜕saleh꜕i dhammehi

\begin{english}
  Well withdrawn from unwho꜓leso꜕me states
\end{english}

Maha꜓ntañ-ca sa꜕da꜕tthaṃ pa꜕ri꜓pūreti

\begin{english}
  And great is th꜓e personal go꜕od tha꜕t h꜕e a꜕chieves.
\end{english}

Na bhi꜓kkh꜕ave hī꜕nena a꜕gga꜕ssa꜕ pa꜕tt꜓i hoti

\begin{english}
  Bhikkhus, it i꜓s not by lo꜕we꜕r means that the supre꜕me i꜕s a꜕ttained
\end{english}

Aggena ca kho bhi꜓kkh꜕ave a꜕gga꜕ssa꜕ pa꜕tt꜓i hoti

\begin{english}
  But, bhikkhus, it is by th꜓e su꜕preme that the supre꜕me i꜕s a꜕ttained.
\end{english}

Maṇḍape꜓yyam-i꜓daṃ bhi꜓kkh꜕ave brahmaca꜕ri꜓yaṃ

\begin{english}
  Bhikkhus, this h꜓ol꜕y life is like the cre꜕am o꜕f t꜕he milk:
\end{english}

Satthā sammukhī꜓-bhū꜕to

\begin{english}
  The Te꜕ach꜕er i꜕s pr꜕esent,
\end{english}

Tasmāti꜕ha bhi꜓kkh꜕ave vī꜓riyaṃ ārabha꜕tha

\begin{english}
  Therefore, bh꜕ikkhus, sta꜕rt t꜕o a꜕rou꜕se your e꜓ne꜕rgy
\end{english}

{\setlength{\tabcolsep}{0.1em}

\begin{tabular}{@{}p{0.45\linewidth} p{0.6\linewidth}@{}}

A꜕ppa꜕tta꜕ssa꜕ pa꜕tt꜓iyā &
\tr{For the a꜓ttainment of the as ye꜕t u꜕na꜕ttained,} \\

Anadhi꜓ga꜕tassa a꜕dhiga꜕māya &
\tr{For the a꜓chievement of the as ye꜕t u꜕na꜕chieved,} \\

Asa꜕cchi꜕ka꜕tassa sa꜕cchi꜕ki꜕ri꜓yāya &
\tr{For the reali꜓zation of the as ye꜕t u꜕nre꜕alized.} \\

\end{tabular}

}

`Evaṃ no ayaṃ amhākaṃ pa꜕bb꜕ajjā a꜕vaṅka꜕tā a꜕vañjhā bha꜕vi꜓ssati

\begin{english}
  Thinking, in su꜕ch a꜕ way: `Our Go꜓i꜕ng Forth will no꜕t b꜕e ba꜕rren
\end{english}

Sa꜕phalā s꜕a-u꜕dra꜓yā.

\begin{english}
  But will be꜓come fru꜕itfu꜕l an꜕d fe꜕rtile,
\end{english}

Yesa꜓ṃ mayaṃ pa꜕ribhuñjāma cīva꜓ra-piṇḍa꜕pāta-\\
Se꜓nāsana-gi꜓lānappa꜕ccaya-bhesa꜕jja-parikkhā꜓raṃ\\
Tesaṃ te kārā a꜕mhesu

\begin{english}
  And all our us꜕e o꜕f robes, a꜕lmsfood,\\
  l꜕odgings, and medici꜕nal re꜓qui꜕sites,\\
  Given by o꜕th꜕ers fo꜕r ou꜕r su꜕pport,
\end{english}

Ma꜕happh꜕alā bhavissanti ma꜕hā-ni꜕sa꜓ṃsā-ti

\begin{english}
  Will rewa꜕rd th꜕em wi꜕th gre꜕at fruit and great be꜓ne꜕fit.'
\end{english}

Evaṃ hi꜕ vo bhi꜓kkh꜕ave si꜕kkh꜕it꜕abbaṃ

\begin{english}
  Bhikkhus, you should tra꜕in yo꜕urse꜕lves thus:
\end{english}

A꜕tt'atthaṃ vā hi bhi꜓kk꜕have sa꜓mpassa꜕mānena

\begin{english}
  Co꜓nsidering your ow꜕n good,
\end{english}

A꜕lam-eva a꜕ppamādena sa꜓mpādetuṃ

\begin{english}
  It is e꜓nough to str꜕ive fo꜕r th꜕e go꜕al wi꜕tho꜕ut ne꜕gligence;
\end{english}

Pa꜕r'atthaṃ vā hi bhi꜓kkh꜕ave sa꜓mpass꜕amānena

\begin{english}
  Bhikkhus, co꜓nsidering the go꜕od o꜕f o꜕thers,
\end{english}

A꜕lam-eva a꜕ppamādena sa꜓mpāde꜕tuṃ

\begin{english}
  It is e꜓noughto str꜕ive fo꜕r th꜕e go꜕al wi꜕tho꜕ut ne꜕gligence;
\end{english}

U꜕bhay'atthaṃ vā hi bhi꜓kkh꜕ave sa꜓mpassa꜕mānena

\begin{english}
  Bhikkhus, co꜓nsidering the go꜕od o꜕f both,
\end{english}

Alam-eva a꜕ppamādena sa꜓mpāde꜕tun-ti

\begin{english}
  It is e꜓nough to str꜕ive fo꜕r th꜕e go꜕al wi꜕tho꜕ut ne꜕gligence.
\end{english}

\chapter{The Verses of Tāyana}% {{{1

\begin{leader}
  [Handa mayaṃ tāyana-gāthāyo bhaṇāmase]
\end{leader}

\begin{twochants}
  Chi꜓nda so꜕taṃ pa꜕rakkamma & kā꜕me panūda brā꜓hm꜕aṇa \\
  Nappahā꜓ya mu꜕ni kāme & nekattam-upa꜕pajja꜕ti \\
\end{twochants}

\begin{english}
  Exert yourself a꜕nd cut t꜕he stream.\\
  Discard sense-pl꜓easu꜕res, Holy꜕ Man;\\
  Not letting sensu꜕al pleasu꜕res go,\\
  A sage will no꜓t re꜕ach uni꜕ty.
\end{english}

\begin{twochants}
  Kayirā ce ka꜕yirāthe꜓naṃ & da꜕ḷham-enaṃ pa꜕rakka꜕me \\
  Sithilo hi pa꜕ribbājo & bh꜕iyyo ākira꜕te ra꜕jaṃ \\
\end{twochants}

\begin{english}
  Vigorously, wi꜕th all on꜕e's strength,\\
  It should be do꜓ne, wh꜕at should b꜕e done;\\
  A lax monast꜕ic life sti꜕rs up\\
  The dust of pa꜓ssio꜕ns all th꜕e more
\end{english}

\begin{twochants}
  A꜕kataṃ dukkaṭaṃ se꜓yyo & pacchā tappati du꜓kk꜕aṭaṃ \\
  Katañ-ca su꜕ka꜓taṃ seyyo & yaṃ ka꜕tvā nānuta꜕ppa꜕ti \\
\end{twochants}

\begin{english}
  Better is not t꜕o do ba꜕d deeds\\
  That afterwa꜓rds wo꜕uld bring re꜕morse;\\
  It's rather go꜓od de꜕eds one sho꜕uld do\\
  Which having done on꜕e won't re꜕gret.
\end{english}

\begin{twochants}
  Kuso꜓ ya꜕thā du꜕ggahi꜕to & hattham-evā꜓nu꜕kant꜕ati \\
  Sā꜓maññaṃ du꜕pparāma꜕ṭṭhaṃ & nirayāyūpa꜕kaḍḍh꜕ati \\
\end{twochants}

\begin{english}
  As Kusa-grass, wh꜕en wrongly꜕ grasped,\\
  Will only cu꜓t i꜕nto on꜕e's hand\\
  So does th꜕e monk's lif꜕e wrongl꜕y led\\
  Indeed drag on꜓e t꜕o hell꜕ish states.
\end{english}

\begin{twochants}
  Yaṃ-kiñci si꜕thi꜓laṃ kammaṃ & sa꜓ṅki꜕liṭṭha꜓ñ-ca꜕ yaṃ va꜓taṃ \\
  Sa꜓ṅka꜕ssaraṃ brahma-ca꜕ri꜓yaṃ & na taṃ ho꜓ti ma꜕happh꜕alan-ti \\
\end{twochants}

\begin{english}
  Whateve꜕r deed tha꜕t's slackl꜕y done,\\
  Whatever vo꜓w co꜕rruptl꜕y kept,\\
  The Holy Life le꜕d in꜕ doubtf꜕ul ways --\\
  All these will ne꜓ve꜕r bear gr꜕eat fruits.
\end{english}

% }}}1

% End of reflections-and-recollections-p3.tex
