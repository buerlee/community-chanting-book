% vim: foldmethod=marker foldlevel=0 foldtext=FoldText()

\chapter[The Noble Eightfold Path]{The Teaching on the Noble Eightfold Path}% {{{1

\begin{leader}
  [Handa mayaṃ ariyaṭṭhaṅgika-magga-pāṭham bhaṇāmase]
\end{leader}

Ayam-eva \cD{a}riyo aṭṭh\cU{a}ṅg\cD{i}ko maggo

\begin{english}
  This is the N\cD{o}b\cD{l}e \cD{E}ightf\cD{o}ld Path,
\end{english}

\begin{twochants}

S\cU{e}yyath\cU{ī}daṃ &
\tr{Which is as f\cD{o}llows:} \\

S\cU{a}mmā-diṭṭhi &
\tr{R\cD{i}ght View,} \\

S\cU{a}mmā-s\cU{a}ṅk\cD{a}ppo &
\tr{Right Int\cD{e}ntion} \\

S\cU{a}mmā-vācā &
\tr{R\cD{i}ght Speech,} \\

S\cU{a}mmā-kammanto &
\tr{Right \cD{A}ction,} \\

S\cU{a}mmā-ājīvo &
\tr{Right L\cU{i}ve\cD{l}ihood,} \\

S\cU{a}mmā-v\cD{ā}yāmo &
\tr{Right \cD{E}ffort,} \\

S\cU{a}mmā-s\cD{a}ti &
\tr{Right M\cU{i}ndf\cD{u}lness,} \\

S\cU{a}mmā-s\cD{a}mādhi &
\tr{R\cD{i}ght C\cD{o}nc\cD{e}nt\cD{r}ation.} \\

\end{twochants}

K\cD{a}tamā ca bh\cU{i}kk\cD{h}ave sammā-diṭṭhi

\begin{english}
  And what bhikkhus \cD{i}s R\cD{i}ght View?
\end{english}

\begin{tabular}{@{}p{0.5\linewidth} p{0.5\linewidth}@{}}

Yaṃ kho bh\cU{i}kk\cD{h}ave dukkhe ñāṇaṃ &
\tr{Knowledge of s\cD{u}ffering;} \\

Dukkha-s\cD{a}m\cD{u}daye ñāṇaṃ &
\tr{Knowledge of the \cU{o}rigin of s\cD{u}ffering;} \\

Dukkha-n\cU{i}rodhe ñāṇaṃ &
\tr{Knowledge of the cess\cD{a}ti\cD{o}n \cD{o}f s\cD{u}ff\cD{e}ring;} \\

Dukkha-n\cU{i}rodha-gām\cU{i}n\cU{i}yā\newline p\cD{a}ṭip\cD{a}dāya ñāṇaṃ &
\tr{Knowledge of t\cU{h}e path\newline Leading to the ces\cD{s}at\cD{i}on \cD{o}f s\cD{u}ffering:} \\

\end{tabular}

\cD{A}yaṃ vuccati bh\cU{i}kk\cD{h}ave s\cU{a}mmā-diṭṭhi

\begin{english}
  This bhikkhus is c\cD{a}lled R\cD{i}ght View.
\end{english}

Katamo ca bh\cU{i}kk\cD{h}ave sammā-s\cU{a}ṅk\cD{a}ppo

\begin{english}
  And what bhikkhus is R\cD{i}ght \cD{I}nt\cD{e}ntion?
\end{english}

\begin{twochants}

Nekkhamma-s\cU{a}ṅk\cD{a}ppo &
\tr{The intention of ren\cD{u}n\cD{c}i\cD{a}tion;} \\

\cD{A}byāpāda-s\cU{a}ṅk\cD{a}ppo &
\tr{The intention of n\cD{o}n-i\cD{l}l-will;} \\

\cD{A}vihiṃsā-s\cU{a}ṅk\cD{a}ppo &
\tr{The intention of non-cr\cU{u}\cD{e}lty:} \\

\end{twochants}

Ayaṃ vuccati bh\cU{i}kk\cD{h}ave s\cU{a}mmā-s\cU{a}ṅk\cD{a}ppo

\begin{english}
  This bhikkhus is c\cD{a}lled R\cD{i}ght \cD{I}nt\cD{e}ntion.
\end{english}

Katamā ca bh\cU{i}kk\cD{h}ave s\cU{a}mmā-vācā

\begin{english}
  And what bhikkhus \cD{i}s R\cD{i}ght Speech?
\end{english}

\begin{twochants}

Musā-vādā ver\cU{a}m\cD{a}ṇī &
\tr{Abstaining fr\cD{o}m f\cD{a}lse speech;} \\

Pisuṇāya vācāya ver\cU{a}m\cD{a}ṇī &
\tr{Abstain\cU{i}ng from mal\cU{i}ci\cD{o}us speech;} \\

Pharusāya vācāya ver\cU{a}m\cD{a}ṇī &
\tr{Abstaining fr\cD{o}m h\cD{a}rsh speech;} \\

S\cU{a}mphapp\cD{a}lāpā ver\cU{a}m\cD{a}ṇī. &
\tr{Abstaining from \cD{i}d\cD{l}e ch\cD{a}tter:} \\

\end{twochants}

Ayaṃ vuccati bh\cU{i}kk\cD{h}ave s\cU{a}mmā-vācā

\begin{english}
  This bhikkhus is c\cD{a}lled R\cD{i}ght Speech.
\end{english}

Katamo ca bh\cU{i}kk\cD{h}ave s\cU{a}mmā-kammanto

\begin{english}
  And what bhikkhus \cD{i}s R\cD{i}ght \cD{A}ction?
\end{english}

\begin{tabular}{@{}p{0.4\linewidth} p{0.6\linewidth}@{}}

Pāṇāt\cD{i}pātā ver\cU{a}m\cD{a}ṇī &
\tr{Abstain\cU{i}ng from k\cD{i}ll\cD{i}ng l\cD{i}v\cD{i}ng b\cD{e}ings;} \\

\cD{A}dinnādānā ver\cU{a}m\cD{a}ṇī &
\tr{Abstain\cU{i}ng from t\cD{a}k\cD{i}ng w\cD{h}at \cD{i}s n\cD{o}t g\cD{i}ven;} \\

Kāmes\cD{u}-micch\cU{ā}cārā ver\cU{a}m\cD{a}ṇī &
\tr{Abstain\cU{i}ng from s\cD{e}x\cD{u}al m\cU{i}sc\cD{o}nduct:} \\

\end{tabular}

Ayaṃ vuccati bh\cU{i}kk\cD{h}ave s\cU{a}mmā-kammanto

\begin{english}
  This bhikkhus is c\cD{a}lled R\cD{i}ght A\cD{c}tion.
\end{english}

Katamo ca bh\cU{i}kkh\cD{a}ve s\cU{a}mmā-ājīvo

\begin{english}
  And what bhikkhus is Right \cU{L}ive\cD{l}ihood?
\end{english}

\begin{tabular}{@{}p{0.45\linewidth} p{0.55\linewidth}@{}}

Idha bh\cU{i}kk\cD{h}ave \cD{a}riya-s\cU{ā}v\cD{a}ko\newline
Micchā-ājīvaṃ p\cD{a}hāya\newline
Sammā-āj\cU{ī}vena jīv\cU{i}taṃ k\cD{a}ppeti &

\tr{Here, bhikkhus, a Nob\cD{l}e D\cD{i}s\cD{c}iple,\newline
Having \cU{a}bandoned wrong l\cU{i}ve\cD{l}ihood,\newline
Earns \cU{h}is living by r\cD{i}ght l\cD{i}ve\cD{l}ihood:} \\

\end{tabular}

Ayaṃ vuccati bh\cU{i}kk\cD{h}ave s\cU{a}mmā-ājīvo

\begin{english}
  This bhikkhus is c\cD{a}lled R\cD{i}ght L\cD{i}ve\cD{l}ihood.
\end{english}

Katamo ca bh\cU{i}kk\cD{h}ave s\cU{a}mmā-vāyāmo

\begin{english}
  And what bhikkhus \cD{i}s R\cD{i}ght \cD{E}ffort?
\end{english}

Idha bh\cU{i}kk\cD{h}ave bhikkhu \cD{a}nuppannānaṃ pāp\cD{a}kānaṃ \cD{a}k\cD{u}salānaṃ dhammānaṃ anuppādāya

\begin{english}
  Here, bhikkhus, \cU{a} bhikkhu aw\cD{a}k\cD{e}ns zeal for the non-\cU{a}rising of unar\cD{i}sen, evil unwh\cU{o}les\cD{o}me states;
\end{english}

Chandaṃ j\cD{a}neti vāyam\cU{a}ti v\cU{ī}r\cU{i}yaṃ ārabh\cD{a}ti c\cD{i}ttaṃ p\cD{a}ggaṇh\cU{ā}ti p\cD{a}dah\cD{a}ti

\begin{english}
  He puts forth \cD{e}ffort, arouses \cU{e}n\cD{e}rgy, exerts \cU{h}is mind a\cD{n}d strives.
\end{english}

\cD{U}ppannānaṃ pāp\cD{a}kānaṃ \cD{a}k\cD{u}salānaṃ dhammānaṃ p\cD{a}hānāya

\begin{english}
  He awak\cU{e}ns zeal for the ab\cD{a}ndoning of \cU{a}risen, evil unwh\cU{o}les\cD{o}me states;
\end{english}

Chandaṃ j\cD{a}neti vāyam\cU{a}ti v\cU{ī}r\cU{i}yaṃ ārabh\cD{a}ti c\cD{i}ttaṃ p\cD{a}ggaṇh\cU{ā}ti p\cD{a}dah\cD{a}ti

\begin{english}
  He puts forth \cD{e}ffort, arouses \cU{e}n\cD{e}rgy, exerts \cU{h}is mind a\cD{n}d strives.
\end{english}

Anuppannānaṃ k\cD{u}salānaṃ dhammānaṃ \cD{u}ppādāya

\begin{english}
  He awak\cU{e}ns zeal for the ar\cD{i}sing of un\cU{a}risen wh\cU{o}les\cD{o}me states;
\end{english}

Chandaṃ j\cD{a}neti vāyam\cU{a}ti v\cU{ī}r\cU{i}yaṃ ārabh\cD{a}ti c\cD{i}ttaṃ p\cD{a}ggaṇh\cU{ā}ti p\cD{a}dah\cD{a}ti

\begin{english}
  He puts forth \cD{e}ffort, arouses \cU{e}n\cD{e}rgy, exerts \cU{h}is mind a\cD{n}d strives.
\end{english}

\cD{U}ppannānaṃ k\cD{u}salānaṃ dhammānaṃ ṭ\cU{h}it\cD{i}yā \cD{a}s\cD{a}mmosāya b\cU{h}iyyobhāvāya vep\cD{u}llāya bhāv\cU{a}nāya pār\cU{i}p\cD{ū}riyā

\begin{english}
  He awakens zeal for the cont\cD{i}nuance, non-dis\cU{a}ppearance, str\cD{e}ngthening, increase and \cU{f}ulfilment by dev\cU{e}l\cD{o}pment of ar\cD{i}sen wh\cD{o}les\cD{o}me states;
\end{english}

Chandaṃ j\cD{a}neti vāyam\cU{a}ti v\cU{ī}r\cU{i}yaṃ ārabh\cD{a}ti c\cD{i}ttaṃ p\cD{a}ggaṇh\cU{ā}ti p\cD{a}dah\cD{a}ti

\begin{english}
  He puts forth \cD{e}ffort, arouses \cU{e}n\cD{e}rgy, exerts \cU{h}is mind a\cD{n}d strives:
\end{english}

Ayaṃ vuccati bh\cU{i}kk\cD{h}ave s\cU{a}mmā-vāyāmo

\begin{english}
  This bhikkhus is c\cD{a}lled R\cD{i}ght \cD{E}ffort.
\end{english}

Katamā ca bh\cU{i}kk\cD{h}ave s\cU{a}mmā-s\cD{a}ti

\begin{english}
  And what bhikkhus is Right M\cU{i}ndf\cD{u}lness?
\end{english}

Idha bh\cU{i}kk\cD{h}ave bhikkhu kāye kāyānup\cD{a}ssī v\cU{i}h\cD{a}rati

\begin{english}
  Here, bhikkhus, a bh\cD{i}kkh\cD{u} \cD{a}bides cont\cU{e}mplating the b\cD{o}dy \cD{a}s \cD{a} b\cD{o}dy,
\end{english}

Ātāpī s\cU{a}mp\cD{a}jāno s\cD{a}timā

\begin{english}
  Ardent, full\cU{y} \cD{a}ware and m\cD{i}ndful,
\end{english}

V\cU{i}neyya loke \cD{a}bhijjhā-domanassaṃ

\begin{english}
  Having p\cD{u}t \cD{a}way c\cD{o}v\cD{e}t\cD{o}us\cD{n}ess a\cD{n}d gr\cD{i}ef f\cD{o}r t\cD{h}e world;
\end{english}

Ved\cD{a}nās\cD{u} ved\cD{a}nān\cU{u}p\cD{a}ssī v\cU{i}h\cD{a}rati

\begin{english}
  He \cD{a}bides cont\cU{e}mplating f\cD{e}el\cD{i}ngs \cD{a}s f\cD{e}elings,
\end{english}

Ātāpī s\cU{a}mp\cD{a}jāno s\cD{a}timā

\begin{english}
  Ardent, full\cU{y} \cD{a}ware and m\cD{i}ndful,
\end{english}

V\cU{i}neyya loke \cD{a}bhijjhā-domanassaṃ

\begin{english}
  Having p\cD{u}t \cD{a}way c\cD{o}v\cD{e}t\cD{o}us\cD{n}ess a\cD{n}d gr\cD{i}ef f\cD{o}r t\cD{h}e world;
\end{english}

C\cD{i}tte c\cD{i}ttān\cU{u}p\cD{a}ssī v\cU{i}h\cD{a}rati

\begin{english}
  He \cD{a}bides cont\cU{e}mplating m\cD{i}nd \cD{a}s mind,
\end{english}

Ātāpī s\cU{a}mp\cD{a}jāno s\cD{a}timā

\begin{english}
  Ardent, full\cU{y} \cD{a}ware and m\cD{i}ndful,
\end{english}

V\cU{i}neyya loke \cD{a}bhijjhā-domanassaṃ

\begin{english}
  Having p\cD{u}t \cD{a}way c\cD{o}v\cD{e}t\cD{o}usn\cD{e}ss a\cD{n}d gr\cD{i}ef f\cD{o}r t\cD{h}e world.
\end{english}

Dhammes\cD{u} dhammān\cU{u}p\cD{a}ssī v\cU{i}h\cD{a}rati

\begin{english}
  He \cD{a}bides cont\cU{e}mplating mind-\cD{o}bj\cD{e}cts \cD{a}s m\cD{i}nd-\cD{o}bjects,
\end{english}

Ātāpī s\cU{a}mp\cD{a}jāno s\cD{a}timā

\begin{english}
  Arden,t full\cU{y} \cD{a}ware and m\cD{i}ndful,
\end{english}

V\cU{i}neyya loke \cD{a}bhijjhā-domanassaṃ

\begin{english}
  Having p\cD{u}t \cD{a}way c\cD{o}v\cD{e}t\cD{o}us\cD{n}ess a\cD{n}d gr\cD{i}ef f\cD{o}r t\cD{h}e world:
\end{english}

Ayaṃ vuccati bh\cU{i}kk\cD{h}ave s\cU{a}mmā-s\cD{a}ti

\begin{english}
  This bhikkhus is c\cD{a}lled R\cD{i}ght M\cD{i}ndf\cD{u}lness.
\end{english}

Katamo ca bh\cU{i}kk\cD{h}ave s\cU{a}mmā-s\cD{a}mādhi

\begin{english}
  And what bhikkhus is R\cD{i}ght C\cD{o}nc\cD{e}nt\cD{r}ation?
\end{english}

\begin{twochants}

Idha bh\cU{i}kk\cD{h}ave bhikkhu &
\tr{Here, bhikkhus, a bh\cD{i}kkhu,} \\

Vivicc'eva kāmehi &
\tr{Quite s\cU{e}cluded from s\cD{e}ns\cD{u}al p\cD{l}easures,} \\

Vivicca \cD{a}k\cD{u}s\cD{a}lehi d\cD{h}ammehi &
\tr{Secluded from unwh\cU{o}les\cD{o}me states,} \\

\end{twochants}

S\cD{a}v\cU{i}t\cD{a}kkaṃ s\cD{a}v\cU{i}cāraṃ vivek\cD{a}-jaṃ pīt\cD{i}-sukhaṃ p\cD{a}ṭhamaṃ jhānaṃ upas\cU{a}mpajja v\cU{i}h\cD{a}rati

\begin{english}
  Enters \cU{u}pon and \cD{a}bides i\cD{n} t\cD{h}e f\cD{i}rst Jh\cD{ā}na --\\
  Accomp\cU{a}nied by appl\cD{i}ed a\cD{n}d s\cD{u}sta\cD{i}ned thought,\\
  With rapt\cU{u}re and pl\cD{e}asure b\cD{o}rn \cD{o}f s\cD{e}cl\cD{u}sion.
\end{english}

V\cU{i}takka-vicārānaṃ vūp\cD{a}samā

\begin{english}
  With the stilling of appl\cD{i}ed a\cD{n}d s\cD{u}sta\cD{i}ned thought,
\end{english}

Ajjhattaṃ s\cU{a}mp\cD{a}sādanaṃ cet\cD{a}so ekod\cU{i}bhāvaṃ av\cU{i}t\cD{a}kkaṃ av\cU{i}cāraṃ s\cD{a}mādh\cU{i}-jaṃ pīt\cD{i}-sukhaṃ d\cD{u}tiyaṃ jhānaṃ upas\cU{a}mp\cD{a}jja v\cU{i}h\cD{a}rati

\begin{english}
  He enters \cU{u}pon and \cD{a}bides i\cD{n} t\cD{h}e s\cD{e}c\cD{o}nd Jh\cD{ā}na --\\
  Accomp\cU{a}nied by self-c\cU{o}n\cD{f}idence and s\cD{i}ngl\cD{e}n\cD{e}ss \cD{o}f mind,\\
  Without appli\cD{e}d a\cD{n}d s\cD{u}sta\cD{i}ned thought,\\
  With rapt\cU{u}re and pl\cD{e}asure b\cD{o}rn \cD{o}f c\cD{o}nc\cD{e}nt\cD{r}ation.
\end{english}

\begin{twochants}

Pītiyā c\cD{a} v\cU{i}rāgā &
\tr{With the fad\cU{i}ng \cD{a}way as w\cD{e}ll \cD{o}f r\cD{a}pture} \\

\cD{U}pekkhako ca v\cU{i}h\cD{a}rati &
\tr{He abides in equan\cU{i}m\cD{i}ty,} \\

S\cD{a}to c\cD{a} s\cU{a}mp\cD{a}jāno &
\tr{Mindful and full\cU{y} \cD{a}ware,} \\

S\cD{u}khañ-ca kāyena p\cD{a}ṭis\cU{a}ṃvedeti &
\tr{Still fe\cD{e}l\cD{i}ng pl\cD{e}as\cD{u}re w\cD{i}th t\cD{h}e b\cD{o}dy,} \\

\end{twochants}

Yaṃ taṃ \cD{a}riyā āc\cD{i}kkhanti `\cD{U}pekkh\cU{a}ko s\cD{a}timā s\cD{u}kha-v\cU{i}h\cD{ā}rī'ti tatiyaṃ jhānaṃ \cD{u}pas\cU{a}mp\cD{a}jja v\cU{i}h\cD{a}rati

\begin{english}
  He enters \cU{u}pon and \cD{a}bides i\cD{n} t\cD{h}e th\cD{i}rd J\cD{h}āna --\\
  On account \cU{o}f which the N\cD{o}b\cD{l}e \cU{O}nes \cD{a}nnounce,\\
  `He has \cU{a} pleasant ab\cD{i}ding,\\
  With equan\cU{i}m\cD{i}ty and is m\cD{i}ndful.'
\end{english}

\begin{twochants}

Sukhassa c\cD{a} pahānā &
\tr{With the ab\cU{a}ndoning of pl\cD{e}asure} \\

Dukkhassa c\cD{a} pahānā &
\tr{And the ab\cD{a}nd\cD{o}n\cD{i}ng \cD{o}f pain,} \\

\end{twochants}

\clearpage

P\cD{u}bb'eva somanass\cD{a} domanass\cU{ā}naṃ \cD{a}tthaṅg\cD{a}mā

\begin{english}
  With the previous dis\cU{a}ppearance of j\cD{o}y a\cD{n}d grief,
\end{english}

Adukkham-as\cD{u}khaṃ \cD{u}pekkhā-s\cD{a}ti-p\cD{ā}r\cU{i}s\cD{u}ddhiṃ c\cD{a}tutthaṃ jhānaṃ \cD{u}pas\cU{a}mp\cD{a}jja v\cU{i}h\cD{a}rati

\begin{english}
  He enters \cU{u}pon and \cD{a}bides \cD{i}n t\cD{h}e fo\cD{u}rth J\cD{h}āna --\\
  Accomp\cU{a}nied by neither p\cD{a}in n\cD{o}r-p\cD{l}easure,\\
  And purity of m\cU{i}ndf\cD{u}lness d\cD{u}e t\cD{o} \cD{e}q\cD{u}a\cD{n}imity:
\end{english}

Ayaṃ vuccati bh\cU{i}kk\cD{h}ave s\cU{a}mmā-s\cD{a}mādhi

\begin{english}
  This bhikkhus is c\cD{a}lled R\cD{i}ght C\cD{o}nc\cD{e}nt\cD{r}ation.
\end{english}

Ayam-eva \cD{a}riyo aṭṭh\cU{a}ṅg\cD{i}ko maggo

\begin{english}
  This is the N\cD{o}b\cD{l}e \cD{E}ightf\cD{o}ld Path.
\end{english}

\chapter[The Wheel of Dhamma]{Teachings from the Discourse on Setting in Motion the Wheel of Dhamma}% {{{1

\begin{leader}
  [H\cU{a}nda mayaṃ dhamma-cakkappavattana s\cD{u}tta-pāṭhaṃ bh\cD{a}ṇāmase]
\end{leader}

Dve me bh\cU{i}kkh\cD{a}ve antā

\begin{english}
  Bhikkhus, there are these t\cD{w}o e\cD{x}tremes
\end{english}

Pabbaj\cU{i}tena na sev\cU{i}t\cD{a}bbā

\begin{english}
  That sho\cD{u}ld n\cD{o}t \cD{b}e p\cD{u}rsued by one who h\cU{a}s g\cD{o}ne forth:
\end{english}

Yo cāyaṃ kāmes\cD{u} kāma-s\cD{u}kh'all\cU{i}kān\cU{u}yogo

\begin{english}
  That is, whatever is tied \cD{u}p \cD{t}o s\cD{e}nse p\cD{l}easures,\\
  Within the r\cD{e}alm \cD{o}f s\cD{e}ns\cD{u}\cD{a}l\cD{i}ty,
\end{english}

\begin{twochants}

Hīno &
\tr{Wh\cD{i}ch \cD{i}s low,} \\

Gammo &
\tr{C\cD{o}mmon,} \\

Pothuj\cU{j}an\cD{i}ko &
\tr{The way of the c\cD{o}mm\cD{o}n folks,} \\

Ana\cU{r}iyo &
\tr{Not the w\cD{a}y \cD{o}f t\cD{h}e N\cU{o}b\cD{l}e Ones} \\

Anattha-s\cU{a}ñ\cD{h}ito &
\tr{And p\cD{o}intless;} \\

\end{twochants}

Yo cāyaṃ atta-kilam\cU{a}thān\cU{u}yogo

\begin{english}
  Then there is what\cD{e}v\cD{e}r \cD{i}s \cD{t}ied up\\
  With s\cD{e}lf-d\cD{e}p\cD{r}iv\cD{a}tion,
\end{english}

\begin{twochants}

Dukkho &
\tr{Which is p\cD{a}inful,} \\

Ana\cU{r}iyo &
\tr{Not the w\cD{a}y \cD{o}f t\cD{h}e N\cU{o}b\cD{l}e Ones} \\

\cD{A}natth\cD{a}-s\cU{a}ñ\cD{h}ito &
\tr{And p\cD{o}intless.} \\

\end{twochants}

Ete te bh\cU{i}kkh\cD{a}ve u\cD{b}h\cD{o} ante \cD{a}nup\cD{a}gamma majjh\cU{i}mā p\cD{a}ṭ\cD{i}p\cD{a}dā tath\cU{ā}g\cD{a}tena \cD{a}bhi\cU{s}ambuddhā

\begin{english}
  Bhikkhus, without g\cD{o}ing \cD{t}o \cD{e}it\cD{h}er \cD{o}f t\cD{h}ese \cD{e}xtremes,\\
  The Tath\cU{ā}g\cD{a}ta has \cD{u}l\cD{t}im\cD{a}te\cD{l}y \cD{a}w\cD{a}kened\\
  To \cU{a} middle w\cD{a}y \cD{o}f p\cD{r}actice,
\end{english}

\begin{twochants}

Cakkhu-k\cD{a}r\cU{a}ṇī &
\tr{Giv\cU{i}ng rise to \cD{v}ision,} \\

Ñāṇa-k\cD{a}r\cU{a}ṇī &
\tr{M\cD{a}k\cD{i}ng f\cD{o}r \cD{i}nsight,} \\

\cD{U}pas\cD{a}māya &
\tr{Lead\cU{i}ng \cD{t}o calm,} \\

\cD{A}bhiññāya &
\tr{To h\cD{e}igh\cD{t}ened k\cD{n}owing,} \\

S\cU{a}mbodhāya &
\tr{Aw\cD{a}k\cD{e}ning} \\

N\cU{i}bbānāya s\cU{a}ṃva\cD{t}tati &
\tr{A\cD{n}d \cD{t}o Nibb\cD{ā}na.} \\

\end{twochants}

Katamā ca sā bh\cU{i}kk\cD{h}ave majjh\cD{i}mā \cD{p}a\cD{ṭ}i\cD{p}adā

\begin{english}
  And what, bhikkhus, \cD{i}s th\cD{a}t m\cD{i}dd\cD{l}e w\cD{a}y \cD{o}f pr\cD{a}ctice?
\end{english}

Ayam-eva \cD{a}riyo aṭṭh\cU{a}ṅg\cD{i}ko maggo

\begin{english}
  It is this N\cD{o}b\cD{l}e E\cD{i}ghtf\cD{o}ld Path,
\end{english}

\begin{twochants}

S\cU{e}yyath\cU{ī}daṃ &
\tr{Which is as f\cD{o}llows:} \\

S\cU{a}mmā-diṭṭhi &
\tr{R\cD{i}ght View,} \\

S\cU{a}mmā-s\cU{a}ṅk\cD{a}ppo &
\tr{Right Int\cD{e}ntion} \\

S\cU{a}mmā-vācā &
\tr{R\cD{i}ght Speech,} \\

S\cU{a}mmā-kammanto &
\tr{Right \cD{A}ction,} \\

S\cU{a}mmā-ājīvo &
\tr{Right L\cU{i}ve\cD{l}ihood,} \\

S\cU{a}mmā-v\cD{ā}yāmo &
\tr{Right \cD{E}ffort,} \\

S\cU{a}mmā-s\cD{a}ti &
\tr{Right M\cU{i}ndf\cD{u}lness,} \\

S\cU{a}mmā-s\cD{a}mādhi &
\tr{R\cD{i}ght C\cD{o}nc\cD{e}nt\cD{r}ation.} \\

\end{twochants}

Ayaṃ kho sā bh\cU{i}kk\cD{h}ave majj\cD{h}imā \cD{p}a\cD{ṭ}i\cD{p}adā tath\cU{ā}g\cD{a}tena abhis\cU{a}mbuddhā

\begin{english}
  This, bhikkhus, is the m\cD{i}dd\cD{l}e w\cD{a}y \cD{o}f p\cD{r}actice\\
  That the Tath\cU{ā}g\cD{a}ta has \cD{u}lt\cD{i}m\cD{a}te\cD{l}y \cD{a}w\cD{a}k\cD{e}ned to,
\end{english}

\begin{twochants}

Cakkhu-k\cD{a}r\cU{a}ṇī &
\tr{Giv\cU{i}ng rise to \cD{v}ision,} \\

Ñāṇa-k\cD{a}r\cU{a}ṇī &
\tr{M\cD{a}k\cD{i}ng f\cD{o}r \cD{i}nsight,} \\

\cD{U}pas\cD{a}māya &
\tr{Lead\cU{i}ng \cD{t}o calm,} \\

\cD{A}bhiññāya &
\tr{To h\cD{e}igh\cD{t}ened k\cD{n}owing,} \\

S\cU{a}mbodhāya &
\tr{Aw\cD{a}k\cD{e}ning} \\

N\cU{i}bbānāya s\cU{a}ṃva\cD{t}tati &
\tr{A\cD{n}d \cD{t}o Nibb\cD{ā}na.} \\

\end{twochants}

Idaṃ kho pana bh\cU{i}kk\cD{h}ave dukkhaṃ \cD{a}riy\cU{a}-\cD{s}accaṃ

\begin{english}
  This bhikkhus is the N\cD{o}bl\cD{e} T\cD{r}uth \cD{o}f d\cD{u}kkha:
\end{english}

\begin{twochants}

Jātip\cD{i} dukkhā &
\tr{Birth is d\cD{u}kkha,} \\

Jarāp\cD{i} dukkhā &
\tr{Ageing is d\cD{u}kkha} \\

Maraṇamp\cD{i} dukkhaṃ &
\tr{And death is d\cD{u}kkha;} \\

\end{twochants}

S\cU{o}ka-p\cD{a}rideva-dukkh\cD{a}-domanass\cD{u}pāyās\cU{ā}p\cD{i} dukkhā

\begin{english}
  S\cU{o}rrow lament\cD{a}tion pain grief and d\cD{e}spair are d\cD{u}kkha,
\end{english}

Appiyeh\cD{i} s\cU{a}mp\cD{a}yogo dukkho

\begin{english}
  Association with the d\cD{i}sliked is d\cD{u}kkha,
\end{english}

Piyeh\cD{i} v\cU{i}pp\cD{a}yogo dukkho

\begin{english}
  Separ\cU{a}tion from t\cD{h}e liked is d\cD{u}kkha,
\end{english}

Yampicch\cU{a}ṃ n\cD{a} labhat\cD{i} tamp\cD{i} dukkhaṃ

\begin{english}
  Not attaining one's w\cU{i}shes is d\cD{u}kkha;
\end{english}

S\cU{a}ṅkh\cD{i}ttena pañc\cD{u}pādānakkh\cU{a}ndhā dukkhā

\begin{english}
  In brief t\cD{h}e five focuses of ide\cU{n}tity are d\cD{u}kkha.
\end{english}

Idaṃ kho p\cD{a}na bh\cU{i}kk\cD{h}ave dukkha-s\cD{a}m\cD{u}dayo \cD{a}riy\cU{a}-s\cD{a}ccaṃ

\begin{english}
  This bhikkhus is the N\cD{o}b\cD{l}e T\cD{r}uth \cD{o}f t\cD{h}e ca\cD{u}se \cD{o}f d\cD{u}kkha:
\end{english}

\begin{twochants}

Yā'yaṃ taṇhā &
\tr{It is this cr\cD{a}ving} \\

Ponobbh\cU{a}v\cD{i}kā &
\tr{Which le\cD{a}ds \cD{t}o r\cD{e}birth,} \\

Nand\cU{i}-rāga-s\cD{a}h\cD{a}g\cD{a}tā &
\tr{Accompanied by del\cU{i}ght \cD{a}nd lust,} \\

T\cD{a}tra-t\cD{a}trābh\cU{i}nand\cD{i}nī &
\tr{Delighting now h\cD{e}re, n\cD{o}w there,} \\

S\cU{e}yyath\cU{ī}daṃ &
\tr{N\cD{a}mely:} \\

Kāma-taṇhā &
\tr{Craving f\cD{o}r s\cD{e}ns\cD{u}\cD{a}lity,} \\

Bhava-taṇhā &
\tr{Craving \cU{t}o b\cD{e}come,} \\

V\cU{i}bhava-taṇhā &
\tr{Craving n\cD{o}t \cD{t}o b\cD{e}come.} \\

\end{twochants}

Idaṃ kho p\cD{a}na bh\cU{i}kk\cD{h}ave dukkha-nirodho \cD{a}riy\cU{a}-s\cD{a}ccaṃ

\begin{english}
  This bhikkhus is the N\cD{o}b\cD{l}e T\cD{r}uth \cD{o}f t\cD{h}e c\cD{e}ss\cD{a}t\cD{i}on \cD{o}f d\cD{u}kkha:
\end{english}

Yo tassāy'eva taṇhāya \cD{a}sesa-v\cU{i}rāga-nirodho

\begin{english}
  It is the remainderless f\cD{a}d\cD{i}ng \cD{a}w\cD{a}y a\cD{n}d c\cD{e}ss\cD{a}tion of t\cU{h}at very c\cD{r}aving,
\end{english}

\begin{twochants}

Cāgo &
\tr{Its rel\cU{i}nq\cD{u}ishment,} \\

P\cD{a}ṭin\cU{i}ss\cD{a}ggo &
\tr{L\cD{e}tt\cD{i}ng go,} \\

Mutti &
\tr{R\cD{e}lease,} \\

\cD{A}nāl\cU{a}yo &
\tr{Without \cD{a}n\cD{y} \cD{a}tt\cD{a}chment.} \\

\end{twochants}

Idaṃ kho p\cD{a}na bh\cU{i}kk\cD{h}ave dukkha-nirodh\cD{a}-gām\cU{i}nī-p\cD{a}ṭ\cD{i}p\cD{a}dā \cD{a}riy\cU{a}-s\cD{a}ccaṃ

\begin{english}
  This bhikkhus is the N\cD{o}bl\cD{e} Tr\cD{u}th \cD{o}f t\cD{h}e w\cD{a}y \cD{o}f pr\cD{a}ctice leading to the c\cU{e}ssation of d\cD{u}kkha:
\end{english}

Ayam-eva \cD{a}riyo aṭṭh'\cU{a}ṅg\cD{i}ko maggo

\begin{english}
  It is just this N\cD{o}bl\cD{e} \cD{E}ightf\cD{o}ld Path,
\end{english}

\begin{twochants}

S\cU{e}yyath\cU{ī}daṃ &
\tr{Which is as f\cD{o}llows:} \\

S\cU{a}mmā-diṭṭhi &
\tr{R\cD{i}ght View,} \\

S\cU{a}mmā-s\cU{a}ṅk\cD{a}ppo &
\tr{Right Int\cD{e}ntion} \\

S\cU{a}mmā-vācā &
\tr{R\cD{i}ght Speech,} \\

S\cU{a}mmā-kammanto &
\tr{Right \cD{A}ction,} \\

S\cU{a}mmā-ājīvo &
\tr{Right L\cU{i}ve\cD{l}ihood,} \\

S\cU{a}mmā-v\cD{ā}yāmo &
\tr{Right \cD{E}ffort,} \\

S\cU{a}mmā-s\cD{a}ti &
\tr{Right M\cU{i}ndf\cD{u}lness,} \\

S\cU{a}mmā-s\cD{a}mādhi &
\tr{R\cD{i}ght C\cD{o}nc\cD{e}nt\cD{r}ation.} \\

\end{twochants}

Idaṃ dukkhaṃ \cD{a}riya-s\cD{a}ccan-\cD{t}i me bh\cU{i}kk\cD{h}ave\\
Pubbe \cD{a}nanuss\cD{u}tes\cD{u} dhammesu\\
Cakkhuṃ \cD{u}d\cU{a}pādi\\
Ñāṇaṃ \cD{u}d\cU{a}pādi\\
Paññā \cD{u}d\cU{a}pādi\\
Vijjā \cD{u}d\cU{a}pādi\\
Āloko \cD{u}d\cU{a}pādi

\begin{english}
  Bhikkhus, in reg\cD{a}rd \cD{t}o th\cD{i}ngs \cD{u}nh\cD{e}ard-\cD{o}f b\cD{e}fore,\\
  Visi\cU{o}n \cD{a}rose,\\
  \cD{I}n\cD{s}ight \cD{a}rose,\\
  Disc\cD{e}rn\cD{m}ent \cD{a}rose,\\
  Knowl\cU{e}dge \cD{a}rose,\\
  \cD{L}ight \cD{a}rose:\\
  This is the N\cD{o}b\cD{l}e Tr\cD{u}th \cD{o}f d\cD{u}kkha;
\end{english}

Taṃ kho p\cD{a}n'idaṃ dukkhaṃ \cD{a}riy\cU{a}-s\cD{a}ccaṃ p\cD{a}riññeyyan-ti

\begin{english}
  Now this N\cD{o}bl\cD{e} T\cD{r}uth \cD{o}f d\cD{u}kkha\\
  Should be completely \cU{u}nd\cD{e}rstood;
\end{english}

Taṃ kho p\cD{a}n'idaṃ dukkhaṃ \cD{a}riy\cU{a}-s\cD{a}ccaṃ p\cD{a}riññātan-ti

\begin{english}
  Now this N\cD{o}bl\cD{e} T\cD{r}uth \cD{o}f d\cD{u}kkha
  Has b\cD{e}en c\cD{o}mpl\cD{e}te\cD{l}y \cD{u}nd\cD{e}rstood.
\end{english}

Idaṃ dukkha-s\cD{a}m\cD{u}dayo \cD{a}riy\cU{a}-s\cD{a}ccan-\cD{t}i me bh\cU{i}kk\cD{h}ave\\
Pubbe \cD{a}nanuss\cD{u}tes\cD{u} dhammesu\\
Cakkhuṃ \cD{u}d\cU{a}pādi\\
Ñāṇaṃ \cD{u}d\cU{a}pādi\\
Paññā \cD{u}d\cU{a}pādi\\
Vijjā \cD{u}d\cU{a}pādi\\
Āloko \cD{u}d\cU{a}pādi

\begin{english}
  Bhikkhus, in reg\cD{a}rd \cD{t}o th\cD{i}ngs \cD{u}nh\cD{e}ard-\cD{o}f b\cD{e}fore,\\
  Visi\cU{o}n \cD{a}rose,\\
  \cD{I}n\cD{s}ight \cD{a}rose,\\
  Disc\cD{e}rn\cD{m}ent \cD{a}rose,\\
  Knowl\cU{e}dge \cD{a}rose,\\
  \cD{L}ight \cD{a}rose:\\
  This is the N\cD{o}bl\cD{e} Tr\cD{u}th \cD{o}f t\cD{h}e c\cD{a}use \cD{o}f d\cD{u}kkha.
\end{english}

\clearpage

Taṃ kho p\cD{a}n'idaṃ dukkha-s\cD{a}m\cD{u}dayo \cD{a}riy\cU{a}-s\cD{a}ccaṃ p\cD{a}hāt\cD{a}bban-ti

\begin{english}
  Now this c\cD{a}use \cD{o}f d\cD{u}kkha\\
  Sh\cD{o}uld \cD{b}e \cD{a}b\cD{a}ndoned;
\end{english}

Taṃ kho p\cD{a}n'idaṃ dukkha-s\cD{a}m\cD{u}dayo \cD{a}riy\cU{a}-s\cD{a}ccaṃ p\cD{a}hīnan-ti

\begin{english}
  Now this c\cD{a}use \cD{o}f d\cD{u}kkha\\
  H\cD{a}s b\cD{e}en \cD{a}b\cD{a}ndoned.
\end{english}

Idaṃ dukkh\cD{a}-nirodho \cD{a}riy\cU{a}-s\cD{a}ccan-\cD{t}i me bh\cU{i}kk\cD{h}ave\\
Pubbe \cD{a}nanuss\cD{u}tes\cD{u} dhammesu\\
Cakkhuṃ \cD{u}d\cU{a}pādi\\
Ñāṇaṃ \cD{u}d\cU{a}pādi\\
Paññā \cD{u}d\cU{a}pādi\\
Vijjā \cD{u}d\cU{a}pādi\\
Āloko \cD{u}d\cU{a}pādi

\begin{english}
  Bhikkhus, in reg\cD{a}rd \cD{t}o th\cD{i}ngs \cD{u}nh\cD{e}ard-\cD{o}f b\cD{e}fore,\\
  Visi\cU{o}n \cD{a}rose,\\
  \cD{I}n\cD{s}ight \cD{a}rose,\\
  Disc\cD{e}rn\cD{m}ent \cD{a}rose,\\
  Knowl\cU{e}dge \cD{a}rose,\\
  \cD{L}ight \cD{a}rose:\\
  This is the N\cD{o}bl\cD{e} T\cD{r}uth \cD{o}f t\cD{h}e c\cD{e}ss\cD{a}t\cD{i}on \cD{o}f d\cD{u}kkha;
\end{english}

Taṃ kho p\cD{a}n'idaṃ dukkha-nirodho \cD{a}riy\cU{a}-s\cD{a}ccaṃ sacc\cD{h}i-kāt\cU{a}bban-ti

\begin{english}
  Now the c\cU{e}ssation \cD{o}f d\cD{u}kkha\\
  Should be exp\cD{e}ri\cD{e}nced d\cD{i}r\cD{e}ctly;
\end{english}

Taṃ kho p\cD{a}n'idaṃ dukkha-nirodho \cD{a}riy\cU{a}-s\cD{a}ccaṃ sacc\cD{h}ik\cD{a}tan-ti

\begin{english}
  Now the c\cU{e}ssation \cD{o}f d\cD{u}kkha\\
  H\cD{a}s b\cD{e}en \cD{e}xp\cD{e}ri\cD{e}nced d\cD{i}r\cD{e}ctly.
\end{english}

Idaṃ dukkh\cD{a}-nirodh\cD{a}-gām\cU{i}nī-p\cD{a}ṭ\cD{i}p\cD{a}dā \cD{a}riy\cU{a}-s\cD{a}ccan-\cD{t}i me bh\cU{i}kk\cD{h}ave\\
Pubbe \cD{a}nanuss\cD{u}tes\cD{u} dhammesu\\
Cakkhuṃ \cD{u}d\cU{a}pādi\\
Ñāṇaṃ \cD{u}d\cU{a}pādi\\
Paññā \cD{u}d\cU{a}pādi\\
Vijjā \cD{u}d\cU{a}pādi\\
Āloko \cD{u}d\cU{a}pādi

\begin{english}
  Bhikkhus, in reg\cD{a}rd \cD{t}o th\cD{i}ngs \cD{u}nh\cD{e}ard-\cD{o}f b\cD{e}fore,\\
  Visi\cU{o}n \cD{a}rose,\\
  \cD{I}n\cD{s}ight \cD{a}rose,\\
  Disc\cD{e}rn\cD{m}ent \cD{a}rose,\\
  Knowl\cU{e}dge \cD{a}rose,\\
  \cD{L}ight \cD{a}rose:\\
  This is the N\cD{o}bl\cD{e} Tr\cD{u}th \cD{o}f t\cD{h}e w\cD{a}y \cD{o}f pr\cD{a}ctice leading to the c\cU{e}ssation of d\cD{u}kkha;
\end{english}

Taṃ kho p\cD{a}n'idaṃ dukkha-nirodha-gām\cU{i}nī-p\cD{a}ṭ\cD{i}p\cD{a}dā \cD{a}riy\cU{a}-s\cD{a}ccaṃ bhāvetabban-ti

\begin{english}
  Now this w\cD{a}y \cD{o}f pr\cD{a}ctice leading to the c\cU{e}ssation of d\cD{u}kkha\\
  Sh\cD{o}uld b\cD{e} d\cD{e}v\cD{e}loped;
\end{english}

Taṃ kho p\cD{a}n'idaṃ dukkha-nirodha-gām\cU{i}nī-p\cD{a}ṭ\cD{i}p\cD{a}dā \cD{a}riy\cU{a}-s\cD{a}ccaṃ bhāv\cU{i}tan-ti

\begin{english}
  Now this w\cD{a}y \cD{o}f pr\cD{a}ctice leading to the c\cU{e}ssation of d\cD{u}kkha\\
  H\cD{a}s b\cD{e}en d\cD{e}v\cD{e}loped.
\end{english}

Yāva-k\cD{ī}vañ-c\cD{a} me bh\cU{i}kk\cD{h}ave \cD{i}mes\cD{u} c\cD{a}tūsu \cD{a}riy\cU{a}-s\cD{a}ccesu\\
Evan-\cD{t}i-p\cD{a}rivaṭṭaṃ dvādas'\cU{ā}k\cU{ā}raṃ yath\cU{ā}-bhūtaṃ ñāṇa-dass\cD{a}naṃ na s\cD{u}v\cU{i}s\cD{u}ddhaṃ \cD{a}hosi

\begin{english}
  As long, bh\cD{i}kkhus, as my knowledge and underst\cD{a}nding,\\
  As it a\cD{c}tu\cD{a}ll\cD{y} is,\\
  Of these Four N\cU{o}b\cD{l}e Truths,\\
  With their three ph\cD{a}s\cD{e}s a\cD{n}d tw\cD{e}lve \cD{a}spects,\\
  Was n\cD{o}t \cD{e}n\cU{t}ire\cD{l}y pure,
\end{english}

N'eva tāvāhaṃ bh\cU{i}kk\cD{h}ave s\cD{a}dev\cU{a}ke loke s\cD{a}mār\cU{a}ke s\cD{a}brahm\cU{a}ke\\
Sassamaṇa-brāhmaṇiyā p\cD{a}jāya s\cD{a}deva-m\cD{a}nuss\cU{ā}ya\\
An\cU{u}tt\cD{a}raṃ s\cU{a}mmā-s\cU{a}mbodhiṃ \cD{a}bhis\cU{a}mbuddho p\cD{a}ccaññāsiṃ

\begin{english}
  Did I not cl\cD{a}im, bh\cD{i}kkhus,\\
  In this world of d\cD{e}vas M\cD{ā}r\cD{a} a\cD{n}d B\cD{r}ahmā,\\
  Amongst m\cD{a}nkind with its priests and ren\cU{u}nc\cD{i}ants,\\
  Kings and c\cU{o}mm\cD{o}ners,\\
  An \cD{u}lt\cD{i}m\cD{a}te \cD{a}w\cD{a}k\cD{e}ning\\
  To uns\cU{u}rpassed p\cD{e}rf\cD{e}ct \cD{e}nl\cU{i}ght\cD{e}nment.
\end{english}

Y\cD{a}to c\cD{a} kho me bh\cU{i}kk\cD{h}ave \cD{i}mes\cD{u} c\cD{a}tūsu \cD{a}riy\cU{a}-s\cD{a}ccesu\\
Evan-\cD{t}i-p\cD{a}rivaṭṭaṃ dvādas'\cU{ā}k\cU{ā}raṃ yath\cU{ā}-bhūtaṃ ñāṇa-dassanaṃ s\cD{u}v\cU{i}s\cD{u}ddhaṃ ahosi

\begin{english}
  But when, bh\cD{i}kkhus, my knowledge and underst\cD{a}nding\\
  As it a\cD{c}tu\cD{a}ll\cD{y} is,\\
  Of these Four N\cU{o}b\cD{l}e Truths,\\
  With their three ph\cD{a}s\cD{e}s a\cD{n}d tw\cD{e}lve \cD{a}spects,\\
  Was ind\cD{e}ed \cD{e}n\cU{t}ire\cD{l}y pure,
\end{english}

Athāhaṃ bh\cU{i}kk\cD{h}ave s\cD{a}dev\cU{a}ke loke s\cD{a}mā\cU{r}ake s\cD{a}brahm\cU{a}ke\\
Sassamaṇa-brāhmaṇiyā p\cD{a}jāya s\cD{a}deva-m\cD{a}nuss\cU{ā}ya\\
An\cU{u}tt\cD{a}raṃ s\cU{a}mmā-s\cU{a}mbodhiṃ \cD{a}bhis\cU{a}mbuddho p\cD{a}cc\cD{a}ññ\cD{ā}siṃ

\begin{english}
  T\cD{h}en \cD{i}ndeed did I cl\cD{a}im, bh\cD{i}kkhus,\\
  In this world of d\cD{e}vas, M\cD{ā}r\cD{a} a\cD{n}d B\cD{r}ahmā,\\
  Amongst m\cD{a}nkind with its priests and ren\cU{u}nc\cD{i}ants,\\
  Kings and c\cU{o}mm\cD{o}ners,\\
  An \cD{u}lt\cD{i}m\cD{a}te \cD{a}w\cD{a}k\cD{e}ning\\
  To uns\cU{u}rpassed, p\cD{e}rf\cD{e}ct \cD{e}nl\cU{i}ght\cD{e}nment.
\end{english}

Ñāṇañ-ca pana me das\cD{s}anaṃ \cD{u}d\cU{a}pādi

\begin{english}
  Now kn\cD{o}wl\cD{e}dge a\cD{n}d \cD{u}nd\cD{e}rs\cD{t}anding ar\cD{o}se \cD{i}n me:
\end{english}

\cD{A}kuppā me v\cU{i}mutti

\begin{english}
  My release \cD{i}s uns\cU{h}akeable,
\end{english}

\cD{A}yam-an\cD{t}imā jāti

\begin{english}
  This is my l\cD{a}st birth,
\end{english}

N'atth\cU{i}dāni p\cD{u}nabb\cD{h}avo-ti

\begin{english}
  There won't be \cD{a}n\cD{y} f\cD{u}rt\cD{h}er b\cD{e}c\cD{o}ming.
\end{english}

\chapter[Striving according to Dhamma]{The Teaching on striving according to Dhamma}% {{{1

\begin{leader}
  [Handa mayaṃ dhamma-pahaṃsāna-pāṭham bhaṇāmase]
\end{leader}

Evaṃ sv\cD{ā}kkhāto bh\cU{i}kk\cD{h}ave mayā dhammo

\begin{english}
  Bhikkhus, t\cD{h}e Dhamma has thus been w\cD{e}ll exp\cU{o}unded by me,
\end{english}

\begin{twochants}
Uttāno &
\tr{El\cD{u}c\cD{i}d\cD{a}ted,} \\

V\cU{i}v\cD{a}ṭo &
\tr{D\cD{i}sclosed,} \\

P\cD{a}kās\cU{i}to &
\tr{R\cD{e}vealed} \\

Ch\cU{i}nna-p\cD{i}lot\cU{i}ko &
\tr{A\cD{n}d st\cD{r}ipped \cD{o}f p\cD{a}tchwork --} \\
\end{twochants}

Alam-eva s\cD{a}ddhā-p\cD{a}bba\cU{j}itena kula-p\cD{u}ttena v\cU{ī}riyaṃ \cD{ā}rab\cD{h}ituṃ

\begin{english}
  This is eno\cD{u}gh f\cD{o}r \cD{a} c\cD{l}ansman,\\
  Who has g\cD{o}ne forth out \cD{o}f faith,\\
  To aro\cD{u}se \cD{h}is \cU{e}n\cU{e}rg\cD{y} thus:
\end{english}

Kāmaṃ t\cD{a}co ca nah\cU{ā}ru \cD{c}a aṭṭhi \cD{c}a \cD{a}vas\cU{i}s\cD{s}atu

\begin{english}
  `Willingly let \cD{o}nl\cD{y} m\cD{y} skin, s\cD{i}n\cD{e}ws \cD{a}nd b\cD{o}nes r\cD{e}main,
\end{english}

S\cD{a}rīre \cD{u}pasus\cU{s}atu maṃs\cD{a}-loh\cD{i}taṃ

\begin{english}
  And let t\cU{h}e flesh and bl\cD{o}od \cD{i}n t\cD{h}is b\cD{o}dy w\cD{i}t\cD{h}er \cD{a}way.
\end{english}

\begin{tabular}{@{}p{0.5\linewidth} p{0.5\linewidth}@{}}

Yaṃ taṃ &
\tr{As long as what\cD{e}v\cD{e}r \cD{i}s \cD{t}o \cD{b}e \cD{a}ttained}\\

P\cD{u}risa-thāmena &
\tr{By hum\cU{a}n strength,} \\

P\cD{u}risa-v\cU{ī}riyena &
\tr{By human \cU{e}n\cD{e}rgy,} \\

\end{tabular}

\begin{tabular}{@{}p{0.5\linewidth} p{0.5\linewidth}@{}}

P\cD{u}risa-p\cD{a}rak\cD{k}amena &
\tr{\cD{B}y h\cD{u}m\cD{a}n \cD{e}ffort} \\

P\cD{a}tt\cD{a}bbaṃ na taṃ \cD{a}pāp\cD{u}ṇitvā &
\tr{Has not b\cU{e}en \cD{a}ttained,} \\

V\cU{ī}riyassa s\cU{a}ṇṭhānaṃ bh\cD{a}viss\cD{a}tī-ti &
\tr{Let n\cD{o}t \cD{m}y \cD{e}f\cD{f}orts s\cD{t}and still.'} \\

\end{tabular}

Dukkhaṃ bh\cU{i}kk\cD{h}ave kus\cU{ī}to v\cU{i}h\cD{a}rati

\begin{english}
  Bhikkhus, the la\cU{z}y person dw\cD{e}lls \cD{i}n s\cU{u}ff\cD{e}ring,
\end{english}

Vok\cD{i}ṇṇo pāp\cD{a}kehi \cD{a}k\cD{u}sale\cD{h}i dhammehi

\begin{english}
  Soiled by \cD{e}v\cD{i}l, \cD{u}nwh\cD{o}les\cD{o}me states
\end{english}

Mah\cU{a}ntañ-ca s\cD{a}d\cD{a}tthaṃ p\cD{a}r\cU{i}hāpeti

\begin{english}
  And great is t\cU{h}e personal g\cD{o}od th\cD{a}t \cD{h}e n\cD{e}glects.
\end{english}

Āraddha-v\cU{ī}riyo \cD{c}a kho bh\cU{i}kk\cD{h}ave s\cD{u}khaṃ v\cU{i}h\cD{a}rati

\begin{english}
  The en\cU{e}rgetic p\cD{e}r\cD{s}on th\cD{o}ugh d\cD{w}ells h\cD{a}pp\cD{i}ly,
\end{english}

P\cD{a}vivitto pāp\cD{a}ke\cD{h}i \cD{a}k\cD{u}sale\cD{h}i dhammehi

\begin{english}
  Well withdrawn from unwh\cU{o}les\cD{o}me states
\end{english}

Mah\cU{a}ntañ-ca s\cD{a}d\cD{a}tthaṃ p\cD{a}r\cU{i}pūreti

\begin{english}
  And great is t\cU{h}e personal g\cD{o}od th\cD{a}t \cD{h}e \cD{a}chieves.
\end{english}

Na bh\cU{i}kk\cD{h}ave h\cD{ī}nena \cD{a}gg\cD{a}ss\cD{a} p\cD{a}t\cU{t}i hoti

\begin{english}
  Bhikkhus, it \cU{i}s not by l\cD{o}w\cD{e}r means that the supr\cD{e}me \cD{i}s \cD{a}ttained
\end{english}

Aggena ca kho bh\cU{i}kk\cD{h}ave \cD{a}gg\cD{a}ss\cD{a} p\cD{a}t\cU{t}i hoti

\begin{english}
  But, bhikkhus, it is by t\cU{h}e s\cD{u}preme that the supr\cD{e}me \cD{i}s \cD{a}ttained.
\end{english}

Maṇḍap\cU{e}yyam-\cU{i}daṃ bh\cU{i}kk\cD{h}ave brahmac\cD{a}r\cU{i}yaṃ

\begin{english}
  Bhikkhus, this \cU{h}o\cD{l}y life is like the cr\cD{e}am \cD{o}f \cD{t}he milk:
\end{english}

Satthā sammukh\cU{ī}-bh\cD{ū}to

\begin{english}
  The T\cD{e}ac\cD{h}er \cD{i}s p\cD{r}esent,
\end{english}

Tasmāt\cD{i}ha bh\cU{i}kk\cD{h}ave v\cU{ī}riyaṃ ārabh\cD{a}tha

\begin{english}
  Therefore, b\cD{h}ikkhus, st\cD{a}rt \cD{t}o \cD{a}ro\cD{u}se your \cU{e}n\cD{e}rgy
\end{english}

{\setlength{\tabcolsep}{0.1em}

\begin{tabular}{@{}p{0.45\linewidth} p{0.6\linewidth}@{}}

\cD{A}pp\cD{a}tt\cD{a}ss\cD{a} p\cD{a}t\cU{t}iyā &
\tr{For the \cU{a}ttainment of the as y\cD{e}t \cD{u}n\cD{a}ttained,} \\

Anadh\cU{i}g\cD{a}tassa \cD{a}dhig\cD{a}māya &
\tr{For the \cU{a}chievement of the as y\cD{e}t \cD{u}n\cD{a}chieved,} \\

As\cD{a}cch\cD{i}k\cD{a}tassa s\cD{a}cch\cD{i}k\cD{i}r\cU{i}yāya &
\tr{For the real\cU{i}zation of the as y\cD{e}t \cD{u}nr\cD{e}alized.} \\

\end{tabular}

}

`Evaṃ no ayaṃ amhākaṃ p\cD{a}b\cD{b}ajjā \cD{a}vaṅk\cD{a}tā \cD{a}vañjhā bh\cD{a}v\cU{i}ssati

\begin{english}
  Thinking, in s\cD{u}ch \cD{a} way: `Our G\cU{o}\cD{i}ng Forth will n\cD{o}t \cD{b}e b\cD{a}rren
\end{english}

S\cD{a}phalā \cD{s}a-\cD{u}dr\cU{a}yā.

\begin{english}
  But will b\cU{e}come fr\cD{u}itf\cD{u}l a\cD{n}d f\cD{e}rtile,
\end{english}

Yes\cU{a}ṃ mayaṃ p\cD{a}ribhuñjāma cīv\cU{a}ra-piṇḍ\cD{a}pāta-\\
S\cU{e}nāsana-g\cU{i}lānapp\cD{a}ccaya-bhes\cD{a}jja-parikkh\cU{ā}raṃ\\
Tesaṃ te kārā \cD{a}mhesu

\begin{english}
  And all our u\cD{s}e \cD{o}f robes, \cD{a}lmsfood,\\
  \cD{l}odgings, and medic\cD{i}nal r\cU{e}qu\cD{i}sites,\\
  Given by \cD{o}t\cD{h}ers f\cD{o}r o\cD{u}r s\cD{u}pport,
\end{english}

M\cD{a}happ\cD{h}alā bhavissanti m\cD{a}hā-n\cD{i}s\cU{a}ṃsā-ti

\begin{english}
  Will rew\cD{a}rd t\cD{h}em w\cD{i}th gr\cD{e}at fruit and great b\cU{e}n\cD{e}fit.'
\end{english}

Evaṃ h\cD{i} vo bh\cU{i}kk\cD{h}ave s\cD{i}kk\cD{h}i\cD{t}abbaṃ

\begin{english}
  Bhikkhus, you should tr\cD{a}in y\cD{o}urs\cD{e}lves thus:
\end{english}

\cD{A}tt'atthaṃ vā hi bh\cU{i}k\cD{k}have s\cU{a}mpass\cD{a}mānena

\begin{english}
  C\cU{o}nsidering your o\cD{w}n good,
\end{english}

\cD{A}lam-eva \cD{a}ppamādena s\cU{a}mpādetuṃ

\begin{english}
  It is \cU{e}nough to st\cD{r}ive f\cD{o}r t\cD{h}e g\cD{o}al w\cD{i}th\cD{o}ut n\cD{e}gligence;
\end{english}

P\cD{a}r'atthaṃ vā hi bh\cU{i}kk\cD{h}ave s\cU{a}mpas\cD{s}amānena

\begin{english}
  Bhikkhus, c\cU{o}nsidering the g\cD{o}od \cD{o}f \cD{o}thers,
\end{english}

\cD{A}lam-eva \cD{a}ppamādena s\cU{a}mpād\cD{e}tuṃ

\begin{english}
  It is \cU{e}noughto st\cD{r}ive f\cD{o}r t\cD{h}e g\cD{o}al w\cD{i}th\cD{o}ut n\cD{e}gligence;
\end{english}

\cD{U}bhay'atthaṃ vā hi bh\cU{i}kk\cD{h}ave s\cU{a}mpass\cD{a}mānena

\begin{english}
  Bhikkhus, c\cU{o}nsidering the g\cD{o}od \cD{o}f both,
\end{english}

Alam-eva \cD{a}ppamādena s\cU{a}mpād\cD{e}tun-ti

\begin{english}
  It is \cU{e}nough to st\cD{r}ive f\cD{o}r t\cD{h}e g\cD{o}al w\cD{i}th\cD{o}ut n\cD{e}gligence.
\end{english}

\chapter{The Verses of Tāyana}% {{{1

\begin{leader}
  [Handa mayaṃ tāyana-gāthāyo bhaṇāmase]
\end{leader}

\begin{twochants}
  Ch\cU{i}nda s\cD{o}taṃ p\cD{a}rakkamma & K\cD{ā}me panūda br\cU{ā}h\cD{m}aṇa \\
  Nappah\cU{ā}ya m\cD{u}ni kāme & Nekattam-up\cD{a}pajj\cD{a}ti \\
\end{twochants}

\begin{english}
  Exert yourself \cD{a}nd cut \cD{t}he stream.\\
  Discard sense-p\cU{l}eas\cD{u}res, Hol\cD{y} Man;\\
  Not letting sens\cD{u}al pleas\cD{u}res go,\\
  A sage will n\cU{o}t r\cD{e}ach un\cD{i}ty.
\end{english}

\begin{twochants}
  Kayirā ce k\cD{a}yirāth\cU{e}naṃ & D\cD{a}ḷham-enaṃ p\cD{a}rakk\cD{a}me \\
  Sithilo hi p\cD{a}ribbājo & B\cD{h}iyyo ākir\cD{a}te r\cD{a}jaṃ \\
\end{twochants}

\begin{english}
  Vigorously, w\cD{i}th all o\cD{n}e's strength,\\
  It should be d\cU{o}ne, w\cD{h}at should \cD{b}e done;\\
  A lax monas\cD{t}ic life st\cD{i}rs up\\
  The dust of p\cU{a}ssi\cD{o}ns all t\cD{h}e more
\end{english}

\begin{twochants}
  \cD{A}kataṃ dukkaṭaṃ s\cU{e}yyo & Pacchā tappati d\cU{u}k\cD{k}aṭaṃ \\
  Katañ-ca s\cD{u}k\cU{a}taṃ seyyo & Yaṃ k\cD{a}tvā nānut\cD{a}pp\cD{a}ti \\
\end{twochants}

\begin{english}
  Better is not \cD{t}o do b\cD{a}d deeds\\
  That afterw\cU{a}rds w\cD{o}uld bring r\cD{e}morse;\\
  It's rather g\cU{o}od d\cD{e}eds one sh\cD{o}uld do\\
  Which having done o\cD{n}e won't r\cD{e}gret.
\end{english}

\begin{twochants}
  Kus\cU{o} y\cD{a}thā d\cD{u}ggah\cD{i}to & Hattham-ev\cU{ā}n\cD{u}kan\cD{t}ati \\
  S\cU{ā}maññaṃ d\cD{u}pparām\cD{a}ṭṭhaṃ & Nirayāyūp\cD{a}kaḍḍ\cD{h}ati \\
\end{twochants}

\begin{english}
  As Kusa-grass, w\cD{h}en wrongl\cD{y} grasped,\\
  Will only c\cU{u}t \cD{i}nto o\cD{n}e's hand\\
  So does t\cD{h}e monk's li\cD{f}e wrong\cD{l}y led\\
  Indeed drag o\cU{n}e \cD{t}o hel\cD{l}ish states.
\end{english}

\begin{twochants}
  Yaṃ-kiñci s\cD{i}th\cU{i}laṃ kammaṃ & S\cU{a}ṅk\cD{i}liṭṭh\cU{a}ñ-c\cD{a} yaṃ v\cU{a}taṃ \\
  S\cU{a}ṅk\cD{a}ssaraṃ brahma-c\cD{a}r\cU{i}yaṃ & Na taṃ h\cU{o}ti m\cD{a}happ\cD{h}alan-ti \\
\end{twochants}

\begin{english}
  Whatev\cD{e}r deed th\cD{a}t's slack\cD{l}y done,\\
  Whatever v\cU{o}w c\cD{o}rrupt\cD{l}y kept,\\
  The Holy Life l\cD{e}d i\cD{n} doubt\cD{f}ul ways --\\
  All these will n\cU{e}v\cD{e}r bear g\cD{r}eat fruits.
\end{english}

% }}}1

% End of reflections-and-recollections-p3.tex
