\chapter{Pāli Pronunciation}

The Roman lettering of Pāli words is based on the pronunciation of
English, with the following clarifications:

\section{Vowels}

\begin{tabular}{@{} ll @{}}

\begin{tabular}{@{} ll @{}}
  Short & Long\\
  \textbf{a} as in \prul{a}bout &
  \textbf{ā} as in f\prul{a}ther\\
  \textbf{i} as in h\prul{i}t &
  \textbf{ī} as in mach\prul{i}ne\\
  \textbf{u} as in p\prul{u}t &
  \textbf{ū} as in r\prul{u}le\\
  & \textbf{e} as in gr\prul{e}y\\
  & \textbf{o} as in m\prul{o}re\\
\end{tabular} &

\parbox[b][][t]{0.5\linewidth}{%
Exceptions: \textbf{e} and \textbf{o} change to short sounds in syllables
ending in consonants. They are then pronounced as in “g\prul{e}t” and
“\prul{o}x”, respectively.} \\

\end{tabular}

\section{Consonants}

\textbf{c} as in an\prul{c}ient (like \prul{ch} but unaspirated)

\textbf{ṃ, ṅ} as \prul{ng} in sa\prul{ng}

\textbf{ñ} as \prul{ny} in ca\prul{ny}on

\textbf{v} rather softer than the English \prul{v}; near \prul{w}

\textbf{cc} is a double \prul{c} as in Fibona\prul{cc}i, never pronounced as in a\prul{cc}ount

\subsection{Aspirated consonants}

\textbf{bh ch dh ḍh gh jh kh ph th ṭh}

These two-lettered notations with \prul{h} denote an aspirated, airy sound,
distinct from the hard, crisp sound of the single consonant. They should be
considered as one unit.

However, the other combinations with \textbf{h,} i.e., \textbf{lh, mh, ñh,} and
\textbf{vh,} do count as two consonants (for example in the Pāli words
‘ji\textbf{vh}ā’ or ‘mu\textbf{ḷh}o’).

\clearpage

\subsection{Examples}

\textbf{th} as \prul{t} in \prul{t}ongue. (Never pronounced as in `\prul{th}e'.)

\textbf{ph} as \prul{p} in \prul{p}alate. (Never pronounced as in `\prul{ph}oto'.)

These are distinct from the hard, crisp sound of the single consonant, e.g.
\textbf{th} as in “\prul{Th}omas” (not as in “\prul{th}in”) or \textbf{ph} as
in “\prul{p}uff” (not as in “\prul{ph}one”).

\subsection{Retroflex consonants}

\textbf{ḍ ḍh ḷ ṇ ṭ ṭh}

These retroflex consonants have no English equivalents. They are sounded
by curling the tip of the tongue back against the palate.

\section{Separating the syllables}

\textbf{bh, ch, dh, kh\ldots} are aspirated consonants. They count as
\textit{one consonant} and are not divided.

\textbf{bbh, cch, ddh, kkh\ldots} are double (aspirated) consonants. They count as
\textit{two consonants} and \textit{are} divided.

Therefore \textbf{am·hā·kaṃ}, but \textbf{sa·dham·maṃ}, not \textbf{sad·ham·maṃ}.
\textbf{Bud·dho} and not \textbf{Bu·ddho}.

Precise pronunciation and correct separation of the syllables is
especially important when someone is interested in learning Pāli and to
understand and memorize the meaning of Suttas and other chants,
otherwise the meaning of it will get distorted.

\textbf{An example to illustrate this:}

The Pāli word ‘\textbf{sukka}’ means ‘bright’; ‘\textbf{sukkha}’ means
‘dry’; ‘\textbf{sukha}’ --- ‘happiness’; ‘\textbf{suka}’ --- ‘parrot’ and
‘\textbf{sūka}’ --- ‘bristles on an ear of barley’.

So if you chant ‘\textbf{sukha}’ with a ‘\textbf{k}’ instead of a
‘\textbf{kh}’, you would chant ‘parrot’ instead of ‘happiness’.

% End of pali-pronunciation.tex
