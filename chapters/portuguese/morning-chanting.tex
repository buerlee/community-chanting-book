\chapter{Dedicação de Oferendas}

[Yo so] bha꜕gavā a꜕rahaṃ sammāsambuddho

\begin{english}
Ao Excelso, o Mestre, que totalmente alcançou a iluminação perfeita, 
\end{english}

Svākkhā꜓to yena bha꜕gava꜓tā dhammo

\begin{english}
Ao ensinamento, que Ele tão bem explicou,
\end{english}

Supaṭi꜕panno yassa bha꜕gava꜕to sāvaka꜕saṅgho

\begin{english}
E aos discípulos do Excelso, que tão bem praticaram,
\end{english}

Tam-ma꜓yaṃ bha꜕gavantaṃ sa꜕dhammaṃ sa꜕saṅghaṃ

\begin{english}
A estes – ao Buddha, ao Dhamma e ao Sangha ---
\end{english}

Imehi꜓ sakkārehi꜕ yathārahaṃ āropi꜕tehi a꜕bhi꜓pūja꜕yāma

\begin{english}
Apresentamos a devida homenagem com oferendas.
\end{english}

Sādhu꜓ no bhante bha꜕gavā su꜕cira-parinibbu꜕topi

\begin{english}
Para nós, é bom que tendo o Excelso se libertado,
\end{english}

Pacchi꜓mā-ja꜕na꜓tānu꜓kampa꜕-mānasā

\begin{english}
Ainda teve compaixão pelas gerações futuras.
\end{english}

Ime sakkāre dugga꜕ta꜕-paṇṇākāra꜓-bhūte pa꜕ṭiggaṇhātu

\begin{english}
Que estas simples oferendas sejam aceites
\end{english}

Amhā꜓kaṃ dīgha꜕rattaṃ hi꜕tāya su꜕khāya

\begin{english}
Pelo nosso duradouro benefício e pela felicidade que nos dá.
\end{english}

\clearpage

Arahaṃ sammāsambuddho bha꜕gavā

\begin{english}
Ao Mestre, O perfeitamente Iluminado e Excelso ---
\end{english}

Buddhaṃ bha꜕gavantaṃ a꜕bhi꜓vādemi

\begin{english}
  Ao Buddha, o Excelso, eu presto homenagem.
  \instr{Vénia}
\end{english}

[Svākkhā꜓to] bha꜕gava꜓tā dhammo

\begin{english}
 Ao ensinamento, tão plenamente explicado por Ele ---
\end{english}

Dhammaṃ namassāmi

\begin{english}
  Ao Dhamma, eu presto homenagem.
  \instr{Vénia}
\end{english}

[Supaṭi꜕panno] bha꜕gava꜕to sāvaka꜕saṅgho

\begin{english}
Aos discípulos do Excelso que tão bem praticaram ---
\end{english}

Sa꜓ṅghaṃ na꜕māmi

\begin{english}
  Ao Sangha, eu presto homenagem.
  \instr{Vénia}
\end{english}

\chapter{Homenagem Preliminar}

\begin{leader}
  [Ha꜓nda mayaṃ buddhassa꜕ bha꜕gavato pubbabhāga-namakā꜕raṃ karomase]
\end{leader}

\begin{english}
  [Prestemos agora homenagem preliminar ao Buddha.]
\end{english}

\vspace{\baselineskip}

Namo tassa bha꜕gava꜕to araha꜕to sa꜓mmāsa꜓mbuddha꜕ssa

\instr{Três vezes}

\begin{english}
  Homenagem ao Excelso, Nobre e Perfeitamente Iluminado.

  \instr{Três vezes}
\end{english}

\clearpage

\chapter{Homenagem ao Buddha}

\begin{leader}
  [Ha꜓nda mayaṃ buddhābhi꜕tthu꜕tiṃ karomase]
\end{leader}

\begin{english}
  [Cantemos agora em elogio ao Buddha.]
\end{english}

Yo so tathā꜓ga꜕to a꜕rahaṃ sammāsambuddho

\begin{english}
  O Tathāgata é puro e perfeitamente iluminado.
\end{english}

Vijjāca꜕raṇa꜓-sampanno

\begin{english}
  Impecável em conduta e compreensão,
\end{english}

Su꜕ga꜕to

\begin{english}
  Realizado,
\end{english}

Loka꜕vi꜓dū

\begin{english}
  Conhecedor dos mundos.
\end{english}

Anu꜓tta꜕ro purisa꜕damma-sārathi

\begin{english}
  Ele treina perfeitamente aqueles que desejam treinar-se.
\end{english}

Satthā deva-ma꜕nussānaṃ

\begin{english}
  Ele é Professor de deuses e humanos.
\end{english}

Buddho bha꜕gavā

\begin{english}
  Ele é desperto e sagrado.
\end{english}

Yo imaṃ lokaṃ sa꜕devakaṃ sa꜕mārakaṃ sa꜕brahma꜕kaṃ

\begin{english}
  Neste mundo com seus deuses, demónios e espíritos gentis,
\end{english}

Sassa꜓maṇa-brāhmaṇiṃ pa꜕jaṃ sa꜕deva-ma꜕nussa꜓ṃ sa꜕yaṃ a꜕bhiññā sacchika꜕tv꜓ā pa꜕vedesi

\begin{english}
  Seus buscadores e sábios, seres celestiais e humanos,\\Ele revelou a verdade por compreensão profunda.
\end{english}

Yo dhammaṃ dese꜓si ā꜕di꜓-kalyāṇaṃ majjhe꜓-ka꜕lyāṇaṃ \\pa꜕riyosāna-k꜕alyāṇaṃ

\begin{english}
  Ele indicou o Dhamma: Sublime no início, \\Sublime no meio, Sublime no fim.
\end{english}

Sāttha꜓ṃ sa꜕byañjanaṃ kevala-pa꜕ripuṇṇaṃ pa꜕risuddhaṃ \\brahma-ca꜕ri꜓yaṃ pa꜕kāsesi

\begin{english}
  Ele explicou a vida espiritual de completa pureza,\\Na sua essência e convenções.
\end{english}

Tam-aha꜓ṃ bha꜕gavantaṃ a꜕bhi꜓pūja꜕yāmi tam-aha꜓ṃ bha꜕gavantaṃ \\si꜕rasā꜓ na꜕māmi

\begin{english}
  Eu canto o meu elogio ao Excelso, Eu saúdo respeitosamente \\o Excelso.
  \instr{Vénia}
\end{english}

\clearpage

\chapter{Homenagem ao Dhamma}

\begin{leader}
  [Ha꜓nda mayaṃ dhammābhi꜕tthu꜕tiṃ karomase]
\end{leader}

\begin{english}
  [Cantemos agora em elogio ao Dhamma.]
\end{english}

Yo so svākkhā꜓to bha꜕gava꜓tā dhammo

\begin{english}
  O Dhamma é bem explicado pelo Excelso,
\end{english}

Sa꜓ndiṭṭhi꜕ko

\begin{english}
  Imanente aqui e agora,
\end{english}

A꜕kāli꜕ko

\begin{english}
  Intemporal,
\end{english}

Ehi꜕passi꜕ko

\begin{english}
  Encorajando investigação,
\end{english}

Opanayi꜕ko

\begin{english}
  Conduzindo ao interior,
\end{english}

Pa꜕cca꜕ttaṃ vedi꜓ta꜕bbo viññūhi

\begin{english}
  Para ser experimentado individualmente pelos sábios.
\end{english}

Tam-aha꜓ṃ dhammaṃ a꜕bhi꜓pūja꜕yāmi tam-aha꜓ṃ dhammaṃ \\si꜕rasā꜓ na꜕māmi

\begin{english}
  Eu canto o meu elogio a este ensinamento, eu reverencio\\ esta verdade.
  \instr{Vénia}
\end{english}

\clearpage

\chapter{Homenagem ao Saṅgha}

\begin{leader}
  [Ha꜓nda mayaṃ saṅghābhi꜕tthu꜕tiṃ karomase]
\end{leader}

\begin{english}
  [Cantemos agora em elogio ao Sangha.]
\end{english}

Yo so supaṭi꜕panno bha꜕gava꜕to sāvaka꜕saṅgho

\begin{english}
  São os discípulos do Excelso que praticaram correctamente,
\end{english}

Ujupaṭi꜕panno bha꜕gava꜕to sāvaka꜕saṅgho

\begin{english}
  Que praticaram directamente,
\end{english}

Ñāyapaṭi꜕panno bha꜕gava꜕to sāvaka꜕saṅgho

\begin{english}
  Que praticaram reflectidamente,
\end{english}

Sā꜓mīci꜕pa꜕ṭi꜕panno bha꜕gava꜕to sāvaka꜕saṅgho

\begin{english}
  Aqueles que praticaram com integridade ---
\end{english}

Yadidaṃ cattāri purisa꜕yugāni aṭṭha꜓ purisa꜕pugga꜕lā

\begin{english}
  Isto é, os quatro pares, os oito tipos de Seres Nobres ---
\end{english}

Esa bha꜕gava꜕to sāvaka꜕saṅgho

\begin{english}
 Estes são os discípulos do Excelso.
\end{english}

Āhu꜕neyyo

\begin{english}
  Tais discípulos são merecedores de presentes,
\end{english}

Pāhu꜕neyyo

\begin{english}
  Merecedores de hospitalidade,
\end{english}

\clearpage

Dakkhi꜕ṇeyyo

\begin{english}
  Merecedores de oferendas,
\end{english}

Añja꜕li-ka꜕ra꜓ṇīyo

\begin{english}
  Merecedores de respeito;
\end{english}

Anu꜓tta꜕raṃ puññakkhe꜕ttaṃ lokassa

\begin{english}
  Eles promovem o surgimento do bem incomparável \\no mundo.
\end{english}

Tam-aha꜓ṃ saṅghaṃ a꜕bhi꜓pūja꜕yāmi tam-aha꜓ṃ saṅghaṃ \\si꜕rasā꜓ na꜕māmi

\begin{english}
  Eu canto o meu elogio a este Saṅgha,\\
  Eu reverencio este Saṅgha.
  \instr{Vénia}
\end{english}

\clearpage

\chapter{Saudação à Jóia Tríplice}

\begin{leader}
  [Ha꜓nda mayaṃ ratanattaya-paṇāma-gāthā꜓yo c'eva\\
  sa꜓ṃvega-parikittana-pāṭhañca꜕ bhaṇāmase]
\end{leader}

\begin{english}
  [Cantemos agora a nossa saudação à Jóia Tríplice e à passagem \\que estimula o sentido de urgência.]
\end{english}

Buddho su꜕suddho ka꜕ruṇā-maha꜓ṇṇavo

\begin{english}
  O Buddha absolutamente puro, com compaixão como um Oceano,
\end{english}

Yo'ccanta꜕-suddhabba꜕ra-ñāṇa꜕-loca꜕no

\begin{english}
 Possuindo a visão clara da Sabedoria,
\end{english}

Lokassa꜕ pāpūpa꜕ki꜓lesa꜕-ghāta꜕ko

\begin{english}
  Destruidor da corrupção egoísta mundana ---
\end{english}

Vandāmi꜓ buddhaṃ a꜕ha꜓m-āda꜕rena꜕ taṃ

\begin{english}
  Em plena devoção, esse Buddha eu reverencio.
\end{english}

Dhammo pa꜕dīpo vi꜕ya tassa꜕ satthu꜕no

\begin{english}
  O ensinamento do Mestre, como uma lâmpada,
\end{english}

Yo magga꜓-pākāma꜕ta꜕-bheda꜕-bhinna꜕ko

\begin{english}
  Iluminando o caminho e o seu fruto: a Realidade Imortal,
\end{english}

Lokuttaro yo ca꜕ ta꜕d-attha꜕-dīpa꜕no

\begin{english}
  Aquilo que está para além do mundo condicionado ---
\end{english}

Vandāmi꜓ dhammaṃ a꜕ha꜓m-āda꜕rena꜕ taṃ

\begin{english}
  Em plena devoção, esse Dhamma eu reverencio.
\end{english}

Sa꜓ṅgho su꜕khettābhyati-khe꜕tta-sa꜓ññito

\begin{english}
  O Sangha, o melhor terreno para cultivo,
\end{english}

Yo diṭṭha꜓-santo su꜕ga꜕tānu꜕bodha꜕ko

\begin{english}
  Aqueles que realizaram a paz, despertaram a seguir ao \\Realizado,
\end{english}

Lolappa꜕hīno a꜕ri꜓yo su꜕medha꜕so

\begin{english}
  Nobres e Sábios, tendo abandonado todo o anseio, ---
\end{english}

Vandāmi꜓ saṅghaṃ a꜕ha꜓m-āda꜕rena꜕ taṃ

\begin{english}
 Em plena devoção, esse Sangha eu reverencio.
\end{english}

Iccevam-ekanta꜕bhi꜓pūja꜕-neyya꜕kaṃ vatthuttayaṃ \\vanda꜕ya꜕tābhi꜕saṅkha꜕taṃ

\begin{english}
 Esta saudação devia de ser feita ao que é valoroso.
\end{english}

Puññaṃ ma꜕yā yaṃ ma꜕ma꜕ sabbu꜕padda꜕vā mā ho꜓ntu꜕ ve tassa꜕ pa꜕bhāva꜕-siddhi꜕yā

\begin{english}
  Que através do poder desta boa acção, possam todos os obstáculos serem vencidos.
\end{english}

Idha tathā꜓ga꜕to loke u꜕ppanno a꜕rahaṃ sammāsambuddho

\begin{english}
  Aquele que conhece as coisas como são, veio a este mundo e é um Arahant, um ser perfeitamente desperto.
\end{english}

Dhammo ca꜕ desi꜕to niyyāni꜕ko u꜕pa꜕sa꜕miko pa꜕rinibbāni꜕ko sa꜓mbodha꜕gāmī su꜕ga꜕tappa꜕vedi꜕to

\begin{english}
  Purificando o caminho, conduzindo para fora da ilusão, tranquilizando e dirigindo-se para a paz perfeita, conduzindo à Iluminação  --- Este Caminho Ele deu a conhecer.
\end{english}

Ma꜓yan-taṃ dhammaṃ su꜕tvā evaṃ jānāma

\begin{english}
  Tendo ouvido o Ensinamento sabemos o seguinte:
\end{english}

Jātipi꜕ dukkhā

\begin{english}
  O nascimento é dukkha,
\end{english}

Jarāpi꜕ dukkhā

\begin{english}
  O envelhecimento é dukkha,
\end{english}

Ma꜕raṇampi꜕ dukkhaṃ

\begin{english}
  E morte é dukkha;
\end{english}

So꜓ka-pa꜕rideva-dukkha꜕-domanass'u꜕pāyāsā꜓pi꜕ dukkhā

\begin{english}
  Tristeza, lamento, dor, mágoa e desespero são dukkha;
\end{english}

Appiyehi꜕ sa꜓mpa꜕yogo dukkho

\begin{english}
  Associação com o que não se gosta é dukkha;
\end{english}

Piyehi꜕ vi꜓ppa꜕yogo dukkho

\begin{english}
  Separação do que se gosta é dukkha;
\end{english}

Yamp'iccha꜓ṃ na꜕ labhati tampi꜕ dukkhaṃ

\begin{english}
  Não alcançar aquilo que se quer é dukkha.
\end{english}

Sa꜓ṅkhittena pañcu꜕pādānakkha꜓ndhā dukkhā

\begin{english}
  Resumindo, as cinco ópticas da identidade são dukkha.
\end{english}

Seyya꜕thīdaṃ

\begin{english}
  Estas são como se segue:
\end{english}

Rūpūpādāna꜕kkha꜓ndho

\begin{english}
  Apego à forma,
\end{english}

Vedanūpādāna꜕kkha꜓ndho

\begin{english}
  Apego à sensação,
\end{english}

Sa꜓ññūpādāna꜕kkha꜓ndho

\begin{english}
  Apego à percepção,
\end{english}

Sa꜓ṅkhā꜓rūpādāna꜕kkha꜓ndho

\begin{english}
  Apego às formações mentais,
\end{english}

Viññāṇūpādāna꜕kkha꜓ndho

\begin{english}
  Apego à consciência sensorial.
\end{english}

Yesaṃ pa꜕riññāya

\begin{english}
  Para se compreender isto completamente,
\end{english}

Dha꜕ramāno so꜓ bha꜕gavā evaṃ ba꜕hulaṃ sā꜓va꜕ke vi꜕neti

\begin{english}
  O Excelso, durante a sua vida frequentemente instruiu os seus discípulos \\simplesmente desta forma.
\end{english}

Evaṃ bhāgā ca꜕ panassa bha꜕gava꜕to sā꜓va꜕kesu a꜕nusā꜓sa꜕nī ba꜕hulā pa꜕vatta꜕ti

\begin{english}
  Para além disso, Ele ainda instruiu:
\end{english}


Rūpaṃ a꜕niccaṃ

\begin{english}
  A forma é impermanente,
\end{english}

Vedanā a꜕niccā

\begin{english}
  A sensação é impermanente,
\end{english}

Sa꜓ññā a꜕niccā

\begin{english}
  A percepção é impermanente,
\end{english}

Sa꜓ṅkhā꜓rā a꜕niccā

\begin{english}
  As formações mentais são impermanentes,
\end{english}

Viññāṇaṃ a꜕niccaṃ

\begin{english}
  A consciência sensorial é impermanente;
\end{english}

Rūpaṃ a꜕nattā

\begin{english}
  A forma é não-eu,
\end{english}

Vedanā a꜕nattā

\begin{english}
  A sensação é não-eu,
\end{english}

Sa꜓ññā a꜕nattā

\begin{english}
  A percepção é não-eu,
\end{english}

Sa꜓ṅkhā꜓rā a꜕nattā

\begin{english}
  As formações mentais são não-eu,
\end{english}

Viññāṇaṃ a꜕nattā

\begin{english}
  A consciência sensorial é não-eu;
\end{english}

Sa꜕bbe sa꜓ṅkhā꜓rā a꜕niccā

\begin{english}
  Todas as condições são transitórias,
\end{english}

Sa꜕bbe dhammā a꜕nattā'ti

\begin{english}
  Não existe eu no criado ou no incriado.
\end{english}

Te ma꜓yaṃ otiṇṇāmha jāti꜕yā ja꜕rā-maraṇena

\begin{english}
  Todos nós estamos presos pelo nascimento, envelhecimento e morte,
\end{english}

So꜓kehi꜕ pa꜕ridevehi꜕ dukkhe꜓hi꜕ domanassehi꜕ u꜕pāyāsehi

\begin{english}
  Pela tristeza, lamentação, dor, mágoa e desespero,
\end{english}

Dukkho꜓tiṇṇā dukkha꜕-pa꜕retā

\begin{english}
  Presos por dukkha e obstruídos por dukkha.
\end{english}

Appeva nāmi꜓massa꜕ kevalassa꜕ dukkha-kkha꜓ndhassa꜕ anta꜕kiri꜓yā \\paññāyethā'ti

\begin{english}
  Aspiremos todos à total libertação do sofrimento.
\end{english}

\begin{instruction}
  A parte que se segue é cantada somente pelos monges.
\end{instruction}

Ci꜓ra꜓-pari꜕nibbutampi꜓ taṃ bha꜕gava꜓ntaṃ uddissa a꜕raha꜓ntaṃ sammāsambuddhaṃ

\begin{english}
  Relembrando o Excelso, o Nobre Mestre, O Perfeitamente Iluminado, que há muito atingiu o Paranibbana,
\end{english}

Saddhā a꜕gārasmā anagāri꜓yaṃ pabba꜕ji꜕tā

\begin{english}
  Partimos com fé do lar para a vida sem lar monástica,
\end{english}

Tasmi꜓ṃ bha꜕gavati brahma-ca꜕ri꜓yaṃ ca꜕rāma

\begin{english}
  E tal como o Iluminado, praticamos a Vida Sagrada, 
\end{english}

Bhikkhū꜓naṃ si꜓kkhāsā꜕jīva꜕-samāpannā

\begin{english}
  Completamente equipados com o sistema de treino dos Bhikkhus.
\end{english}

Taṃ no brahma-ca꜕ri꜓yaṃ imassa꜕ kevalassa꜕ dukkha-kkha꜓ndhassa꜕ anta꜕kiri꜓yāya sa꜓ṃva꜓tta꜕tu

\begin{english}
  Possa esta Vida Sagrada conduzir-nos ao término de toda esta massa\\  de sofrimento.
\end{english}

\begin{instruction}
  Uma versão alternativa da secção anterior, que pode também ser cantada por leigos.
\end{instruction}

Ci꜓ra꜓-pari꜕nibbutampi꜓ taṃ bha꜕gava꜓ntaṃ saraṇaṃ ga꜕tā

\begin{english}
  O Excelso, que há muito atingiu o Paranibbana, é o nosso refúgio.
\end{english}

Dha꜓mmañca sa꜓ṅghañca

\begin{english}
  Assim também são o Dhamma e o Saṅgha.
\end{english}

Tassa bha꜕gavato sā꜓sanaṃ yathā꜓-sati yathā꜓-balaṃ manasika꜕roma a꜕nupaṭipa꜓jjāma

\begin{english}
  Atentamente seguimos o caminho daquele Excelso, com \prul{toda} a \\nossa plena consciência e força.
\end{english}

Sā꜓ sā꜓ no pa꜕ṭi꜓patti

\begin{english}
  Que então o cultivo desta prática
\end{english}

Imassa꜕ kevalassa꜕ dukkha-kkha꜓ndhassa꜕ anta꜕kiri꜓yāya sa꜓ṃva꜓tta꜕tu

\begin{english}
  Nos conduza ao término de todo o tipo de sofrimento.
\end{english}

\clearpage

\chapter{Homenagem de Encerramento}

[Arahaṃ] sammāsambuddho bha꜕gavā

\begin{english}
  Ao Mestre, O perfeitamente Iluminado e Excelso ---
\end{english}

Buddhaṃ bha꜕gavantaṃ a꜕bhi꜓vādemi

\begin{english}
  Ao Buddha, o Excelso, eu presto homenagem.
  \instr{Vénia}
\end{english}

[Svākkhā꜓to] bha꜕gava꜓tā dhammo

\begin{english}
  Ao ensinamento, tão plenamente explicado por Ele ---
\end{english}

Dhammaṃ namassāmi

\begin{english}
  Ao Dhamma, eu presto homenagem.
  \instr{Vénia}
\end{english}


[Supaṭi꜕panno] bha꜕gava꜕to sāvaka꜕saṅgho

\begin{english}
  Aos discípulos do Excelso que tão bem praticaram ---
\end{english}

Sa꜓ṅghaṃ na꜕māmi

\begin{english}
  Ao Sangha, eu presto homenagem.
  \instr{Vénia}
\end{english}
