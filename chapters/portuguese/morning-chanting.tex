% vim: foldmethod=marker foldlevel=0 foldtext=FoldText()

\chapter{Dedicação de ofertas}   % {{{1

[Yo so] bh\cD{a}gavā \cD{a}rahaṃ sammāsambuddho\\
Svākkh\cU{ā}to yena bh\cD{a}gav\cU{a}tā dhammo\\
Supaṭ\cD{i}panno yassa bh\cD{a}gav\cD{a}to sāvak\cD{a}saṅgho\\
Tam-m\cU{a}yaṃ bh\cD{a}gavantaṃ s\cD{a}dhammaṃ s\cD{a}saṅghaṃ\\
Imeh\cU{i} sakkāreh\cD{i} yathārahaṃ ārop\cD{i}tehi \cD{a}bh\cU{i}pūj\cD{a}yāma\\
Sādh\cU{u} no bhante bh\cD{a}gavā s\cD{u}cira-parinibb\cD{u}topi\\
Pacch\cU{i}mā-j\cD{a}n\cU{a}tān\cU{u}kamp\cD{a}-mānasā\\
Ime sakkāre dugg\cD{a}t\cD{a}-paṇṇākār\cU{a}-bhūte p\cD{a}ṭiggaṇhātu\\
Amh\cU{ā}kaṃ dīgh\cD{a}rattaṃ h\cD{i}tāya s\cD{u}khāya

Arahaṃ sammāsambuddho bh\cD{a}gavā\\
Buddhaṃ bh\cD{a}gavantaṃ \cD{a}bh\cU{i}vādemi \instr{Bow}

[Svākkh\cU{ā}to] bh\cD{a}gav\cU{a}tā dhammo\\
Dhammaṃ namassāmi \instr{Bow}

[Supaṭ\cD{i}panno] bh\cD{a}gav\cD{a}to sāvak\cD{a}saṅgho\\
S\cU{a}ṅghaṃ n\cD{a}māmi \instr{Bow}

\chapter{Homenagem Preliminar}

\begin{leader}
  [H\cU{a}nda mayaṃ buddhass\cD{a} bh\cD{a}gavato pubbabhāga-namak\cD{ā}raṃ karomase]
\end{leader}

Namo tassa bh\cD{a}gav\cD{a}to arah\cD{a}to s\cU{a}mmās\cU{a}mbuddh\cD{a}ssa

\instr{Três tempos}

%}}}1
\clearpage

\chapter{Lembrança do Buddha}     % {{{1

\begin{leader}
  [H\cU{a}nda mayaṃ buddhābh\cD{i}tth\cD{u}tiṃ karomase]
\end{leader}

Yo so tath\cU{ā}g\cD{a}to \cD{a}rahaṃ sammāsambuddho\\
Vijjāc\cD{a}raṇ\cU{a}-sampanno S\cD{u}g\cD{a}to Lok\cD{a}v\cU{i}dū\\
An\cU{u}tt\cD{a}ro puris\cD{a}damma-sārathi\\
Satthā deva-m\cD{a}nussānaṃ Buddho bh\cD{a}gavā

Yo imaṃ lokaṃ s\cD{a}devakaṃ s\cD{a}mārakaṃ s\cD{a}brahm\cD{a}kaṃ\\
Sass\cU{a}maṇa-brāhmaṇiṃ p\cD{a}jaṃ s\cD{a}deva-m\cD{a}nuss\cU{a}ṃ s\cD{a}yaṃ \cD{a}bhiññā sacchik\cD{a}t\cU{v}ā p\cD{a}vedesi\\
Yo dhammaṃ des\cU{e}si \cD{ā}d\cU{i}-kalyāṇaṃ majjh\cU{e}-k\cD{a}lyāṇaṃ\\
p\cD{a}riyosāṇa-\cD{k}alyāṇaṃ\\
Sātth\cU{a}ṃ s\cD{a}byañjaṇaṃ kevala-p\cD{a}ripuṇṇaṃ p\cD{a}risuddhaṃ\\
brahma-c\cD{a}r\cU{i}yaṃ p\cD{a}kāsesi\\

Tam-ah\cU{a}ṃ bh\cD{a}gavantaṃ \cD{a}bh\cU{i}pūj\cD{a}yāmi tam-ah\cU{a}ṃ bh\cD{a}gavantaṃ\\
s\cD{i}ras\cU{ā} n\cD{a}māmi \instr{Bow}

\chapter{Lembrança do Dhamma}     % {{{1

\begin{leader}
  [H\cU{a}nda mayaṃ dhammābh\cD{i}tth\cD{u}tiṃ karomase]
\end{leader}

Yo so svākkh\cU{ā}to bh\cD{a}gav\cU{a}tā dhammo\\
S\cU{a}ndiṭṭh\cD{i}ko \cD{a}kāl\cD{i}ko eh\cD{i}pass\cD{i}ko opanay\cD{i}ko p\cD{a}cc\cD{a}ttaṃ ved\cU{i}t\cD{a}bbo viññūhi

Tam-ah\cU{a}ṃ dhammaṃ \cD{a}bh\cU{i}pūj\cD{a}yāmi\\
\vin tam-ah\cU{a}ṃ dhammaṃ s\cD{i}ras\cU{ā} n\cD{a}māmi \instr{Bow}

%}}}1
\enlargethispage{\baselineskip}
\clearpage

\chapter{Lembrança do Sangha}     % {{{1

\begin{leader}
  [H\cU{a}nda mayaṃ saṅghābh\cD{i}tth\cD{u}tiṃ karomase]
\end{leader}

Yo so supaṭ\cD{i}panno bh\cD{a}gav\cD{a}to sāvak\cD{a}saṅgho\\
Ujupaṭ\cD{i}panno bh\cD{a}gav\cD{a}to sāvak\cD{a}saṅgho\\
Ñāyapaṭ\cD{i}panno bh\cD{a}gav\cD{a}to sāvak\cD{a}saṅgho\\
S\cU{ā}mīc\cD{i}p\cD{a}ṭ\cD{i}panno bh\cD{a}gav\cD{a}to sāvak\cD{a}saṅgho\\
Yadidaṃ cattāri puris\cD{a}yugāni aṭṭh\cU{a} puris\cD{a}pugg\cD{a}lā\\
Esa bh\cD{a}gav\cD{a}to sāvak\cD{a}saṅgho

Āh\cD{u}ṇeyyo pāh\cD{u}ṇeyyo dakkh\cD{i}ṇeyyo añj\cD{a}li-k\cD{a}r\cU{a}ṇīyo\\
An\cU{u}tt\cD{a}raṃ puññakkh\cD{e}ttaṃ lokassa

Tam-ah\cU{a}ṃ saṅghaṃ \cD{a}bh\cU{i}pūj\cD{a}yāmi\\
\vin tam-ah\cU{a}ṃ saṅghaṃ s\cD{i}ras\cU{ā} n\cD{a}māmi \instr{Bow}

\chapter{Saudação ao Tripla Gema}% {{{1

\begin{leader}
  [H\cU{a}nda mayaṃ ratanattaya-paṇāma-gāth\cU{ā}yo ceva\\
  s\cU{a}ṃvega-parikittana-pāṭhañc\cD{a} bhaṇāmase]
\end{leader}

Buddho s\cD{u}suddho k\cD{a}ruṇāmah\cU{a}ṇṇavo\\
Yoccant\cD{a}-suddhabb\cD{a}ra-ñāṇ\cD{a}-loc\cD{a}no\\
Lokass\cD{a} pāpūp\cD{a}k\cU{i}les\cD{a}-ghāt\cD{a}ko\\
Vandām\cU{i} buddhaṃ \cD{a}h\cU{a}m-ād\cD{a}ren\cD{a} taṃ\\
Dhammo p\cD{a}dīpo v\cD{i}ya tass\cD{a} satth\cD{u}no\\
Yo magg\cU{a}pākām\cD{a}t\cD{a}-bhed\cD{a}-bhinn\cD{a}ko\\
Lokuttaro yo c\cD{a} t\cD{a}datth\cD{a}-dīp\cD{a}no\\
Vandām\cU{i} dhammaṃ \cD{a}h\cU{a}m-ād\cD{a}ren\cD{a} taṃ\\
S\cU{a}ṅgho s\cD{u}khettābhyati-kh\cD{e}tta-s\cU{a}ññito\\
Yo diṭṭh\cU{a}santo s\cD{u}g\cD{a}tān\cD{u}bodh\cD{a}ko\\
Lolapp\cD{a}hīno \cD{a}r\cU{i}yo s\cD{u}medh\cD{a}so\\
Vandām\cU{i} saṅghaṃ \cD{a}h\cU{a}m-ād\cD{a}ren\cD{a} taṃ\\
Iccevam-ekant\cD{a}bh\cU{i}pūj\cD{a}-neyy\cD{a}kaṃ vatthuttayaṃ\\
vand\cD{a}y\cD{a}tābh\cD{i}saṅkh\cD{a}taṃ\\
Puññaṃ m\cD{a}yā yaṃ m\cD{a}m\cD{a} sabb\cD{u}padd\cD{a}vā\\
\vin mā h\cU{o}nt\cD{u} ve tass\cD{a} p\cD{a}bhāv\cD{a}siddh\cD{i}yā

Idha tath\cU{ā}g\cD{a}to loke \cD{u}ppanno \cD{a}rahaṃ sammāsambuddho\\
Dhammo c\cD{a} des\cD{i}to niyyān\cD{i}ko \cD{u}p\cD{a}s\cD{a}miko p\cD{a}rinibbān\cD{i}ko\\
\vin s\cU{a}mbodh\cD{a}gāmī s\cD{u}g\cD{a}tapp\cD{a}ved\cD{i}to

M\cU{a}yantaṃ dhammaṃ s\cD{u}tvā evaṃ jānāma

Jātip\cD{i} dukkhā jarāp\cD{i} dukkhā m\cD{a}raṇamp\cD{i} dukkhaṃ\\
S\cU{o}ka-p\cD{a}rideva-dukkh\cD{a}-domanass\cD{u}pāyās\cU{ā}p\cD{i} dukkhā\\
Appiyeh\cD{i} s\cU{a}mp\cD{a}yogo dukkho\\
Piyeh\cD{i} v\cU{i}pp\cD{a}yogo dukkho\\
Yampicch\cU{a}ṃ n\cD{a} labhati tamp\cD{i} dukkhaṃ\\
S\cU{a}ṅkhittena pañc\cD{u}pādānakkh\cU{a}ndhā dukkhā

Seyy\cD{a}thīdaṃ

\begin{twochants}
Rūpūpādān\cD{a}kkh\cU{a}ndho & Vedanūpādān\cD{a}kkh\cU{a}ndho\\
S\cU{a}ññūpādān\cD{a}kkh\cU{a}ndho & S\cU{a}ṅkh\cU{ā}rūpādān\cD{a}kkh\cU{a}ndho\\
Viññāṇūpādān\cD{a}kkh\cU{a}ndho & \\
\end{twochants}

Yesaṃ p\cD{a}riññāya\\
Dh\cD{a}ramāno s\cU{o} bh\cD{a}gavā evaṃ b\cD{a}hulaṃ s\cU{ā}v\cD{a}ke v\cD{i}neti\\
Evaṃ bhāgā c\cD{a} panassa bh\cD{a}gav\cD{a}to s\cU{ā}v\cD{a}kesu \cD{a}nus\cU{ā}s\cD{a}nī b\cD{a}hulā p\cD{a}vatt\cD{a}ti

\clearpage

\begin{threechants}
Rūpaṃ \cD{a}niccaṃ & Vedanā \cD{a}niccā & S\cU{a}ññā \cD{a}niccā\\
S\cU{a}ṅkh\cU{ā}rā \cD{a}niccā & Viññāṇaṃ \cD{a}niccaṃ & \\
Rūpaṃ \cD{a}nattā & Vedanā \cD{a}nattā & S\cU{a}ññā \cD{a}nattā\\
S\cU{a}ṅkh\cU{ā}rā \cD{a}nattā & Viññāṇaṃ \cD{a}nattā & \\
\end{threechants}

S\cD{a}bbe s\cU{a}ṅkh\cU{ā}rā \cD{a}niccā\\
S\cD{a}bbe dhammā \cD{a}nattā'ti

Te m\cU{a}yaṃ otiṇṇāmha-jāt\cD{i}yā j\cD{a}rāmaraṇena\\
S\cU{o}keh\cD{i} p\cD{a}rideveh\cD{i} dukkh\cU{e}h\cD{i} domanasseh\cD{i} \cD{u}pāyāsehi\\
Dukkh\cU{o}tiṇṇā dukkh\cD{a}p\cD{a}retā\\
Appevanām\cU{i}mass\cD{a} kevalass\cD{a} dukkhakkh\cU{a}ndhass\cD{a}\\
\vin ant\cD{a}kir\cU{i}yā paññāyethā'ti

\begin{instruction}
  O que se segue é entoado apenas pelos monges e monjas.
\end{instruction}

C\cU{i}r\cU{a}par\cD{i}nibbutamp\cU{i} taṃ bh\cD{a}gav\cU{a}ntaṃ uddissa\\
\vin \cD{a}rah\cU{a}ntaṃ sammāsambuddhaṃ\\
Saddhā \cD{a}gārasmā anagār\cU{i}yaṃ pabb\cD{a}j\cD{i}tā\\
Tasm\cU{i}ṃ bh\cD{a}gavati brahma-c\cD{a}r\cU{i}yaṃ c\cD{a}rāma\\
Bhikkh\cU{ū}naṃ/Sīladhar\cU{ī}naṃ s\cU{i}kkhās\cD{ā}jīv\cD{a}-samāpannā\\
Taṃ no brahma-c\cD{a}r\cU{i}yaṃ imass\cD{a} kevalass\cD{a} dukkhakkh\cU{a}ndhass\cD{a}\\
\vin ant\cD{a}kir\cU{i}yāya s\cU{a}ṃv\cU{a}tt\cD{a}tu

\clearpage

\begin{instruction}
  Uma versão alternativa da secção precedente, o que pode ser recitado por leigos bem.
\end{instruction}

C\cU{i}r\cU{a}par\cD{i}nibbutamp\cU{i} taṃ bh\cD{a}gav\cU{a}ntaṃ saraṇaṃ g\cD{a}tā\\
Dh\cU{a}mmañca S\cU{a}ṅghañca\\
Tassa bh\cD{a}gavato s\cU{ā}sanaṃ yath\cU{ā}sati yath\cU{ā}balaṃ\\
\vin manasik\cD{a}roma \cD{a}nupaṭip\cU{a}jjāma\\
S\cU{ā} s\cU{ā} no p\cD{a}ṭ\cU{i}patti\\
Imass\cD{a} kevalass\cD{a} dukkhakkh\cU{a}ndhass\cD{a}\\
\vin ant\cD{a}kir\cU{i}yāya s\cU{a}ṃv\cU{a}tt\cD{a}tu

\chapter{Fechando homenagem}       % {{{1

[Arahaṃ] sammāsambuddho bh\cD{a}gavā\\
Buddhaṃ bh\cD{a}gavantaṃ \cD{a}bh\cU{i}vādemi \instr{Bow}

[Svākkh\cU{ā}to] bh\cD{a}gav\cU{a}tā dhammo\\
Dhammaṃ namassāmi \instr{Bow}

[Supaṭ\cD{i}panno] bh\cD{a}gav\cD{a}to sāvak\cD{a}saṅgho\\
S\cU{a}ṅghaṃ n\cD{a}māmi \instr{Bow}

%}}}1

% End of morning-chanting.tex
