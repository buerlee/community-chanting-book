\setlength{\englishIndent}{0pt}

\chapter{Añjali}

Os cânticos e os pedidos formais são feitos com as mãos em añjali.
Este é um gesto de respeito, executado pondo as palmas das mãos juntas
directamente à frente do peito, com os dedos alinhados a apontar
para cima.

\chapter{Pedindo uma Palestra de Dhamma}

\begin{instruction}
  Depois de fazer a vénia três vezes, com as mãos unidas em añjali, recitar o seguinte:
\end{instruction}

Brahmā ca꜕ lokādhipa꜕tī sa꜕hampa꜕ti\\
Ka꜕tañja꜕lī a꜕nadhiva꜕raṃ a꜕yāca꜕tha

Santī꜓dha sa꜕ttāppa꜕ra꜕jakkha꜕-jātikā\\
Desetu꜕ dhammaṃ a꜕nu꜕kampi꜕maṃ pa꜕jaṃ

\begin{instruction}
  Fazer as três vénias outra vez.
\end{instruction}

\begin{english}
O deus Brahmā Sahampati, Senhor do mundo,\\
Com as palmas das mãos juntas em reverência, pediu um favor:

`Há seres aqui com pouco pó apenas nos seus olhos,\\
Por favor, por compaixão ensina-lhes o Dhamma.'
\end{english}

\chapter{Reconhecendo o Ensinamento}

\enlargethispage{2\baselineskip}

\begin{tabular}{@{} ll @{}}
Uma pessoa: & Ha꜓nda mayaṃ dhammakathā꜓ya sā꜓dhukā꜕raṃ dadāmase \\
& \hspace*{1em}\tr{Expressemos agora  nossa aprovação}\\
& \hspace*{1em}\tr{deste Ensinamento do Dhamma.}\\
Resposta: & Sādhu, sādhu, sādhu, anu꜓modāmi \\
& \hspace*{1em}\tr{É bom, eu o valorizo.} \\
\end{tabular}

\clearpage
\chapter{Pedindo o Cântico dos Parittas}

\begin{instruction}
  Depois de fazer a vénia três vezes, com as mãos unidas em añjali, recitar o seguinte:
\end{instruction}

Vipatti-paṭibāhā꜓ya sabba꜕-sampatti꜕-siddhi꜕yā\\
Sabbadukkha-vināsā꜓ya\\
Parittaṃ brūtha꜕ maṅga꜕laṃ

Vipatti-paṭibāhā꜓ya sabba꜕-sampatti꜕-siddhi꜕yā\\
Sabbabhaya-vināsā꜓ya\\
Parittaṃ brūtha꜕ maṅga꜕laṃ

Vipatti-paṭibāhā꜓ya sabba꜕-sampatti꜕-siddhi꜕yā\\
Sabbaroga-vināsā꜓ya\\
Parittaṃ brūtha꜕ maṅga꜕laṃ

\begin{instruction}
  Vénia três vezes
\end{instruction}

\begin{english}
Para desviar o infortúnio, para o surgimento da boa fortuna,\\
Para o desvanecimento de todo o dukkha,\\
Por favor cantai uma bênção e protecção.

Para desviar o infortúnio, para o surgimento da boa fortuna,\\
Para o afastamento de todo o medo,\\
Por favor cantai uma bênção e protecção.

Para desviar o infortúnio, para o surgimento da boa fortuna,\\
Para o afastamento de toda a doença,\\
Por favor cantai uma bênção e protecção.
\end{english}

\setlength{\englishIndent}{\leaderIndent}

\clearpage
\chapter[Três Refúgios \& Cinco Preceitos]{Pedido dos Três Refúgios\newline \& Cinco Preceitos}

\begin{instruction}
  Após fazer três vénias, com as palmas\\
  das mão unidas em añjali, recita-se o pedido:
\end{instruction}

\subsection{Em grupo}

Mayaṃ bhante tisaraṇena sa꜕ha pañca sī꜓lāni yā꜕cāma\\
Dutiyampi mayaṃ bhante tisaraṇena sa꜕ha pañca sī꜓lāni yā꜕cāma\\
Tatiyampi mayaṃ bhante tisaraṇena sa꜕ha pañca sī꜓lāni yā꜕cāma

\subsection{Individualmente}

Ahaṃ bhante tisaraṇena sa꜕ha pañca sī꜓lāni yā꜕cāmi\\
Dutiyampi ahaṃ bhante tisaraṇena sa꜕ha pañca sī꜓lāni yā꜕cāmi\\
Tatiyampi ahaṃ bhante tisaraṇena sa꜕ha pañca sī꜓lāni yā꜕cāmi

\subsection{Tradução}

\begin{english}
  Pedimos/Peço, Venerável Mestre,\\
  \vin os Três Refúgios e os Cinco Preceitos.

  Pela segunda vez, pedimos/peço, Venerável Mestre,\\
  \vin os Três Refúgios e os Cinco Preceitos.

  Pela terceira vez, pedimos/peço, Venerável Mestre,\\
  \vin os Três Refúgios e os Cinco Preceitos.
\end{english}

\clearpage
\chapter{Os Três Refúgios}

\begin{instruction}
  Repetir, depois de o líder ter\\
  cantado as primeiras três linhas
\end{instruction}

Namo tassa bhagavato arahato sammāsambuddhassa\\
Namo tassa bhagavato arahato sammāsambuddhassa\\
Namo tassa bhagavato arahato sammāsambuddhassa

\begin{english}
  Homenagem ao Excelso, Nobre e Perfeitamente Iluminado.\\
  Homenagem ao Excelso, Nobre e Perfeitamente Iluminado.\\
  Homenagem ao Excelso, Nobre e Perfeitamente Iluminado.
\end{english}

Buddhaṃ saraṇaṃ gacchāmi\\
Dhammaṃ saraṇaṃ gacchāmi\\
Saṅghaṃ saraṇaṃ gacchāmi

\begin{english}
  Tenho o Buddha como refúgio.\\
  Tenho o Dhamma como refúgio.\\
  Tenho o Saṅgha como refúgio.
\end{english}

Dutiyampi buddhaṃ saraṇaṃ gacchāmi\\
Dutiyampi dhammaṃ saraṇaṃ gacchāmi\\
Dutiyampi saṅghaṃ saraṇaṃ gacchāmi

\begin{english}
  Pela segunda vez, tenho o Buddha como refúgio.\\
  Pela segunda vez, tenho o Dhamma como refúgio.\\
  Pela segunda vez, tenho o Saṅgha como refúgio.
\end{english}

Tatiyampi buddhaṃ saraṇaṃ gacchāmi\\
Tatiyampi dhammaṃ saraṇaṃ gacchāmi\\
Tatiyampi saṅghaṃ saraṇaṃ gacchāmi

\clearpage

\begin{english}
  Pela terceira vez, tenho o Buddha como refúgio.\\
  Pela terceira vez, tenho o Dhamma como refúgio.\\
  Pela terceira vez, tenho o Saṅgha como refúgio.
\end{english}

\begin{instruction}
  Líder:
\end{instruction}

[Tisaraṇa-gamanaṃ niṭṭhitaṃ]

\begin{english}
  Fica assim completo o Triplo Refúgio.
\end{english}

\begin{instruction}
  Resposta:
\end{instruction}

Āma bhante

\begin{english}
  Sim, Venerável Mestre.
\end{english}

\chapter{Os Cinco Preceitos}

\begin{instruction}
  Repetir cada preceito depois do líder
\end{instruction}

\begin{precept}
  \setcounter{enumi}{0}
  \item Pāṇātipātā vera꜓maṇī sikkhā꜓padaṃ sa꜓mādi꜕yāmi
\end{precept}

\begin{english}
  Observo o preceito de me abster de matar qualquer criatura viva.
\end{english}

\begin{precept}
  \setcounter{enumi}{1}
  \item Adinnādānā vera꜓maṇī sikkhā꜓padaṃ sa꜓mādi꜕yāmi
\end{precept}

\begin{english}
  Observo o preceito de não tirar aquilo que não me for oferecido.
\end{english}

\begin{precept}
  \setcounter{enumi}{2}
  \item Kāmesu micchā꜓cārā vera꜓maṇī sikkhā꜓padaṃ sa꜓mādi꜕yāmi
\end{precept}

\begin{english}
  Observo o preceito de me abster de ter uma conduta sexual imprórpia.
\end{english}

\begin{precept}
  \setcounter{enumi}{3}
  \item Musā꜓vādā vera꜓maṇī sikkhā꜓padaṃ sa꜓mādi꜕yāmi
\end{precept}

\enlargethispage{\baselineskip}

\begin{english}
  Observo o preceito de me abster de mentir.
\end{english}

\clearpage

\begin{precept}
  \setcounter{enumi}{4}
  \item Surāmeraya-majja-pamādaṭṭhā꜓nā vera꜓maṇī sikkhā꜓padaṃ sa꜓mādi꜕yāmi
\end{precept}

\begin{english}
  Observo o preceito de me abster de consumir bebidas\\
  e drogas intoxicantes que deturpem a mente.
\end{english}

\begin{instruction}
  Líder:
\end{instruction}

[Imāni pañca sikkhā꜓padāni\\
Sī꜓lena suga꜕tiṃ yanti\\
Sī꜓lena bhoga꜕sa꜓mpadā\\
Sī꜓lena nibbu꜕tiṃ yanti\\
Tasmā꜓ sī꜓laṃ viso꜓dhaye]

\begin{english}
  Estes são os Cinco Preceitos;\\
  A virtude é fonte de felicidade,\\
  A virtude é fonte de verdadeira riqueza,\\
  A virtude é fonte de paz ---\\
  Que a virtude seja assim purificada.
\end{english}

\begin{instruction}
  Resposta:
\end{instruction}

Sādhu, sādhu, sādhu

\begin{instruction}
  Fazer três vénias
\end{instruction}

\clearpage
\chapter[Três Refúgios \& Oito Preceitos]{Pedido dos Três Refúgios\newline \& Oito Preceitos}

\begin{instruction}
  Após fazer três vénias, com as palmas\\
  das mão unidas em añjali, recita-se o pedido:
\end{instruction}

\subsection{Em grupo}

Mayaṃ bhante tisaraṇena sa꜕ha aṭṭha sī꜓lāni yā꜕cāma\\
Dutiyampi mayaṃ bhante tisaraṇena sa꜕ha aṭṭha sī꜓lāni yā꜕cāma\\
Tatiyampi mayaṃ bhante tisaraṇena sa꜕ha aṭṭha sī꜓lāni yā꜕cāma

\subsection{Individualmente}

Ahaṃ bhante tisaraṇena sa꜕ha aṭṭha sī꜓lāni yā꜕cāmi\\
Dutiyampi ahaṃ bhante tisaraṇena sa꜕ha aṭṭha sī꜓lāni yā꜕cāmi\\
Tatiyampi ahaṃ bhante tisaraṇena sa꜕ha aṭṭha sī꜓lāni yā꜕cāmi

\subsection{Tradução}

\begin{english}
  Pedimos/Peço, Venerável Mestre,\\
  \vin os Três Refúgios e os Oito Preceitos.

  Pela segunda vez, pedimos/peço, Venerável Mestre,\\
  \vin os Três Refúgios e os Oito Preceitos.

  Pela terceira vez, pedimos/peço, Venerável Mestre,\\
  \vin os Três Refúgios e os Oito Preceitos.
\end{english}

\clearpage
\chapter{Os Três Refúgios}

\begin{instruction}
  Repetir, depois de o líder ter\\
  cantado as primeiras três linhas
\end{instruction}

Namo tassa bhagavato arahato sammāsambuddhassa\\
Namo tassa bhagavato arahato sammāsambuddhassa\\
Namo tassa bhagavato arahato sammāsambuddhassa

\begin{english}
  Homenagem ao Excelso, Nobre e Perfeitamente Iluminado.\\
  Homenagem ao Excelso, Nobre e Perfeitamente Iluminado.\\
  Homenagem ao Excelso, Nobre e Perfeitamente Iluminado.
\end{english}

Buddhaṃ saraṇaṃ gacchāmi\\
Dhammaṃ saraṇaṃ gacchāmi\\
Saṅghaṃ saraṇaṃ gacchāmi

\begin{english}
  Tenho o Buddha como refúgio.\\
  Tenho o Dhamma como refúgio.\\
  Tenho o Saṅgha como refúgio.
\end{english}

Dutiyampi buddhaṃ saraṇaṃ gacchāmi\\
Dutiyampi dhammaṃ saraṇaṃ gacchāmi\\
Dutiyampi saṅghaṃ saraṇaṃ gacchāmi

\begin{english}
  Pela segunda vez, tenho o Buddha como refúgio.\\
  Pela segunda vez, tenho o Dhamma como refúgio.\\
  Pela segunda vez, tenho o Saṅgha como refúgio.
\end{english}

Tatiyampi buddhaṃ saraṇaṃ gacchāmi\\
Tatiyampi dhammaṃ saraṇaṃ gacchāmi\\
Tatiyampi saṅghaṃ saraṇaṃ gacchāmi

\clearpage

\begin{english}
  Pela terceira vez, tenho o Buddha como refúgio.\\
  Pela terceira vez, tenho o Dhamma como refúgio.\\
  Pela terceira vez, tenho o Saṅgha como refúgio.
\end{english}

\begin{instruction}
  Líder:
\end{instruction}

[Tisaraṇa-gamanaṃ niṭṭhitaṃ]

\begin{english}
  Fica assim completo o Triplo Refúgio.
\end{english}

\begin{instruction}
  Resposta:
\end{instruction}

Āma bhante

\begin{english}
  Sim, Venerável Mestre.
\end{english}

\chapter{Os Oito Preceitos}

\begin{instruction}
  Repetir cada preceito depois do líder
\end{instruction}

\begin{precept}
  \setcounter{enumi}{0}
  \item Pāṇātipātā vera꜓maṇī sikkhā꜓padaṃ sa꜓mādi꜕yāmi
\end{precept}

\begin{english}
  Observo o preceito de me abster de matar qualquer criatura viva.
\end{english}

\begin{precept}
  \setcounter{enumi}{1}
  \item Adinnādānā vera꜓maṇī sikkhā꜓padaṃ sa꜓mādi꜕yāmi
\end{precept}

\begin{english}
  Observo o preceito de não tirar aquilo que não me for oferecido.
\end{english}

\begin{precept}
  \setcounter{enumi}{2}
  \item Abrahmacariyā vera꜓maṇī sikkhā꜓padaṃ sa꜓mādi꜕yāmi
\end{precept}

\begin{english}
  Observo o preceito de me abster de qualquer tipo de actividade sexual.
\end{english}

\begin{precept}
  \setcounter{enumi}{3}
  \item Musā꜓vādā vera꜓maṇī sikkhā꜓padaṃ sa꜓mādi꜕yāmi
\end{precept}

\begin{english}
  Observo o preceito de me abster de mentir.
\end{english}

\clearpage

\begin{precept}
  \setcounter{enumi}{4}
  \item Surāmeraya-majja-pamādaṭṭhā꜓nā vera꜓maṇī sikkhā꜓padaṃ sa꜓mādi꜕yāmi
\end{precept}

\begin{english}
  Observo o preceito de me abster de consumir bebidas\\
  e drogas intoxicantes que deturpem a mente.
\end{english}

\begin{precept}
  \setcounter{enumi}{5}
  \item Vikālabhojanā vera꜓maṇī sikkhā꜓padaṃ sa꜓mādi꜕yāmi.
\end{precept}

\begin{english}
  Observo o preceito de me abster de comer em alturas indevidas.
\end{english}

\begin{precept}
  \setcounter{enumi}{6}
  \item Nacca-gīta-vādita-visūkada꜓ssanā mālā-gandha-vilepana-dhāraṇa-maṇḍana-vibhūsanaṭṭhā꜓nā vera꜓maṇī sikkhā꜓padaṃ sa꜓mādi꜕yāmi.
\end{precept}

\begin{english}
  Observo o preceito de me abster de qualquer tipo de entretenimento, embelezamento e adornamento.
\end{english}

\begin{precept}
  \setcounter{enumi}{7}
  \item Uccāsayana-mahā꜓sayanā vera꜓maṇī sikkhā꜓padaṃ sa꜓mādi꜕yāmi.
\end{precept}

\begin{english}
  Observo o preceito de me abster de dormir em camas elevadas e luxuosas.
\end{english}

\begin{instruction}
  Líder:
\end{instruction}

[Imāni aṭṭha sikkhā꜓padāni\\
Sī꜓lena suga꜕tiṃ yanti\\
Sī꜓lena bhoga꜕sa꜓mpadā\\
Sī꜓lena nibbu꜕tiṃ yanti\\
Tasmā꜓ sī꜓laṃ viso꜓dhaye]

\clearpage

\begin{english}
  Estes são os Oito Preceitos;\\
  A virtude é fonte de felicidade,\\
  A virtude é fonte de verdadeira riqueza,\\
  A virtude é fonte de paz ---\\
  Que a virtude seja assim purificada.
\end{english}

\begin{instruction}
  Resposta:
\end{instruction}

Sādhu, sādhu, sādhu.

\begin{instruction}
  Fazer três vénias
\end{instruction}

