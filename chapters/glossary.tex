\chapter{Glossary of Pāli Terms}

\begin{description}

\item[Anattā] Literally, “not-self,” i.e. impersonal, without individual
  essence; neither a person nor belonging to a person. One of the three
  characteristics of conditioned phenomena.

\item[Anicca] Transient, impermanent, unstable, having the nature to
  arise and pass away. One of the three characteristics of conditioned
  phenomena.

\item[Añjali] A gesture of respect. The palms of both hands join
  together directly in front of the chest, with the fingers aligned and
  pointing upwards.

\item[Arahaṃ/Arahant] Literally, ‘worthy one’ --- a term applied to all
  enlightened beings. As an epithet of the Buddha alone, “Lord” is used.

\item[Ariyapuggalā] ‘Noble Beings’ or ‘Noble Disciples’ --- there are
  eight kinds: those who are working on or who have achieved the four
  different stages of realization.

\item[Bhagavā] Bountiful, with good fortune --- when used as an epithet of
  the Buddha, “the Fortunate One,” “the Blessed One.”

\item[Bhikkhu] A Buddhist monk who lives as an alms mendicant, abiding
  by 227 training precepts that define a life of renunciation and
  simplicity.

\item[Brahmā] Celestial being; a god in one of the higher spiritual
  realms.

\item[Buddha] The Understanding One, the Awakened One, who knows things
  as they are; a potential in every human being. The historical Buddha,
  Siddhattha Gotama, lived and taught in India in the 5th century B.C.E.

\item[Deva] A celestial being. Less refined than a brahmā; as a deva is
  still in a sensual realm, albeit a very refined one.

\item[Dhamma] (Sanskrit: Dharma) The Teaching of the Buddha as contained
  in the scriptures; not dogmatic in character, but more like a raft or
  vehicle to convey the disciple to deliverance. Also, the Truth towards
  which that Teaching points; that which is beyond words, concepts or
  intellectual understanding. When written as ‘\emph{dhamma}’, i.e.
  with lower case `d', this refers to an ‘item’ or ‘thing’.

\item[Dukkha] Literally, ‘hard to bear’ --- dis-ease, restlessness of
  mind, anguish, conflict, unsatisfactoriness, discontent, stress,
  suffering. One of the three characteristics of conditioned phenomena.

\item[Factors of Awakening (bojjhaṅga)] 1. mindfulness, 2. investigation
of truth, 3. effort, 4. rapture, 5. tranquility, 6. concentration, 7.
equanimity.

\item[Foundations of Mindfulness (satipaṭṭhāna)] Mindfulness of 1.
\emph{kāya} (body), 2. \emph{vedanā} (feelings), 3. \emph{citta} (mind),
4. \emph{dhamma} (mind-objects).

\item[Grounds of Birth (yoni)] The four modes of generation by which
  beings take birth: womb-born, egg-born, putrescence-born
  (moisture-born) and spontaneously born (the apparitional).

\item[Holy Life (brahmacariyā)] Literally: the Brahma-conduct; usually
  referring to the monastic life. Using this term emphasizes the vow of
  celibacy.

\item[Jhāna] Mental absorption. A state of strong concentration focused
  on a single physical or sensation or mental notion.

\item[Kamma] (Sanskrit: karma) Action, deed; actions created by habitual
  impulse, intention, volition, natural energies.

\item[Māra] Personification of evil forces. During the Buddha’s struggle
  for enlightenment, Māra manifested frightening and enticing forms to
  try to turn him back from his goal.

\item[Nibbāna] (Sanskrit: Nirvāṇa) Literally, ‘coolness’ --- the state of
  liberation from all suffering and defilements, the goal of the
  Buddhist path.

\item[Paccekabuddha] Solitary Buddha --- someone enlightened by their own
  efforts without relying on a teacher but who, unlike the Buddha, has
  no following of disciples.

\item[Pañc’upādānakkhandhā] The five aggregates, physical or mental ---
  that is: \emph{rūpa, vedanā, saññā, saṅkhārā, viññāṇa.} Attachment to
  any of these as, ‘This is mine’, ‘I am this’ or, ‘This is my self’ is
  \emph{upādāna} --- clinging or grasping.

\item[Paritta] Verses chanted particularly for blessing and protection.

\item[Parinibbāna] The Buddha’s final passing away, i.e. final entering
  nibbāna.

\item[Peaceful Sage (muni)] An epithet of the Buddha

\item[Planes of Birth (bhūmi)] The three planes where rebirth takes
  place: \emph{kāmāvacarabhūmi}: the sensuous plane;
  \emph{rūpāvacara-bhūmi}: form-plane; \emph{arūpāvacarabhūmi}: formless
  plane.

\item[Puñña] Merit, the accumulation of good fortune, blessings, or
  well-being resulting from the practice of Dhamma.

\item[Rūpa] Form or matter. The physical elements that make up the body,
  i.e. earth, water, fire and air (solidity, cohesion, temperature and
  vibration).

\item[Saṅgha] The community of those who practise the Buddha’s Way.

  More specifically, those who have formally committed themselves to the
lifestyle of mendicant monks and nuns. The “four pairs, the eight kinds of
noble beings” are those who are on the path to or who have realized the
fruition of the four stages of enlightenment: stream entry, once return,
non-return and arahantship.

\item[Saṅkhārā] Formations. all conditioned things, or volitional
  impulses, that is all mental states apart from feeling and perception
  that colour one’s thoughts and make them either good, bad or neutral.

\item[Saññā] Perception, the mental function of recognition.

\item[Tathāgata] ‘Thus gone’ or ‘Thus come’ --- one who has gone beyond
  suffering and mortality; one who experiences things as they are,
  without delusion. The epithet that the Buddha applied to himself.

\item[Threefold bliss] Mundane bliss, celestial bliss and Nibbānic
  bliss.

\item[Triple Gem] Buddha, Dhamma and Sangha.

\item[Vedanā] Feeling --- physical and mental feelings that may be either
  pleasant, unpleasant or neutral.

\item[Viññāṇa] Sense consciousness --- the process whereby there is
  seeing, hearing, smelling, tasting, touching and thinking.

\end{description}

