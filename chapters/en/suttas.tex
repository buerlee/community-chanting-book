\chapter[The Wheel of Dhamma]{The Discourse on Setting in Motion the Wheel of Dhamma}

\begin{leader}
{\AlegreyaSans\addfontfeature{LetterSpace=3.0}\color[gray]{0.5}\textbf{[SOLO INTRODUCTION]}}

This is the first teaching of the Tathāgata on attaining to unexcelled, perfect enlightenment.\\
Here is the perfect turning of the incomparable wheel of Truth,\\
Inestimable wherever it is expounded in the world.\\
Disclosed here are two extremes, and the Middle Way,\\
With the Four Noble Truths and the purified knowledge and vision\\
Pointed out by the Lord of Dhamma.\\
Let us chant together this Sutta proclaiming the supreme, independent enlightenment that is widely renowned as\\
``The Turning of the Wheel of the Dhamma.''
\end{leader}

[Thus have I heard.] Once when the Blessed One was staying in the deer sanctuary at Isipatana, near Benares, he spoke to the group of five bhikkhus:

``These two extremes, bhikkhus, should not be followed by one who has gone forth: sensual indulgence, which is low, coarse, vulgar, ignoble, and unprofitable; and self-torture, which is painful, ignoble, and unprofitable.

``Bhikkhus, by avoiding these two extremes, the Tathāgata has realized the Middle Way, which gives vision and understanding, which leads to calm, penetration, enlightenment, to Nibbāna.

``And what, bhikkhus, is the Middle Way realized by the Tathāgata, which gives vision and understanding, which leads to calm, penetration, enlightenment, to Nibbāna?

%``It is just this Noble Eightfold Path, namely:
%
%``Right View, Right Intention, Right Speech, Right Action, Right
%
%Livelihood, Right Effort, Right Mindfulness, and Right Concentration.
%
%``Truly, bhikkhus, this Middle Way understood by the Tathāgata produces vision, produces knowledge, and leads to calm, penetration, enlightenment, to Nibbāna.
%
%``This, bhikkhus, is the Noble Truth of dukkha:
%
%``Birth is dukkha, ageing is dukkha, death is dukkha, grief, lamentation, pain, sorrow and despair are dukkha, association with the disliked is dukkha, separation from the liked is dukkha, not to get what one wants is dukkha. In brief, clinging to the five khandhas is dukkha.
%
%``This, bhikkhus, is the Noble Truth of the cause of dukkha:
%
%``The craving which causes rebirth and is bound up with pleasure and lust, ever seeking fresh delight, now here, now there; namely, craving for sense pleasure, craving for existence, and craving for annihilation.
%
%``This, bhikkhus, is the Noble Truth of the cessation of dukkha:
%
%``The complete cessation, giving up, abandonment of that craving, complete release from that craving, and complete detachment from it.
%
%``This, bhikkhus, is the Noble Truth of the way leading to the cessation of dukkha:
%
%``Only this Noble Eightfold Path; namely, Right View, Right Intention, Right Speech, Right Action, Right Livelihood, Right Effort, Right Mindfulness, and Right Concentration.
%
%``With the thought, `This is the Noble Truth of dukkha,' there arose in me, bhikkhus, vision, knowledge, insight, wisdom, light, concerning things unknown before.
%
%``With the thought, `This is the Noble Truth of dukkha, and this dukkha has to be understood,' there arose in me, bhikkhus, vision, knowledge, insight, wisdom, light, concerning things unknown before.
%
%``With the thought, `This is the Noble Truth of dukkha, and this dukkha has been understood,' there arose in me, bhikkhus, vision, knowledge, insight, wisdom, light, concerning things unknown before.
%
%``With the thought, `This is the Noble Truth of the cause of dukkha,' there arose in me, bhikkhus, vision, knowledge, insight, wisdom, light, concerning things unknown before.
%
%``With the thought, `This is the Noble Truth of the cause of dukkha, and this cause of dukkha has to be abandoned,' there arose in me, bhikkhus, vision, knowledge, insight, wisdom, light, concerning things unknown before.
%
%``With the thought, `This is the Noble Truth of the cause of dukkha, and this cause of dukkha has been abandoned,' there arose in me, bhikkhus, vision, knowledge, insight, wisdom, light, concerning things unknown before.
%
%``With the thought, `This is the Noble Truth of the cessation of dukkha,' there arose in me vision, knowledge, insight, wisdom, light, concerning things unknown before.
%
%``With the thought, `This is the Noble Truth of the cessation of dukkha, and this cessation of dukkha has to be realized,' there arose in me vision, knowledge, insight, wisdom, light, concerning things unknown before.
%
%``With the thought, `This is the Noble Truth of the cessation of dukkha, and this cessation of dukkha has been realized,' there arose in me vision, knowledge, insight, wisdom, light, concerning things unknown before.
%
%``With the thought, `This is the Noble Truth of the way leading to the cessation of dukkha,' there arose in me vision, knowledge, insight, wisdom, light, concerning things unknown before.
%
%``With the thought, `This Noble Truth of the way leading to the cessation of dukkha has to be developed,' there arose in me vision, knowledge, insight, wisdom, light, concerning things unknown before.
%
%``With the thought, `This Noble Truth of the way leading to the cessation of dukkha has been developed,' there arose in me vision, knowledge, insight, wisdom, light, concerning things unknown before.
%
%``So long, bhikkhus, as my knowledge and vision of reality regarding these Four Noble Truths, in their three phases and twelve aspects, was not fully clear to me, I did not declare to the world of spirits, demons, and gods, with its seekers and sages, celestial and human beings, the realization of incomparable, perfect enlightenment.
%
%``But when, bhikkhus, my knowledge and vision of reality regarding these Four Noble Truths, in their three phases and twelve aspects, was fully clear to me, I declared to the world of spirits, demons, and gods, with its seekers and sages, celestial and human beings, that I understood incomparable, perfect enlightenment.
%
%``Knowledge and vision arose: `Unshakeable is my deliverance; this is the last birth, there will be no more renewal of being.'\,''
%
%Thus spoke the Blessed One. Glad at heart, the group of five bhikkhus approved of the words of the Blessed One.
%
%As this exposition was proceeding, the spotless, immaculate vision of the Dhamma appeared to the Venerable Koṇḍañña and he knew: ``Everything that has the nature to arise has the nature to cease.''
%
%When the Blessed One had set in motion the Wheel of Dhamma, the Earthbound devas proclaimed with one voice,
%
%``The incomparable Wheel of Dhamma has been set in motion by the Blessed One in the deer sanctuary at Isipatana, near Benares, and no seeker, brahmin, celestial being, demon, god, or any other being in the world can stop it.''
%
%Having heard what the Earthbound devas said, the devas of the Four Great Kings proclaimed with one voice. . . .
%
%Having heard what the devas of the Four Great Kings said, the devas of the Thirty-three proclaimed with one voice. . . .
%
%Having heard what the devas of the Thirty-three said, the Yāma devas proclaimed with one voice. . . .
%
%Having heard what the Yāma devas said, the Devas of Delight proclaimed with one voice. . . .
%
%Having heard what the Devas of Delight said, the Devas Who Delight in Creating, proclaimed with one voice. . . .
%
%Having heard what the Devas Who Delight in Creating said, the
%
%Devas Who Delight in the Creations of Others proclaimed with one voice. . . .
%
%Having heard what the Devas Who Delight in the Creations of Others said, the Brahma gods proclaimed in one voice,
%
%``The incomparable Wheel of Dhamma has been set in motion by the Blessed One in the deer sanctuary at Isipatana, near Benares, and no seeker, brahmin, celestial being, demon, god, or any other being in the world can stop it.''
%
%Thus in a moment, an instant, a flash, word of the Setting in Motion of the Wheel of Dhamma went forth up to the Brahma world, and the ten-thousandfold universal system trembled and quaked and shook, and a boundless, sublime radiance surpassing the power of devas appeared on earth.
%
%Then the Blessed One made the utterance,
%
%``Truly, Koṇḍañña has understood, Koṇḍañña has understood!''
%
%Thus it was that the Venerable Koṇḍañña got the name Aññā-Koṇḍañña: ``Koṇḍañña Who Understands.''

\chapter{Dhammacakkappavattana Sutta}

%\begin{tabular}{l l}
%Anuttaraṃ abhisambodhiṃ & sambujjhitvā Tathāgato\\
%Paṭhamaṃ yaṃ adesesi & Dhammacakkaṃ anuttaraṃ\\
%Sammadeva pavattento & loke appaṭivattiyaṃ\\
%Yatthākkhātā ubho antā & paṭipatti ca majjhimā\\
%Catūsvāriyasaccesu & visuddhaṃ ñāṇadassanaṃ\\
%Desitaṃ dhammarājena & sammāsambodhikittanaṃ\\
%Nāmena vissutaṃ suttaṃ & Dhammacakkappavattanam\\
%Veyyākaraṇapāthena & saṅgītantambhaṇāma se.
%\end{tabular}

\begin{leader}
\instr{Solo introduction}

\begin{tabular}{l l}
Anuttaraṃ abhisambodhiṃ & sambujjhitvā Tathāgato\\
Pathamaṃ yaṃ adesesi & Dhammacakkaṃ anuttaraṃ\\
Sammadeva pavattento & loke appativattiyaṃ\\
Yatthākkhātā ubho antā & patipatti ca majjhimā\\
Catūsvāriyasaccesu & visuddhaṃ ñāṇadassanaṃ\\
Desitaṃ dhammarājena & sammāsambodhikittanaṃ\\
Nāmena vissutaṃ suttaṃ & Dhammacakkappavattanaṃ\\
Veyyākaraṇapāthena & saṅgītantam bhaṇāma se.
\end{tabular}
\end{leader}

[Evam-me sutaṃ.] Ekaṃ samayaṃ Bhagavā, Bārāṇasiyaṃ viharati Isipatane
Migadāye.  Tatra kho Bhagavā pañca-vaggiye bhikkhū āmantesi:

“Dve’me bhikkhave antā pabbajitena na sevitabbā.
%/Katame dve?/
Yo cāyaṃ
kāmesu kāma-sukh’allikānuyogo, hīno gammo pothujjaniko anariyo
anattha-sañhito, yo cāyaṃ atta-kilamathānuyogo, dukkho anariyo
anattha-sañhito.

Ete’te bhikkhave ubho ante anupagamma, majjhimā
paṭipadā Tathāgatena abhisambuddhā, cakkhu-karaṇī ñāṇa-karaṇī upasamāya
abhiññāya sambodhāya nibbānāya saṃvattati.

Katamā ca sā bhikkhave majjhimā paṭipadā Tathāgatena abhisambuddhā,
cakkhu-karaṇī ñāṇa-karaṇī upasamāya abhiññāya sambodhāya nibbānāya
saṃvattati?

%Ayam-eva ariyo aṭṭh’aṅgiko maggo, seyyathīdaṃ, sammādiṭṭhi
%sammā-saṅkappo, sammā-vācā sammā-kammanto sammā-ājīvo, sammāvāyāmo
%sammā-sati sammā-samādhi. Ayaṃ kho sā bhikkhave majjhimā paṭipadā
%Tathāgatena abhisambuddhā, cakkhu-karaṇī ñāṇa-karaṇī upasamāya abhiññāya
%sambodhāya nibbānāya saṃvattati.

%Idaṃ kho pana bhikkhave dukkhaṃ ariyasaccaṃ: Jāti pi dukkhā jarā pi
%dukkhā /vyādhi pi dukkhā/ maraṇam pi dukkhaṃ,
%sokaparideva-dukkha-domanass’upāyāsā pi dukkhā, appiyehi sampayogo
%dukkho, piyehi vippayogo dukkho, yam-p’icchaṃ na labhati tam pi dukkhaṃ,
%saṅkhittena pañc’upādānakkhandhā dukkhā. Idaṃ kho pana bhikkhave
%dukkhasamudayo /aṃ/ ariya-saccaṃ: Yā’yaṃ taṇhā ponobbhavikā
%nandi-rāga-sahagatā tatratatrābhinandinī, seyyathīdaṃ, kāma-taṇhā
%bhava-taṇhā vibhava-taṇhā. Idaṃ kho pana bhikkhave dukkha-nirodho /aṃ/
%ariya-saccaṃ: Yo tassā yeva taṇhāya asesa-virāga-nirodho cāgo
%paṭinissaggo mutti anālayo. Idaṃ kho pana bhikkhave
%dukkha-nirodha-gāminīpaṭipadā ariya-saccaṃ: ayam-eva ariyo aṭṭh’aṅgiko
%maggo, seyyathīdaṃ, sammā-diṭṭhi sammā-saṅkappo, sammā-vācā
%sammākammanto sammā-ājīvo, sammā-vāyāmo sammā-sati sammā-samādhi.
%
%Idaṃ dukkhaṃ ariya-saccan-ti me bhikkhave, pubbe ananussutesu dhammesu,
%cakkhuṃ udapādi ñāṇaṃ udapādi paññā udapādi vijjā udapādi āloko udapādi.
%Taṃ kho pan’idaṃ dukkhaṃ ariya-saccaṃ pariññeyyan-ti me bhikkhave, pubbe
%ananussutesu dhammesu, cakkhuṃ udapādi ñāṇaṃ udapādi paññā udapādi vijjā
%udapādi āloko udapādi. Taṃ kho pan’idaṃ dukkhaṃ ariya-saccaṃ
%pariññātan-ti me bhikkhave, pubbe ananussutesu dhammesu, cakkhuṃ udapādi
%ñāṇaṃ udapādi paññā udapādi vijjā udapādi āloko udapādi.
%
%Idaṃ dukkha-samudayo /aṃ/ ariya-saccan-ti me bhikkhave, pubbe
%ananussutesu dhammesu, cakkhuṃ udapādi ñāṇaṃ udapādi paññā udapādi vijjā
%udapādi āloko udapādi. Taṃ kho pan’idaṃ dukkha-samudayo /aṃ/
%ariya-saccaṃ pahātabban-ti me bhikkhave, pubbe ananussutesu dhammesu,
%cakkhuṃ udapādi ñāṇaṃ udapādi paññā udapādi vijjā udapādi āloko udapādi.
%Taṃ kho pan’idaṃ dukkhasamudayo /aṃ/ ariya-saccaṃ pahīnan-ti me
%bhikkhave, pubbe ananussutesu dhammesu, cakkhuṃ udapādi ñāṇaṃ udapādi
%paññā udapādi vijjā udapādi āloko udapādi.
%
%Idaṃ dukkha-nirodho /aṃ/ ariya-saccan-ti me bhikkhave, pubbe
%ananussutesu dhammesu, cakkhuṃ udapādi ñāṇaṃ udapādi paññā udapādi vijjā
%udapādi āloko udapādi. Taṃ kho pan’idaṃ dukkha-nirodho /aṃ/ ariya-saccaṃ
%sacchikātabban-ti me bhikkhave, pubbe ananussutesu dhammesu, cakkhuṃ
%udapādi ñāṇaṃ udapādi paññā udapādi vijjā udapādi āloko udapādi. Taṃ kho
%pan’idaṃ dukkhanirodho /aṃ/ ariya-saccaṃ sacchikatan-ti me bhikkhave,
%pubbe ananussutesu dhammesu, cakkhuṃ udapādi ñāṇaṃ udapādi paññā udapādi
%vijjā udapādi āloko udapādi.
%
%Idaṃ dukkha-nirodha-gāminī-paṭipadā ariyasaccan-ti me bhikkhave, pubbe
%ananussutesu dhammesu, cakkhuṃ udapādi ñāṇaṃ udapādi paññā udapādi vijjā
%udapādi āloko udapādi.  Taṃ kho pan’idaṃ dukkha-nirodha-gāminīpaṭipadā
%ariya-saccaṃ bhāvetabban-ti me bhikkhave, pubbe ananussutesu dhammesu,
%cakkhuṃ udapādi ñāṇaṃ udapādi paññā udapādi vijjā udapādi āloko udapādi.
%Taṃ kho pan’idaṃ dukkha-nirodha-gāminī-paṭipadā ariya-saccaṃ bhāvitan-ti
%me bhikkhave, pubbe ananussutesu dhammesu, cakkhuṃ udapādi ñāṇaṃ udapādi
%paññā udapādi vijjā udapādi āloko udapādi.
%
%Yāva-kīvañ-ca me bhikkhave imesu catūsu ariya-saccesu,
%evan-ti-parivaṭṭaṃ dvādas’ākāraṃ yathā-bhūtaṃ ñāṇa-dassanaṃ na
%suvisuddhaṃ ahosi, n’eva tāvāhaṃ bhikkhave sadevake loke samārake
%sabrahmake, sassamaṇa-brāhmaṇiyā pajāya sadevamanussāya, anuttaraṃ
%sammā-sambodhiṃ abhi-sambuddho paccaññāsiṃ. Yato ca kho me bhikkhave
%imesu catūsu ariya-saccesu, evan-ti-parivaṭṭaṃ dvā-das’ākāraṃ
%yathābhūtaṃ ñāṇa-dassanaṃ suvisuddhaṃ ahosi, athāhaṃ bhikkhave sadevake
%loke samārake sabrahmake, sassamaṇa-brāhmaṇiyā pajāya sadeva-manussāya,
%anuttaraṃ sammāsambodhiṃ abhisambuddho paccaññāsiṃ.  ¥āṇañ-ca pana me
%dassanaṃ udapādi: akuppā me vimutti, ayam-antimā jāti, n’atthi dāni
%punabbhavo-ti.” Idam-avoca Bhagavā.  Attamanā pañca-vaggiyā bhikkhū
%Bhagavato bhāsitaṃ abhinanduṃ. Imasmiñ-ca pana veyyākaraṇasmiṃ
%bhaññamāne, Āyasmato Koṇḍaññassa virajaṃ vīta-malaṃ dhammacakkhuṃ
%udapādi. Yaṅ-kiñci samudayadhammaṃ sabban-taṃ nirodha-dhamman-ti.
%
%Pavattite ca Bhagavatā dhamma-cakke, bhummā devā saddam-anussāvesuṃ:
%“Etam-Bhagavatā Bārāṇasiyaṃ Isipatane Migadāye anuttaraṃ dhamma-cakkaṃ
%pavattitaṃ, appaṭivattiyaṃ samaṇena vā brāhmaṇena vā devena vā mārena vā
%brahmunā vā kenaci vā lokasmin-ti.” Bhummānaṃ devānaṃ saddaṃ sutvā,
%Cātummahā-rājikā devā saddam-anussāvesuṃ.  Cātummahā-rājikānaṃ devānaṃ
%saddaṃ sutvā, Tāvatiṃsā devā saddam-anussāvesuṃ.  Tāvatiṃsānaṃ devānaṃ
%saddaṃ sutvā, Yāmā devā saddam-anussāvesuṃ. Yāmānaṃ devānaṃ saddaṃ
%sutvā, Tusitā devā saddamanussāvesuṃ. Tusitānaṃ devānaṃ saddaṃ sutvā,
%Nimmāna-ratī devā saddamanussāvesuṃ. Nimmāna-ratīnaṃ devānaṃ saddaṃ
%sutvā, Para-nimmita-vasa-vattī devā saddam-anussāvesuṃ.
%Para-nimmita-vasavattīnaṃ devānaṃ saddaṃ sutvā, Brahma-kāyikā devā
%saddam-anussāvesuṃ: “Etam-Bhagavatā Bārāṇasiyaṃ Isipatane Migadāye
%anuttaraṃ dhamma-cakkaṃ pavattitaṃ, appaṭivattiyaṃ samaṇena vā
%brāhmaṇena vā devena vā mārena vā brahmunā vā kenaci vā lokasmin-ti.”
%
%Iti-ha tena khaṇena tena muhuttena, yāva brahma-lokā saddo abbhuggacchi.
%Ayañ-ca dasa-sahassī loka-dhātu, saṅkampi sampakampi sampavedhi.
%Appamāṇo ca oḷāro /uḷāro/ obhāso loke pātur-ahosi, atikkamm’eva devānaṃ
%dev’ānubhāvaṃ.
%
%Atha kho Bhagavā udānaṃ udānesi: “Aññāsi vata bho Koṇḍañño, aññāsi vata
%bho Koṇḍañño-ti.” Iti-h’idaṃ āyasmato Koṇḍaññassa, Aññā-koṇḍañño-tveva
%nāmaṃ, ahosī-ti.
%
%\emph{Dhamma-cakkappavattana-suttaṃ Niṭṭhitaṃ.}

% [S.V.420f; Vin.I.10f]
% ♦ /…/ shows variants to Thai Pāli usage.