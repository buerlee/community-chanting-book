\chapter{Chanting Guide}

Chanting is used to settle the mind and bring the attention to suitable
qualities.

To learn to chant well, the most important thing is to listen.

Also bear in mind that chanting customs differ in the various monasteries.

% TODO: get it ready: www.fsaudio.org/chanting
% TODO: trailing slash breaks the page (not a folder)

Download chanting recordings at \prul{www.fsaudio.org/chanting}

\section{Reading the marks}

[Square brackets] indicate parts chanted only by the leader.

The slash / indicates differences in male/female, and singular/plural forms.

The cantillation marks indicate changes in tone:

\begin{tabular}{llll}
High tone: & no꜓ble & Long low tone: & ho꜖mage\\
Low tone: & ble꜕ssed & Long mid tone: & \prul{guides}\\
\end{tabular}

The hyphens in Pāli words do not affect the pronunciation.

\section{Stress and rhythm}

The syllables are chanted for either one or half of a unit of rhythm:

\begin{centering}

{\setlength{\tabcolsep}{1.8pt}%
\begin{tabular}{ccccccccccccccc}
NA & · & MO & \hsp & TAS & · & SA & \hsp & BHA & · & GA & · & VA & · & TO\\
 ½ &   &  1 &      &  1  &   &  ½ &      &   ½ &   &  ½ &   &  ½ &   &  1\\
\end{tabular}%

\begin{tabular}{ccccccccccccccccccc}
A & · & RA & · & HA & · & TO & \hsp & SAM & · & MĀ & · & SAM & · & BUD & · & DHA & · & SSA\\
½ &   &  ½ &   &  ½ &   &  1 &      &  1  &   &  1 &   &  1  &   &  1  &   &  1  &   &  ½ \\
\end{tabular}%

\begin{tabular}{ccc c ccccc c ccccc c ccccccc}
BUD & · & DHO & \hsp & SU & · & SUD & · & DHO & \hsp & KA & · & RU & · & ṆĀ & \hsp & MA & · & HAṆ & · & ṆA & · & VO\\
 1  &   & 1   &      & ½  &   & 1   &   & 1   &      & ½  &   & ½  &   & 1  &      & ½  &   & 1   &   & ½  &   & 1\\
\end{tabular}%
}

\end{centering}

\begin{tabular}{@{} lll @{}}
  1 & Stressed syllables take one unit of rhythm,\\
  ½ & unstressed syllables take half a unit.\\
\end{tabular}

\textbf{Unstressed syllables} end in a short \textbf{a, i} or
\textbf{u}. All other syllables are stressed.

\clearpage

A syllable is \textbf{long} when:

\begin{itemize}
  \item the vowel is long: \textbf{ā ī ū e o}
  \item a short vowel is followed by two consonants, or one double consonant: \prul{sa}mmā, su\prul{su}ddho
\end{itemize}

\section{Separating the syllables}

\textbf{bh, ch, dh, kh\ldots} are aspirated consonants. They count as
\textit{one consonant} and are not divided.

\textbf{bbh, cch, ddh, kkh\ldots} are double (aspirated) consonants. They count as
\textit{two consonants} and \textit{are} divided.

Therefore \textbf{am·hā·kaṃ}, but \textbf{sa·dham·maṃ}, not \textbf{sad·ham·maṃ}.
\textbf{Bud·dho} and not \textbf{Bu·ddho}.

\begin{centering}

\begin{minipage}{0.8\linewidth}
\begin{multicols}{2}
\setlength{\tabcolsep}{1.8pt}%

\begin{tabular}{rrcccl}
     & A & · & NIC & · & CA   \\
     & ½ &   &  1  &   & ½    \\
(not & A & · & NI  & · & CCA) \\
     & ½ &   & ½   &   & ½    \\
\end{tabular}

\columnbreak

\begin{tabular}{rrcccl}
     & PUG & · & GA  & · & LĀ \\
     &  1  &   &  ½  &   &  1 \\
(not & PU  & · & GGA & · & LĀ)\\
     &  ½  &   &  ½  &   &  1 \\
\end{tabular}

\end{multicols}
\end{minipage}

\end{centering}

They are always enunciated separately, e.g. \textbf{dd} in ‘uddeso’ as
in ‘mad dog’, or \textbf{gg} in ‘maggo’ as in ‘big gun’.

\section{Listening}

A general rule of thumb for understanding the practice of chanting is to
listen carefully to what the leader and the group are chanting and to
follow, keeping the same pitch, tempo and speed. All voices should blend
together as one.

% End of chanting-guide.tex
