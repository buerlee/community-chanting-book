\chapter[Partilha de bênçãos]{Reflexões sobre a partilha de bênçãos}

\begin{leader}
[Cantêmos agora reflexões sobre a partilha de bênçãos]
\end{leader}

Através do bem que resulta da minha prática,\\
Que os meus mestres e guias espirituais de grande virtude,\\
Minha mãe, meu pai e meus familiares,\\
\ldots

% TODO

\chapter[Amor e a compaixão]{Palavras do Buddha sobre o amor e a compaixão}

\begin{leader}
[Cantêmos agora as palavras do Buddha sobre o Amor e a Compaixão]
\end{leader}

Eis o que se deve fazer\\
Para cultivar a bondade\\
E seguir a via da paz:\\
Ser capaz e ser honesto,\\
Franco e gentil no falar.\\
Humilde e não arrogante,\\
Contente, facilmente satisfeito,\\
Aliviado de deveres e frugal no seu caminho.

Pacífico e sereno, sábio e inteligente,\\
Sem orgulho, sem exigência por natureza.\\
Que ele nada faça\\
Que os sábios possam vir a reprovar.\\
Desejando: Na alegria e na segurança,\\
Que todos os seres sejam felizes.\\
Quaisquer que sejam os seres vivos,\\
Fracos, fortes, sem excepção\\
Dos maiores aos mais pequenos,\\
Visíveis ou invisíveis,\\
Estejam perto ou estejam longe,\\
Nascidos ou por nascer --\\
Que todos os seres sejam felizes!

Que ninguém engane ninguém,\\
Ou despreze alguém em que estado fôr.\\
Que ninguém por raiva ou má-fé,\\
Deseje mal a alguém.\\
Assim como uma Mãe protege o filho,\\
Com sua vida, seu único filho,\\
Assim de coração infinito,\\
Se deveria estimar todo o ser vivo;\\
Irradiando ternura por todo o mundo:\\
Acima ao mais alto céu,\\
E abaixo às profundezas;\\
Irradiante e sem limites,\\
Livre de ódio e má-fé.\\
Seja parado ou a andar,\\
Sentado ou deitado,\\
Livre de torpor,\\
Esta é uma lembrança a manter.

Diz-se esta ser a sublime permanência.\\
O puro de coração, com clareza de visão,\\
Ao não insistir em ideias fixas,\\
Liberto dos desejos dos sentidos,\\
Não voltará a nascer neste mundo.