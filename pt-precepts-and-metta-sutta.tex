\documentclass[
  babelLanguage=portuguese,
  final,
  %showtrims,
  %showwirebinding,
]{chantingbook}

\usepackage{local}

\title{Livro de Cânticos}
\subtitle{}

\begin{document}

\frontmatter

\thispagestyle{empty}

{\centering
\mbox{}
\vfill

\begin{minipage}{0.7\linewidth}

\parttitlefont\color{chaptertitle}
%\addfontfeature{LetterSpace=2.0}

Cânticos

\bigskip

{\normalsize
Pedido dos Três Refúgios \& Cinco Preceitos

Metta Sutta
}

\end{minipage}

\vspace*{4\baselineskip}

\vfill

\mbox{}
}

\mainmatter

\artopttrue

\clearpage
\chapter[Três Refúgios \& Cinco Preceitos]{Pedido dos Três Refúgios\newline \& Cinco Preceitos}

\begin{instruction}
  Após fazer três prostrações, com as palmas\\
  das mão unidas em añjali, recita-se o pedido:
\end{instruction}

\subsection{Em grupo}

Mayaṃ bhante tisaraṇena sa꜕ha pañca sī꜓lāni yā꜕cāma\\
Dutiyampi mayaṃ bhante tisaraṇena sa꜕ha pañca sī꜓lāni yā꜕cāma\\
Tatiyampi mayaṃ bhante tisaraṇena sa꜕ha pañca sī꜓lāni yā꜕cāma

\subsection{Individualmente}

Ahaṃ bhante tisaraṇena sa꜕ha pañca sī꜓lāni yā꜕cāmi\\
Dutiyampi ahaṃ bhante tisaraṇena sa꜕ha pañca sī꜓lāni yā꜕cāmi\\
Tatiyampi ahaṃ bhante tisaraṇena sa꜕ha pañca sī꜓lāni yā꜕cāmi

\subsection{Tradução}

\begin{english}
  Pedimos/Peço, Venerável Mestre,\\
  \vin os Três Refúgios e os Cinco Preceitos.

  Pela segunda vez, pedimos/peço, Venerável Mestre,\\
  \vin os Três Refúgios e os Cinco Preceitos.

  Pela terceira vez, pedimos/peço, Venerável Mestre,\\
  \vin os Três Refúgios e os Cinco Preceitos.
\end{english}

\clearpage
\chapter{Os Três Refúgios}

\begin{instruction}
  Repetir, depois de o líder ter\\
  cantado as primeiras três linhas
\end{instruction}

Namo tassa bhagavato arahato sammāsambuddhassa\\
Namo tassa bhagavato arahato sammāsambuddhassa\\
Namo tassa bhagavato arahato sammāsambuddhassa

\begin{english}
  Homenagem ao Abençoado, Nobre e Perfeitamente Iluminado.\\
  Homenagem ao Abençoado, Nobre e Perfeitamente Iluminado.\\
  Homenagem ao Abençoado, Nobre e Perfeitamente Iluminado.
\end{english}

Buddhaṃ saraṇaṃ gacchāmi\\
Dhammaṃ saraṇaṃ gacchāmi\\
Saṅghaṃ saraṇaṃ gacchāmi

\begin{english}
  Tenho o Buddha como refúgio.\\
  Tenho o Dhamma como refúgio.\\
  Tenho o Saṅgha como refúgio.
\end{english}

Dutiyampi buddhaṃ saraṇaṃ gacchāmi\\
Dutiyampi dhammaṃ saraṇaṃ gacchāmi\\
Dutiyampi saṅghaṃ saraṇaṃ gacchāmi

\begin{english}
  Pela segunda vez, tenho o Buddha como refúgio.\\
  Pela segunda vez, tenho o Dhamma como refúgio.\\
  Pela segunda vez, tenho o Saṅgha como refúgio.
\end{english}

Tatiyampi buddhaṃ saraṇaṃ gacchāmi\\
Tatiyampi dhammaṃ saraṇaṃ gacchāmi\\
Tatiyampi saṅghaṃ saraṇaṃ gacchāmi

\clearpage

\begin{english}
  Pela terceira vez, tenho o Buddha como refúgio.\\
  Pela terceira vez, tenho o Dhamma como refúgio.\\
  Pela terceira vez, tenho o Saṅgha como refúgio.
\end{english}

\begin{instruction}
  Líder:
\end{instruction}

[Tisaraṇa-gamanaṃ niṭṭhitaṃ]

\begin{english}
  Fica assim completo o Triplo Refúgio.
\end{english}

\begin{instruction}
  Responso:
\end{instruction}

Āma bhante

\begin{english}
  Sim, Venerável Mestre.
\end{english}

\chapter{Os Cinco Preceitos}

\begin{instruction}
  Repetir cada preceito depois do líder
\end{instruction}

\begin{precept}
  \setcounter{enumi}{0}
  \item Pāṇātipātā vera꜓maṇī sikkhā꜓padaṃ sa꜓mādi꜕yāmi
\end{precept}

\begin{english}
  Observo o preceito de me abster de matar qualquer criatura viva.
\end{english}

\begin{precept}
  \setcounter{enumi}{1}
  \item Adinnādānā vera꜓maṇī sikkhā꜓padaṃ sa꜓mādi꜕yāmi
\end{precept}

\begin{english}
  Observo o preceito de tirar aquilo que não me for oferecido.
\end{english}

\begin{precept}
  \setcounter{enumi}{2}
  \item Kāmesu micchā꜓cārā vera꜓maṇī sikkhā꜓padaṃ sa꜓mādi꜕yāmi
\end{precept}

\begin{english}
  Observo o preceito de me abster de ter uma conduta sexual imprórpia.
\end{english}

\begin{precept}
  \setcounter{enumi}{3}
  \item Musā꜓vādā vera꜓maṇī sikkhā꜓padaṃ sa꜓mādi꜕yāmi
\end{precept}

\enlargethispage{\baselineskip}

\begin{english}
  Observo o preceito de me abster de mentir.
\end{english}

\clearpage

\begin{precept}
  \setcounter{enumi}{4}
  \item Surāmeraya-majja-pamādaṭṭhā꜓nā vera꜓maṇī sikkhā꜓padaṃ sa꜓mādi꜕yāmi
\end{precept}

\begin{english}
  Observo o preceito de me abster de consumir bebidas\\
  e drogas intoxicantes que deturpem a mente.
\end{english}

\begin{instruction}
  Líder:
\end{instruction}

[Imāni pañca sikkhā꜓padāni\\
Sī꜓lena suga꜕tiṃ yanti\\
Sī꜓lena bhoga꜕sa꜓mpadā\\
Sī꜓lena nibbu꜕tiṃ yanti\\
Tasmā꜓ sī꜓laṃ viso꜓dhaye]

\begin{english}
  Estes são os Cinco Preceitos;\\
  A virtude é fonte de felicidade,\\
  A virtude é fonte de verdadeira riqueza,\\
  A virtude é fonte de paz ---\\
  Que a virtude seja assim purificada.
\end{english}

\begin{instruction}
  Responso:
\end{instruction}

Sādhu, sādhu, sādhu

\begin{instruction}
  Fazer três prostrações
\end{instruction}

\artoptfalse

\chapter[Metta Sutta]{Metta Sutta}

\begin{leader}
  [Cantemos agora as palavras do Buddha\\ sobre o Amor e a Compaixão]
\end{leader}

Eis o que se deve fazer\\
Para cultivar a bondade\\
E seguir a via da paz:\\
Ser capaz e ser honesto,\\
Franco e gentil no falar.\\
Humilde e não arrogante,\\
Contente, facilmente satisfeito,\\
Aliviado de deveres e frugal no seu caminho.

Pacífico e sereno, sábio e inteligente,\\
Sem orgulho, sem exigência por natureza.\\
Que ele nada faça\\
Que os sábios possam vir a reprovar.\\
Desejando: Na alegria e na segurança,\\
Que todos os seres sejam felizes.\\
Quaisquer que sejam os seres vivos,\\
Fracos, fortes, sem excepção\\
Dos maiores aos mais pequenos,\\
Visíveis ou invisíveis,\\
Estejam perto ou estejam longe,\\
Nascidos ou por nascer ---\\
Que todos os seres sejam felizes!

Que ninguém engane ninguém,\\
Ou despreze alguém em que estado fôr.\\
Que ninguém por raiva ou má-fé,\\
Deseje mal a alguém.\\
Assim como uma Mãe protege o filho,\\
Com sua vida, seu único filho,\\
Assim de coração infinito,\\
Se deveria estimar todo o ser vivo;\\
Irradiando ternura por todo o mundo:\\
Acima ao mais alto céu,\\
E abaixo às profundezas;\\
Irradiante e sem limites,\\
Livre de ódio e má-fé.\\
Seja parado ou a andar,\\
Sentado ou deitado,\\
Livre de torpor,\\
Esta é uma lembrança a manter.

Diz-se esta ser a sublime permanência.\\
O puro de coração, com clareza de visão,\\
Ao não insistir em ideias fixas,\\
Liberto dos desejos dos sentidos,\\
Não voltará a nascer neste mundo.

\artopttrue

\chapter{Homenagem de encerramento}

\enlargethispage{3\baselineskip}

[Arahaṃ] sammāsambuddho bha꜕gavā\\
Buddhaṃ bha꜕gavantaṃ a꜕bhi꜓vādemi

[Svākkhā꜓to] bha꜕gava꜓tā dhammo\\
Dhammaṃ namassāmi

[Supaṭi꜕panno] bha꜕gava꜕to sāvaka꜕saṅgho\\
Sa꜓ṅghaṃ na꜕māmi


\end{document}
